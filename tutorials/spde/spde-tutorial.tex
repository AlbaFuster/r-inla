\documentclass[a4paper,11pt]{report}
\usepackage[scale={0.6,0.9},centering,includeheadfoot]{geometry}
\usepackage[T1]{fontenc}
\usepackage[utf8]{inputenc}
\usepackage{amsmath,upquote,Rd,Sweave,hyperref} 

\def\mb#1{\ensuremath{\boldsymbol{#1}}} % version: amsmath

\newcommand{\bA}{\mbox{\protect \boldmath $A$}}
\newcommand{\bB}{\mbox{\protect \boldmath $B$}}
\newcommand{\bF}{\mbox{\protect \boldmath $F$}}
\newcommand{\bD}{\mbox{\protect \boldmath $D$}}
\newcommand{\bW}{\mbox{\protect \boldmath $W$}}
\newcommand{\bC}{\mbox{\protect \boldmath $C$}}
\newcommand{\bG}{\mbox{\protect \boldmath $G$}}
\newcommand{\bK}{\mbox{\protect \boldmath $K$}}
\newcommand{\bI}{\mbox{\protect \boldmath $I$}}
\newcommand{\bQ}{\mbox{\protect \boldmath $Q$}}
\newcommand{\bM}{\mbox{\protect \boldmath $M$}}
\newcommand{\bU}{\mbox{\protect \boldmath $U$}}
\newcommand{\bV}{\mbox{\protect \boldmath $V$}}

\newcommand{\be}{\mbox{\protect \boldmath $e$}}
\newcommand{\bu}{\mbox{\protect \boldmath $u$}}
\newcommand{\bv}{\mbox{\protect \boldmath $v$}}
\newcommand{\bx}{\mbox{\protect \boldmath $x$}}
\newcommand{\by}{\mbox{\protect \boldmath $y$}}

\newcommand{\btheta}{\mbox{\boldmath $\theta$}}
\newcommand{\bbeta}{\mbox{\boldmath $\beta$}}
\newcommand{\bmu}{\mbox{\boldmath $\mu$}}
\newcommand{\bxi}{\mbox{\boldmath $\xi$}}
\newcommand{\bomega}{\mbox{\boldmath $\omega$}}
\newcommand{\boldeta}{\mbox{\boldmath $\eta$}}

\title{\textbf{The R-INLA tutorial on SPDE models}\\
Warning: work in progress...\\
Suggestions are welcome to \texttt{elias@r-inla.org}}
\author{Elias T. Krainski, Finn Lindgren, Daniel Simpson and H{\aa}vard Rue}

\begin{document}
\maketitle

\begin{abstract}
  In this tutorial we present how to fit models 
  to spatial point-referenced data, the so-called 
  geostatistical model, using INLA and SPDE. 
  After a fast introduction to such models, 
  we start with the mesh building 
  (necessary to fit the models) and show the application 
  to a toy example with details. 
  In the following examples we presents some funcionalities 
  to fit more complex models, including non-Gaussian models, 
  survival and those with two likelihoods, such as semi-continuous,  
  missaligned and preferential sampling, where we joingly model 
  the point pattern process. Also, we introduce 
  non-stationary SPDE models, and show hot wo fit 
  some space-time models, including point process, 
  lowering time dimention and dynamic spatio-temporal regression model. 
  We also briefly present the data-cloning approach.
\end{abstract} 

\section{Acknowledgments}
To Helen Sofaer for valuable English review in 
Chapters~1~and~2. 

\begin{verbatim}
2013 October 02:
 * mesh improvements:
  . use the new Finn's functions: inla.mesh.2d and inla.noncovexhull 
  . non-convex, SpatialPolygons 
 * toy-example improved (maybe more clear...) 
 * new chapters: 
  . simulation and likelihood through SPDE
  . point process 
  . preferential sampling 
  . spatio temporal example 
  . data cloning
2013 October 08: 
 * Finn's suggestions to make clear the stack with two likelihoods examples and on non-stationary

to do: 
 ** English improvement ** 

\end{verbatim}
 

\tableofcontents

\input{spde-tutorial-likelihoodintro} 
\input{spde-tutorial-spde-introduction}
\input{spde-tutorial-toy} 
\input{spde-tutorial-rain} 
\input{spde-tutorial-survival}
\input{spde-tutorial-semicontinuous} 
\input{spde-tutorial-jcovar} 
\input{spde-tutorial-nonstationar} 
\input{spde-tutorial-spacetime}
\input{spde-tutorial-lower-spatio-temporal}
\input{spde-tutorial-coregionalization}
\input{spde-tutorial-prefsampl} 
\input{spde-tutorial-stpp}
\input{spde-tutorial-dynamic}
\input{spde-tutorial-datacloning} 

\bibliographystyle{apalike}
\bibliography{spde-tutorial}

\end{document}


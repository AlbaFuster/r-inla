
\begin{abstract}
This tutorial will show you how to fit models 
that contains at least one effect specified 
from an SPDE using the `R-INLA`. 
Up to now, it can an SPDE based model can be applied to model 
random effects over continuous one- or two- dimensional domains. 
However, the theory works for higher dimensional cases. 
The usual application is data whose geographical 
location is explicitly considered in the analysis. 
This tutorial explores 'R-INLA' functionalities by using examples. 
It starts with simple models and increases in complexity. 

In Chapter~\ref{ch:intro} includes a section introducing the 
random field models and the Mat\'{e}rn class. 
We illustrate some features of this class in figures. 
The we introduce the main results in \cite{lindgrenRL:2011}
intuitively linking to images of the matrices involved being 
computed for a illustrative small case with few spatial locations.
We show how to fit a geostatistical model for a simulated data, 
the toy example, covering from the mesh building, model definition, 
data preparation, showing results, doing predictions and 
considering results from different meshes. 
We also show how to build a mesh considering non-convex domains, 
spatial polygons objects and domains with holes or physical boundaries. 

In Chapter~\ref{ch:ngns} we consider three examples. 
We consider the daily average rainfall from rainfall collected at 
616 gauge stations in the Paran\'{a} state in Brasil over year 2011. 
For this data we show a detailed analysis including code to compute 
geographical covariates, smoothed regression an prediction. 
The second example in this Chapter consider survival analysis for 
the Leukaemia dataset, analysed in \cite{hendersonSG:2003}. 
We show how to consider the parametric Weibull case and 
also the non-parametric Cox proportional hazard case. 
For this case we have the implementation internally considers 
a new structure of the data in order to perform a Poisson regression. 
The last example in this Chapter considers simulated data to 
illustrate the approach of modeling 
the SPDE model parameters by a regression which is the 
case of having covariates in the covariance, 
proposed in \cite{ingebrigtsenLS:2013}. 

In Chapter~\ref{ch:manipula} we have a collection of examples 
were copy random fields to model two or more outcomes jointly. 
It includes a measurement error model in order to account for 
spatially structured measurement error in a covariate. 
A coregionalization model consider the case for three outcomes 
were the fist outcome is in the linear predictor for the second 
one and both are in the predictor for the third outcome, 
as proposed in \cite{schmidtG:2003}. 
An example considering copying a part or the entire 
linear predictor from one outcome in a linear predictor 
to another one ends this chapter. 
It shows a slight different way from the coregionalization model 
to jointly model three outcomes. 

The log Cox point process model is considered in Chapter~\ref{ch:lcox}. 
In this case we show how to fit a Log-cox point process using 
the direct approximation for the likelihood as proposed 
in \cite{simpsonetal:2016}. 
We also take the opportunity to show how to consider the 
joint modeling of the process and the locations, 
under the preferential sampling as proposed in \cite{diggleetal:2010}. 

Finally, Chapter~\ref{ch:spacetime} presents several cases to 
example analysis of space-time data. 
We start by an example having discrete time domain as in 
\cite{camelettietal:2012}. 
We extend this example considering the time as continuous 
by considering time knots along with temporal function basis 
functions for projection. 
We also extend the coregionalization example for the 
space-time in this Chapter. 
The space-time model is also applied for modeling regression coefficients 
in a dynamic regression example having the regression coefficients varying over space-time. 
We consider the space-time version of the log-Cox point process model for 
a dataset and also illustrates an approach to deal whit the case of having 
a large space-time point process data.

Since this tutorial is more a collection of examples, one should 
start with the tutorial marked as \textbf{Read this first!} at the 
tutorials link in the R-INLA web page, \url{http://www.r-inla.org}, 
more precisely at  
\url{http://www.r-inla.org/examples/tutorials/spde-tutorial-from-jss}.
If you are in a rush to fit a simple geostatistical model, 
we made a short tutorial without the details as a vignette in the \textbf{\textsf{INLA}}. Thus one can have it just typing   
\texttt{vignette(SPDEhowto)} for a two dimensional example 
or \texttt{vignette(SPDE1d)} for a one dimensional example. 
We built a Shiny application to help one to understand the mesh building. 
It depends on the \textbf{\textsf{shiny}} package. 
This application opens by typing \texttt{demo(mesh2d)}. 

This content is part of the book available at 
\url{http://www.r-inla.org/spde-book}, 
whose Gitbook version is freely available 
along all the code and datasets. 

\end{abstract}

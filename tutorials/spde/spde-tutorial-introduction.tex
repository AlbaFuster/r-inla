
\chapter{Introduction} 

A point-referenced dataset is made up of any 
data measured at known locations. 
These locations may be in any coordinate reference system, 
most often the longitude and latitude coordinates. 
Point-referenced data are very common in 
many areas of science. 
This type of data appears in mining, climate modeling, 
ecology, agriculture and other areas. 
If we want to model the data while incorporating 
the information about where the data are from, 
we need a model for point referenced data. 

It is possible to build a regression model 
using each coordinate as a covariate. 
But in some cases it is necessary to include a very 
complicated function based on the coordinates to 
get an adequate description of the mean. 
For example, we may need complex non-linear function or 
a non-parametric function. 
This type of model only incorporates a
trend on the mean based on the coordinates. 
Also, this type of model is a fixed effect model. 

Instead, it is more natural for a model to measure 
the first law of geography in a simple way. 
This law says: ``Everything is related to everything else, 
but near things are more related than distant things'', 
\cite{tobler:1970}. 
So, we need a model that incorporates the property that an 
observation is more correlated with an observation 
collected at a neighboring location than with another that 
is collected at more distant location. 
One option to model this dependence is to use 
a spatially-structured random effect. 
This type of model incorporates spatial dependency, 
rather than simply the spatial trend. 
However, it is possible to include both terms in a model. 
Spatial dependency can be accounted for 
within more general models using 
spatially structured random effects. 

In spatial statistics, different models are used to incorporate 
spatial dependency depending on whether the locations are areas 
(states, cities, etc.) or whether the locations are points. 
In the latter case, the locations can be fixed or random. 
Models of point-referenced data that include a 
spatially-structured random effect are commonly called 
geostatistical models. 
Geostatistics is the specific area of spatial statistics 
that these models. 
See \cite{cressie:1993} for a good introduction to spatial statistics. 

\section{The Gaussian random field}

To introduce some notation, let $s$ be any 
location in the study area and let $X(s)$ be 
the random effect at that location. 
$X(s)$ is a stochastic process, with $s\in \bD$, 
were $\bD$ is the domain of the study area and $\bD\in \Re^d$. 
Suppose, for example, that $\bD$ is a specific country 
and we have data measured at geographical 
locations, over $d=2$ dimensions, within this country. 

Suppose we assume that we have a realization 
of $x(s_i)$, $i=1,2,...,n$, a realization of $X(s)$ 
in $n$ locations. It is commonlly assumed that 
$x(s)$ has a multivariate Gaussian distribution. 
Also, if we assume that $X(s)$ is continuous over space, 
we have a continuously indexed Gaussian field (GF). 
This implies that it is possible to collect these 
data at any location within the study region. 
To complete the specification of the distribution of 
$x(s)$, is necessary to define its mean and covariance. 

A very simple option is to define a 
correlation function based only on euclidean 
distance between locations. 
This assumes that if we have two pairs of 
points separated by the same distance $h$, both pairs 
have same correlation. 
It is intuitive to choose any function 
decreasing with $h$. 
There is some work about the GF and correlation 
functions in \cite{abrahamsen:1997}. 

A very popular correlation function is the 
Mat\'ern correlation function, which depends 
on a scale parameter $\kappa>0$ and a smoothness 
parameter $\nu>0$. 
Considering two locations $s_i$ and $s_j$, the 
stationary and isotropic Mat\'ern correlation function is: 
\begin{equation}
Cor_M(X(s_i), X(s_j)) = 
\frac{2^{1-\nu}}{\Gamma(\nu)}
(\kappa \parallel s_i - s_j\parallel)^\nu 
K_\nu(\kappa \parallel s_i - s_j \parallel)
\end{equation}
where $\parallel . \parallel$ denotes 
the euclidean distance and $K_\nu$ is the modified 
Bessel function of the second order. 
The Mat\'ern covariance function is
$\sigma_x Cor(X(s_i), X(s_j))$, where 
$\sigma_x$ is the marginal variance of the process. 

If we have a realization $x(s)$ at $n$ locations, 
we can define its joint covariance matrix. 
Each entry of this joint covariance matrix $\Sigma$ is 
$\Sigma_{i,j} = \sigma_xCor_M(X(s_i), X(s_j))$. 
It is common to assume that $X(x)$ has a zero mean. 
We have now completely defined a multivariate 
distribution for $x(s)$. 

Now, suppose that we have data $y_i$ observed at 
locations $s_i$, $i=1,...,n$. 
If an underlying GF generated these data, we can fit the 
parameters of this process, based on the identity 
$y(s_i) = x(s_i)$, where $y(s_i)$ is a 
realization of the GF. 
In this case, the likelihood function 
is the multivariate distribution with 
mean $\mu_x$ and covariance $\Sigma$. 
If we assume $\mu_x = \beta_0$, 
we have four parameters to estimate. 

In many situations we assume that we have an underlying 
GF but cannot directly observe it and instead observe 
data with measurement error, i.e., 
$y(s_i) = x(s_i) + e_i$. 
It is common to assume that $e_i$ is 
independent of $e_j$ for all $i\neq j$ and 
$e_i \sim N(0, \sigma_e)$. 
This additional parameter, $\sigma_e$, measures 
the noise, called the nugget effect. 
In this case the covariance of the marginal distribution 
of $y(s)$ is $\sigma^2_eI + \Sigma$. 
This model is a short extension of the basic GF model, 
and in this case, we have one additional parameter 
to estimate. 
To look more about this model 
see \cite{diggleribeiro:2007}. 

It is possible to describe this model within a 
larger class of models, the hierarchical models. 
Suppose that we have observations $y_i$ on locations 
$s_i$, $i=1,...,n$. We start with 
\begin{equation}\begin{array}{c}
y_i|\theta,\beta,x_i,F_i \sim P(y_i|\mu_i,\phi) \\
\bbx \sim GF(0, \Sigma)
\end{array}\end{equation}
where $\mu_i = h(F_i^{T}\beta + x_i)$, 
$F$ is a matrix of covariates, 
$x$ is the random effect, 
$\theta$ are parameters of the random effect, 
$\beta$ are covariate coefficients, 
$h()$ is a function mapping the linear predictor 
$F_i^{T}\beta + x_i$ to $E(y_i) = \mu_i$ and 
$\phi$ is a dispersion parameter of the distribution, 
in the exponential family, which is assumed for $y_i$. 
To write the GF with a nugget effect, 
we replace $\beta_0$ with $F_i^{T}\beta$, 
assume a Gaussian distribution for $y_i$, with 
variance $\sigma_e^2$ and $x$ as a GF.

We can extend this basic hierarchical model in many ways,
and we return to some extensions in later sections. 
If we know the properties of the GF, 
we can study all the practical models 
that contain, or are based on, this random effect. 

It is worth mentioning that the data, or the random effect,  
on a finite number of $n$ points where we have observed 
data are considered a realization of a 
multivariate Gaussian distribution. 
Therefore, to evaluate the likelihood function, or the random 
effect distribution, we need to compute the 
multivariate Gaussian density. 
So, we have, in the log scale, the expression 
\begin{equation}
-\frac{1}{2}\left(n\log(2\pi) + \log(|\Sigma|) 
+ [x(s) - \mu_x]^{T}\Sigma^{-1}[x(s)-\mu_x]\right)
\end{equation}
where $\Sigma$ is a dense $n\times n$. 
To compute this, we need a factorization of this matrix. 
Because this matrix is a dense, this is a operation of 
order $O(n^3)$, so this is a 'big n problem'. 

An alternative used in some software for 
geostatistical analysis is to use the empirical 
variogram to fit the parameters of the correlation 
function. This option does not use any likelihood for 
the data or uses a multivariate Gaussian 
distribution for the spatially structured random effect. 
A good description of these techniques is made 
in \cite{cressie:1993}. 

However, it is adequate to assume a likelihood for the 
data and a GF for the spatial dependence as in the model 
based approach proposed for geostatistics, \cite{diggleribeiro:2007}. 
At times we need to use the multivariate 
Gaussian distribution for the random effect. 
But, if the dimension of the GF is big, 
it becomes impractical for model-based inference 
to directy use the covariance as defined previously. 

In another area of the spatial statistics, the 
analysis of areal data, there are models specified 
by conditional distributions that imply a joint 
distribution with a sparse precision matrix. 
These models are called Gaussian Markov random 
fields (GMRF), \cite{RueHeld:2005}. 
It is easier to make Bayesian inference when we use a GMRF 
than when we use the GF, because to work with two dimensional 
GMRF models we have a cost of $O(n^{3/2})$ 
on the computations with its precision matrix. 
This makes it easier to conduct analyses with big 'n'. 

\documentclass[a4paper,11pt]{article}
\usepackage[scale={0.8,0.9},centering,includeheadfoot]{geometry}
\usepackage{amstext}
\usepackage{listings}
\begin{document}

\section*{Linkmodel: robit}

\subsection*{Parametrization}

This is the link that map $p\in (0,1)$ into $x\in\Re$, where
\begin{displaymath}
    F_{\nu}(x) = p
\end{displaymath}
and $F_{\nu}$ is the cummulative distribution function for Student-t
with $\nu$ degrees of freedom, normalized to have unit variance and
$\nu > 2$.

\subsection*{Hyperparameters}

The  parameter $\nu$ represented as
\begin{displaymath}
    \nu = 2 + \exp(\theta)
\end{displaymath}
and the prior is defined on $\theta$. $\nu$ is default fixed and set
to $7$ (to estimate $\nu$ is somewhat challenging).

\subsection*{Specification}

Use \texttt{model="robit"} within \texttt{control.link}.  

\subsubsection*{Hyperparameter spesification and default values}
%% DO NOT EDIT!
%% This file is generated automatically from models.R
\begin{description}
	\item[doc] \verb!Robit link!
	\item[hyper]\ 
	 \begin{description}
	 	\item[theta]\ 
	 	 \begin{description}
	 	 	\item[hyperid] \verb!49021!
	 	 	\item[name] \verb!log degrees of freedom!
	 	 	\item[short.name] \verb!dof!
	 	 	\item[initial] \verb!1.6094379124341!
	 	 	\item[fixed] \verb!TRUE!
	 	 	\item[prior] \verb!pc.dof!
	 	 	\item[param] \verb!50 0.5!
	 	 	\item[to.theta] \verb!function(x) log(x - 2)!
	 	 	\item[from.theta] \verb!function(x) 2 + exp(x)!
	 	 \end{description}
	 \end{description}
	\item[status] \verb!experimental!
	\item[pdf] \verb!robit!
\end{description}


\subsection*{Example}

\begin{verbatim}
n = 300
Nt = 2
x = rnorm(n, sd = 0.3)
eta = 1 + x
df = 7
y = rbinom(n, size=Nt, prob = inla.link.invrobit(eta, df = df))

r = inla(y ~ 1 + x,
         family = "binomial",
          Ntrials = Nt,
          data = data.frame(y, x, Nt),
          control.family = list(
              control.link = list(
                  model = "robit",
                  hyper = list(dof = list(
                      initial = log(df - 2), 
                      fixed = FALSE)))))
summary(r)
\end{verbatim}

\subsection*{Notes}

\begin{itemize}
\item The link-function is also available as R-functions
    \texttt{inla.link.robit} and \texttt{inla.link.invrobit}
\item This link-model is experimental for the moment.
\end{itemize}

\end{document}


% LocalWords: 

%%% Local Variables: 
%%% TeX-master: t
%%% End: 

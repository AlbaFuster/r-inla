\documentclass[a4paper,11pt]{article}
\usepackage[scale={0.8,0.9},centering,includeheadfoot]{geometry}
\usepackage{amstext}
\usepackage{listings}
\begin{document}

\section*{Linkmodel: sn}

\subsection*{Parametrization}

This is the link that map $p\in (0,1)$ into $x\in\Re$, where
\begin{displaymath}
    F_{a}(x) = p
\end{displaymath}
and $F_{a}$ is the cummulative distribution function for the
skew-normal distribution,
\begin{displaymath}
    2\phi(x)\Phi(a^{1/3}x)
\end{displaymath}
which is renormalized to have zero mean and unit variance.

\subsection*{Hyperparameters}

The  parameter $a$ represented as
\begin{displaymath}
    a = a_{\text{max}} \left(2\frac{\exp(\theta)}{1+ \exp(\theta)} -1\right)
\end{displaymath}
and the prior is defined on $\theta$. There is a PC prior available
for $\theta$. The (absolute) bound of
$a_{\text{max}} = 3.2^{3} = 32.768$, is there for for stability
reasons\footnote{This constant is defined as \texttt{LINK\_SN\_AMAX}
    in the file \texttt{inla.h}.}. The PC-prior is corrected for
this bound, whereas the pc-prior in the R-functions
\texttt{inla.pc.\{r,p,q,d\}sn} does not define a such bound.

\subsection*{Specification}

Use \texttt{model="sn"} within \texttt{control.link}.  

\subsubsection*{Hyperparameter spesification and default values}
%% DO NOT EDIT!
%% This file is generated automatically from models.R
\begin{description}
	\item[doc] Skew-normal link
	\item[hyper]\ 
	 \begin{description}
	 	\item[theta]\ 
	 	 \begin{description}
	 	 	\item[hyperid] 49031
	 	 	\item[name] skew
	 	 	\item[short.name] skew
	 	 	\item[initial] 0
	 	 	\item[fixed] TRUE
	 	 	\item[prior] pc.sn
	 	 	\item[param] 50
	 	 	\item[to.theta] \verb!function(x, skew.max = 0.99) log((1+x/skew.max)/(1-x/skew.max))!
	 	 	\item[from.theta] \verb!function(x, skew.max = 0.99) skew.max*(2*exp(x)/(1+exp(x))-1)!
	 	 \end{description}
	 \end{description}
	\item[status] experimental
	\item[pdf] linksn
\end{description}


\subsection*{Example}

\begin{verbatim}
\end{verbatim}

\subsection*{Notes}

\begin{itemize}
\item The link-function is also available as R-functions
    \texttt{inla.link.sn} and \texttt{inla.link.invsn}
\item This link-model is experimental for the moment.
\end{itemize}

\end{document}


% LocalWords: 

%%% Local Variables: 
%%% TeX-master: t
%%% End: 

\documentclass[a4paper,11pt]{article}
\usepackage[scale={0.8,0.9},centering,includeheadfoot]{geometry}
\usepackage{amstext}
\usepackage{listings}
\usepackage{verbatim}
\begin{document}

\section*{Linkmodel: powerlogit}

\subsection*{Parametrization}

This is the link that map $p\in (0,1)$ into $x\in\Re$, where
\begin{displaymath}
    F^{\beta}(x) = p, \qquad \beta > 0
\end{displaymath}
and $F(x)$ is the cummulative distribution function for the
logit,
\begin{displaymath}
    F(x) = \frac{1}{1+\exp(-x)}.
\end{displaymath}
This link is renormalized so its corresponding density have zero mean
and unit variance for every value of $\beta$.

\subsection*{Hyperparameters}

The parameter $\beta$ represent the power
\begin{displaymath}
    \beta = \exp(\theta_1)
\end{displaymath}
and the prior is defined on $\theta_1$. 

The intercept is represented by a quantile level $\alpha$, where
\begin{displaymath}
    \alpha = \frac{\exp(\theta_2)}{1 + \exp(\theta_2)}
\end{displaymath}
and the prior is defined on $\theta_2$.

\subsection*{Specification}

Use \texttt{model="powerlogit"} within \texttt{control.link}.  

\subsubsection*{Hyperparameter spesification and default values}
%% DO NOT EDIT!
%% This file is generated automatically from models.R
\begin{description}
	\item[doc] Power logit link
	\item[hyper]\ 
	 \begin{description}
	 	\item[theta1]\ 
	 	 \begin{description}
	 	 	\item[hyperid] 49131
	 	 	\item[name] power
	 	 	\item[short.name] power
	 	 	\item[initial] 0.00123456789
	 	 	\item[fixed] FALSE
	 	 	\item[prior] normal
	 	 	\item[param] 0 10
	 	 	\item[to.theta] \verb!function(x) log(x)!
	 	 	\item[from.theta] \verb!function(x) exp(x)!
	 	 \end{description}
	 	\item[theta2]\ 
	 	 \begin{description}
	 	 	\item[hyperid] 49132
	 	 	\item[name] intercept
	 	 	\item[short.name] intercept
	 	 	\item[initial] 0
	 	 	\item[fixed] FALSE
	 	 	\item[prior] logitbeta
	 	 	\item[param] 1 1
	 	 	\item[to.theta] \verb!function(x) log(x / (1 - x))!
	 	 	\item[from.theta] \verb!function(x) exp(x) / (1 + exp(x))!
	 	 \end{description}
	 \end{description}
	\item[pdf] powerlogit
\end{description}


\subsection*{Example}

\verbatiminput{powerlogit-example.R}
    
\subsection*{Notes}

\begin{itemize}
\item This link is EXPERIMENTAL, and new default priors will be added
    in the near future
\item Setting the initial value for the hyperparameter ``intercept''
    to infinity, will remove the intercept from the link-model.
\end{itemize}

\end{document}


% LocalWords: 

%%% Local Variables: 
%%% TeX-master: t
%%% End: 

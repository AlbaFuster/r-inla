\documentclass[a4paper,11pt]{article}
\usepackage[scale={0.8,0.9},centering,includeheadfoot]{geometry}
\usepackage{amstext}
\usepackage{amsmath}
\usepackage{verbatim}

\begin{document}
\section*{Negative Binomial}

\subsection*{Parametrisation}

The negative Binomial distribution is
\begin{displaymath}
    \text{Prob}(y) = \frac{\Gamma(y+n)}{\Gamma(n) \Gamma(y+1)} p^n (1-p)^y
\end{displaymath}
for responses $y=0, 1, 2, \ldots$, where
\begin{description}
\item[$n$:] number of successful trials (\emph{size}), or dispersion
    parameter. Must be strictly positive, need not be integer.
\item[$p$:] probability of success in each trial.
\end{description}

\subsection*{Link-function}

The mean and variance of $y$ are given as
\begin{displaymath}
    \mu = n \frac{1-p}{p} \qquad\text{and}\qquad \sigma^{2} = \mu(1 + \frac{\mu}{n})
\end{displaymath}
and the mean is linked to the linear predictor by
\begin{displaymath}
    \mu = E \exp(\eta)
\end{displaymath}
where the hyperparameter $n$ (\emph{size}) plays the role of an
dispersion parameter. $E$ represents knows constant and $\log(E)$ is
the offset of $\eta$.

\subsection*{Hyperparameters}

The default parameterization (\verb|variant=0|) is that the dispersion
parameter $n$ (\emph{size}) is represented as
\begin{displaymath}
    n = \exp(\theta)
\end{displaymath}
and the prior is defined on $\theta$.

If \verb|variant=1|, then dispersion parameter $n$ (\emph{size}) is
represented as
\begin{displaymath}
    n = E\exp(\theta)
\end{displaymath}
and the prior is defined on $\theta$.

If \verb|variant=2|, then dispersion parameter $n$ (\emph{size}) is
represented as
\begin{displaymath}
    n = S\exp(\theta)
\end{displaymath}
where $S$ is \texttt{scale}, and the prior is defined on $\theta$.

\subsection*{Specification}

\begin{itemize}
\item $\text{family}=\texttt{nbinomial}$
\item Required arguments: $y$ and $E$ (default $E=1$) and
    \texttt{scale} (default \texttt{scale}=1)
\item Chose variant with either
    \verb|control.family = list(variant=0)| (default) or\\
    \verb|control.family = list(variant=1)| or \verb|control.family = list(variant=2)|.
    
\end{itemize}

\subsubsection*{Hyperparameter spesification and default values}
\documentclass[a4paper,11pt]{article}
\usepackage[compat2]{geometry}
\usepackage{amstext}
\usepackage{amsmath}
\usepackage{verbatim}

\begin{document}
\section*{Negative Binomial}

\subsection*{Parametrisation}

The negative Binomial distribution is
\begin{displaymath}
    \text{Prob}(y) = \frac{\Gamma(y+n)}{\Gamma(n) \Gamma(y+1)} p^n (1-p)^y
\end{displaymath}
for responses $y=0, 1, 2, \ldots$, where
\begin{description}
\item[$n$:] number of successful trials, or dispersion
    parameter. Must be strictly positive, need not be integer.
\item[$p$:] probability of success in each trial.
\end{description}

\subsection*{Link-function}

The mean and variance of $y$ are given as
\begin{displaymath}
    \mu = n \frac{1-p}{p} \qquad\text{and}\qquad \sigma^{2} = \mu(1 + \frac{\mu}{n})
\end{displaymath}
and the mean is linked to the linear predictor by
\begin{displaymath}
    \mu = E \exp(\eta)
\end{displaymath}
where the hyperparameter $n$ (or the \emph{size}) plays the role of an
overdispersion parameter. $E$ represents knows constant and $\log(E)$
is the offset of $\eta$.

\subsection*{Hyperparameters}

The overdispersion parameter $n$ is represented as
\begin{displaymath}
    \theta = \log(n)
\end{displaymath}
and the prior is defined on $\theta$. 

\subsection*{Specification}

\begin{itemize}
\item $\text{family}=\texttt{nbinomial}$
\item Required arguments: $y$ and $E$ (default $E=1$).
\end{itemize}

\subsection*{Example}

In the following example we estimate the parameters in a simulated
example with negative binomial responses and assign the hyperparameter
$\theta$ a Gaussian prior with mean $0$ and precision $0.01$
\verbatiminput{example-nbinomial.R}

\subsection*{Notes}

As $n\rightarrow\infty$, the negative Binomial converges to the
Poisson distribution. For numerical reasons, if $n$ is too large:
\begin{displaymath}
    \frac{\mu}{n} < 10^{-4},
\end{displaymath}
then the Poisson limit is used.


\end{document}


% LocalWords:  hyperparameter overdispersion Hyperparameters nbinomial

%%% Local Variables: 
%%% TeX-master: t
%%% End: 



\subsection*{Example}

In the following example we estimate the parameters in a simulated
example with negative binomial responses and assign the hyperparameter
$\theta$ a Gaussian prior with mean $0$ and precision $0.01$
\verbatiminput{example-nbinomial.R}

\subsection*{Notes}

As $n\rightarrow\infty$, the negative Binomial converges to the
Poisson distribution. For numerical reasons, if $n$ is too large:
\begin{displaymath}
    \frac{\mu}{n} < 10^{-4},
\end{displaymath}
then the Poisson limit is used.

The PC-prior is available for \verb|variant=1|, see
\verb|inla.doc("pc.gamma")|.

\clearpage


\subsection*{The \texttt{nbinomial2} distribution}

The negative Binomial distribution is also available in its ``pure
form'', as the number of excess experiments to get $n$ successes with
a success in the last experiment,
\begin{displaymath}
    \text{Prob}(y) = {y + n -1 \choose n-1} (1-p)^{y} p^{n}
\end{displaymath}
for $y=0, 1, 2, \ldots$, where
\begin{description}
\item[$n$:] the (fixed) number of successes before stopping,
    $n=1, 2, \ldots$,
\item[$p$:] the probability of success in each independent trial.
\end{description}

\subsection*{Link-function}

The probability $p$ is (by default) linked to the linear predictor
$\eta$ as
\begin{displaymath}
    p = \frac{\exp(\eta)}{1+\exp(\eta)}.
\end{displaymath}

\subsection*{Hyperparameters}
None.

\subsection*{Specification}

\begin{itemize}
\item $\text{family}=\texttt{nbinomial2}$
\item Required arguments: $y$ and $Ntrials$
\end{itemize}
The argument \texttt{Ntrials} gives the value of $n$ for each
observation.

\subsubsection*{Hyperparameter spesification and default values}
%% DO NOT EDIT!
%% This file is generated automatically from models.R
\begin{description}
	\item[doc] The negBinomial2 likelihood
	\item[hyper]\ 
	\item[survival] FALSE
	\item[discrete] TRUE
	\item[link] default logit loga cauchit probit cloglog loglog
	\item[pdf] nbinomial
\end{description}


\subsection*{Example}
\verbatiminput{example-nbinomial2.R}



\end{document}


% LocalWords:  hyperparameter dispersion Hyperparameters nbinomial

%%% Local Variables: 
%%% TeX-master: t
%%% End: 

\documentclass[a4paper,11pt]{article}
\usepackage[scale={0.8,0.9},centering,includeheadfoot]{geometry}
\usepackage{amstext}
\usepackage{amsmath}
\usepackage{verbatim}

\begin{document}
\section*{The Kumaraswamy distribution}

\subsection*{Parametrisation}

The Kumaraswamy distribution is
\begin{displaymath}
    f(y) = \alpha \beta y^{\alpha-1}(1-y^{\alpha})^{\beta-1}
\end{displaymath}
for $0<y<1$ and $\alpha, \beta > 0$. The cummulative distribution
function is
\begin{displaymath}
    F(y) = 1-(1-y^{\alpha})^{\beta}.
\end{displaymath}
The parametrisation is given in terms of the quantile function
\begin{displaymath}
    \kappa(q) = \left(1-(1-q)^{1/\beta}\right)^{1/\alpha}
\end{displaymath}
and the precision parameter $\phi$,
\begin{displaymath}
    \phi(q) = -\ln\left(1-(1-q)^{1/\beta}\right)
\end{displaymath}
for \emph{fixed} value of $0<q<1$.

\subsection*{Link-function}

The quantile $\kappa$ to the linear predictor by
\begin{displaymath}
    \text{logit}(\kappa) = \eta
\end{displaymath}
using the default logit link-function. 

\subsection*{Hyperparameters}

The hyperparameter is
\begin{displaymath}
    \theta_1 = log(\phi)
\end{displaymath}
and the prior is given for $\theta_1$.

For technical reasons, the fixed value of the quantile $q$ is given as
a second hyperparameter,
\begin{displaymath}
    \theta_2 = q
\end{displaymath}
for which its initial value must be given. 


\subsection*{Specification}

\begin{itemize}
\item $\text{family}=\texttt{kumar}$
\item Required arguments: $y$ 
\end{itemize}

\subsubsection*{Hyperparameter spesification and default values}
%% DO NOT EDIT!
%% This file is generated automatically from models.R
\begin{description}
	\item[hyper]\ 
	 \begin{description}
	 	\item[theta1]\ 
	 	 \begin{description}
	 	 	\item[hyperid] 60001
	 	 	\item[name] precision parameter
	 	 	\item[short.name] prec
	 	 	\item[initial] 0
	 	 	\item[fixed] FALSE
	 	 	\item[prior] loggamma
	 	 	\item[param] 1 0.001
	 	 	\item[to.theta] \verb!function(x) log(x)!
	 	 	\item[from.theta] \verb!function(x) exp(x)!
	 	 \end{description}
	 	\item[theta2]\ 
	 	 \begin{description}
	 	 	\item[hyperid] 60002
	 	 	\item[name] quantile
	 	 	\item[short.name] q
	 	 	\item[initial] 0.5
	 	 	\item[fixed] TRUE
	 	 	\item[prior] invalid
	 	 	\item[param] 
	 	 	\item[to.theta] \verb!function(x) x!
	 	 	\item[from.theta] \verb!function(x) x!
	 	 \end{description}
	 \end{description}
	\item[survival] FALSE
	\item[discrete] FALSE
	\item[link] default logit
	\item[pdf] kumar
\end{description}




\subsection*{Example}

\verbatiminput{example-kumar.R}

\subsection*{Notes}

None.


\end{document}


% LocalWords: 

%%% Local Variables: 
%%% TeX-master: t
%%% End: 

\documentclass[a4paper,11pt]{article}
\usepackage[scale={0.8,0.9},centering,includeheadfoot]{geometry}
\usepackage{amstext}
\usepackage{amsmath}
\usepackage{verbatim}

\begin{document}
\section*{Skew-Normal likelihood}

\subsection*{Parametrisation}

The standardised Skew-Normal distribution is
\begin{displaymath}
    f(z) = \frac{2}{\omega_{\alpha}}
    \phi\left(\frac{z -\xi_{\alpha}}{\omega_{\alpha}}\right)
    \Phi\left(\alpha \frac{z -\xi_{\alpha}}{\omega_{\alpha}}\right)
\end{displaymath}
where $\omega_\alpha$ and $\xi_\alpha$ are so that the mean is zero
and variance is one, and they depends both on the skewness parameter
$\alpha$.

The skew-normal likelihood is defined as the density wrt $y$, where
\begin{displaymath}
    z = (y - \eta)\sqrt{w \tau} \;\sim\; f(z)
\end{displaymath}
and
\begin{description}
\item[$\eta$:] is the the linear predictor
\item[$\tau$:] is the precision
\item[$w$:] is a fixed scale or weight, $w>0$,
\end{description}

\subsection*{Link-function}

The mean equals the linear predictor
\begin{displaymath}
    \mu = \eta
\end{displaymath}

\subsection*{Hyperparameters}

The precision is represented as
\begin{displaymath}
    \theta_{1} = \log \tau
\end{displaymath}
and the prior is defined on $\theta_{1}$. 

The (standardised) skewness $\gamma$, is represented as
\begin{displaymath}
    \gamma=0.988 (2\frac{\exp(\theta_2)}{1+\exp(\theta_2)}-1)
\end{displaymath}
and the prior is defined on $\theta_{2}$. The standardised skewness
depends on $\alpha$ as
\begin{displaymath}
\gamma ={\frac  {4-\pi }{2}}{\frac{\left(\delta {\sqrt  {2/\pi
              }}\right)^{3}}{\left(1-2\delta^{2}/\pi
        \right)^{{3/2}}}}, \qquad
    \delta ={\frac{\alpha}{{\sqrt{1+\alpha^{2}}}}}
\end{displaymath}


\subsection*{Specification}

\begin{itemize}
\item \texttt{family="sn"}
\item Required arguments: $y$ and $w$ (keyword \texttt{scale}, and 
    $w=1$ by default).
\end{itemize}

\subsubsection*{Hyperparameter specification and default values}
%% DO NOT EDIT!
%% This file is generated automatically from models.R
\begin{description}
	\item[doc] Skew-normal link
	\item[hyper]\ 
	 \begin{description}
	 	\item[theta]\ 
	 	 \begin{description}
	 	 	\item[hyperid] 49031
	 	 	\item[name] skew
	 	 	\item[short.name] skew
	 	 	\item[initial] 0
	 	 	\item[fixed] TRUE
	 	 	\item[prior] pc.sn
	 	 	\item[param] 50
	 	 	\item[to.theta] \verb!function(x, skew.max = 0.99) log((1+x/skew.max)/(1-x/skew.max))!
	 	 	\item[from.theta] \verb!function(x, skew.max = 0.99) skew.max*(2*exp(x)/(1+exp(x))-1)!
	 	 \end{description}
	 \end{description}
	\item[status] experimental
	\item[pdf] linksn
\end{description}



\subsection*{Example}

\verbatiminput{example-sn.R}

\subsection*{Notes}

\begin{itemize}
\item This implementation replaces older ones ("sn" and "sn2") from
    $16^{\text{\emph{th}}}$ September 2020.
\item A $N(a,0)$ prior is interpreted as a constant prior with density
    equal to one.
\end{itemize}

\end{document}

% LocalWords:  np Hyperparameters Ntrials gaussian Parametrisation skewness sn

%%% Local Variables: 
%%% TeX-master: t
%%% End: 
% LocalWords:  quantile probit Hyperparameter th

\documentclass[a4paper,11pt]{article}
\usepackage[scale={0.8,0.9},centering,includeheadfoot]{geometry}
\usepackage{amstext}
\usepackage{amsmath}
\usepackage{verbatim}

\begin{document}
\section*{Censored Negative Binomial (version2)}

\subsection*{Parametrisation}

The negative Binomial distribution is
\begin{displaymath}
    \text{Prob}(y) = \frac{\Gamma(y+n)}{\Gamma(n) \Gamma(y+1)} p^n (1-p)^y
\end{displaymath}
for responses $y=0, 1, 2, \ldots$, where
\begin{description}
\item[$n$:] number of successful trials (\emph{size}), or dispersion
    parameter. Must be strictly positive, need not be integer.
\item[$p$:] probability of success in each trial.
\end{description}

The cencoring is that response values in the interval $L \le y \le H$
are cencored (and reported as a value in the interval, like $y = L$,
say), whereas other values are reported as is.

This function is the nbinomial version of \texttt{cenpoisson2}.

\subsection*{Link-function}

The mean and variance of (the uncensored) $y$ are given as
\begin{displaymath}
    \mu = n \frac{1-p}{p} \qquad\text{and}\qquad \sigma^{2} = \mu(1 + \frac{\mu}{n})
\end{displaymath}
and the mean is linked to the linear predictor by
\begin{displaymath}
    \mu = E \exp(\eta)
\end{displaymath}
where the hyperparameter $n$ (\emph{size}) plays the role of an
dispersion parameter. $E$ represents knows constant and $\log(E)$ is
the offset of $\eta$.

\subsection*{Hyperparameters}

The default parameterization (\verb|variant=0|) is that the dispersion
parameter $n$ (\emph{size}) is represented as
\begin{displaymath}
    n = \exp(\theta)
\end{displaymath}
and the prior is defined on $\theta$.

If \verb|variant=1|, then dispersion parameter $n$ (\emph{size}) is
represented as
\begin{displaymath}
    n = E\exp(\theta)
\end{displaymath}
and the prior is defined on $\theta$.

If \verb|variant=2|, then dispersion parameter $n$ (\emph{size}) is
represented as
\begin{displaymath}
    n = S\exp(\theta)
\end{displaymath}
where $S$ is \texttt{scale}, and the prior is defined on $\theta$.

\subsection*{Specification}

\begin{itemize}
\item \texttt{family="nbinomial"}
\item Required arguments: $(y,L,H)$ [see below] and $E$ (default
    $E=1$) and \texttt{scale} (default \texttt{scale}=1)
\item Chose variant with either
    \verb|control.family = list(variant=0)| (default) or\\
    \verb|control.family = list(variant=1)| or
    \verb|control.family = list(variant=2)|.
\item Data and cencoring information is given by the triplet: $y$, $L$
    and $H$, and must be defined as a \texttt{inla.mdata}-object. $L$
    and $H$ are vectors of same length as $y$ giving the cencoring
    information for each observation. $L$ and $H$ must be integer
    valued or \texttt{Inf}.

    If $y[i]$ is not in the interval, then the observation is not
    cencored.

    $L[i]=$\texttt{Inf} and/or $H[i]=$\texttt{Inf} is equivalent to
    $L[i]= -1$ and/or $H[i] = -1$.

    $L[i]=$\texttt{Inf} implies no interval censoring.
    $H[i]=$\texttt{Inf} and $0\leq L[i]<\infty$ implies right
    censoring.

\end{itemize}
    

\subsubsection*{Hyperparameter spesification and default values}
%% DO NOT EDIT!
%% This file is generated automatically from models.R
\begin{description}
	\item[doc] The CenNegBinomial2 likelihood (similar to cenpoisson2)
	\item[hyper]\ 
	 \begin{description}
	 	\item[theta]\ 
	 	 \begin{description}
	 	 	\item[hyperid] 63101
	 	 	\item[name] size
	 	 	\item[short.name] size
	 	 	\item[initial] 2.30258509299405
	 	 	\item[fixed] FALSE
	 	 	\item[prior] pc.mgamma
	 	 	\item[param] 7
	 	 	\item[to.theta] \verb!function(x) log(x)!
	 	 	\item[from.theta] \verb!function(x) exp(x)!
	 	 \end{description}
	 \end{description}
	\item[survival] FALSE
	\item[discrete] TRUE
	\item[link] default log logoffset quantile
	\item[pdf] cennbinomial2
\end{description}



\subsection*{Example}

In the following example we estimate the parameters in a simulated
example. {\small \verbatiminput{example-cennbinomial2.R} }

\subsection*{Notes}


\end{document}


% LocalWords:  hyperparameter dispersion Hyperparameters nbinomial

%%% Local Variables: 
%%% TeX-master: t
%%% End: 

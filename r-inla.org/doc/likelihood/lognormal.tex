\documentclass[a4paper,11pt]{article}
\usepackage[scale={0.8,0.9},centering,includeheadfoot]{geometry}
\usepackage{amstext}
\usepackage{amsmath}
\usepackage{verbatim}
\newcommand{\vect}[1]{\boldsymbol{#1}}
\begin{document}
\section*{LogNormal}

\subsection*{Parametrisation}

The LogNormal has density
\begin{displaymath}
    f(y) = \frac{1}{y\sqrt{2\pi}} \sqrt{\tau} \exp\left(
      -\frac{1}{2} \tau (\log y - \mu)^{2}
    \right), \qquad y > 0
\end{displaymath}
where
\begin{description}
\item[$\tau > 0$] is the precision parameter,
\item[$\mu$] is the mean parameter.
\end{description}

\subsection*{Link-function}

The parameter $\mu$ is linked to the linear predictor as:
\[
\eta = \mu
\]
\subsection*{Hyperparameters}

The $\tau$ parameter is represented as
\[
\theta = \log\tau
\]
and the prior is defined on $\theta$.

\subsection*{Specification}

\begin{itemize}
\item $\text{family}=\texttt{lognormal}$ for regression models and
    $\text{family}=\text{lognormalsurv}$ for survival models.
\item Required arguments: $y$ Given in a format by using
    $\texttt{inla.surv()}$ function for
    $\text{family}=\text{lognormal.surv}$
\end{itemize}

\subsubsection*{Hyperparameter spesification and default values}
\begin{description}
	\item[hyper]\ 
	 \begin{description}
	 	\item[theta]\ 
	 	 \begin{description}
	 	 	 \item[ name ] log precision 
	 	 	 \item[ short.name ] prec 
	 	 	 \item[ initial ] 2 
	 	 	 \item[ fixed ] FALSE 
	 	 	 \item[ prior ] loggamma 
	 	 	 \item[ param ] 1 5e-05 
	 	 	 \item[ to.theta ] \verb|| 
	 	 	 \item[ from.theta ] \verb|| 
	 	 \end{description}
	 \end{description}
	 \item[ survival ] TRUE 
	 \item[ discrete ] FALSE 
	 \item[ link ] default identity 
	 \item[ pdf ] lognormal 
\end{description}

%% DO NOT EDIT!
%% This file is generated automatically from models.R
\begin{description}
	\item[doc] \verb!The log-Normal likelihood (survival)!
	\item[hyper]\ 
	 \begin{description}
	 	\item[theta1]\ 
	 	 \begin{description}
	 	 	\item[hyperid] \verb!78001!
	 	 	\item[name] \verb!log precision!
	 	 	\item[short.name] \verb!prec!
	 	 	\item[output.name] \verb!Precision for the lognormalsurv observations!
	 	 	\item[output.name.intern] \verb!Log precision for the lognormalsurv observations!
	 	 	\item[initial] \verb!0!
	 	 	\item[fixed] \verb!FALSE!
	 	 	\item[prior] \verb!loggamma!
	 	 	\item[param] \verb!1 5e-05!
	 	 	\item[to.theta] \verb!function(x) log(x)!
	 	 	\item[from.theta] \verb!function(x) exp(x)!
	 	 \end{description}
	 	\item[theta2]\ 
	 	 \begin{description}
	 	 	\item[hyperid] \verb!78002!
	 	 	\item[name] \verb!beta1!
	 	 	\item[short.name] \verb!beta1!
	 	 	\item[output.name] \verb!beta1 for logNormal-Cure!
	 	 	\item[output.name.intern] \verb!beta1 for logNormal-Cure!
	 	 	\item[initial] \verb!-7!
	 	 	\item[fixed] \verb!FALSE!
	 	 	\item[prior] \verb!normal!
	 	 	\item[param] \verb!-4 100!
	 	 	\item[to.theta] \verb!function(x) x!
	 	 	\item[from.theta] \verb!function(x) x!
	 	 \end{description}
	 	\item[theta3]\ 
	 	 \begin{description}
	 	 	\item[hyperid] \verb!78003!
	 	 	\item[name] \verb!beta2!
	 	 	\item[short.name] \verb!beta2!
	 	 	\item[output.name] \verb!beta2 for logNormal-Cure!
	 	 	\item[output.name.intern] \verb!beta2 for logNormal-Cure!
	 	 	\item[initial] \verb!0!
	 	 	\item[fixed] \verb!FALSE!
	 	 	\item[prior] \verb!normal!
	 	 	\item[param] \verb!0 100!
	 	 	\item[to.theta] \verb!function(x) x!
	 	 	\item[from.theta] \verb!function(x) x!
	 	 \end{description}
	 	\item[theta4]\ 
	 	 \begin{description}
	 	 	\item[hyperid] \verb!78004!
	 	 	\item[name] \verb!beta3!
	 	 	\item[short.name] \verb!beta3!
	 	 	\item[output.name] \verb!beta3 for logNormal-Cure!
	 	 	\item[output.name.intern] \verb!beta3 for logNormal-Cure!
	 	 	\item[initial] \verb!0!
	 	 	\item[fixed] \verb!FALSE!
	 	 	\item[prior] \verb!normal!
	 	 	\item[param] \verb!0 100!
	 	 	\item[to.theta] \verb!function(x) x!
	 	 	\item[from.theta] \verb!function(x) x!
	 	 \end{description}
	 	\item[theta5]\ 
	 	 \begin{description}
	 	 	\item[hyperid] \verb!78005!
	 	 	\item[name] \verb!beta4!
	 	 	\item[short.name] \verb!beta4!
	 	 	\item[output.name] \verb!beta4 for logNormal-Cure!
	 	 	\item[output.name.intern] \verb!beta4 for logNormal-Cure!
	 	 	\item[initial] \verb!0!
	 	 	\item[fixed] \verb!FALSE!
	 	 	\item[prior] \verb!normal!
	 	 	\item[param] \verb!0 100!
	 	 	\item[to.theta] \verb!function(x) x!
	 	 	\item[from.theta] \verb!function(x) x!
	 	 \end{description}
	 	\item[theta6]\ 
	 	 \begin{description}
	 	 	\item[hyperid] \verb!78006!
	 	 	\item[name] \verb!beta5!
	 	 	\item[short.name] \verb!beta5!
	 	 	\item[output.name] \verb!beta5 for logNormal-Cure!
	 	 	\item[output.name.intern] \verb!beta5 for logNormal-Cure!
	 	 	\item[initial] \verb!0!
	 	 	\item[fixed] \verb!FALSE!
	 	 	\item[prior] \verb!normal!
	 	 	\item[param] \verb!0 100!
	 	 	\item[to.theta] \verb!function(x) x!
	 	 	\item[from.theta] \verb!function(x) x!
	 	 \end{description}
	 	\item[theta7]\ 
	 	 \begin{description}
	 	 	\item[hyperid] \verb!78007!
	 	 	\item[name] \verb!beta6!
	 	 	\item[short.name] \verb!beta6!
	 	 	\item[output.name] \verb!beta6 for logNormal-Cure!
	 	 	\item[output.name.intern] \verb!beta6 for logNormal-Cure!
	 	 	\item[initial] \verb!0!
	 	 	\item[fixed] \verb!FALSE!
	 	 	\item[prior] \verb!normal!
	 	 	\item[param] \verb!0 100!
	 	 	\item[to.theta] \verb!function(x) x!
	 	 	\item[from.theta] \verb!function(x) x!
	 	 \end{description}
	 	\item[theta8]\ 
	 	 \begin{description}
	 	 	\item[hyperid] \verb!78008!
	 	 	\item[name] \verb!beta7!
	 	 	\item[short.name] \verb!beta7!
	 	 	\item[output.name] \verb!beta7 for logNormal-Cure!
	 	 	\item[output.name.intern] \verb!beta7 for logNormal-Cure!
	 	 	\item[initial] \verb!0!
	 	 	\item[fixed] \verb!FALSE!
	 	 	\item[prior] \verb!normal!
	 	 	\item[param] \verb!0 100!
	 	 	\item[to.theta] \verb!function(x) x!
	 	 	\item[from.theta] \verb!function(x) x!
	 	 \end{description}
	 	\item[theta9]\ 
	 	 \begin{description}
	 	 	\item[hyperid] \verb!78009!
	 	 	\item[name] \verb!beta8!
	 	 	\item[short.name] \verb!beta8!
	 	 	\item[output.name] \verb!beta8 for logNormal-Cure!
	 	 	\item[output.name.intern] \verb!beta8 for logNormal-Cure!
	 	 	\item[initial] \verb!0!
	 	 	\item[fixed] \verb!FALSE!
	 	 	\item[prior] \verb!normal!
	 	 	\item[param] \verb!0 100!
	 	 	\item[to.theta] \verb!function(x) x!
	 	 	\item[from.theta] \verb!function(x) x!
	 	 \end{description}
	 	\item[theta10]\ 
	 	 \begin{description}
	 	 	\item[hyperid] \verb!78010!
	 	 	\item[name] \verb!beta9!
	 	 	\item[short.name] \verb!beta9!
	 	 	\item[output.name] \verb!beta9 for logNormal-Cure!
	 	 	\item[output.name.intern] \verb!beta9 for logNormal-Cure!
	 	 	\item[initial] \verb!0!
	 	 	\item[fixed] \verb!FALSE!
	 	 	\item[prior] \verb!normal!
	 	 	\item[param] \verb!0 100!
	 	 	\item[to.theta] \verb!function(x) x!
	 	 	\item[from.theta] \verb!function(x) x!
	 	 \end{description}
	 	\item[theta11]\ 
	 	 \begin{description}
	 	 	\item[hyperid] \verb!78011!
	 	 	\item[name] \verb!beta10!
	 	 	\item[short.name] \verb!beta10!
	 	 	\item[output.name] \verb!beta10 for logNormal-Cure!
	 	 	\item[output.name.intern] \verb!beta10 for logNormal-Cure!
	 	 	\item[initial] \verb!0!
	 	 	\item[fixed] \verb!FALSE!
	 	 	\item[prior] \verb!normal!
	 	 	\item[param] \verb!0 100!
	 	 	\item[to.theta] \verb!function(x) x!
	 	 	\item[from.theta] \verb!function(x) x!
	 	 \end{description}
	 \end{description}
	\item[survival] \verb!TRUE!
	\item[discrete] \verb!FALSE!
	\item[link] \verb!default identity!
	\item[pdf] \verb!lognormal!
\end{description}



\subsection*{Example}

In the following example we estimate the parameters in a simulated
case \verbatiminput{example-lognormal.R}

\subsection*{Notes}

\begin{itemize}
\item lognormalsurv can be used for right censored, left censored,
    interval censored data. A general framework to represent time is
    given by $\texttt{inla.surv}$. If the observed times $y$ are
    large/huge, then this can cause numerical overflow, and if you
    encounter this problem, try to scale the observatios, like
    \verb|time = time / max(time)|.
\end{itemize}


\end{document}



%%% Local Variables: 
%%% TeX-master: t
%%% End: 


\documentclass[a4paper,11pt]{article}
\usepackage[scale={0.8,0.9},centering,includeheadfoot]{geometry}
\usepackage{amstext}
\usepackage{amsmath}
\usepackage{verbatim}
\newcommand{\vect}[1]{\boldsymbol{#1}}
\begin{document}
\section*{LogNormal}

\subsection*{Parametrisation}

The LogNormal has density
\begin{displaymath}
    f(y) = \frac{1}{y\sqrt{2\pi}} \sqrt{\tau} \exp\left(
      -\frac{1}{2} \tau (\log y - \mu)^{2}
    \right), \qquad y > 0
\end{displaymath}
where
\begin{description}
\item[$\tau > 0$] is the precision parameter,
\item[$\mu$] is the mean parameter.
\end{description}

\subsection*{Link-function}

The parameter $\mu$ is linked to the linear predictor as:
\[
\eta = \mu
\]
\subsection*{Hyperparameters}

The $\tau$ parameter is represented as
\[
\theta = \log\tau
\]
and the prior is defined on $\theta$.

\subsection*{Specification}

\begin{itemize}
\item $\text{family}=\texttt{lognormal}$
\item Required arguments: $y$ (to be given in a format by using
    $\texttt{inla.surv()}$ function )
\end{itemize}

\subsubsection*{Hyperparameter spesification and default values}
\begin{description}
	\item[hyper]\ 
	 \begin{description}
	 	\item[theta]\ 
	 	 \begin{description}
	 	 	 \item[ name ] log precision 
	 	 	 \item[ short.name ] prec 
	 	 	 \item[ initial ] 2 
	 	 	 \item[ fixed ] FALSE 
	 	 	 \item[ prior ] loggamma 
	 	 	 \item[ param ] 1 5e-05 
	 	 	 \item[ to.theta ] \verb|| 
	 	 	 \item[ from.theta ] \verb|| 
	 	 \end{description}
	 \end{description}
	 \item[ survival ] TRUE 
	 \item[ discrete ] FALSE 
	 \item[ link ] default identity 
	 \item[ pdf ] lognormal 
\end{description}



\subsection*{Example}

In the following example we estimate the parameters in a simulated
case \verbatiminput{example-lognormal.R}

\subsection*{Notes}

\begin{itemize}
\item Lognormal model can be used for right censored, left censored,
    interval censored data.
\item A general frame work to represent time is given by
    $\texttt{inla.surv}$
\item If the observed times $y$ are large/huge, then this can cause
    numerical overflow in the likelihood routines giving error
    messages like
\begin{verbatim}
        file: smtp-taucs.c  hgid: 891deb69ae0c  date: Tue Nov 09 22:34:28 2010 +0100
        Function: GMRFLib_build_sparse_matrix_TAUCS(), Line: 611, Thread: 0
        Variable evaluates to NAN/INF. This does not make sense. Abort...
\end{verbatim}
    If you encounter this problem, try to scale the observatios,
    \verb|time = time / max(time)| or similar, before running
    \verb|inla()|. 
\end{itemize}


\end{document}



%%% Local Variables: 
%%% TeX-master: t
%%% End: 


\documentclass[a4paper,11pt]{article}
\usepackage[scale={0.8,0.9},centering,includeheadfoot]{geometry}
\usepackage{amstext}
\usepackage{amsmath}
\usepackage{verbatim}

\begin{document}
\section*{The Kumaraswamy distribution}

\subsection*{Parametrisation}

The Kumaraswamy distribution is
\begin{displaymath}
    f(y) = \alpha \beta y^{\alpha-1}(1-y^{\alpha})^{\beta-1}
\end{displaymath}
for $0<y<1$ and $\alpha, \beta > 0$. The cummulative distribution
function is
\begin{displaymath}
    F(y) = 1-(1-y^{\alpha})^{\beta}.
\end{displaymath}
The parametrisation is given in terms of the quantile function
\begin{displaymath}
    \kappa(q) = \left(1-(1-q)^{1/\beta}\right)^{1/\alpha}
\end{displaymath}
and the precision parameter $\phi$,
\begin{displaymath}
    \phi(q) = -\ln\left(1-(1-q)^{1/\beta}\right)
\end{displaymath}
for \emph{fixed} value of $0<q<1$.

\subsection*{Link-function}

The quantile $\kappa$ to the linear predictor by
\begin{displaymath}
    \text{logit}(\kappa) = \eta
\end{displaymath}
using the default logit link-function. 

\subsection*{Hyperparameters}

The hyperparameter is
\begin{displaymath}
    \phi = \exp(S \theta)
\end{displaymath}
and the prior is defined on $\theta$. The constant $S$ currently set
to $0.1$ to avoid numerical instabilities in the optimization, since
small changes of $\alpha$ can make a huge difference.


\subsection*{Specification}

\begin{itemize}
\item $\text{family}=\texttt{qkumar}$
\item Required arguments: $y$ and the quantile $q$. 
\end{itemize}
The quantile is given as \texttt{control.family=list(control.link = list(quantile=$q$))}.

\subsubsection*{Hyperparameter spesification and default values}
\documentclass[a4paper,11pt]{article}
\usepackage[scale={0.8,0.9},centering,includeheadfoot]{geometry}
\usepackage{amstext}
\usepackage{amsmath}
\usepackage{verbatim}

\begin{document}
\section*{The Kumaraswamy distribution}

\subsection*{Parametrisation}

The Kumaraswamy distribution is
\begin{displaymath}
    f(y) = \alpha \beta y^{\alpha-1}(1-y^{\alpha})^{\beta-1}
\end{displaymath}
for $0<y<1$ and $\alpha, \beta > 0$. The cummulative distribution
function is
\begin{displaymath}
    F(y) = 1-(1-y^{\alpha})^{\beta}.
\end{displaymath}
The parametrisation is given in terms of the quantile function
\begin{displaymath}
    \kappa(q) = \left(1-(1-q)^{1/\beta}\right)^{1/\alpha}
\end{displaymath}
and the precision parameter $\phi$,
\begin{displaymath}
    \phi(q) = -\ln\left(1-(1-q)^{1/\beta}\right)
\end{displaymath}
for \emph{fixed} value of $0<q<1$.

\subsection*{Link-function}

The quantile $\kappa$ to the linear predictor by
\begin{displaymath}
    \text{logit}(\kappa) = \eta
\end{displaymath}
using the default logit link-function. 

\subsection*{Hyperparameters}

The hyperparameter is
\begin{displaymath}
    \theta = log(\phi)
\end{displaymath}
and the prior is given for $\theta$.

\subsection*{Specification}

\begin{itemize}
\item $\text{family}=\texttt{qkumar}$
\item Required arguments: $y$ and the quantile $q$. 
\end{itemize}
The quantile is given as \texttt{control.family=list(quantile=$q$)}.

\subsubsection*{Hyperparameter spesification and default values}
\documentclass[a4paper,11pt]{article}
\usepackage[scale={0.8,0.9},centering,includeheadfoot]{geometry}
\usepackage{amstext}
\usepackage{amsmath}
\usepackage{verbatim}

\begin{document}
\section*{The Kumaraswamy distribution}

\subsection*{Parametrisation}

The Kumaraswamy distribution is
\begin{displaymath}
    f(y) = \alpha \beta y^{\alpha-1}(1-y^{\alpha})^{\beta-1}
\end{displaymath}
for $0<y<1$ and $\alpha, \beta > 0$. The cummulative distribution
function is
\begin{displaymath}
    F(y) = 1-(1-y^{\alpha})^{\beta}.
\end{displaymath}
The parametrisation is given in terms of the quantile function
\begin{displaymath}
    \kappa(q) = \left(1-(1-q)^{1/\beta}\right)^{1/\alpha}
\end{displaymath}
and the precision parameter $\phi$,
\begin{displaymath}
    \phi(q) = -\ln\left(1-(1-q)^{1/\beta}\right)
\end{displaymath}
for \emph{fixed} value of $0<q<1$.

\subsection*{Link-function}

The quantile $\kappa$ to the linear predictor by
\begin{displaymath}
    \text{logit}(\kappa) = \eta
\end{displaymath}
using the default logit link-function. 

\subsection*{Hyperparameters}

The hyperparameter is
\begin{displaymath}
    \theta = log(\phi)
\end{displaymath}
and the prior is given for $\theta$.

\subsection*{Specification}

\begin{itemize}
\item $\text{family}=\texttt{qkumar}$
\item Required arguments: $y$ and the quantile $q$. 
\end{itemize}
The quantile is given as \texttt{control.family=list(quantile=$q$)}.

\subsubsection*{Hyperparameter spesification and default values}
\documentclass[a4paper,11pt]{article}
\usepackage[scale={0.8,0.9},centering,includeheadfoot]{geometry}
\usepackage{amstext}
\usepackage{amsmath}
\usepackage{verbatim}

\begin{document}
\section*{The Kumaraswamy distribution}

\subsection*{Parametrisation}

The Kumaraswamy distribution is
\begin{displaymath}
    f(y) = \alpha \beta y^{\alpha-1}(1-y^{\alpha})^{\beta-1}
\end{displaymath}
for $0<y<1$ and $\alpha, \beta > 0$. The cummulative distribution
function is
\begin{displaymath}
    F(y) = 1-(1-y^{\alpha})^{\beta}.
\end{displaymath}
The parametrisation is given in terms of the quantile function
\begin{displaymath}
    \kappa(q) = \left(1-(1-q)^{1/\beta}\right)^{1/\alpha}
\end{displaymath}
and the precision parameter $\phi$,
\begin{displaymath}
    \phi(q) = -\ln\left(1-(1-q)^{1/\beta}\right)
\end{displaymath}
for \emph{fixed} value of $0<q<1$.

\subsection*{Link-function}

The quantile $\kappa$ to the linear predictor by
\begin{displaymath}
    \text{logit}(\kappa) = \eta
\end{displaymath}
using the default logit link-function. 

\subsection*{Hyperparameters}

The hyperparameter is
\begin{displaymath}
    \theta = log(\phi)
\end{displaymath}
and the prior is given for $\theta$.

\subsection*{Specification}

\begin{itemize}
\item $\text{family}=\texttt{qkumar}$
\item Required arguments: $y$ and the quantile $q$. 
\end{itemize}
The quantile is given as \texttt{control.family=list(quantile=$q$)}.

\subsubsection*{Hyperparameter spesification and default values}
\input{../hyper/likelihood/qkumar.tex}



\subsection*{Example}

\verbatiminput{example-qkumar.R}

\subsection*{Notes}

None.


\end{document}


% LocalWords: 

%%% Local Variables: 
%%% TeX-master: t
%%% End: 




\subsection*{Example}

\verbatiminput{example-qkumar.R}

\subsection*{Notes}

None.


\end{document}


% LocalWords: 

%%% Local Variables: 
%%% TeX-master: t
%%% End: 




\subsection*{Example}

\verbatiminput{example-qkumar.R}

\subsection*{Notes}

None.


\end{document}


% LocalWords: 

%%% Local Variables: 
%%% TeX-master: t
%%% End: 




\subsection*{Example}

\verbatiminput{example-qkumar.R}

\subsection*{Notes}

None.


\end{document}


% LocalWords: 

%%% Local Variables: 
%%% TeX-master: t
%%% End: 

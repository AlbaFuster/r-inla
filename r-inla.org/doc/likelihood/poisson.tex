\documentclass[a4paper,11pt]{article}
\usepackage[scale={0.8,0.9},centering,includeheadfoot]{geometry}
\usepackage{amstext}
\usepackage{amsmath}
\usepackage{verbatim}

\begin{document}
\section*{Poisson}

\subsection*{Parametrisation}

The Poisson distribution is
\begin{displaymath}
    \text{Prob}(y) = \frac{\lambda^{y}}{y!}\exp(-\lambda)
\end{displaymath}
for responses $y=0, 1, 2, \ldots$, where
\begin{description}
\item[$\lambda$:] the expected value.
\end{description}

\subsection*{Link-function}

The mean and variance of $y$ are given as
\begin{displaymath}
    \mu = \lambda \qquad\text{and}\qquad \sigma^{2} = \lambda
\end{displaymath}
and the mean is linked to the linear predictor by
\begin{displaymath}
    \lambda(\eta) = E \exp(\eta)
\end{displaymath}
where $E>0$ is a known constant (or $\log(E)$ is the offset of $\eta$).

\subsection*{Hyperparameters}

None.

\subsection*{Specification}

\begin{itemize}
\item $\text{family}=\texttt{poisson}$
\item Required arguments: $y$ and $E$
\end{itemize}

\subsection*{Example}

In the following example we estimate the parameters in a simulated
example with Poisson responses.
\verbatiminput{example-poisson.R}

\subsection*{Notes}

This likelihood also accept $E=0$ and in this case $\log(E)$ is
\emph{defined} to be $0$. Non-integer values of $y \ge 0$ is accepted
although the normalising constant of the likelihood is then wrong (but
its a constant only).

For the quantile-link, then \texttt{model="quantile"} is applied to
$\lambda$ only and this is then scaled with \texttt{E}. A more
population version, can be achived moving the constant \texttt{E} into
the linear predictor by 
\begin{verbatim}
     ~ offset(log(E)) + ...
\end{verbatim}
Note there is no link-model \texttt{pquantile} for the Poisson, as
this would disable the \texttt{E} argument.
\end{document}


% LocalWords:  np Hyperparameters Ntrials poisson

%%% Local Variables: 
%%% TeX-master: t
%%% End: 

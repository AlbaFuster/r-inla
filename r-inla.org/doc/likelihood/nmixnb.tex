\documentclass[a4paper,11pt]{article}
\usepackage[scale={0.8,0.9},centering,includeheadfoot]{geometry}
\usepackage{amstext}
\usepackage{amsmath}
\usepackage{verbatim}
\def\mmax{5}
\def\mmaxp1{6}

\begin{document}
\section*{NMixNB}

\subsection*{Parametrisation}

The N-MixtureNB distribution is a negative Binomial mixture of the Binomials, as
\begin{displaymath}
    \text{Prob}(y) = \sum_{n=y}^{\infty} {n \choose y} \ p^n
    (1-p)^{n-y} \times \frac{\Gamma(n + \delta)}{\Gamma(\delta) n!}
    q^{\delta}(1-q)^{n}
\end{displaymath}
for responses $y=0, 1, 2, \ldots,n$, where $n$ is Poisson number of
trials, and $p$ is probability of success. For $\delta$ and $q$, see below.
Replicated reponses
$y_1, y_2, \ldots, y_r$, are iid from the Binomial, given (a common)
$n$ from the negative Binomial,
\begin{displaymath}
    \text{Prob}(y_1, \ldots, y_r) = \sum_{n=\max\{y_1, \ldots,
        y_r\}}^{\infty} \left\{\prod_{i=1}^{r}
    {n \choose y_i} \ p^n
    (1-p)^{n-y_i}\right\} \times \frac{\Gamma(n + \delta)}{\Gamma(\delta) n!}
    q^{\delta}(1-q)^{n}
\end{displaymath}
The negative binomial is parameterisized in terms of the mean
$\lambda$ and overdispersion $1/\delta$, where
$q = \delta/(\delta + \mu)$; see the \texttt{R} documentation
\texttt{?dnbinom} for this parameterisation (where $\delta=\texttt{size}$).

\subsection*{Link-function}

The probability $p$ is linked to the linear predictor by
\begin{displaymath}
    p(\eta) = \frac{\exp(\eta)}{1+\exp(\eta)}
\end{displaymath}
for the default logit link, while $\lambda$ depends on fixed
covariates
\begin{displaymath}
    \log(\lambda) = \sum_{j=1}^{m} \beta_j x_j
\end{displaymath}
with one vector of covariates for each observation. $m$ can be maximum
$\mmax$ and minimum $1$.

\subsubsection*{Hyperparameters}
The parameters
$\theta_1=\beta_1, \theta_2=\beta_2, \ldots, \theta_m=\beta_m$, and
overdispersion $\theta_{\mmaxp1} = \log(1/\delta)$.

\subsubsection*{Hyperparameter spesification and default values}
\documentclass[a4paper,11pt]{article}
\usepackage[scale={0.8,0.9},centering,includeheadfoot]{geometry}
\usepackage{amstext}
\usepackage{amsmath}
\usepackage{verbatim}
\def\mmax{10}
\def\mmaxp1{11}

\begin{document}
\section*{NMixNB}

\subsection*{Parametrisation}

The N-MixtureNB distribution is a negative Binomial mixture of the
Binomials, as
\begin{displaymath}
    \text{Prob}(y) = \sum_{n=y}^{\infty} {n \choose y} \ p^n
    (1-p)^{n-y} \times \frac{\Gamma(n + \delta)}{\Gamma(\delta) n!}
    q^{\delta}(1-q)^{n}
\end{displaymath}
for responses $y=0, 1, 2, \ldots,n$, where $n$ is Poisson number of
trials, and $p$ is probability of success. For $\delta$ and $q$, see
below. Replicated reponses $y_1, y_2, \ldots, y_r$, are iid from the
Binomial with the same $p$, conditioned on the same $n$ from the
negative Binomial,
\begin{displaymath}
    \text{Prob}(y_1, \ldots, y_r) = \sum_{n=\max\{y_1, \ldots,
        y_r\}}^{\infty} \left\{\prod_{i=1}^{r}
      {n \choose y_i} \ p^n
      (1-p)^{n-y_i}\right\} \times \frac{\Gamma(n + \delta)}{\Gamma(\delta) n!}
    q^{\delta}(1-q)^{n}
\end{displaymath}
The negative binomial is parameterisized in terms of the mean
$\lambda$ and overdispersion $1/\delta$, where
$q = \delta/(\delta + \mu)$; see the \texttt{R} documentation
\texttt{?dnbinom} for this parameterisation (where
$\delta=\texttt{size}$).

\subsection*{Link-function}

The probability $p$ is linked to the linear predictor by
\begin{displaymath}
    p(\eta) = \frac{\exp(\eta)}{1+\exp(\eta)}
\end{displaymath}
for the default logit link, while $\lambda$ depends on fixed
covariates
\begin{displaymath}
    \log(\lambda) = \sum_{j=1}^{m} \beta_j x_j
\end{displaymath}
with one vector of covariates for each observation. $m$ can be maximum
$\mmax$ and minimum $1$.

\subsubsection*{Hyperparameters}
The parameters
$\theta_1=\beta_1, \theta_2=\beta_2, \ldots, \theta_m=\beta_m$, and
overdispersion $\theta_{\mmaxp1} = \log(1/\delta)$.

\subsubsection*{Hyperparameter spesification and default values}
\documentclass[a4paper,11pt]{article}
\usepackage[scale={0.8,0.9},centering,includeheadfoot]{geometry}
\usepackage{amstext}
\usepackage{amsmath}
\usepackage{verbatim}
\def\mmax{10}
\def\mmaxp1{11}

\begin{document}
\section*{NMixNB}

\subsection*{Parametrisation}

The N-MixtureNB distribution is a negative Binomial mixture of the
Binomials, as
\begin{displaymath}
    \text{Prob}(y) = \sum_{n=y}^{\infty} {n \choose y} \ p^n
    (1-p)^{n-y} \times \frac{\Gamma(n + \delta)}{\Gamma(\delta) n!}
    q^{\delta}(1-q)^{n}
\end{displaymath}
for responses $y=0, 1, 2, \ldots,n$, where $n$ is Poisson number of
trials, and $p$ is probability of success. For $\delta$ and $q$, see
below. Replicated reponses $y_1, y_2, \ldots, y_r$, are iid from the
Binomial with the same $p$, conditioned on the same $n$ from the
negative Binomial,
\begin{displaymath}
    \text{Prob}(y_1, \ldots, y_r) = \sum_{n=\max\{y_1, \ldots,
        y_r\}}^{\infty} \left\{\prod_{i=1}^{r}
      {n \choose y_i} \ p^n
      (1-p)^{n-y_i}\right\} \times \frac{\Gamma(n + \delta)}{\Gamma(\delta) n!}
    q^{\delta}(1-q)^{n}
\end{displaymath}
The negative binomial is parameterisized in terms of the mean
$\lambda$ and overdispersion $1/\delta$, where
$q = \delta/(\delta + \mu)$; see the \texttt{R} documentation
\texttt{?dnbinom} for this parameterisation (where
$\delta=\texttt{size}$).

\subsection*{Link-function}

The probability $p$ is linked to the linear predictor by
\begin{displaymath}
    p(\eta) = \frac{\exp(\eta)}{1+\exp(\eta)}
\end{displaymath}
for the default logit link, while $\lambda$ depends on fixed
covariates
\begin{displaymath}
    \log(\lambda) = \sum_{j=1}^{m} \beta_j x_j
\end{displaymath}
with one vector of covariates for each observation. $m$ can be maximum
$\mmax$ and minimum $1$.

\subsubsection*{Hyperparameters}
The parameters
$\theta_1=\beta_1, \theta_2=\beta_2, \ldots, \theta_m=\beta_m$, and
overdispersion $\theta_{\mmaxp1} = \log(1/\delta)$.

\subsubsection*{Hyperparameter spesification and default values}
\documentclass[a4paper,11pt]{article}
\usepackage[scale={0.8,0.9},centering,includeheadfoot]{geometry}
\usepackage{amstext}
\usepackage{amsmath}
\usepackage{verbatim}
\def\mmax{10}
\def\mmaxp1{11}

\begin{document}
\section*{NMixNB}

\subsection*{Parametrisation}

The N-MixtureNB distribution is a negative Binomial mixture of the
Binomials, as
\begin{displaymath}
    \text{Prob}(y) = \sum_{n=y}^{\infty} {n \choose y} \ p^n
    (1-p)^{n-y} \times \frac{\Gamma(n + \delta)}{\Gamma(\delta) n!}
    q^{\delta}(1-q)^{n}
\end{displaymath}
for responses $y=0, 1, 2, \ldots,n$, where $n$ is Poisson number of
trials, and $p$ is probability of success. For $\delta$ and $q$, see
below. Replicated reponses $y_1, y_2, \ldots, y_r$, are iid from the
Binomial with the same $p$, conditioned on the same $n$ from the
negative Binomial,
\begin{displaymath}
    \text{Prob}(y_1, \ldots, y_r) = \sum_{n=\max\{y_1, \ldots,
        y_r\}}^{\infty} \left\{\prod_{i=1}^{r}
      {n \choose y_i} \ p^n
      (1-p)^{n-y_i}\right\} \times \frac{\Gamma(n + \delta)}{\Gamma(\delta) n!}
    q^{\delta}(1-q)^{n}
\end{displaymath}
The negative binomial is parameterisized in terms of the mean
$\lambda$ and overdispersion $1/\delta$, where
$q = \delta/(\delta + \mu)$; see the \texttt{R} documentation
\texttt{?dnbinom} for this parameterisation (where
$\delta=\texttt{size}$).

\subsection*{Link-function}

The probability $p$ is linked to the linear predictor by
\begin{displaymath}
    p(\eta) = \frac{\exp(\eta)}{1+\exp(\eta)}
\end{displaymath}
for the default logit link, while $\lambda$ depends on fixed
covariates
\begin{displaymath}
    \log(\lambda) = \sum_{j=1}^{m} \beta_j x_j
\end{displaymath}
with one vector of covariates for each observation. $m$ can be maximum
$\mmax$ and minimum $1$.

\subsubsection*{Hyperparameters}
The parameters
$\theta_1=\beta_1, \theta_2=\beta_2, \ldots, \theta_m=\beta_m$, and
overdispersion $\theta_{\mmaxp1} = \log(1/\delta)$.

\subsubsection*{Hyperparameter spesification and default values}
\input{../hyper/likelihood/nmixnb.tex}

\subsection*{Specification}

\begin{itemize}
\item $\text{family}=\texttt{nmixnb}$
\item Required arguments: the response $Y$ and covariates $X$ as\\
    \verb|inla.mdata(Y, X [, additional.covariates])|
\end{itemize}
The response is a matrix where each row are replicates, where
responses that are \texttt{NA}'s are ignored. The covariates is one or
many vectors, matrices or data.frames. Each row of the covariates
$(x_{i1}, x_{i2}, \ldots, x_{im})$ defines the covariates used for the
$i$'th response(s) (the $i$'th row of \texttt{Y}). Note that
$\beta_{m+1}, \ldots, \beta_{\mmax}$ are fixed to zero.


\subsection*{Example}

In the following example we estimate the parameters in a simulated
example with replications.

\verbatiminput{example-nmixnb.R}

\subsection*{Notes}

\end{document}


% LocalWords:  np Hyperparameters Ntrials

%%% Local Variables: 
%%% TeX-master: t
%%% End: 



% LocalWords: 

%%% Local Variables: 
%%% TeX-master: t
%%% End: 


\subsection*{Specification}

\begin{itemize}
\item $\text{family}=\texttt{nmixnb}$
\item Required arguments: the response $Y$ and covariates $X$ as\\
    \verb|inla.mdata(Y, X [, additional.covariates])|
\end{itemize}
The response is a matrix where each row are replicates, where
responses that are \texttt{NA}'s are ignored. The covariates is one or
many vectors, matrices or data.frames. Each row of the covariates
$(x_{i1}, x_{i2}, \ldots, x_{im})$ defines the covariates used for the
$i$'th response(s) (the $i$'th row of \texttt{Y}). Note that
$\beta_{m+1}, \ldots, \beta_{\mmax}$ are fixed to zero.


\subsection*{Example}

In the following example we estimate the parameters in a simulated
example with replications.

\verbatiminput{example-nmixnb.R}

\subsection*{Notes}

\end{document}


% LocalWords:  np Hyperparameters Ntrials

%%% Local Variables: 
%%% TeX-master: t
%%% End: 



% LocalWords: 

%%% Local Variables: 
%%% TeX-master: t
%%% End: 


\subsection*{Specification}

\begin{itemize}
\item $\text{family}=\texttt{nmixnb}$
\item Required arguments: the response $Y$ and covariates $X$ as\\
    \verb|inla.mdata(Y, X [, additional.covariates])|
\end{itemize}
The response is a matrix where each row are replicates, where
responses that are \texttt{NA}'s are ignored. The covariates is one or
many vectors, matrices or data.frames. Each row of the covariates
$(x_{i1}, x_{i2}, \ldots, x_{im})$ defines the covariates used for the
$i$'th response(s) (the $i$'th row of \texttt{Y}). Note that
$\beta_{m+1}, \ldots, \beta_{\mmax}$ are fixed to zero.


\subsection*{Example}

In the following example we estimate the parameters in a simulated
example with replications.

\verbatiminput{example-nmixnb.R}

\subsection*{Notes}

\end{document}


% LocalWords:  np Hyperparameters Ntrials

%%% Local Variables: 
%%% TeX-master: t
%%% End: 



% LocalWords: 

%%% Local Variables: 
%%% TeX-master: t
%%% End: 


\subsection*{Specification}

\begin{itemize}
\item $\text{family}=\texttt{nmixnb}$
\item Required arguments: the response $Y$ and covariates $X$ as\\
    \verb|inla.mdata(Y, X [, additional.covariates])|
\end{itemize}
The response is a matrix where each row are replicates, where
responses that are \texttt{NA}'s are ignored. The covariates is one or
many vectors, matrices or data.frames. Each row of the covariates
$(x_{i1}, x_{i2}, \ldots, x_{im})$ defines the covariates used for the
$i$'th response(s) (the $i$'th row of \texttt{Y}). Note that 
$\beta_{m+1}, \ldots, \beta_{\mmax}$ are fixed to zero.


\subsection*{Example}

In the following example we estimate the parameters in a simulated
example with replications.

\verbatiminput{example-nmixnb.R}

\subsection*{Notes}

\end{document}


% LocalWords:  np Hyperparameters Ntrials

%%% Local Variables: 
%%% TeX-master: t
%%% End: 



% LocalWords: 

%%% Local Variables: 
%%% TeX-master: t
%%% End: 

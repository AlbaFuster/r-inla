\documentclass[a4paper,11pt]{article}
\usepackage[scale={0.8,0.9},centering,includeheadfoot]{geometry}
\usepackage{amstext}
\usepackage{amsmath}
\usepackage{verbatim}
\newcommand{\vect}[1]{\boldsymbol{#1}}
\begin{document}
\section*{Cox Proportional Hazards Model}

\subsection*{Parametrisation}

In the Cox proportional hazards model, defines the hazard rate as:
\begin{displaymath}
    h(t) = h_{0}(t)\exp(\eta)
\end{displaymath}
where
\begin{description}
\item[$h_{0}(\cdot)$:] baseline hazard
\item[$\eta$:] linear predictor
\end{description}

We start from a finite partition of the time axis
$0=s_{0}<s_{1}<\dots,s_{K}$ and assume the baseline hazard to be
constant in each time interval
\[
h_{0}(t) = \exp(b_{k})\mbox{ for }t\in(s_{k-1},s_{k}],\qquad
k=1,\dots,K
\]
and assign $\mathbf{b} = (b_{1},\dots,b_{K})$ a Gaussian prior (RW1 or
RW2) with unknown precision $\tau_{b}$
\subsection*{Link-function}

The parameter $\eta$ is the linear predictor

\subsection*{Hyperparameters}

The log precision $\log\tau_{b}$ for the piecewise constant hazard

\subsection*{Specification}

\begin{itemize}
\item \texttt{family="coxph"}
\item Required arguments:
    \begin{itemize}
    \item $y$ (to be given in a format by using $\texttt{inla.surv()}$
        function )
    \item $\texttt{control.hazard = list()}$ to control the prior for
        the piecewise constant hazar, see $\texttt{?control.hazard}$
        for more information.
    \end{itemize}
\end{itemize}


\subsubsection*{Hyperparameter spesification and default values}
\paragraph{The ``RW1'' model for the hazard}
\documentclass[a4paper,11pt]{article}
\usepackage[scale={0.8,0.9},centering,includeheadfoot]{geometry}
\usepackage{amstext}
\usepackage{listings}
\begin{document}

\section*{Random walk model of order $1$ (RW1)}

\subsection*{Parametrization}

The random walk model of order $1$ (RW1) for the Gaussian vector
$\mathbf{x}=(x_1,\dots,x_n)$ is constructed assuming independent
increments:
\[
\Delta x_i = x_i-x_{i+1}\sim\mathcal{N}(0,\tau^{-1})
\]
The density for $\mathbf{x}$ is derived from its $n-1$ increments as
\begin{eqnarray*}
    \pi(\mathbf{x}|\tau) &\propto& \tau^{(n-1)/2}
    \exp\left\{-\frac{\tau}{2}\sum (\Delta x_i)^2\right\}\\
    & = &\tau^{(n-1)/2}\exp\left\{-\frac{1}{2}
      \mathbf{x}^T\mathbf{Q}\mathbf{x} \right\}
\end{eqnarray*}
where $\mathbf{Q}=\tau\mathbf{R}$ and $\mathbf{R}$ is the structure
matrix reflecting the neighbourhood structure of the model.

It is also possible to define a {\it cyclic} version of the RW1 model,
in this case the graph is modified so that last node $x_n$ is
neighbour of $x_{n-1}$ and $x_1$.
\subsection*{Hyperparameters}

The precision parameter $\tau$ is represented as
\begin{displaymath}
    \theta =\log \tau
\end{displaymath}
and the prior is defined on $\mathbf{\theta}$.

\subsection*{Specification}

The RW1 model is specified inside the {\tt f()} function as
\begin{verbatim}
 f(<whatever>, model="rw1", values=<values>, cyclic=<TRUE|FALSE>,
              hyper = <hyper>)
\end{verbatim}
The (optional) argument {\tt values } is a numeric or factor vector
giving the values assumed by the covariate for which we want the
effect to be estimated. See next example for an application.
 
\subsubsection*{Hyperparameter spesification and default values}
\documentclass[a4paper,11pt]{article}
\usepackage[scale={0.8,0.9},centering,includeheadfoot]{geometry}
\usepackage{amstext}
\usepackage{listings}
\begin{document}

\section*{Random walk model of order $1$ (RW1)}

\subsection*{Parametrization}

The random walk model of order $1$ (RW1) for the Gaussian vector
$\mathbf{x}=(x_1,\dots,x_n)$ is constructed assuming independent
increments:
\[
\Delta x_i = x_i-x_{i+1}\sim\mathcal{N}(0,\tau^{-1})
\]
The density for $\mathbf{x}$ is derived from its $n-1$ increments as
\begin{eqnarray*}
    \pi(\mathbf{x}|\tau) &\propto& \tau^{(n-1)/2}
    \exp\left\{-\frac{\tau}{2}\sum (\Delta x_i)^2\right\}\\
    & = &\tau^{(n-1)/2}\exp\left\{-\frac{1}{2}
      \mathbf{x}^T\mathbf{Q}\mathbf{x} \right\}
\end{eqnarray*}
where $\mathbf{Q}=\tau\mathbf{R}$ and $\mathbf{R}$ is the structure
matrix reflecting the neighbourhood structure of the model.

It is also possible to define a {\it cyclic} version of the RW1 model,
in this case the graph is modified so that last node $x_n$ is
neighbour of $x_{n-1}$ and $x_1$.
\subsection*{Hyperparameters}

The precision parameter $\tau$ is represented as
\begin{displaymath}
    \theta =\log \tau
\end{displaymath}
and the prior is defined on $\mathbf{\theta}$.

\subsection*{Specification}

The RW1 model is specified inside the {\tt f()} function as
\begin{verbatim}
 f(<whatever>, model="rw1", values=<values>, cyclic=<TRUE|FALSE>,
              hyper = <hyper>)
\end{verbatim}
The (optional) argument {\tt values } is a numeric or factor vector
giving the values assumed by the covariate for which we want the
effect to be estimated. See next example for an application.
 
\subsubsection*{Hyperparameter spesification and default values}
\documentclass[a4paper,11pt]{article}
\usepackage[scale={0.8,0.9},centering,includeheadfoot]{geometry}
\usepackage{amstext}
\usepackage{listings}
\begin{document}

\section*{Random walk model of order $1$ (RW1)}

\subsection*{Parametrization}

The random walk model of order $1$ (RW1) for the Gaussian vector
$\mathbf{x}=(x_1,\dots,x_n)$ is constructed assuming independent
increments:
\[
\Delta x_i = x_i-x_{i+1}\sim\mathcal{N}(0,\tau^{-1})
\]
The density for $\mathbf{x}$ is derived from its $n-1$ increments as
\begin{eqnarray*}
    \pi(\mathbf{x}|\tau) &\propto& \tau^{(n-1)/2}
    \exp\left\{-\frac{\tau}{2}\sum (\Delta x_i)^2\right\}\\
    & = &\tau^{(n-1)/2}\exp\left\{-\frac{1}{2}
      \mathbf{x}^T\mathbf{Q}\mathbf{x} \right\}
\end{eqnarray*}
where $\mathbf{Q}=\tau\mathbf{R}$ and $\mathbf{R}$ is the structure
matrix reflecting the neighbourhood structure of the model.

It is also possible to define a {\it cyclic} version of the RW1 model,
in this case the graph is modified so that last node $x_n$ is
neighbour of $x_{n-1}$ and $x_1$.
\subsection*{Hyperparameters}

The precision parameter $\tau$ is represented as
\begin{displaymath}
    \theta =\log \tau
\end{displaymath}
and the prior is defined on $\mathbf{\theta}$.

\subsection*{Specification}

The RW1 model is specified inside the {\tt f()} function as
\begin{verbatim}
 f(<whatever>, model="rw1", values=<values>, cyclic=<TRUE|FALSE>,
              hyper = <hyper>)
\end{verbatim}
The (optional) argument {\tt values } is a numeric or factor vector
giving the values assumed by the covariate for which we want the
effect to be estimated. See next example for an application.
 
\subsubsection*{Hyperparameter spesification and default values}
\input{../hyper/latent/rw1.tex}


\subsection*{Example}

\begin{verbatim}
n=100
z=seq(0,6,length.out=n)
y=sin(z)+rnorm(n,mean=0,sd=0.5)
data=data.frame(y=y,z=z)

formula=y~f(z,model="rw1",
            hyper = list(prec = list(prior="loggamma",param=c(1,0.01))))
result=inla(formula,data=data,family="gaussian")

#here we estimate the effect only for some of the values in z
formula1=y~f(z,model="rw1",
             hyper = list(prec = list(prior="loggamma",param=c(1,0.01))),
             values=z[seq(1,length(z),2)])
result1=inla(formula1,data=data,family="gaussian")
\end{verbatim}


\subsection*{Notes}

The RW1 is intrinsic with rank deficiency 1.

There exist also support to define irregular RW1 models.
\end{document}


% LocalWords: 

%%% Local Variables: 
%%% TeX-master: t
%%% End: 



\subsection*{Example}

\begin{verbatim}
n=100
z=seq(0,6,length.out=n)
y=sin(z)+rnorm(n,mean=0,sd=0.5)
data=data.frame(y=y,z=z)

formula=y~f(z,model="rw1",
            hyper = list(prec = list(prior="loggamma",param=c(1,0.01))))
result=inla(formula,data=data,family="gaussian")

#here we estimate the effect only for some of the values in z
formula1=y~f(z,model="rw1",
             hyper = list(prec = list(prior="loggamma",param=c(1,0.01))),
             values=z[seq(1,length(z),2)])
result1=inla(formula1,data=data,family="gaussian")
\end{verbatim}


\subsection*{Notes}

The RW1 is intrinsic with rank deficiency 1.

There exist also support to define irregular RW1 models.
\end{document}


% LocalWords: 

%%% Local Variables: 
%%% TeX-master: t
%%% End: 



\subsection*{Example}

\begin{verbatim}
n=100
z=seq(0,6,length.out=n)
y=sin(z)+rnorm(n,mean=0,sd=0.5)
data=data.frame(y=y,z=z)

formula=y~f(z,model="rw1",
            hyper = list(prec = list(prior="loggamma",param=c(1,0.01))))
result=inla(formula,data=data,family="gaussian")

#here we estimate the effect only for some of the values in z
formula1=y~f(z,model="rw1",
             hyper = list(prec = list(prior="loggamma",param=c(1,0.01))),
             values=z[seq(1,length(z),2)])
result1=inla(formula1,data=data,family="gaussian")
\end{verbatim}


\subsection*{Notes}

The RW1 is intrinsic with rank deficiency 1.

There exist also support to define irregular RW1 models.
\end{document}


% LocalWords: 

%%% Local Variables: 
%%% TeX-master: t
%%% End: 

\paragraph{The ``RW2'' model for the hazard}
\begin{description}
	\item[hyper]\ 
	 \begin{description}
	 	\item[theta]\ 
	 	 \begin{description}
	 	 	\item[name] log precision
	 	 	\item[short.name] prec
	 	 	\item[prior] loggamma
	 	 	\item[param] 1 5e-05
	 	 	\item[initial] 4
	 	 	\item[fixed] FALSE
	 	 	\item[to.theta] \verb|function(x) log(x)|
	 	 	\item[from.theta] \verb|function(x) exp(x)|
	 	 \end{description}
	 \end{description}
	\item[constr] TRUE
	\item[nrow.ncol] FALSE
	\item[augmented] FALSE
	\item[aug.factor] 1
	\item[aug.constr] 
	\item[n.div.by] 
	\item[n.required] FALSE
	\item[set.default.values] FALSE
	\item[pdf] rw2
\end{description}



\subsection*{Example}

In the following example we estimate the baseline hazard in a
simulated case \verbatiminput{example-cox.R}

\subsection*{Notes}

\begin{itemize}
\item The Cox model can be used only for uncensored or right censored
    data.
\item The model for the piecewise constant baseline hazard is
    specified through $\texttt{control.hazard}$
\item A general frame work to represent time is given by
    $\texttt{inla.surv}$
\item If the observed times $y$ are large/huge, then this can cause
    numerical overflow in the likelihood routines giving error
    messages like
\begin{verbatim}
        file: smtp-taucs.c  hgid: 891deb69ae0c  date: Tue Nov 09 22:34:28 2010 +0100
        Function: GMRFLib_build_sparse_matrix_TAUCS(), Line: 611, Thread: 0
        Variable evaluates to NAN/INF. This does not make sense. Abort...
\end{verbatim}
    If you encounter this problem, try to scale the observatios,
    \verb|time = time / max(time)| or similar, before running
    \verb|inla()|.
\end{itemize}


\end{document}


% LocalWords:  np Hyperparameters Ntrials

%%% Local Variables: 
%%% TeX-master: t
%%% End: 

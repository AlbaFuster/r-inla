\documentclass[a4paper,11pt]{article}
\usepackage[scale={0.8,0.9},centering,includeheadfoot]{geometry}
\usepackage{ifpdf}
\usepackage{amstext}
\usepackage{amsmath}
\usepackage{verbatim}
\newcommand{\vect}[1]{\boldsymbol{#1}}
\begin{document}
\section*{Gompertz}

\subsection*{Parametrisation}

The Gompertz distribution has log survial function
\begin{displaymath}
    \log S(y) = -\frac{\mu}{\alpha}\left(\exp(\alpha y) -1\right)
\end{displaymath}
for response $y\ge 0$, $\mu>0$ and $\alpha>0$. The cummulative
distribution function and the density then follows as
\begin{displaymath}
    F(y) = 1 - \exp\left[ -\frac{\mu}{\alpha}\left(\exp(\alpha y)
        -1\right) \right]
\end{displaymath}
and
\begin{displaymath}
    f(y) = \mu \exp\left[ \alpha y -\frac{\mu}{\alpha}\left(\exp(\alpha y)
        -1\right) \right].
\end{displaymath}

\subsection*{Link-function}
The parameter $\mu$ is linked to the linear predictor $\eta$ as:
\[
    \mu = \exp(\eta)
\]

\subsection*{Hyperparameters}

The shape parameter $\alpha$ is represented as
\begin{displaymath}
    \alpha = \exp(S\theta)
\end{displaymath}
and the prior is defined on $\theta$. The constant $S$ currently set
to $0.1$ to avoid numerical instabilities in the optimization, since
small changes of $\alpha$ can make a huge difference.

\subsection*{Specification}

\begin{itemize}
\item $\text{family}=\texttt{gompertz}$ for regression models and 
 $\text{family}=\texttt{gompertz.surv}$ for survival models.
\item Required arguments: $y$ (to be given in a format by using
    $\texttt{inla.surv()}$ for survival models )
\end{itemize}

\subsubsection*{Hyperparameter spesification and default values}
%% DO NOT EDIT!
%% This file is generated automatically from models.R
\begin{description}
	\item[doc] gompertz distribution
	\item[hyper]\ 
	 \begin{description}
	 	\item[theta]\ 
	 	 \begin{description}
	 	 	\item[hyperid] 105101
	 	 	\item[name] shape
	 	 	\item[short.name] alpha
	 	 	\item[initial] -1
	 	 	\item[fixed] FALSE
	 	 	\item[prior] normal
	 	 	\item[param] 0 1
	 	 	\item[to.theta] \verb!function(x, sc = 0.1) log(x) / sc!
	 	 	\item[from.theta] \verb!function(x, sc = 0.1) exp(sc * x)!
	 	 \end{description}
	 \end{description}
	\item[status] experimental
	\item[survival] FALSE
	\item[discrete] FALSE
	\item[link] default log neglog
	\item[pdf] gompertz
\end{description}

%% DO NOT EDIT!
%% This file is generated automatically from models.R
\begin{description}
	\item[doc] \verb!gompertz distribution!
	\item[hyper]\ 
	 \begin{description}
	 	\item[theta1]\ 
	 	 \begin{description}
	 	 	\item[hyperid] \verb!106101!
	 	 	\item[name] \verb!shape!
	 	 	\item[short.name] \verb!alpha!
	 	 	\item[output.name.intern] \verb!alpha_intern for Gompertz-surv!
	 	 	\item[output.name] \verb!alpha parameter for Gompertz-surv!
	 	 	\item[initial] \verb!-10!
	 	 	\item[fixed] \verb!FALSE!
	 	 	\item[prior] \verb!normal!
	 	 	\item[param] \verb!0 1!
	 	 	\item[to.theta] \verb!function(x, sc = 0.1) log(x) / sc!
	 	 	\item[from.theta] \verb!function(x, sc = 0.1) exp(sc * x)!
	 	 \end{description}
	 	\item[theta2]\ 
	 	 \begin{description}
	 	 	\item[hyperid] \verb!106102!
	 	 	\item[name] \verb!beta1!
	 	 	\item[short.name] \verb!beta1!
	 	 	\item[output.name] \verb!beta1 for Gompertz-Cure!
	 	 	\item[output.name.intern] \verb!beta1 for Gompertz-Cure!
	 	 	\item[initial] \verb!-5!
	 	 	\item[fixed] \verb!FALSE!
	 	 	\item[prior] \verb!normal!
	 	 	\item[param] \verb!-4 100!
	 	 	\item[to.theta] \verb!function(x) x!
	 	 	\item[from.theta] \verb!function(x) x!
	 	 \end{description}
	 	\item[theta3]\ 
	 	 \begin{description}
	 	 	\item[hyperid] \verb!106103!
	 	 	\item[name] \verb!beta2!
	 	 	\item[short.name] \verb!beta2!
	 	 	\item[output.name] \verb!beta2 for Gompertz-Cure!
	 	 	\item[output.name.intern] \verb!beta2 for Gompertz-Cure!
	 	 	\item[initial] \verb!0!
	 	 	\item[fixed] \verb!FALSE!
	 	 	\item[prior] \verb!normal!
	 	 	\item[param] \verb!0 100!
	 	 	\item[to.theta] \verb!function(x) x!
	 	 	\item[from.theta] \verb!function(x) x!
	 	 \end{description}
	 	\item[theta4]\ 
	 	 \begin{description}
	 	 	\item[hyperid] \verb!106104!
	 	 	\item[name] \verb!beta3!
	 	 	\item[short.name] \verb!beta3!
	 	 	\item[output.name] \verb!beta3 for Gompertz-Cure!
	 	 	\item[output.name.intern] \verb!beta3 for Gompertz-Cure!
	 	 	\item[initial] \verb!0!
	 	 	\item[fixed] \verb!FALSE!
	 	 	\item[prior] \verb!normal!
	 	 	\item[param] \verb!0 100!
	 	 	\item[to.theta] \verb!function(x) x!
	 	 	\item[from.theta] \verb!function(x) x!
	 	 \end{description}
	 	\item[theta5]\ 
	 	 \begin{description}
	 	 	\item[hyperid] \verb!106105!
	 	 	\item[name] \verb!beta4!
	 	 	\item[short.name] \verb!beta4!
	 	 	\item[output.name] \verb!beta4 for Gompertz-Cure!
	 	 	\item[output.name.intern] \verb!beta4 for Gompertz-Cure!
	 	 	\item[initial] \verb!0!
	 	 	\item[fixed] \verb!FALSE!
	 	 	\item[prior] \verb!normal!
	 	 	\item[param] \verb!0 100!
	 	 	\item[to.theta] \verb!function(x) x!
	 	 	\item[from.theta] \verb!function(x) x!
	 	 \end{description}
	 	\item[theta6]\ 
	 	 \begin{description}
	 	 	\item[hyperid] \verb!106106!
	 	 	\item[name] \verb!beta5!
	 	 	\item[short.name] \verb!beta5!
	 	 	\item[output.name] \verb!beta5 for Gompertz-Cure!
	 	 	\item[output.name.intern] \verb!beta5 for Gompertz-Cure!
	 	 	\item[initial] \verb!0!
	 	 	\item[fixed] \verb!FALSE!
	 	 	\item[prior] \verb!normal!
	 	 	\item[param] \verb!0 100!
	 	 	\item[to.theta] \verb!function(x) x!
	 	 	\item[from.theta] \verb!function(x) x!
	 	 \end{description}
	 	\item[theta7]\ 
	 	 \begin{description}
	 	 	\item[hyperid] \verb!106107!
	 	 	\item[name] \verb!beta6!
	 	 	\item[short.name] \verb!beta6!
	 	 	\item[output.name] \verb!beta6 for Gompertz-Cure!
	 	 	\item[output.name.intern] \verb!beta6 for Gompertz-Cure!
	 	 	\item[initial] \verb!0!
	 	 	\item[fixed] \verb!FALSE!
	 	 	\item[prior] \verb!normal!
	 	 	\item[param] \verb!0 100!
	 	 	\item[to.theta] \verb!function(x) x!
	 	 	\item[from.theta] \verb!function(x) x!
	 	 \end{description}
	 	\item[theta8]\ 
	 	 \begin{description}
	 	 	\item[hyperid] \verb!106108!
	 	 	\item[name] \verb!beta7!
	 	 	\item[short.name] \verb!beta7!
	 	 	\item[output.name] \verb!beta7 for Gompertz-Cure!
	 	 	\item[output.name.intern] \verb!beta7 for Gompertz-Cure!
	 	 	\item[initial] \verb!0!
	 	 	\item[fixed] \verb!FALSE!
	 	 	\item[prior] \verb!normal!
	 	 	\item[param] \verb!0 100!
	 	 	\item[to.theta] \verb!function(x) x!
	 	 	\item[from.theta] \verb!function(x) x!
	 	 \end{description}
	 	\item[theta9]\ 
	 	 \begin{description}
	 	 	\item[hyperid] \verb!106109!
	 	 	\item[name] \verb!beta8!
	 	 	\item[short.name] \verb!beta8!
	 	 	\item[output.name] \verb!beta8 for Gompertz-Cure!
	 	 	\item[output.name.intern] \verb!beta8 for Gompertz-Cure!
	 	 	\item[initial] \verb!0!
	 	 	\item[fixed] \verb!FALSE!
	 	 	\item[prior] \verb!normal!
	 	 	\item[param] \verb!0 100!
	 	 	\item[to.theta] \verb!function(x) x!
	 	 	\item[from.theta] \verb!function(x) x!
	 	 \end{description}
	 	\item[theta10]\ 
	 	 \begin{description}
	 	 	\item[hyperid] \verb!106110!
	 	 	\item[name] \verb!beta9!
	 	 	\item[short.name] \verb!beta9!
	 	 	\item[output.name] \verb!beta9 for Gompertz-Cure!
	 	 	\item[output.name.intern] \verb!beta9 for Gompertz-Cure!
	 	 	\item[initial] \verb!0!
	 	 	\item[fixed] \verb!FALSE!
	 	 	\item[prior] \verb!normal!
	 	 	\item[param] \verb!0 100!
	 	 	\item[to.theta] \verb!function(x) x!
	 	 	\item[from.theta] \verb!function(x) x!
	 	 \end{description}
	 	\item[theta11]\ 
	 	 \begin{description}
	 	 	\item[hyperid] \verb!106111!
	 	 	\item[name] \verb!beta10!
	 	 	\item[short.name] \verb!beta10!
	 	 	\item[output.name] \verb!beta10 for Gompertz-Cure!
	 	 	\item[output.name.intern] \verb!beta10 for Gompertz-Cure!
	 	 	\item[initial] \verb!0!
	 	 	\item[fixed] \verb!FALSE!
	 	 	\item[prior] \verb!normal!
	 	 	\item[param] \verb!0 100!
	 	 	\item[to.theta] \verb!function(x) x!
	 	 	\item[from.theta] \verb!function(x) x!
	 	 \end{description}
	 \end{description}
	\item[survival] \verb!TRUE!
	\item[discrete] \verb!FALSE!
	\item[link] \verb!default log neglog!
	\item[pdf] \verb!gompertz!
\end{description}



\subsection*{Example}

In the following example we estimate the parameters in a simulated
case%%
\verbatiminput{example-gompertz.R}

\subsection*{Notes}

\end{document}


%%% Local Variables: 
%%% TeX-master: t
%%% End: 

\documentclass[a4paper,11pt]{article}
\usepackage[scale={0.8,0.9},centering,includeheadfoot]{geometry}
\usepackage{amstext}
\usepackage{amsmath}
\usepackage{verbatim}

\begin{document}
\section*{Beta-binomial Normal-approximation}

\subsection*{Parametrisation}

The Beta-binomial distribution arise from a hierarchical model where
the probability $p$ is Beta-distributed
\begin{displaymath}
    \pi(p) = \frac{1}{B(\alpha, \beta)}
    p^{\alpha-1}(1-p)^{\beta-1},\quad \alpha > 0, \; \beta > 0
\end{displaymath}
and the response $y$ is Binomial
\begin{displaymath}
    \pi(y\mid p) = {n \choose y} p^y (1-p)^{(n-y)}, \quad y=0, 1,\ldots, n.
\end{displaymath}
The marginal distribution for $y$ is the Beta-binomial,
\begin{displaymath}
    \pi(y) = {n \choose y} \frac{
        B(y+\alpha, n-y + \beta)}{B(\alpha, \beta)}.
\end{displaymath}
The mean and variance of $y$ are given as
\begin{displaymath}
    \mu = n \frac{\alpha}{\alpha+\beta} = n \mu_{p}, \quad\text{and}\quad
    \sigma^{2} = n \mu_{p}(1-\mu_{p}) \left(1 + (n-1)\rho\right),
\end{displaymath}
where $\mu_{p} = \frac{\alpha}{\alpha+\beta}$ is the mean of the
probability $p$ from the Beta-distribution, and
$\rho = \frac{1}{\alpha+\beta+1}$ ($0 < \rho < 1$) is the pairwise
correlation between the $n$ Bernoulli draws and an over-dispersion
parameter.

The ``Beta-Binomial Normal approximation''-model implements a
Normal-approximation to this distribution, treating $y$ as a continous
Normal distributed variable, with mean $n\mu_p$ and variance
$n \mu_p (1-\mu_p)(1 + s(n-1)\rho)$. We have added a constant scaling
$s \ge 0$ for more flexibility, where $s=1$ gives the Beta-Binomial above.

\subsection*{Link-function}

The mean probability $\mu_{p}$ is linked to the linear predictor by
\begin{displaymath}
    \mu_p = \frac{\exp(\eta)}{1+\exp(\eta)}
\end{displaymath}
using the default logit-link.


\subsection*{Hyperparameter}

The hyperparameter is the over-dispersion parameter $\rho$, which is
represented as
\begin{displaymath}
    \rho = \frac{\exp(\theta)}{1+\exp(\theta)}
\end{displaymath}
and the prior is defined on $\theta$. 

\subsection*{Specification}

\begin{itemize}
\item \texttt{family="betabinomialna"}
\item Required arguments: $y$ and $\text{Ntrials} = n$ (default
    $\text{Ntrials}=1$), and $\text{scale} = s$ (default $s=1$).
\end{itemize}

\subsubsection*{Hyperparameter spesification and default values}
%% DO NOT EDIT!
%% This file is generated automatically from models.R
\begin{description}
	\item[doc] The Beta-Binomial Normal approximation likelihood
	\item[hyper]\ 
	 \begin{description}
	 	\item[theta]\ 
	 	 \begin{description}
	 	 	\item[hyperid] 62101
	 	 	\item[name] overdispersion
	 	 	\item[short.name] rho
	 	 	\item[initial] 0
	 	 	\item[fixed] FALSE
	 	 	\item[prior] gaussian
	 	 	\item[param] 0 0.4
	 	 	\item[to.theta] \verb!function(x) log(x/(1-x))!
	 	 	\item[from.theta] \verb!function(x) exp(x)/(1+exp(x))!
	 	 \end{description}
	 \end{description}
	\item[survival] FALSE
	\item[discrete] TRUE
	\item[link] default logit cauchit probit cloglog loglog robit sn
	\item[pdf] betabinomialna
\end{description}


\subsection*{Example}

In the following example we estimate the parameters in a simulated
example.
\verbatiminput{example-betabinomialna.R}

\subsection*{Notes}

None.

\end{document}


% LocalWords:  hyperparameter overdispersion Hyperparameters nbinomial

%%% Local Variables: 
%%% TeX-master: t
%%% End: 

\documentclass[a4paper,11pt]{article}
\usepackage[scale={0.8,0.9},centering,includeheadfoot]{geometry}
\usepackage{amstext}
\usepackage{amsmath}
\usepackage{verbatim}

\begin{document}
\section*{Generalised Extreme Value (GEV) distribution}

\subsection*{Parametrisation}

The GEV distribution is defined through the cummulative distribution
function
\begin{displaymath}
    F(y; \eta, \tau, \xi) =
    \exp\left(
      - \left[ 1 + \xi \sqrt{\tau s} (y-\eta)\right]^{-1/\xi}
    \right)
\end{displaymath}
for
\begin{displaymath}
    1 + \xi \sqrt{\tau s} (y-\eta) > 0
\end{displaymath}
and for a continuously response $y$ where
\begin{description}
\item[$\eta$:] is the linear predictor
\item[$\tau$:] is the ``precision''
\item[$s$:] is a fixed scaling, $s>0$.    
\end{description}

\subsection*{Link-function}

The linear predictor is given in the parameterisation of the GEV
distribution.

\subsection*{Hyperparameters}

The GEV-models has two hyperparameters.
The ``precision'' is represented as
\begin{displaymath}
    \theta_{1} = \log \tau
\end{displaymath}
and the prior is defined on $\theta_{1}$.  The shape parameter $\xi$
is represented as
\begin{displaymath}
    \theta_{2} = \xi 
\end{displaymath}
and the prior is defined on $\theta_{2}$.  \footnote{Internally, the
    parameter $\theta_{2}$ is scaled with a fixed scaling $\xi_{s}$
    (default $0.1$), to improve the numerics as the natural ``scale''
    of $\xi$ is small. For this reason the $\theta_{2} (=\xi)$
    reported in \texttt{result\$mode\$theta} will appear as
    $\theta_{2}/\xi_{s}$. For the same reason, if you define the mode
    using \texttt{control.mode = list(theta = ..., ...)} then the
    element representing $\theta_{2}$ should be given as $\theta_{2} /
    \xi_{s}$.}

\subsection*{Specification}

\begin{itemize}
\item $\text{family}=\texttt{gev}$
\item Required arguments: $y$ and $s$ (keyword \texttt{scale})
\item The scaling $\xi_{s}$ is given by the argument
    \texttt{gev.scale.xi} and is default set to $0.1$.
\end{itemize}
The weights has default value 1.

\subsubsection*{Hyperparameter spesification and default values}
\begin{description}
	\item[hyper]\ 
	 \begin{description}
	 	\item[theta1]\ 
	 	 \begin{description}
	 	 	 \item[ name ] precision 
	 	 	 \item[ short.name ] prec 
	 	 	 \item[ initial ] 4 
	 	 	 \item[ fixed ] FALSE 
	 	 	 \item[ prior ] loggamma 
	 	 	 \item[ param ] 1 1e-05 
	 	 \end{description}
	 	\item[theta2]\ 
	 	 \begin{description}
	 	 	 \item[ name ] GEVparameter 
	 	 	 \item[ short.name ] gev 
	 	 	 \item[ initial ] 0 
	 	 	 \item[ fixed ] FALSE 
	 	 	 \item[ prior ] gaussian 
	 	 	 \item[ param ] 0 6.25 
	 	 \end{description}
	 \end{description}
	 \item[ survival ] FALSE 
	 \item[ discrete ] FALSE 
\end{description}


\subsection*{Example}

In the following example, we estimate the parameters of the GEV
distribution on some simulated data.
%%
\verbatiminput{example-gev.R}

\subsection*{Notes}

None.


\end{document}


% LocalWords:  np Hyperparameters Ntrials gaussian

%%% Local Variables: 
%%% TeX-master: t
%%% End: 

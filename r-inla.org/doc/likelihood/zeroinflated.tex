\documentclass[a4paper,11pt]{article}
\usepackage[scale={0.8,0.9},centering,includeheadfoot]{geometry}
\usepackage{amstext}
\usepackage{amsmath}
\usepackage{verbatim}

\begin{document}
\section*{Zero-inflated models: Poisson, Binomial, negative Binomial
    and BetaBinomial}

\subsection*{Parametrisation}

There is support two types of zero-inflated models, which we name type
$0$ and type $1$. These are defined for both the Binomial, the
Poisson, the censored Poisson, the negative Binomial and BetaBinomial
likelihood. For simplicity we will describe only the Poisson as the
other cases are similar.

\subsubsection*{Type 0}

The (type 0) likelihood is defined as
\begin{displaymath}
    \text{Prob}(y \mid \ldots ) = p \times 1_{[y=0]} +
    (1-p)\times \text{Poisson}(y \mid y > 0)
\end{displaymath}
where $p$ is a hyperparameter where
\begin{displaymath}
    p = \frac{\exp(\theta)}{1+\exp(\theta)}
\end{displaymath}
and $\theta$ is the internal representation of $p$; meaning that the
initial value and prior is given for $\theta$. This is model is called
\texttt{zeroinflatedpoisson0} (and \texttt{zeroinflatedbinomial0}).

\subsection*{Type 1}

The (type 1) likelihood is defined as
\begin{displaymath}
    \text{Prob}(y \mid \ldots ) = p \times 1_{[y=0]} +
    (1-p)\times \text{Poisson}(y)
\end{displaymath}
where $p$ is a hyperparameter where
\begin{displaymath}
    p = \frac{\exp(\theta)}{1+\exp(\theta)}
\end{displaymath}
and $\theta$ is the internal representation of $p$; meaning that the
initial value and prior is given for $\theta$. This is model is called
\texttt{zeroinflatedpoisson1} (and \texttt{zeroinflatedbinomial1}).

\subsection*{Link-function}

As for the Poisson, the Binomial the negative Binomial and the BetaBinomial.

\subsection*{Hyperparameters}

For Poisson and the Binomial, there is one hyperparameter; where
\begin{displaymath}
    p = \frac{\exp(\theta)}{1+\exp(\theta)}
\end{displaymath}
and the prior and initial value is is given for $\theta$.

For the negative Binomial and BetaBinomial, there are two
hyperparameters.  The overdispersion parameter $n$ for the negative
Binomial is represented as
\begin{displaymath}
    \theta_{1} = \log(n)
\end{displaymath}
and the prior is defined on $\theta_{1}$. The zero-inflation parameter
$p$, is represented as
\begin{displaymath}
    p = \frac{\exp(\theta_{2})}{1+\exp(\theta_{2})}
\end{displaymath}
and the prior and initial value is is given for $\theta_{2}$. For the
BetaBinomial it is similar.

\subsection*{Specification}

\begin{itemize}
\item $\text{family}=\texttt{zeroinflatedbinomial0}$
\item $\text{family}=\texttt{zeroinflatedbinomial1}$
\item $\text{family}=\texttt{zeroinflatednbinomial0}$
\item $\text{family}=\texttt{zeroinflatednbinomial1}$
\item $\text{family}=\texttt{zeroinflatedpoisson0}$
\item $\text{family}=\texttt{zeroinflatedpoisson1}$
\item $\text{family}=\texttt{zeroinflatedcenpoisson0}$
\item $\text{family}=\texttt{zeroinflatedcenpoisson1}$
\item $\text{family}=\texttt{zeroinflatedbetabinomial0}$
\item $\text{family}=\texttt{zeroinflatedbetabinomial1}$
\item Required arguments: As for the Binomial, the negative Binomial,
    BetaBinomial and Poisson likelihood.
\end{itemize}

\subsubsection*{Hyperparameter spesification and default values}
\paragraph{Zeroinflated Binomial Type 0}
%% DO NOT EDIT!
%% This file is generated automatically from models.R
\begin{description}
	\item[doc] \verb!Zero-inflated Binomial, type 0!
	\item[hyper]\ 
	 \begin{description}
	 	\item[theta]\ 
	 	 \begin{description}
	 	 	\item[hyperid] \verb!90001!
	 	 	\item[name] \verb!logit probability!
	 	 	\item[short.name] \verb!prob!
	 	 	\item[output.name] \verb!zero-probability parameter for zero-inflated binomial_0!
	 	 	\item[output.name.intern] \verb!intern zero-probability parameter for zero-inflated binomial_0!
	 	 	\item[initial] \verb!-1!
	 	 	\item[fixed] \verb!FALSE!
	 	 	\item[prior] \verb!gaussian!
	 	 	\item[param] \verb!-1 0.2!
	 	 	\item[to.theta] \verb!function(x) log(x / (1 - x))!
	 	 	\item[from.theta] \verb!function(x) exp(x) / (1 + exp(x))!
	 	 \end{description}
	 \end{description}
	\item[survival] \verb!FALSE!
	\item[discrete] \verb!FALSE!
	\item[link] \verb!default logit loga cauchit probit cloglog ccloglog loglog robit sn!
	\item[pdf] \verb!zeroinflated!
\end{description}


\paragraph{Zeroinflated Binomial Type 1}
%% DO NOT EDIT!
%% This file is generated automatically from models.R
\begin{description}
	\item[doc] Zero-inflated Binomial, type 1
	\item[hyper]\ 
	 \begin{description}
	 	\item[theta]\ 
	 	 \begin{description}
	 	 	\item[hyperid] 91001
	 	 	\item[name] logit probability
	 	 	\item[short.name] prob
	 	 	\item[initial] -1
	 	 	\item[fixed] FALSE
	 	 	\item[prior] gaussian
	 	 	\item[param] -1 0.2
	 	 	\item[to.theta] \verb!function(x) log(x/(1-x))!
	 	 	\item[from.theta] \verb!function(x) exp(x)/(1+exp(x))!
	 	 \end{description}
	 \end{description}
	\item[survival] FALSE
	\item[discrete] FALSE
	\item[link] default logit cauchit probit cloglog loglog robit
	\item[pdf] zeroinflated
\end{description}


\paragraph{Zeroinflated NegBinomial Type 0}
%% DO NOT EDIT!
%% This file is generated automatically from models.R
\begin{description}
	\item[hyper]\ 
	 \begin{description}
	 	\item[theta1]\ 
	 	 \begin{description}
	 	 	\item[name] log size
	 	 	\item[short.name] size
	 	 	\item[initial] 2.30258509299405
	 	 	\item[fixed] FALSE
	 	 	\item[prior] loggamma
	 	 	\item[param] 1 1
	 	 	\item[to.theta] \verb|function(x) log(x)|
	 	 	\item[from.theta] \verb|function(x) exp(x)|
	 	 \end{description}
	 	\item[theta2]\ 
	 	 \begin{description}
	 	 	\item[name] logit probability
	 	 	\item[short.name] prob
	 	 	\item[initial] -1
	 	 	\item[fixed] FALSE
	 	 	\item[prior] gaussian
	 	 	\item[param] -1 0.2
	 	 	\item[to.theta] \verb|function(x) log(x/(1-x))|
	 	 	\item[from.theta] \verb|function(x) exp(x)/(1+exp(x))|
	 	 \end{description}
	 \end{description}
	\item[survival] FALSE
	\item[discrete] FALSE
	\item[link] default log
	\item[pdf] zeroinflated
\end{description}


\paragraph{Zeroinflated NegBinomial Type 1}
%% DO NOT EDIT!
%% This file is generated automatically from models.R
\begin{description}
	\item[hyper]\ 
	 \begin{description}
	 	\item[theta1]\ 
	 	 \begin{description}
	 	 	\item[name] log size
	 	 	\item[short.name] size
	 	 	\item[initial] 2.30258509299405
	 	 	\item[fixed] FALSE
	 	 	\item[prior] loggamma
	 	 	\item[param] 1 1
	 	 	\item[to.theta] \verb|function(x) log(x)|
	 	 	\item[from.theta] \verb|function(x) exp(x)|
	 	 \end{description}
	 	\item[theta2]\ 
	 	 \begin{description}
	 	 	\item[name] logit probability
	 	 	\item[short.name] prob
	 	 	\item[initial] -1
	 	 	\item[fixed] FALSE
	 	 	\item[prior] gaussian
	 	 	\item[param] -1 0.2
	 	 	\item[to.theta] \verb|function(x) log(x/(1-x))|
	 	 	\item[from.theta] \verb|function(x) exp(x)/(1+exp(x))|
	 	 \end{description}
	 \end{description}
	\item[survival] FALSE
	\item[discrete] FALSE
	\item[link] default log
	\item[pdf] zeroinflated
\end{description}


\paragraph{Zeroinflated BetaBinomial Type 0}
%% DO NOT EDIT!
%% This file is generated automatically from models.R
\begin{description}
	\item[doc] Zero-inflated Beta-Binomial, type 0
	\item[hyper]\ 
	 \begin{description}
	 	\item[theta1]\ 
	 	 \begin{description}
	 	 	\item[hyperid] 88001
	 	 	\item[name] overdispersion
	 	 	\item[short.name] rho
	 	 	\item[initial] 0
	 	 	\item[fixed] FALSE
	 	 	\item[prior] gaussian
	 	 	\item[param] 0 0.4
	 	 	\item[to.theta] \verb!function(x) log(x / (1 - x))!
	 	 	\item[from.theta] \verb!function(x) exp(x) / (1 + exp(x))!
	 	 \end{description}
	 	\item[theta2]\ 
	 	 \begin{description}
	 	 	\item[hyperid] 88002
	 	 	\item[name] logit probability
	 	 	\item[short.name] prob
	 	 	\item[initial] -1
	 	 	\item[fixed] FALSE
	 	 	\item[prior] gaussian
	 	 	\item[param] -1 0.2
	 	 	\item[to.theta] \verb!function(x) log(x / (1 - x))!
	 	 	\item[from.theta] \verb!function(x) exp(x) / (1 + exp(x))!
	 	 \end{description}
	 \end{description}
	\item[survival] FALSE
	\item[discrete] TRUE
	\item[link] default logit loga cauchit probit cloglog ccloglog loglog robit sn
	\item[pdf] zeroinflated
\end{description}


\paragraph{Zeroinflated BetaBinomial Type 1}
%% DO NOT EDIT!
%% This file is generated automatically from models.R
\begin{description}
	\item[hyper]\ 
	 \begin{description}
	 	\item[theta1]\ 
	 	 \begin{description}
	 	 	\item[hyperid] 89001
	 	 	\item[name] overdispersion
	 	 	\item[short.name] rho
	 	 	\item[initial] 0
	 	 	\item[fixed] FALSE
	 	 	\item[prior] gaussian
	 	 	\item[param] 0 0.4
	 	 	\item[to.theta] \verb!function(x) log(x/(1-x))!
	 	 	\item[from.theta] \verb!function(x) exp(x)/(1+exp(x))!
	 	 \end{description}
	 	\item[theta2]\ 
	 	 \begin{description}
	 	 	\item[hyperid] 89002
	 	 	\item[name] logit probability
	 	 	\item[short.name] prob
	 	 	\item[initial] -1
	 	 	\item[fixed] FALSE
	 	 	\item[prior] gaussian
	 	 	\item[param] -1 0.2
	 	 	\item[to.theta] \verb!function(x) log(x/(1-x))!
	 	 	\item[from.theta] \verb!function(x) exp(x)/(1+exp(x))!
	 	 \end{description}
	 \end{description}
	\item[survival] FALSE
	\item[discrete] TRUE
	\item[link] default logit probit cloglog
	\item[pdf] zeroinflated
\end{description}


\paragraph{Zeroinflated Poisson Type 0}
%% DO NOT EDIT!
%% This file is generated automatically from models.R
\begin{description}
	\item[hyper]\ 
	 \begin{description}
	 	\item[theta]\ 
	 	 \begin{description}
	 	 	\item[name] logit probability
	 	 	\item[short.name] prob
	 	 	\item[initial] -1
	 	 	\item[fixed] FALSE
	 	 	\item[prior] gaussian
	 	 	\item[param] -1 0.2
	 	 	\item[to.theta] \verb!function(x) log(x/(1-x))!
	 	 	\item[from.theta] \verb!function(x) exp(x)/(1+exp(x))!
	 	 \end{description}
	 \end{description}
	\item[survival] FALSE
	\item[discrete] FALSE
	\item[link] default log
	\item[pdf] zeroinflated
\end{description}


\paragraph{Zeroinflated Poisson Type 1}
%% DO NOT EDIT!
%% This file is generated automatically from models.R
\begin{description}
	\item[hyper]\ 
	 \begin{description}
	 	\item[theta]\ 
	 	 \begin{description}
	 	 	\item[name] logit probability
	 	 	\item[short.name] prob
	 	 	\item[initial] -1
	 	 	\item[fixed] FALSE
	 	 	\item[prior] gaussian
	 	 	\item[param] -1 0.2
	 	 	\item[to.theta] \verb!function(x) log(x/(1-x))!
	 	 	\item[from.theta] \verb!function(x) exp(x)/(1+exp(x))!
	 	 \end{description}
	 \end{description}
	\item[survival] FALSE
	\item[discrete] FALSE
	\item[link] default log
	\item[pdf] zeroinflated
\end{description}


\paragraph{Zeroinflated Censored Poisson Type 0}
%% DO NOT EDIT!
%% This file is generated automatically from models.R
\begin{description}
	\item[doc] \verb!Zero-inflated censored Poisson, type 0!
	\item[hyper]\ 
	 \begin{description}
	 	\item[theta]\ 
	 	 \begin{description}
	 	 	\item[hyperid] \verb!87101!
	 	 	\item[name] \verb!logit probability!
	 	 	\item[short.name] \verb!prob!
	 	 	\item[output.name] \verb!zero-probability parameter for zero-inflated poisson_0!
	 	 	\item[output.name.intern] \verb!intern zero-probability parameter for zero-inflated poisson_0!
	 	 	\item[initial] \verb!-1!
	 	 	\item[fixed] \verb!FALSE!
	 	 	\item[prior] \verb!gaussian!
	 	 	\item[param] \verb!-1 0.2!
	 	 	\item[to.theta] \verb!function(x) log(x / (1 - x))!
	 	 	\item[from.theta] \verb!function(x) exp(x) / (1 + exp(x))!
	 	 \end{description}
	 \end{description}
	\item[status] \verb!experimental!
	\item[survival] \verb!FALSE!
	\item[discrete] \verb!FALSE!
	\item[link] \verb!default log!
	\item[pdf] \verb!zeroinflated!
\end{description}


\paragraph{Zeroinflated Censored Poisson Type 1}
%% DO NOT EDIT!
%% This file is generated automatically from models.R
\begin{description}
	\item[doc] \verb!Zero-inflated censored Poisson, type 1!
	\item[hyper]\ 
	 \begin{description}
	 	\item[theta]\ 
	 	 \begin{description}
	 	 	\item[hyperid] \verb!87201!
	 	 	\item[name] \verb!logit probability!
	 	 	\item[short.name] \verb!prob!
	 	 	\item[output.name] \verb!zero-probability parameter for zero-inflated poisson_1!
	 	 	\item[output.name.intern] \verb!intern zero-probability parameter for zero-inflated poisson_1!
	 	 	\item[initial] \verb!-1!
	 	 	\item[fixed] \verb!FALSE!
	 	 	\item[prior] \verb!gaussian!
	 	 	\item[param] \verb!-1 0.2!
	 	 	\item[to.theta] \verb!function(x) log(x / (1 - x))!
	 	 	\item[from.theta] \verb!function(x) exp(x) / (1 + exp(x))!
	 	 \end{description}
	 \end{description}
	\item[status] \verb!experimental!
	\item[survival] \verb!FALSE!
	\item[discrete] \verb!FALSE!
	\item[link] \verb!default log!
	\item[pdf] \verb!zeroinflated!
\end{description}



\subsection*{Example}

In the following example we estimate the parameters in a simulated
example for both type 0 and type 1.
\subsubsection*{Poisson}

\verbatiminput{example-zero-inflated-poisson.R}

\subsubsection*{Binomial}

\verbatiminput{example-zero-inflated-binomial.R}

\clearpage

\subsection*{Advanced example}

In the following example we estimate the parameters in a simulated
example for a type0 likelihood, where one linear predictor enters the
zero-probability and one other linear predictor enters the non-zero
Poisson for example. The same trick can be used for other models of
type0.  The trick is that the likelihood
\begin{displaymath}
    p^{*} 1_{[y=0]} + (1-p^{*}) P(y | y > 0)
\end{displaymath}
can be reformulated as a Bernoulli likelihood for the
``class''-variable
\begin{displaymath}
    z =
    \begin{cases}
        1, & \text{if\; }y=0\\
        0, & \text{if\; }y > 0.
    \end{cases}
\end{displaymath}
where $p^{*}$ is the probability for success, and zero-inflated type0
likelihood (with fixed $p = 0$) for those $y > 0$. Since $p^{*}$ and
the linear predictor in $P$ is separated into two likelihoods, we can
apply one linear predictor to each one, hence extend the basic model
to cases where $p^{*}$ also depends on a linear predictor.  Here is a
small simulated example doing this.
\verbatiminput{zeroinflated-example-type0.R}



\subsection*{Notes}

None.




\subsection*{Extentions}

There are some extentions available which currently is only
implemented for the cases where its needed/requested.
\begin{description}
\item[Type 2] Is like Type 1 but where (for the Poisson)
    \begin{displaymath}
        p = 1-\left( \frac{E\exp(x)}{1 + E\exp(x)}\right)^{\alpha}
    \end{displaymath}
    where $\alpha > 0$ is the hyperparameter instead of $p$ (and
    $E\exp(x)$ is the mean).  Available for Poisson as
    \texttt{zeroinflatedpoisson2}, for binomial as
    \texttt{zeroinflatedbinomial2} and for the negative binomial as
    \texttt{zeroinflatednbinomial2}.

    The internal representation is $\theta = \log(\alpha)$ and prior
    is defined on $\log(\alpha)$.
\end{description}

\paragraph{Zeroinflated Poisson Type 2}
%% DO NOT EDIT!
%% This file is generated automatically from models.R
\begin{description}
	\item[hyper]\ 
	 \begin{description}
	 	\item[theta]\ 
	 	 \begin{description}
	 	 	\item[name] log alpha
	 	 	\item[short.name] a
	 	 	\item[initial] 0.693147180559945
	 	 	\item[fixed] FALSE
	 	 	\item[prior] gaussian
	 	 	\item[param] 0.693147180559945 1
	 	 	\item[to.theta] \verb!function(x) log(x)!
	 	 	\item[from.theta] \verb!function(x) exp(x)!
	 	 \end{description}
	 \end{description}
	\item[survival] FALSE
	\item[discrete] FALSE
	\item[link] default log
	\item[pdf] zeroinflated
\end{description}


\paragraph{Zeroinflated Binomial Type 2}
%% DO NOT EDIT!
%% This file is generated automatically from models.R
\begin{description}
	\item[doc] Zero-inflated Binomial, type 2
	\item[hyper]\ 
	 \begin{description}
	 	\item[theta]\ 
	 	 \begin{description}
	 	 	\item[hyperid] 92001
	 	 	\item[name] alpha
	 	 	\item[short.name] alpha
	 	 	\item[initial] -1
	 	 	\item[fixed] FALSE
	 	 	\item[prior] gaussian
	 	 	\item[param] -1 0.2
	 	 	\item[to.theta] \verb!function(x) log(x)!
	 	 	\item[from.theta] \verb!function(x) exp(x)!
	 	 \end{description}
	 \end{description}
	\item[survival] FALSE
	\item[discrete] FALSE
	\item[link] default logit cauchit probit cloglog loglog robit
	\item[pdf] zeroinflated
\end{description}


\paragraph{Zeroinflated Negative Binomial Type 2}
%% DO NOT EDIT!
%% This file is generated automatically from models.R
\begin{description}
	\item[hyper]\ 
	 \begin{description}
	 	\item[theta1]\ 
	 	 \begin{description}
	 	 	\item[hyperid] 99001
	 	 	\item[name] log size
	 	 	\item[short.name] size
	 	 	\item[initial] 2.30258509299405
	 	 	\item[fixed] FALSE
	 	 	\item[prior] loggamma
	 	 	\item[param] 1 1
	 	 	\item[to.theta] \verb!function(x) log(x)!
	 	 	\item[from.theta] \verb!function(x) exp(x)!
	 	 \end{description}
	 	\item[theta2]\ 
	 	 \begin{description}
	 	 	\item[hyperid] 99002
	 	 	\item[name] log alpha
	 	 	\item[short.name] a
	 	 	\item[initial] 0.693147180559945
	 	 	\item[fixed] FALSE
	 	 	\item[prior] gaussian
	 	 	\item[param] 2 1
	 	 	\item[to.theta] \verb!function(x) log(x)!
	 	 	\item[from.theta] \verb!function(x) exp(x)!
	 	 \end{description}
	 \end{description}
	\item[survival] FALSE
	\item[discrete] FALSE
	\item[link] default log
	\item[pdf] zeroinflated
\end{description}


\paragraph{Zeroinflated Negative Binomial Type 1 Strata 2}
%% DO NOT EDIT!
%% This file is generated automatically from models.R
\begin{description}
	\item[hyper]\ 
	 \begin{description}
	 	\item[theta1]\ 
	 	 \begin{description}
	 	 	\item[hyperid] 97001
	 	 	\item[name] log size
	 	 	\item[short.name] size
	 	 	\item[initial] 2.30258509299405
	 	 	\item[fixed] FALSE
	 	 	\item[prior] loggamma
	 	 	\item[param] 1 0.1
	 	 	\item[to.theta] \verb!function(x) log(x)!
	 	 	\item[from.theta] \verb!function(x) exp(x)!
	 	 \end{description}
	 	\item[theta2]\ 
	 	 \begin{description}
	 	 	\item[hyperid] 97002
	 	 	\item[name] logit probability 1
	 	 	\item[short.name] prob1
	 	 	\item[initial] -1
	 	 	\item[fixed] FALSE
	 	 	\item[prior] gaussian
	 	 	\item[param] -1 0.2
	 	 	\item[to.theta] \verb!function(x) log(x/(1-x))!
	 	 	\item[from.theta] \verb!function(x) exp(x)/(1+exp(x))!
	 	 \end{description}
	 	\item[theta3]\ 
	 	 \begin{description}
	 	 	\item[hyperid] 97003
	 	 	\item[name] logit probability 2
	 	 	\item[short.name] prob2
	 	 	\item[initial] -1
	 	 	\item[fixed] FALSE
	 	 	\item[prior] gaussian
	 	 	\item[param] -1 0.2
	 	 	\item[to.theta] \verb!function(x) log(x/(1-x))!
	 	 	\item[from.theta] \verb!function(x) exp(x)/(1+exp(x))!
	 	 \end{description}
	 	\item[theta4]\ 
	 	 \begin{description}
	 	 	\item[hyperid] 97004
	 	 	\item[name] logit probability 3
	 	 	\item[short.name] prob3
	 	 	\item[initial] -1
	 	 	\item[fixed] TRUE
	 	 	\item[prior] gaussian
	 	 	\item[param] -1 0.2
	 	 	\item[to.theta] \verb!function(x) log(x/(1-x))!
	 	 	\item[from.theta] \verb!function(x) exp(x)/(1+exp(x))!
	 	 \end{description}
	 	\item[theta5]\ 
	 	 \begin{description}
	 	 	\item[hyperid] 97005
	 	 	\item[name] logit probability 4
	 	 	\item[short.name] prob4
	 	 	\item[initial] -1
	 	 	\item[fixed] TRUE
	 	 	\item[prior] gaussian
	 	 	\item[param] -1 0.2
	 	 	\item[to.theta] \verb!function(x) log(x/(1-x))!
	 	 	\item[from.theta] \verb!function(x) exp(x)/(1+exp(x))!
	 	 \end{description}
	 	\item[theta6]\ 
	 	 \begin{description}
	 	 	\item[hyperid] 97006
	 	 	\item[name] logit probability 5
	 	 	\item[short.name] prob5
	 	 	\item[initial] -1
	 	 	\item[fixed] TRUE
	 	 	\item[prior] gaussian
	 	 	\item[param] -1 0.2
	 	 	\item[to.theta] \verb!function(x) log(x/(1-x))!
	 	 	\item[from.theta] \verb!function(x) exp(x)/(1+exp(x))!
	 	 \end{description}
	 	\item[theta7]\ 
	 	 \begin{description}
	 	 	\item[hyperid] 97007
	 	 	\item[name] logit probability 6
	 	 	\item[short.name] prob6
	 	 	\item[initial] -1
	 	 	\item[fixed] TRUE
	 	 	\item[prior] gaussian
	 	 	\item[param] -1 0.2
	 	 	\item[to.theta] \verb!function(x) log(x/(1-x))!
	 	 	\item[from.theta] \verb!function(x) exp(x)/(1+exp(x))!
	 	 \end{description}
	 	\item[theta8]\ 
	 	 \begin{description}
	 	 	\item[hyperid] 97008
	 	 	\item[name] logit probability 7
	 	 	\item[short.name] prob7
	 	 	\item[initial] -1
	 	 	\item[fixed] TRUE
	 	 	\item[prior] gaussian
	 	 	\item[param] -1 0.2
	 	 	\item[to.theta] \verb!function(x) log(x/(1-x))!
	 	 	\item[from.theta] \verb!function(x) exp(x)/(1+exp(x))!
	 	 \end{description}
	 	\item[theta9]\ 
	 	 \begin{description}
	 	 	\item[hyperid] 97009
	 	 	\item[name] logit probability 8
	 	 	\item[short.name] prob8
	 	 	\item[initial] -1
	 	 	\item[fixed] TRUE
	 	 	\item[prior] gaussian
	 	 	\item[param] -1 0.2
	 	 	\item[to.theta] \verb!function(x) log(x/(1-x))!
	 	 	\item[from.theta] \verb!function(x) exp(x)/(1+exp(x))!
	 	 \end{description}
	 	\item[theta10]\ 
	 	 \begin{description}
	 	 	\item[hyperid] 97010
	 	 	\item[name] logit probability 9
	 	 	\item[short.name] prob9
	 	 	\item[initial] -1
	 	 	\item[fixed] TRUE
	 	 	\item[prior] gaussian
	 	 	\item[param] -1 0.2
	 	 	\item[to.theta] \verb!function(x) log(x/(1-x))!
	 	 	\item[from.theta] \verb!function(x) exp(x)/(1+exp(x))!
	 	 \end{description}
	 	\item[theta11]\ 
	 	 \begin{description}
	 	 	\item[hyperid] 97011
	 	 	\item[name] logit probability 10
	 	 	\item[short.name] prob10
	 	 	\item[initial] -1
	 	 	\item[fixed] TRUE
	 	 	\item[prior] gaussian
	 	 	\item[param] -1 0.2
	 	 	\item[to.theta] \verb!function(x) log(x/(1-x))!
	 	 	\item[from.theta] \verb!function(x) exp(x)/(1+exp(x))!
	 	 \end{description}
	 \end{description}
	\item[status] experimental
	\item[survival] FALSE
	\item[discrete] FALSE
	\item[link] default log
	\item[pdf] zeroinflated
\end{description}


\paragraph{Zeroinflated Negative Binomial Type 1 Strata 3}
%% DO NOT EDIT!
%% This file is generated automatically from models.R
\begin{description}
	\item[doc] \verb!Zero inflated negBinomial, type 1, strata 3!
	\item[hyper]\ 
	 \begin{description}
	 	\item[theta1]\ 
	 	 \begin{description}
	 	 	\item[hyperid] \verb!98001!
	 	 	\item[name] \verb!logit probability!
	 	 	\item[short.name] \verb!prob!
	 	 	\item[output.name] \verb!zero-probability for zero-inflated nbinomial_1_strata3!
	 	 	\item[output.name.intern] \verb!intern zero-probability for zero-inflated nbinomial_1_strata3!
	 	 	\item[initial] \verb!-1!
	 	 	\item[fixed] \verb!FALSE!
	 	 	\item[prior] \verb!gaussian!
	 	 	\item[param] \verb!-1 0.2!
	 	 	\item[to.theta] \verb!function(x) log(x / (1 - x))!
	 	 	\item[from.theta] \verb!function(x) exp(x) / (1 + exp(x))!
	 	 \end{description}
	 	\item[theta2]\ 
	 	 \begin{description}
	 	 	\item[hyperid] \verb!98002!
	 	 	\item[name] \verb!log size 1!
	 	 	\item[short.name] \verb!size1!
	 	 	\item[output.name] \verb!size1 for zero-inflated nbinomial_1_strata3!
	 	 	\item[output.name.intern] \verb!log_size1 for zero-inflated nbinomial_1_strata3!
	 	 	\item[initial] \verb!2.30258509299405!
	 	 	\item[fixed] \verb!FALSE!
	 	 	\item[prior] \verb!pc.mgamma!
	 	 	\item[param] \verb!7!
	 	 	\item[to.theta] \verb!function(x) log(x)!
	 	 	\item[from.theta] \verb!function(x) exp(x)!
	 	 \end{description}
	 	\item[theta3]\ 
	 	 \begin{description}
	 	 	\item[hyperid] \verb!98003!
	 	 	\item[name] \verb!log size 2!
	 	 	\item[short.name] \verb!size2!
	 	 	\item[output.name] \verb!size2 for zero-inflated nbinomial_1_strata3!
	 	 	\item[output.name.intern] \verb!log_size2 for zero-inflated nbinomial_1_strata3!
	 	 	\item[initial] \verb!2.30258509299405!
	 	 	\item[fixed] \verb!FALSE!
	 	 	\item[prior] \verb!pc.mgamma!
	 	 	\item[param] \verb!7!
	 	 	\item[to.theta] \verb!function(x) log(x)!
	 	 	\item[from.theta] \verb!function(x) exp(x)!
	 	 \end{description}
	 	\item[theta4]\ 
	 	 \begin{description}
	 	 	\item[hyperid] \verb!98004!
	 	 	\item[name] \verb!log size 3!
	 	 	\item[short.name] \verb!size3!
	 	 	\item[output.name] \verb!size3 for zero-inflated nbinomial_1_strata3!
	 	 	\item[output.name.intern] \verb!log_size3 for zero-inflated nbinomial_1_strata3!
	 	 	\item[initial] \verb!2.30258509299405!
	 	 	\item[fixed] \verb!TRUE!
	 	 	\item[prior] \verb!pc.mgamma!
	 	 	\item[param] \verb!7!
	 	 	\item[to.theta] \verb!function(x) log(x)!
	 	 	\item[from.theta] \verb!function(x) exp(x)!
	 	 \end{description}
	 	\item[theta5]\ 
	 	 \begin{description}
	 	 	\item[hyperid] \verb!98005!
	 	 	\item[name] \verb!log size 4!
	 	 	\item[short.name] \verb!size4!
	 	 	\item[output.name] \verb!size4 for zero-inflated nbinomial_1_strata3!
	 	 	\item[output.name.intern] \verb!log_size4 for zero-inflated nbinomial_1_strata3!
	 	 	\item[initial] \verb!2.30258509299405!
	 	 	\item[fixed] \verb!TRUE!
	 	 	\item[prior] \verb!pc.mgamma!
	 	 	\item[param] \verb!7!
	 	 	\item[to.theta] \verb!function(x) log(x)!
	 	 	\item[from.theta] \verb!function(x) exp(x)!
	 	 \end{description}
	 	\item[theta6]\ 
	 	 \begin{description}
	 	 	\item[hyperid] \verb!98006!
	 	 	\item[name] \verb!log size 5!
	 	 	\item[short.name] \verb!size5!
	 	 	\item[output.name] \verb!size5 for zero-inflated nbinomial_1_strata3!
	 	 	\item[output.name.intern] \verb!log_size5 for zero-inflated nbinomial_1_strata3!
	 	 	\item[initial] \verb!2.30258509299405!
	 	 	\item[fixed] \verb!TRUE!
	 	 	\item[prior] \verb!pc.mgamma!
	 	 	\item[param] \verb!7!
	 	 	\item[to.theta] \verb!function(x) log(x)!
	 	 	\item[from.theta] \verb!function(x) exp(x)!
	 	 \end{description}
	 	\item[theta7]\ 
	 	 \begin{description}
	 	 	\item[hyperid] \verb!98007!
	 	 	\item[name] \verb!log size 6!
	 	 	\item[short.name] \verb!size6!
	 	 	\item[output.name] \verb!size6 for zero-inflated nbinomial_1_strata3!
	 	 	\item[output.name.intern] \verb!log_size6 for zero-inflated nbinomial_1_strata3!
	 	 	\item[initial] \verb!2.30258509299405!
	 	 	\item[fixed] \verb!TRUE!
	 	 	\item[prior] \verb!pc.mgamma!
	 	 	\item[param] \verb!7!
	 	 	\item[to.theta] \verb!function(x) log(x)!
	 	 	\item[from.theta] \verb!function(x) exp(x)!
	 	 \end{description}
	 	\item[theta8]\ 
	 	 \begin{description}
	 	 	\item[hyperid] \verb!98008!
	 	 	\item[name] \verb!log size 7!
	 	 	\item[short.name] \verb!size7!
	 	 	\item[output.name] \verb!size7 for zero-inflated nbinomial_1_strata3!
	 	 	\item[output.name.intern] \verb!log_size7 for zero-inflated nbinomial_1_strata3!
	 	 	\item[initial] \verb!2.30258509299405!
	 	 	\item[fixed] \verb!TRUE!
	 	 	\item[prior] \verb!pc.mgamma!
	 	 	\item[param] \verb!7!
	 	 	\item[to.theta] \verb!function(x) log(x)!
	 	 	\item[from.theta] \verb!function(x) exp(x)!
	 	 \end{description}
	 	\item[theta9]\ 
	 	 \begin{description}
	 	 	\item[hyperid] \verb!98009!
	 	 	\item[name] \verb!log size 8!
	 	 	\item[short.name] \verb!size8!
	 	 	\item[output.name] \verb!size8 for zero-inflated nbinomial_1_strata3!
	 	 	\item[output.name.intern] \verb!log_size8 for zero-inflated nbinomial_1_strata3!
	 	 	\item[initial] \verb!2.30258509299405!
	 	 	\item[fixed] \verb!TRUE!
	 	 	\item[prior] \verb!pc.mgamma!
	 	 	\item[param] \verb!7!
	 	 	\item[to.theta] \verb!function(x) log(x)!
	 	 	\item[from.theta] \verb!function(x) exp(x)!
	 	 \end{description}
	 	\item[theta10]\ 
	 	 \begin{description}
	 	 	\item[hyperid] \verb!98010!
	 	 	\item[name] \verb!log size 9!
	 	 	\item[short.name] \verb!size9!
	 	 	\item[output.name] \verb!size9 for zero-inflated nbinomial_1_strata3!
	 	 	\item[output.name.intern] \verb!log_size9 for zero-inflated nbinomial_1_strata3!
	 	 	\item[initial] \verb!2.30258509299405!
	 	 	\item[fixed] \verb!TRUE!
	 	 	\item[prior] \verb!pc.mgamma!
	 	 	\item[param] \verb!7!
	 	 	\item[to.theta] \verb!function(x) log(x)!
	 	 	\item[from.theta] \verb!function(x) exp(x)!
	 	 \end{description}
	 	\item[theta11]\ 
	 	 \begin{description}
	 	 	\item[hyperid] \verb!98011!
	 	 	\item[name] \verb!log size 10!
	 	 	\item[short.name] \verb!size10!
	 	 	\item[output.name] \verb!size10 for zero-inflated nbinomial_1_strata3!
	 	 	\item[output.name.intern] \verb!log_size10 for zero-inflated nbinomial_1_strata3!
	 	 	\item[initial] \verb!2.30258509299405!
	 	 	\item[fixed] \verb!TRUE!
	 	 	\item[prior] \verb!pc.mgamma!
	 	 	\item[param] \verb!7!
	 	 	\item[to.theta] \verb!function(x) log(x)!
	 	 	\item[from.theta] \verb!function(x) exp(x)!
	 	 \end{description}
	 \end{description}
	\item[status] \verb!experimental!
	\item[survival] \verb!FALSE!
	\item[discrete] \verb!FALSE!
	\item[link] \verb!default log!
	\item[pdf] \verb!zeroinflated!
\end{description}


\subsubsection{Zero and $N$-inflated Binomial likelihood: type 3}

This is the case where
\begin{eqnarray*}
  \mbox{Prob}(y|\ldots) &=& p_0 \times 1_{[y=0]}+\\
                        && p_N \times 1_{[y=N]}+\\
                        &&(1-p_0 - p_N) \times \mbox{binomial}(y,N,p)
\end{eqnarray*}
where:
\begin{displaymath}
    p = \frac{\mbox{exp}(\eta)}{1+\mbox{exp}(\eta)}\qquad
    p_0 = \frac{p^{\alpha_0}}{1+p^{\alpha_0}+(1-p)^{\alpha_N}}\qquad
    p_N = \frac{(1-p)^{\alpha_N}}{1+p^{\alpha_0}+(1-p)^{\alpha_N}}
\end{displaymath}

There are 2 hyperparameters, $\alpha_0$ and $\alpha_N$, governing
zero-inflation where: The zero-inflation parameters $\alpha_0$ and
$\alpha_N$ are represented as $\theta_0$ = $\mbox{log}(\alpha_0)$;
$\theta_N$ = $\mbox{log}(\alpha_N)$ and the prior and initial value is
given for $\theta_0$ and $\theta_N$ respectively.

Here is an example
\begin{verbatim}
nsim<-10000
x<-rnorm(nsim)
alpha0<-1.5
alphaN<-2.0
p = exp(x)/(1+exp(x))
p0 = p^alpha0 / (1 + p^alpha0 + (1-p)^alphaN)
pN = (1-p)^alphaN / (1 + p^alpha0 + (1-p)^alphaN)
P<-cbind(p0, pN, (1-p0 -pN))
N<-rpois(nsim,20)
y<-rep(0,nsim)
for(i in 1:nsim)
    y[i]<-sum(rmultinom(1,size = 1,P[i,])*c(0,N[i],rbinom(1,N[i],p[i])))
formula = y ~1 + x  
r = inla(formula, family = "zeroninflatedbinomial3",  Ntrials = N, verbose = TRUE,
           data = data.frame(y, x))
\end{verbatim}
and the default settings
%% DO NOT EDIT!
%% This file is generated automatically from models.R
\begin{description}
	\item[hyper]\ 
	 \begin{description}
	 	\item[theta1]\ 
	 	 \begin{description}
	 	 	\item[hyperid] 93101
	 	 	\item[name] alpha0
	 	 	\item[short.name] alpha0
	 	 	\item[initial] 2
	 	 	\item[fixed] FALSE
	 	 	\item[prior] gaussian
	 	 	\item[param] 4 1
	 	 	\item[to.theta] \verb!function(x) log(x)!
	 	 	\item[from.theta] \verb!function(x) exp(x)!
	 	 \end{description}
	 	\item[theta2]\ 
	 	 \begin{description}
	 	 	\item[hyperid] 93102
	 	 	\item[name] alphaN
	 	 	\item[short.name] alphaN
	 	 	\item[initial] 2
	 	 	\item[fixed] FALSE
	 	 	\item[prior] gaussian
	 	 	\item[param] 4 1
	 	 	\item[to.theta] \verb!function(x) log(x)!
	 	 	\item[from.theta] \verb!function(x) exp(x)!
	 	 \end{description}
	 \end{description}
	\item[survival] FALSE
	\item[discrete] FALSE
	\item[link] default logit cauchit probit cloglog loglog
	\item[pdf] zeroinflated
\end{description}


\end{document}

% LocalWords:  np Hyperparameters Ntrials gaussian hyperparameter

%%% Local Variables: 
%%% TeX-master: t
%%% End: 



\documentclass[a4paper,11pt]{article}
\usepackage[scale={0.8,0.9},centering,includeheadfoot]{geometry}
\usepackage{amstext}
\usepackage{amsmath}
\usepackage{verbatim}
\newcommand{\vect}[1]{\boldsymbol{#1}}
\begin{document}
\section*{Weibull}

\subsection*{Parametrisation}

The Weibull distribution is (\texttt{variant=0})
\begin{displaymath}
    f(y) = \alpha y^{\alpha-1}
    \lambda\exp( - \lambda  y^{\alpha}),
    \qquad \alpha>0, \qquad \lambda>0
\end{displaymath}
and (\texttt{variant=1})
\begin{displaymath}
    f(y) = \alpha y^{\alpha-1}
    \lambda^{\alpha}\exp( - (\lambda  y)^{\alpha}),
    \qquad \alpha>0, \qquad \lambda>0
\end{displaymath}
where
\begin{description}
\item[$\alpha$:] shape parameter.
\end{description}

\subsection*{Link-function}

The parameter $\lambda$ is linked to the linear predictor as:
\[
    \lambda = \exp(\eta)
\]
\subsection*{Hyperparameters}

The $\alpha$ parameter is represented as
\[
    \alpha = \exp(S\theta)
\]
and the prior is defined on $\theta$. The constant $S$ currently set
to $0.1$ to avoid numerical instabilities in the optimization, since
small changes of $\alpha$ can make a huge difference.

\subsection*{Specification}

\begin{itemize}
\item $\text{family}=\texttt{weibull}$ for regression and
 $\text{family}=\texttt{weibullsurv}$ for survival
\item Required arguments: $y$ (to be given using
    $\texttt{inla.surv()}$ for survival models), and
    \texttt{variant=}$0$ (default) or $1$ to define the
    parameterisation.
\end{itemize}

\subsubsection*{Hyperparameter spesification and default values}
\paragraph{weibull}
\begin{quote}
    
\documentclass[a4paper,11pt]{article}
\usepackage[scale={0.8,0.9},centering,includeheadfoot]{geometry}
\usepackage{amstext}
\usepackage{amsmath}
\usepackage{verbatim}
\newcommand{\vect}[1]{\boldsymbol{#1}}
\begin{document}
\section*{Weibull}

\subsection*{Parametrisation}

The Weibull distribution is (\texttt{variant=0})
\begin{displaymath}
    f(y) = \alpha y^{\alpha-1}
    \lambda\exp( - \lambda  y^{\alpha}),
    \qquad \alpha>0, \qquad \lambda>0
\end{displaymath}
and (\texttt{variant=1})
\begin{displaymath}
    f(y) = \alpha y^{\alpha-1}
    \lambda^{\alpha}\exp( - (\lambda  y)^{\alpha}),
    \qquad \alpha>0, \qquad \lambda>0
\end{displaymath}
where
\begin{description}
\item[$\alpha$:] shape parameter.
\end{description}

\subsection*{Link-function}

The parameter $\lambda$ is linked to the linear predictor as:
\[
    \lambda = \exp(\eta)
\]
\subsection*{Hyperparameters}

The $\alpha$ parameter is represented as
\[
    \alpha = \exp(S\theta)
\]
and the prior is defined on $\theta$. The constant $S$ currently set
to $0.1$ to avoid numerical instabilities in the optimization, since
small changes of $\alpha$ can make a huge difference.

\subsection*{Specification}

\begin{itemize}
\item $\text{family}=\texttt{weibull}$ for regression and
 $\text{family}=\texttt{weibullsurv}$ for survival
\item Required arguments: $y$ (to be given using
    $\texttt{inla.surv()}$ for survival models), and
    \texttt{variant=}$0$ (default) or $1$ to define the
    parameterisation.
\end{itemize}

\subsubsection*{Hyperparameter spesification and default values}
\paragraph{weibull}
\begin{quote}
    
\documentclass[a4paper,11pt]{article}
\usepackage[scale={0.8,0.9},centering,includeheadfoot]{geometry}
\usepackage{amstext}
\usepackage{amsmath}
\usepackage{verbatim}
\newcommand{\vect}[1]{\boldsymbol{#1}}
\begin{document}
\section*{Weibull}

\subsection*{Parametrisation}

The Weibull distribution is (\texttt{variant=0})
\begin{displaymath}
    f(y) = \alpha y^{\alpha-1}
    \lambda\exp( - \lambda  y^{\alpha}),
    \qquad \alpha>0, \qquad \lambda>0
\end{displaymath}
and (\texttt{variant=1})
\begin{displaymath}
    f(y) = \alpha y^{\alpha-1}
    \lambda^{\alpha}\exp( - (\lambda  y)^{\alpha}),
    \qquad \alpha>0, \qquad \lambda>0
\end{displaymath}
where
\begin{description}
\item[$\alpha$:] shape parameter.
\end{description}

\subsection*{Link-function}

The parameter $\lambda$ is linked to the linear predictor as:
\[
    \lambda = \exp(\eta)
\]
\subsection*{Hyperparameters}

The $\alpha$ parameter is represented as
\[
    \alpha = \exp(S\theta)
\]
and the prior is defined on $\theta$. The constant $S$ currently set
to $0.1$ to avoid numerical instabilities in the optimization, since
small changes of $\alpha$ can make a huge difference.

\subsection*{Specification}

\begin{itemize}
\item $\text{family}=\texttt{weibull}$ for regression and
 $\text{family}=\texttt{weibullsurv}$ for survival
\item Required arguments: $y$ (to be given using
    $\texttt{inla.surv()}$ for survival models), and
    \texttt{variant=}$0$ (default) or $1$ to define the
    parameterisation.
\end{itemize}

\subsubsection*{Hyperparameter spesification and default values}
\paragraph{weibull}
\begin{quote}
    
\documentclass[a4paper,11pt]{article}
\usepackage[scale={0.8,0.9},centering,includeheadfoot]{geometry}
\usepackage{amstext}
\usepackage{amsmath}
\usepackage{verbatim}
\newcommand{\vect}[1]{\boldsymbol{#1}}
\begin{document}
\section*{Weibull}

\subsection*{Parametrisation}

The Weibull distribution is (\texttt{variant=0})
\begin{displaymath}
    f(y) = \alpha y^{\alpha-1}
    \lambda\exp( - \lambda  y^{\alpha}),
    \qquad \alpha>0, \qquad \lambda>0
\end{displaymath}
and (\texttt{variant=1})
\begin{displaymath}
    f(y) = \alpha y^{\alpha-1}
    \lambda^{\alpha}\exp( - (\lambda  y)^{\alpha}),
    \qquad \alpha>0, \qquad \lambda>0
\end{displaymath}
where
\begin{description}
\item[$\alpha$:] shape parameter.
\end{description}

\subsection*{Link-function}

The parameter $\lambda$ is linked to the linear predictor as:
\[
    \lambda = \exp(\eta)
\]
\subsection*{Hyperparameters}

The $\alpha$ parameter is represented as
\[
    \alpha = \exp(S\theta)
\]
and the prior is defined on $\theta$. The constant $S$ currently set
to $0.1$ to avoid numerical instabilities in the optimization, since
small changes of $\alpha$ can make a huge difference.

\subsection*{Specification}

\begin{itemize}
\item $\text{family}=\texttt{weibull}$ for regression and
 $\text{family}=\texttt{weibullsurv}$ for survival
\item Required arguments: $y$ (to be given using
    $\texttt{inla.surv()}$ for survival models), and
    \texttt{variant=}$0$ (default) or $1$ to define the
    parameterisation.
\end{itemize}

\subsubsection*{Hyperparameter spesification and default values}
\paragraph{weibull}
\begin{quote}
    \input{../hyper/likelihood/weibull.tex}
\end{quote}
\paragraph{weibullsurv}
\begin{quote}
    \input{../hyper/likelihood/weibullsurv.tex}
\end{quote}

\subsection*{Example}

In the following example we estimate the parameters in a simulated
case \verbatiminput{example-weibull.R}

\subsection*{Notes}

\begin{itemize}
\item Weibullsurv model can be used for right censored, left censored,
    interval censored data. If the observed times $y$ are large/huge,
    then this can cause numerical overflow in the likelihood routine.
    If you encounter this problem, try to scale the observatios,
    \verb|time = time / max(time)| or similar.
\end{itemize}


\end{document}


% LocalWords:  np Hyperparameters Ntrials

%%% Local Variables: 
%%% TeX-master: t
%%% End: 

\end{quote}
\paragraph{weibullsurv}
\begin{quote}
    %% DO NOT EDIT!
%% This file is generated automatically from models.R
\begin{description}
	\item[doc] The Weibull likelihood (survival)
	\item[hyper]\ 
	 \begin{description}
	 	\item[theta]\ 
	 	 \begin{description}
	 	 	\item[hyperid] 79101
	 	 	\item[name] log alpha
	 	 	\item[short.name] alpha
	 	 	\item[initial] -2
	 	 	\item[fixed] FALSE
	 	 	\item[prior] pc.alphaw
	 	 	\item[param] 5
	 	 	\item[to.theta] \verb!function(x, sc = 0.1) log(x) / sc!
	 	 	\item[from.theta] \verb!function(x, sc = 0.1) exp(sc * x)!
	 	 \end{description}
	 	\item[theta2]\ 
	 	 \begin{description}
	 	 	\item[hyperid] 79102
	 	 	\item[name] beta1
	 	 	\item[short.name] beta1
	 	 	\item[initial] -7
	 	 	\item[fixed] FALSE
	 	 	\item[prior] normal
	 	 	\item[param] -4 100
	 	 	\item[to.theta] \verb!function(x) x!
	 	 	\item[from.theta] \verb!function(x) x!
	 	 \end{description}
	 	\item[theta3]\ 
	 	 \begin{description}
	 	 	\item[hyperid] 79103
	 	 	\item[name] beta2
	 	 	\item[short.name] beta2
	 	 	\item[initial] 0
	 	 	\item[fixed] FALSE
	 	 	\item[prior] normal
	 	 	\item[param] 0 100
	 	 	\item[to.theta] \verb!function(x) x!
	 	 	\item[from.theta] \verb!function(x) x!
	 	 \end{description}
	 	\item[theta4]\ 
	 	 \begin{description}
	 	 	\item[hyperid] 79104
	 	 	\item[name] beta3
	 	 	\item[short.name] beta3
	 	 	\item[initial] 0
	 	 	\item[fixed] FALSE
	 	 	\item[prior] normal
	 	 	\item[param] 0 100
	 	 	\item[to.theta] \verb!function(x) x!
	 	 	\item[from.theta] \verb!function(x) x!
	 	 \end{description}
	 	\item[theta5]\ 
	 	 \begin{description}
	 	 	\item[hyperid] 79105
	 	 	\item[name] beta4
	 	 	\item[short.name] beta4
	 	 	\item[initial] 0
	 	 	\item[fixed] FALSE
	 	 	\item[prior] normal
	 	 	\item[param] 0 100
	 	 	\item[to.theta] \verb!function(x) x!
	 	 	\item[from.theta] \verb!function(x) x!
	 	 \end{description}
	 	\item[theta6]\ 
	 	 \begin{description}
	 	 	\item[hyperid] 79106
	 	 	\item[name] beta5
	 	 	\item[short.name] beta5
	 	 	\item[initial] 0
	 	 	\item[fixed] FALSE
	 	 	\item[prior] normal
	 	 	\item[param] 0 100
	 	 	\item[to.theta] \verb!function(x) x!
	 	 	\item[from.theta] \verb!function(x) x!
	 	 \end{description}
	 	\item[theta7]\ 
	 	 \begin{description}
	 	 	\item[hyperid] 79107
	 	 	\item[name] beta6
	 	 	\item[short.name] beta6
	 	 	\item[initial] 0
	 	 	\item[fixed] FALSE
	 	 	\item[prior] normal
	 	 	\item[param] 0 100
	 	 	\item[to.theta] \verb!function(x) x!
	 	 	\item[from.theta] \verb!function(x) x!
	 	 \end{description}
	 	\item[theta8]\ 
	 	 \begin{description}
	 	 	\item[hyperid] 79108
	 	 	\item[name] beta7
	 	 	\item[short.name] beta7
	 	 	\item[initial] 0
	 	 	\item[fixed] FALSE
	 	 	\item[prior] normal
	 	 	\item[param] 0 100
	 	 	\item[to.theta] \verb!function(x) x!
	 	 	\item[from.theta] \verb!function(x) x!
	 	 \end{description}
	 	\item[theta9]\ 
	 	 \begin{description}
	 	 	\item[hyperid] 79109
	 	 	\item[name] beta8
	 	 	\item[short.name] beta8
	 	 	\item[initial] 0
	 	 	\item[fixed] FALSE
	 	 	\item[prior] normal
	 	 	\item[param] 0 100
	 	 	\item[to.theta] \verb!function(x) x!
	 	 	\item[from.theta] \verb!function(x) x!
	 	 \end{description}
	 	\item[theta10]\ 
	 	 \begin{description}
	 	 	\item[hyperid] 79110
	 	 	\item[name] beta9
	 	 	\item[short.name] beta9
	 	 	\item[initial] 0
	 	 	\item[fixed] FALSE
	 	 	\item[prior] normal
	 	 	\item[param] 0 100
	 	 	\item[to.theta] \verb!function(x) x!
	 	 	\item[from.theta] \verb!function(x) x!
	 	 \end{description}
	 	\item[theta11]\ 
	 	 \begin{description}
	 	 	\item[hyperid] 79111
	 	 	\item[name] beta10
	 	 	\item[short.name] beta10
	 	 	\item[initial] 0
	 	 	\item[fixed] FALSE
	 	 	\item[prior] normal
	 	 	\item[param] 0 100
	 	 	\item[to.theta] \verb!function(x) x!
	 	 	\item[from.theta] \verb!function(x) x!
	 	 \end{description}
	 \end{description}
	\item[survival] TRUE
	\item[discrete] FALSE
	\item[link] default log neglog quantile
	\item[pdf] weibull
\end{description}

\end{quote}

\subsection*{Example}

In the following example we estimate the parameters in a simulated
case \verbatiminput{example-weibull.R}

\subsection*{Notes}

\begin{itemize}
\item Weibullsurv model can be used for right censored, left censored,
    interval censored data. If the observed times $y$ are large/huge,
    then this can cause numerical overflow in the likelihood routine.
    If you encounter this problem, try to scale the observatios,
    \verb|time = time / max(time)| or similar.
\end{itemize}


\end{document}


% LocalWords:  np Hyperparameters Ntrials

%%% Local Variables: 
%%% TeX-master: t
%%% End: 

\end{quote}
\paragraph{weibullsurv}
\begin{quote}
    %% DO NOT EDIT!
%% This file is generated automatically from models.R
\begin{description}
	\item[doc] The Weibull likelihood (survival)
	\item[hyper]\ 
	 \begin{description}
	 	\item[theta]\ 
	 	 \begin{description}
	 	 	\item[hyperid] 79101
	 	 	\item[name] log alpha
	 	 	\item[short.name] alpha
	 	 	\item[initial] -2
	 	 	\item[fixed] FALSE
	 	 	\item[prior] pc.alphaw
	 	 	\item[param] 5
	 	 	\item[to.theta] \verb!function(x, sc = 0.1) log(x) / sc!
	 	 	\item[from.theta] \verb!function(x, sc = 0.1) exp(sc * x)!
	 	 \end{description}
	 	\item[theta2]\ 
	 	 \begin{description}
	 	 	\item[hyperid] 79102
	 	 	\item[name] beta1
	 	 	\item[short.name] beta1
	 	 	\item[initial] -7
	 	 	\item[fixed] FALSE
	 	 	\item[prior] normal
	 	 	\item[param] -4 100
	 	 	\item[to.theta] \verb!function(x) x!
	 	 	\item[from.theta] \verb!function(x) x!
	 	 \end{description}
	 	\item[theta3]\ 
	 	 \begin{description}
	 	 	\item[hyperid] 79103
	 	 	\item[name] beta2
	 	 	\item[short.name] beta2
	 	 	\item[initial] 0
	 	 	\item[fixed] FALSE
	 	 	\item[prior] normal
	 	 	\item[param] 0 100
	 	 	\item[to.theta] \verb!function(x) x!
	 	 	\item[from.theta] \verb!function(x) x!
	 	 \end{description}
	 	\item[theta4]\ 
	 	 \begin{description}
	 	 	\item[hyperid] 79104
	 	 	\item[name] beta3
	 	 	\item[short.name] beta3
	 	 	\item[initial] 0
	 	 	\item[fixed] FALSE
	 	 	\item[prior] normal
	 	 	\item[param] 0 100
	 	 	\item[to.theta] \verb!function(x) x!
	 	 	\item[from.theta] \verb!function(x) x!
	 	 \end{description}
	 	\item[theta5]\ 
	 	 \begin{description}
	 	 	\item[hyperid] 79105
	 	 	\item[name] beta4
	 	 	\item[short.name] beta4
	 	 	\item[initial] 0
	 	 	\item[fixed] FALSE
	 	 	\item[prior] normal
	 	 	\item[param] 0 100
	 	 	\item[to.theta] \verb!function(x) x!
	 	 	\item[from.theta] \verb!function(x) x!
	 	 \end{description}
	 	\item[theta6]\ 
	 	 \begin{description}
	 	 	\item[hyperid] 79106
	 	 	\item[name] beta5
	 	 	\item[short.name] beta5
	 	 	\item[initial] 0
	 	 	\item[fixed] FALSE
	 	 	\item[prior] normal
	 	 	\item[param] 0 100
	 	 	\item[to.theta] \verb!function(x) x!
	 	 	\item[from.theta] \verb!function(x) x!
	 	 \end{description}
	 	\item[theta7]\ 
	 	 \begin{description}
	 	 	\item[hyperid] 79107
	 	 	\item[name] beta6
	 	 	\item[short.name] beta6
	 	 	\item[initial] 0
	 	 	\item[fixed] FALSE
	 	 	\item[prior] normal
	 	 	\item[param] 0 100
	 	 	\item[to.theta] \verb!function(x) x!
	 	 	\item[from.theta] \verb!function(x) x!
	 	 \end{description}
	 	\item[theta8]\ 
	 	 \begin{description}
	 	 	\item[hyperid] 79108
	 	 	\item[name] beta7
	 	 	\item[short.name] beta7
	 	 	\item[initial] 0
	 	 	\item[fixed] FALSE
	 	 	\item[prior] normal
	 	 	\item[param] 0 100
	 	 	\item[to.theta] \verb!function(x) x!
	 	 	\item[from.theta] \verb!function(x) x!
	 	 \end{description}
	 	\item[theta9]\ 
	 	 \begin{description}
	 	 	\item[hyperid] 79109
	 	 	\item[name] beta8
	 	 	\item[short.name] beta8
	 	 	\item[initial] 0
	 	 	\item[fixed] FALSE
	 	 	\item[prior] normal
	 	 	\item[param] 0 100
	 	 	\item[to.theta] \verb!function(x) x!
	 	 	\item[from.theta] \verb!function(x) x!
	 	 \end{description}
	 	\item[theta10]\ 
	 	 \begin{description}
	 	 	\item[hyperid] 79110
	 	 	\item[name] beta9
	 	 	\item[short.name] beta9
	 	 	\item[initial] 0
	 	 	\item[fixed] FALSE
	 	 	\item[prior] normal
	 	 	\item[param] 0 100
	 	 	\item[to.theta] \verb!function(x) x!
	 	 	\item[from.theta] \verb!function(x) x!
	 	 \end{description}
	 	\item[theta11]\ 
	 	 \begin{description}
	 	 	\item[hyperid] 79111
	 	 	\item[name] beta10
	 	 	\item[short.name] beta10
	 	 	\item[initial] 0
	 	 	\item[fixed] FALSE
	 	 	\item[prior] normal
	 	 	\item[param] 0 100
	 	 	\item[to.theta] \verb!function(x) x!
	 	 	\item[from.theta] \verb!function(x) x!
	 	 \end{description}
	 \end{description}
	\item[survival] TRUE
	\item[discrete] FALSE
	\item[link] default log neglog quantile
	\item[pdf] weibull
\end{description}

\end{quote}

\subsection*{Example}

In the following example we estimate the parameters in a simulated
case \verbatiminput{example-weibull.R}

\subsection*{Notes}

\begin{itemize}
\item Weibullsurv model can be used for right censored, left censored,
    interval censored data. If the observed times $y$ are large/huge,
    then this can cause numerical overflow in the likelihood routine.
    If you encounter this problem, try to scale the observatios,
    \verb|time = time / max(time)| or similar.
\end{itemize}


\end{document}


% LocalWords:  np Hyperparameters Ntrials

%%% Local Variables: 
%%% TeX-master: t
%%% End: 

\end{quote}
\paragraph{weibullsurv}
\begin{quote}
    %% DO NOT EDIT!
%% This file is generated automatically from models.R
\begin{description}
	\item[doc] The Weibull likelihood (survival)
	\item[hyper]\ 
	 \begin{description}
	 	\item[theta]\ 
	 	 \begin{description}
	 	 	\item[hyperid] 79101
	 	 	\item[name] log alpha
	 	 	\item[short.name] alpha
	 	 	\item[initial] -2
	 	 	\item[fixed] FALSE
	 	 	\item[prior] pc.alphaw
	 	 	\item[param] 5
	 	 	\item[to.theta] \verb!function(x, sc = 0.1) log(x) / sc!
	 	 	\item[from.theta] \verb!function(x, sc = 0.1) exp(sc * x)!
	 	 \end{description}
	 	\item[theta2]\ 
	 	 \begin{description}
	 	 	\item[hyperid] 79102
	 	 	\item[name] beta1
	 	 	\item[short.name] beta1
	 	 	\item[initial] -7
	 	 	\item[fixed] FALSE
	 	 	\item[prior] normal
	 	 	\item[param] -4 100
	 	 	\item[to.theta] \verb!function(x) x!
	 	 	\item[from.theta] \verb!function(x) x!
	 	 \end{description}
	 	\item[theta3]\ 
	 	 \begin{description}
	 	 	\item[hyperid] 79103
	 	 	\item[name] beta2
	 	 	\item[short.name] beta2
	 	 	\item[initial] 0
	 	 	\item[fixed] FALSE
	 	 	\item[prior] normal
	 	 	\item[param] 0 100
	 	 	\item[to.theta] \verb!function(x) x!
	 	 	\item[from.theta] \verb!function(x) x!
	 	 \end{description}
	 	\item[theta4]\ 
	 	 \begin{description}
	 	 	\item[hyperid] 79104
	 	 	\item[name] beta3
	 	 	\item[short.name] beta3
	 	 	\item[initial] 0
	 	 	\item[fixed] FALSE
	 	 	\item[prior] normal
	 	 	\item[param] 0 100
	 	 	\item[to.theta] \verb!function(x) x!
	 	 	\item[from.theta] \verb!function(x) x!
	 	 \end{description}
	 	\item[theta5]\ 
	 	 \begin{description}
	 	 	\item[hyperid] 79105
	 	 	\item[name] beta4
	 	 	\item[short.name] beta4
	 	 	\item[initial] 0
	 	 	\item[fixed] FALSE
	 	 	\item[prior] normal
	 	 	\item[param] 0 100
	 	 	\item[to.theta] \verb!function(x) x!
	 	 	\item[from.theta] \verb!function(x) x!
	 	 \end{description}
	 	\item[theta6]\ 
	 	 \begin{description}
	 	 	\item[hyperid] 79106
	 	 	\item[name] beta5
	 	 	\item[short.name] beta5
	 	 	\item[initial] 0
	 	 	\item[fixed] FALSE
	 	 	\item[prior] normal
	 	 	\item[param] 0 100
	 	 	\item[to.theta] \verb!function(x) x!
	 	 	\item[from.theta] \verb!function(x) x!
	 	 \end{description}
	 	\item[theta7]\ 
	 	 \begin{description}
	 	 	\item[hyperid] 79107
	 	 	\item[name] beta6
	 	 	\item[short.name] beta6
	 	 	\item[initial] 0
	 	 	\item[fixed] FALSE
	 	 	\item[prior] normal
	 	 	\item[param] 0 100
	 	 	\item[to.theta] \verb!function(x) x!
	 	 	\item[from.theta] \verb!function(x) x!
	 	 \end{description}
	 	\item[theta8]\ 
	 	 \begin{description}
	 	 	\item[hyperid] 79108
	 	 	\item[name] beta7
	 	 	\item[short.name] beta7
	 	 	\item[initial] 0
	 	 	\item[fixed] FALSE
	 	 	\item[prior] normal
	 	 	\item[param] 0 100
	 	 	\item[to.theta] \verb!function(x) x!
	 	 	\item[from.theta] \verb!function(x) x!
	 	 \end{description}
	 	\item[theta9]\ 
	 	 \begin{description}
	 	 	\item[hyperid] 79109
	 	 	\item[name] beta8
	 	 	\item[short.name] beta8
	 	 	\item[initial] 0
	 	 	\item[fixed] FALSE
	 	 	\item[prior] normal
	 	 	\item[param] 0 100
	 	 	\item[to.theta] \verb!function(x) x!
	 	 	\item[from.theta] \verb!function(x) x!
	 	 \end{description}
	 	\item[theta10]\ 
	 	 \begin{description}
	 	 	\item[hyperid] 79110
	 	 	\item[name] beta9
	 	 	\item[short.name] beta9
	 	 	\item[initial] 0
	 	 	\item[fixed] FALSE
	 	 	\item[prior] normal
	 	 	\item[param] 0 100
	 	 	\item[to.theta] \verb!function(x) x!
	 	 	\item[from.theta] \verb!function(x) x!
	 	 \end{description}
	 	\item[theta11]\ 
	 	 \begin{description}
	 	 	\item[hyperid] 79111
	 	 	\item[name] beta10
	 	 	\item[short.name] beta10
	 	 	\item[initial] 0
	 	 	\item[fixed] FALSE
	 	 	\item[prior] normal
	 	 	\item[param] 0 100
	 	 	\item[to.theta] \verb!function(x) x!
	 	 	\item[from.theta] \verb!function(x) x!
	 	 \end{description}
	 \end{description}
	\item[survival] TRUE
	\item[discrete] FALSE
	\item[link] default log neglog quantile
	\item[pdf] weibull
\end{description}

\end{quote}

\subsection*{Example}

In the following example we estimate the parameters in a simulated
case \verbatiminput{example-weibull.R}

\subsection*{Notes}

\begin{itemize}
\item Weibullsurv model can be used for right censored, left censored,
    interval censored data. If the observed times $y$ are large/huge,
    then this can cause numerical overflow in the likelihood routine.
    If you encounter this problem, try to scale the observatios,
    \verb|time = time / max(time)| or similar.
\end{itemize}


\end{document}


% LocalWords:  np Hyperparameters Ntrials

%%% Local Variables: 
%%% TeX-master: t
%%% End: 

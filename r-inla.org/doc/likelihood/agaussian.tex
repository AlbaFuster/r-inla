\documentclass[a4paper,11pt]{article}
\usepackage[scale={0.8,0.9},centering,includeheadfoot]{geometry}
\usepackage{amstext}
\usepackage{amsmath}
\usepackage{verbatim}

\begin{document}
\section*{Aggregated Gaussian}

\subsection*{Parametrisation}

This is shorthand to allow for aggregated Gaussian observations, where
we have repeated observations with the same mean and (known scaled)
precision. These can be aggregated into an equivalent likelihood
reducing the computional effort.

Let $y=(y_1, \ldots, y_n)$ be iid observations of a Gaussian with mean
$\mu$ and precisions $s_i\tau$, where the $s_i$'s are fixed and known
scaling parameters (default $s_i=1$), then
\begin{eqnarray}
    f(y|\mu,\tau) &=& \prod_{i=1}^{n}
    (2\pi)^{-1/2}(s_i\tau)^{1/2} \exp\left( -\frac{1}{2}
                       (s_i\tau) \left(y_i-\mu\right)^{2}\right) \\
                   &=& (2\pi)^{-n/2} \tau^{n/2} \left(\prod_{i=1}^{n} s_i^{1/2}\right)
                       \exp\left(-\frac{1}{2} \tau \sum_{i=1}^{n} s_i
                       (y_i-\mu)^{2}\right)\\
                   &=& (2\pi)^{-n/2} \tau^{n/2} \left(\prod_{i=1}^{n} s_i^{1/2}\right)
                       \exp\left(-\frac{1}{2} m\tau \left[
                       (\bar{y} - \mu)^{2} + v \right]\right)
\end{eqnarray}
where
\begin{eqnarray}
  m &=& \sum_{i=1}^{n} s_i\nonumber\\
  \bar{y} &=& \frac{1}{m}\sum_{i=1}^{n} s_i y_i\nonumber\\
  v & =&  \frac{1}{m}\sum_{i=1}^{n} s_i y_i^2 -  {\bar{y}}^2 \nonumber
\end{eqnarray}

\subsection*{Link-function}

The mean $\mu$ is linked to the linear predictor $\eta$ by
\begin{displaymath}
    \mu = \eta
\end{displaymath}

\subsection*{Hyperparameters}

The hyperparameter is $\theta$, where
\begin{displaymath}
    \theta = \log \tau
\end{displaymath}
and the prior is defined on $\theta$.

\subsection*{Specification}

This family require the response to be an \texttt{inla.mdata}-object,
where each row\footnote{It is a list of vectors, so not strictly a
    ``row''.} is
\begin{displaymath}
    (v, \,
    \frac{1}{2}\sum_{i=1}^{n} \log(s_i), \,
    m, \,
    n, \,
    \bar{y})
\end{displaymath}
This object is most easily constructed using the
\texttt{inla.agaussian()} function, which gives the object to use
directly.

\begin{itemize}
\item $\text{family}=\texttt{agaussian}$
\item Required arguments: \texttt{inla.mdata(}$Y,S$\texttt{)}.
    Argument $S$ is optional, and is not given, then $s_i$ for $i$.
\end{itemize}
Here, $Y$ is a matrix where each row is a vector of observations. NA's
in each row are remove. The matrix $S$ is optional

\subsubsection*{Hyperparameter spesification and default values}
%% DO NOT EDIT!
%% This file is generated automatically from models.R
\begin{description}
	\item[doc] The aggregated Gaussian likelihoood
	\item[hyper]\ 
	 \begin{description}
	 	\item[theta]\ 
	 	 \begin{description}
	 	 	\item[hyperid] 66001
	 	 	\item[name] log precision
	 	 	\item[short.name] prec
	 	 	\item[initial] 4
	 	 	\item[fixed] FALSE
	 	 	\item[prior] loggamma
	 	 	\item[param] 1 5e-05
	 	 	\item[to.theta] \verb!function(x) log(x)!
	 	 	\item[from.theta] \verb!function(x) exp(x)!
	 	 \end{description}
	 \end{description}
	\item[status] experimental
	\item[survival] FALSE
	\item[discrete] FALSE
	\item[link] default identity logit loga cauchit log logoffset
	\item[pdf] agaussian
\end{description}


\subsection*{Example}

\verbatiminput{example-agaussian.R}

\subsection*{Notes}

\begin{itemize}
\item Thanks to JW for suggesting this formulation and for providing
    the details.
\end{itemize}


\end{document}


% LocalWords: 

%%% Local Variables: 
%%% TeX-master: t
%%% End: 

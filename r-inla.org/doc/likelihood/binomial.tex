\documentclass[a4paper,11pt]{article}
\usepackage[scale={0.8,0.9},centering,includeheadfoot]{geometry}
\usepackage{amstext}
\usepackage{amsmath}
\usepackage{verbatim}

\begin{document}
\section*{Binomial and negative binomial distribution}

\subsection*{Parametrisation}

The Binomial distribution is
\begin{displaymath}
    \text{Prob}(y) = {n \choose y} \ p^y (1-p)^{n-y}
\end{displaymath}
for responses $y=0, 1, 2, \ldots,n$, where
\begin{description}
\item[$n$:] number of trials.
\item[$p$:] probability of success in each trial.
\end{description}
The negative binomial distribution is
\begin{displaymath}
    \text{Prob}(n) = {n-1 \choose y-1} \ p^y (1-p)^{n-y}
\end{displaymath}
for given $y=1, 2, \dots$ and response $n-y=0,1, 2, \ldots$.

\subsection*{Link-function}

The mean and variance of $y$ are given in the binomial case as
\begin{displaymath}
    \mu = np \qquad\text{and}\qquad \sigma^{2} = np(1-p)
\end{displaymath}
and the probability $p$ is linked to the linear predictor by
\begin{displaymath}
    p(\eta) = \frac{\exp(\eta)}{1+\exp(\eta)}
\end{displaymath}

\subsubsection*{Hyperparameters}
None.

\subsubsection*{Hyperparameter spesification and default values}
%% DO NOT EDIT!
%% This file is generated automatically from models.R
\begin{description}
	\item[doc] The Binomial likelihood
	\item[hyper]\ 
	\item[survival] FALSE
	\item[discrete] TRUE
	\item[link] default logit cauchit probit cloglog loglog log sslogit logitoffset
	\item[pdf] binomial
\end{description}


\subsection*{Specification}

\begin{itemize}
\item \texttt{family="binomial"}
\item Required arguments: $y$ and $n$ (keyword \texttt{Ntrials})
\item Optional argument: \texttt{variant=0} for binomial (default),
    and \texttt{variant=1} for the negative binomial.
\end{itemize}


\subsection*{Expert version}

There is also an ``expert'' version were you are supposed to know what
you are doing. Here, we allow $y$ and $n$ to be non-integers (whatever
that means), however, the condition $0\le y \le n$ apply. The
normalizing constant is computed as above using the integer part of
$y$ and $n$. This is similar to using \verb|floor(y)| and
\verb|floor(n)| in \verb|R|. The marginal likelihood estimate will in
this case make less sense.
\begin{itemize}
\item \texttt{family="xbinomial"}
\item Required arguments: $y$ and $n$ (keyword \texttt{Ntrials})
\item Optional argument: \texttt{scale=q}, which scales the
    probability with $0<q\le1$ into $p'$, where
    \begin{displaymath}
        p' = q p(\eta).
    \end{displaymath}
    By default, $q=1$. Note that ``fitted values'' will still be be
    $p(\eta)$. 
\end{itemize}
%% DO NOT EDIT!
%% This file is generated automatically from models.R
\begin{description}
	\item[doc] \verb!The Binomial likelihood (expert version)!
	\item[hyper]\ 
	\item[survival] \verb!FALSE!
	\item[discrete] \verb!TRUE!
	\item[link] \verb!default logit loga cauchit probit cloglog ccloglog loglog log sslogit logitoffset quantile pquantile robit sn powerlogit gev!
	\item[pdf] \verb!binomial!
	\item[status] \verb!experimental!
\end{description}


\subsection*{Examples}

In the following example we estimate the parameters in a simulated
example with binomial responses.
\verbatiminput{example-binomial.R}

In the following example we estimate the parameters in a simulated
example with binomial responses using the \texttt{scale}-argument as
well. This require the use of the expert-version ``xbinomial''.
\verbatiminput{example-xbinomial.R}

\subsection*{Notes}

\begin{itemize}
\item If the response is a \verb|factor| it must be converted to
    $\{0,1\}$ before calling \verb|inla()|, as this conversion is not
    done automatic (as for example in \verb|glm()|).
\item This version of the negative binomial mimics the binomial
    distribution, and the ``data'' kind of enter in the
    \texttt{Ntrials} argument (as $y$ is pre-determinded) which both
    can appear, and should appear, strange. There is also an
    alternative implementation, \texttt{family="nbinomial"}, which
    mimics the Poisson distribution.
\end{itemize}
\end{document}


% LocalWords:  np Hyperparameters Ntrials

%%% Local Variables: 
%%% TeX-master: t
%%% End: 

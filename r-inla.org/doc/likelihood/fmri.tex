\documentclass[a4paper,11pt]{article}
\usepackage[scale={0.8,0.9},centering,includeheadfoot]{geometry}
\usepackage{amstext}
\usepackage{amsmath}
\usepackage{verbatim}

\begin{document}
\section*{Likelihood \textbf{fmri}}

\subsection*{Parametrisation}

This is special parameterisation of the non-central $\chi_\nu$
distribution. Let $\{x_i\}$ are iid Gaussians with mean $\mu$ and
precision $\tau$, then
\begin{displaymath}
    z = \sqrt{\sum_{i=1}^{\nu} \tau x_i^{2}}
\end{displaymath}
is non-central $\chi$-distribution with (integer and fixed by design) $\nu>0$ degrees of
freedom, and non-centrality parameter
\begin{displaymath}
    \rho = \sqrt{\nu \tau \mu^{2}}.
\end{displaymath}
The observation $y$ is $y=z/\sqrt{\tau}$
and we're interested in the underlying true signal
\begin{displaymath}
    \lambda=\rho/\sqrt{\tau} = \sqrt{\nu\mu^{2}}
\end{displaymath}

\subsection*{Link-function}

The linkfunction is given as 
\begin{displaymath}
    \log(\lambda) =  \eta
\end{displaymath}
where $\eta$  is the linear predictor.

\subsection*{Hyperparameters}

The hyperparameters are ${\theta}=(\theta_1,\theta_2)$, where
\begin{displaymath}
    \tau = \exp(\theta_1)
\end{displaymath}
is the precision, and
\begin{displaymath}
    \nu = \theta_2
\end{displaymath}
For technical reasons, $\nu$ is implemented as a hyper-parameter, but
is required to be fixed. Hence, the initial value for $\theta_2$
defines the (fixed) value for $\nu$.

The prior is given on $\theta_1$.

\subsection*{Specification}

\begin{itemize}
\item \texttt{family="fmri"} or \texttt{family="fmrisurv"}
\item Required arguments: $y$ (and optional \texttt{scale} for
    \texttt{fmri} to scale $\tau$)
\end{itemize}

\subsubsection*{Hyperparameter spesification and default values}
%% DO NOT EDIT!
%% This file is generated automatically from models.R
\begin{description}
	\item[doc] \verb!fmri distribution (special nc-chi)!
	\item[hyper]\ 
	 \begin{description}
	 	\item[theta1]\ 
	 	 \begin{description}
	 	 	\item[hyperid] \verb!103101!
	 	 	\item[name] \verb!precision!
	 	 	\item[short.name] \verb!prec!
	 	 	\item[output.name] \verb!Precision for fmri!
	 	 	\item[output.name.intern] \verb!Log precision for fmri!
	 	 	\item[initial] \verb!0!
	 	 	\item[fixed] \verb!FALSE!
	 	 	\item[prior] \verb!loggamma!
	 	 	\item[param] \verb!10 10!
	 	 	\item[to.theta] \verb!function(x) log(x)!
	 	 	\item[from.theta] \verb!function(x) exp(x)!
	 	 \end{description}
	 	\item[theta2]\ 
	 	 \begin{description}
	 	 	\item[hyperid] \verb!103202!
	 	 	\item[name] \verb!dof!
	 	 	\item[short.name] \verb!df!
	 	 	\item[output.name] \verb!NOT IN USE!
	 	 	\item[output.name.intern] \verb!NOT IN USE!
	 	 	\item[initial] \verb!4!
	 	 	\item[fixed] \verb!TRUE!
	 	 	\item[prior] \verb!normal!
	 	 	\item[param] \verb!0 1!
	 	 	\item[to.theta] \verb!function(x) x!
	 	 	\item[from.theta] \verb!function(x) x!
	 	 \end{description}
	 \end{description}
	\item[status] \verb!experimental!
	\item[survival] \verb!FALSE!
	\item[discrete] \verb!FALSE!
	\item[link] \verb!default log!
	\item[pdf] \verb!fmri!
\end{description}

%% DO NOT EDIT!
%% This file is generated automatically from models.R
\begin{description}
	\item[doc] \verb!fmri distribution (special nc-chi)!
	\item[hyper]\ 
	 \begin{description}
	 	\item[theta1]\ 
	 	 \begin{description}
	 	 	\item[hyperid] \verb!104101!
	 	 	\item[name] \verb!precision!
	 	 	\item[short.name] \verb!prec!
	 	 	\item[output.name] \verb!Precision for fmrisurv!
	 	 	\item[output.name.intern] \verb!Log precision for fmrisurv!
	 	 	\item[initial] \verb!0!
	 	 	\item[fixed] \verb!FALSE!
	 	 	\item[prior] \verb!loggamma!
	 	 	\item[param] \verb!10 10!
	 	 	\item[to.theta] \verb!function(x) log(x)!
	 	 	\item[from.theta] \verb!function(x) exp(x)!
	 	 \end{description}
	 	\item[theta2]\ 
	 	 \begin{description}
	 	 	\item[hyperid] \verb!104201!
	 	 	\item[name] \verb!dof!
	 	 	\item[short.name] \verb!df!
	 	 	\item[output.name] \verb!NOT IN USE!
	 	 	\item[output.name.intern] \verb!NOT IN USE!
	 	 	\item[initial] \verb!4!
	 	 	\item[fixed] \verb!TRUE!
	 	 	\item[prior] \verb!normal!
	 	 	\item[param] \verb!0 1!
	 	 	\item[to.theta] \verb!function(x) x!
	 	 	\item[from.theta] \verb!function(x) x!
	 	 \end{description}
	 \end{description}
	\item[status] \verb!experimental!
	\item[survival] \verb!TRUE!
	\item[discrete] \verb!FALSE!
	\item[link] \verb!default log!
	\item[pdf] \verb!fmri!
\end{description}


\subsection*{Example}

In the following example we estimate the parameters in a simulated
example.
\verbatiminput{example-fmri.R}

\subsection*{Notes}

Thanks to LS for providing all the details and a robust implementation
of this likelihood.

\end{document}

% LocalWords: 

%%% Local Variables: 
%%% TeX-master: t
%%% End: 

\documentclass[a4paper,11pt]{article}
\usepackage[scale={0.8,0.9},centering,includeheadfoot]{geometry}
\usepackage{amstext}
\usepackage{amsmath,amssymb}
\usepackage{verbatim}

\begin{document}

\section*{Generalized Poisson}

The generalized Poisson distribution is given by
\begin{displaymath}
    f(y|\lambda,w) = \frac{\lambda(\lambda+wy)^{y-1}}{y!}
    \exp(-(\lambda+wy))
\end{displaymath}
for $y=0, 1, 2, \ldots$ and where $\lambda>0$ and
$\max(-1,-\lambda/4)\leq w\leq 1$. The mean and variance of $y$ are
\begin{displaymath}
    \mu =\lambda(1-w)^{-1} \qquad\text{and}\qquad
    \sigma^{2} = \lambda(1-w)^{-3}= \mu(1-w)^{-2}.
\end{displaymath}
Since the dispersion parameter $w$ influence the mean as well as the
variance, we will use the following parameterisation (ADD REFERENCE?)
\begin{displaymath}
    w=\frac{\varphi\mu^{p-1}}{1+\varphi\mu^{p-1}}, 
\end{displaymath}
which gives the following density
\begin{displaymath} 
    f(y|\mu,\varphi,p) =\frac{ \mu(\mu+\varphi\mu^{p-1} y)^{y-1}}%%
    {(1+\varphi\mu^{p-1})^{y}y!}
    \exp\left(-\frac{\mu+\varphi\mu^{p-1} y}{1+\varphi\mu^{p-1}}\right)
\end{displaymath}
for $y=0, 1, 2, \ldots$. We assume $\varphi \ge 0$.

\subsection*{Link-function}

The mean and variance of $y$ are given as
\begin{displaymath}
    \text{E}(y|.) = \mu \qquad\text{and}\qquad
    \text{Var}(y|.) = \mu\left(1+\varphi\mu^{p-1}\right)^{2}
\end{displaymath}
and the mean is linked to the linear predictor by
\begin{displaymath}
    \mu = E \exp(\eta)
\end{displaymath}

\subsection*{Hyperparameters}

The overdispersion parameter $\varphi \ge 0$ is represented as
\begin{displaymath}
    \varphi = \exp(\theta_{1}) 
\end{displaymath}
The ``shape'' parameter $p$ is represented as
\begin{displaymath}
    p = \theta_{2}
\end{displaymath}
Note that $\theta_{2}=1$ and \texttt{fixed = TRUE}, default. The
prior is defined on ${\theta}=(\theta_1,\theta_2)$.


\subsection*{Specification}

\begin{itemize}
\item \texttt{family="gpoisson"}
\end{itemize}


\subsubsection*{Hyperparameter spesification and default values}
%% DO NOT EDIT!
%% This file is generated automatically from models.R
\begin{description}
	\item[doc] \verb!The generalized Poisson likelihood!
	\item[hyper]\ 
	 \begin{description}
	 	\item[theta1]\ 
	 	 \begin{description}
	 	 	\item[hyperid] \verb!56001!
	 	 	\item[name] \verb!overdispersion!
	 	 	\item[short.name] \verb!phi!
	 	 	\item[output.name] \verb!Overdispersion for gpoisson!
	 	 	\item[output.name.intern] \verb!Log overdispersion for gpoisson!
	 	 	\item[initial] \verb!0!
	 	 	\item[fixed] \verb!FALSE!
	 	 	\item[prior] \verb!loggamma!
	 	 	\item[param] \verb!1 1!
	 	 	\item[to.theta] \verb!function(x) log(x)!
	 	 	\item[from.theta] \verb!function(x) exp(x)!
	 	 \end{description}
	 	\item[theta2]\ 
	 	 \begin{description}
	 	 	\item[hyperid] \verb!56002!
	 	 	\item[name] \verb!p!
	 	 	\item[short.name] \verb!p!
	 	 	\item[output.name] \verb!Parameter p for gpoisson!
	 	 	\item[output.name.intern] \verb!Parameter p_intern for gpoisson!
	 	 	\item[initial] \verb!1!
	 	 	\item[fixed] \verb!TRUE!
	 	 	\item[prior] \verb!normal!
	 	 	\item[param] \verb!1 100!
	 	 	\item[to.theta] \verb!function(x) x!
	 	 	\item[from.theta] \verb!function(x) x!
	 	 \end{description}
	 \end{description}
	\item[survival] \verb!FALSE!
	\item[discrete] \verb!TRUE!
	\item[link] \verb!default log logoffset!
	\item[pdf] \verb!gpoisson!
	\item[status] \verb!experimental!
\end{description}

    
\subsection*{Example}

In the following example we estimate the parameters in a simulated
example with generalized Poisson responses.
\verbatiminput{example-gpoisson.R}


\subsection*{Notes}

The parameter $p$ is default fixed to be $1$. Allowing it to be
estimated jointly with the overdispersion parameter, please note the
following.
\begin{itemize}
\item The parameter $p$ and the overdispersion parameter are strongly
    correlated when estimated jointly.
\item You may want to chose an informative prior for $p$, as the shape
    of the likelihood might not be want you expect for ``extreme'' $p$.
\item You may experience problems in the numerical optimization (fail
    to converge); a more informative prior (if available) for $p$ will
    help with this issue.
\end{itemize}


\end{document}

\documentclass[a4paper,11pt]{article}
\usepackage[scale={0.8,0.9},centering,includeheadfoot]{geometry}
\usepackage{amstext}
\usepackage{verbatim}
\begin{document}

\section*{Gaussian prior}

\subsection*{Parametrisation}
The normal/Gaussian distribution has density
\begin{equation}
    \pi(\theta)=\left(\frac{\tau}{2\pi}\right)^{1/2}
    \exp\left(-\frac{\tau}{2}(\theta-\mu)^2\right)
\end{equation}
for continuous $\theta$ where
\begin{description}
\item[$\mu$:] is the mean
\item[$\tau$:] is precision.
\end{description}

\subsection*{Specification}
The Gaussian prior for the hyperparameters is specified inside the
\texttt{f()} function as following using the old style
\begin{center}
    \texttt{f( <whatever> , prior="normal", param=c(<mean>, <precision>) )}
\end{center}
\begin{center}
    \texttt{f( <whatever> , prior="gaussian", param=c(<mean>, <precision>) )}
\end{center}
or better, using the new style
\begin{center}
    \texttt{f( <whatever> , hyper = list(<theta> = list(prior="gaussian", param=c(<mean>, <precision>))))}
\end{center}
Similar if there are more than one hyperparameter for that particular
\texttt{f}-model. In the case where we want to specify the prior for
the hyperparameter of an observation model, for example the negative
Binomial, the the prior spesification will appear inside the
\texttt{control.family()}-argument; see the following example for
illustration.

\subsection*{Example}
\verbatiminput{example-gaussian.R}

\subsection*{Notes}

None.

\end{document}


% LocalWords:  hyperparameters param gaussian hyperparameter

%%% Local Variables: 
%%% TeX-master: t
%%% End: 

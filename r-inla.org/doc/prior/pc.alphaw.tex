\documentclass[a4paper,11pt]{article}
\usepackage[scale={0.8,0.9},centering,includeheadfoot]{geometry}
\usepackage{amstext}
\usepackage{listings}
\begin{document}

\section*{PC prior for $\alpha$ in the Weibull likelihood}

\subsection*{Parametrization}
The PC prior for the $\alpha$ parameter in the Weibull likelihood, has
\begin{displaymath}
    \text{KLD}(\alpha) = (\Gamma((1 + \alpha)/\alpha)\alpha + \alpha
    \log(\alpha) -
    \alpha\gamma + \gamma - \alpha)/\alpha
\end{displaymath}
where $\gamma = 0.5772156649...$ is Euler's constant. The base-model is
$\alpha=1$, and the expression for the prior follow from
\begin{displaymath}
    d(\alpha) = \sqrt{2\text{KLD}(\alpha)}
\end{displaymath}
and
\begin{displaymath}
    \pi(\alpha) = \frac{\lambda}{2} \exp(-\lambda d(\alpha)) \left|\frac{\partial
    d(\alpha)}{\partial\alpha}\right|
\end{displaymath}
for $\lambda>0$ and $\alpha{} > 0$. The density, cumulative
distribution function, quantile function, and a random number
generator for this distribution are implemented in the
\texttt{inla.pc.\{d,p,q,r\}alphaw} functions. Internally, R-INLA uses
$\theta = \log(\alpha)$ rather than $\alpha$, and the prior for
$\theta$ follows accordingly.

\subsection*{Specification}
This prior for the hyperparameters is specified inside the
\texttt{hyper}-specification, as
\begin{center}
    \texttt{hyper = list(<theta> =
        list(prior="pc.alphaw", param= <lambda>))}
\end{center}

\subsection*{Example}

\subsection*{Notes}

\begin{itemize}
\item The default value is $\lambda = 5$.
\item See also functions \texttt{inla.pc.\{d,p,q,r\}alphaw}
\end{itemize}

\end{document}

% LocalWords:  Weibull Parametrization KLD quantile inla pc alphaw INLA param
% LocalWords:  hyperparameters logalphaw

%%% Local Variables: 
%%% TeX-master: t
%%% End: 

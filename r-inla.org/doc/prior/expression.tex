\documentclass[a4paper,11pt]{article}
\usepackage[scale={0.8,0.9},centering,includeheadfoot]{geometry}
\usepackage{amstext}
\usepackage{verbatim}
\begin{document}

\section*{``Expression'': a do-it-yourself prior}

This prior allow the user to define an expression for the log-density
of any (univariate) prior $\log\pi(\theta)$, as a function of the
corresponding $\theta$ (which is in the internal scale; be aware).

The expression is evaluated using the
\texttt{muparser}-library\footnote{See
    \texttt{http://muparser.sourceforge.net/} for more documentation},
with some local configuration changes to make it more
``\texttt{R}''-like in style. The format is
\begin{quote}
    \texttt{expression: <statement>; <statement>; ...; return(<value>)}
\end{quote}
where ``\texttt{<statement>}'' is any regular statement (more below)
and value returned, ``\texttt{<value>}'' is the value for the
log-density of the prior, evaluated at the current value for $\theta$.

The following expression implements the normal prior (in not a good
way...)
\begin{verbatim}
expression:
        mean = 0; sigma = 1;
        dens = 1/sqrt(2*pi) * 1/sigma * exp(-0.5*(x-mean)^2/sigma^2);
        logdens = log(dens);
        return(logdens)
\end{verbatim}
All variables in the expression are initialised with the current value
of $\theta$ before the expression is evaluated. In this way, the
variable \verb|x| in this example will be $\theta$.

\subsection*{Notes}
\begin{enumerate}
\item \verb|return (x)| (with a space before ``(.)'') is NOT allowed,
    it must be \verb|return(x)|.
\item A ``;'' is needed to terminate each expression, a newline DOES
    NOT terminate an expression.
\item You can use ``\verb|_|'' in variable-names, like
    \verb|log_precision = <whatever>|; see the following example.
\end{enumerate}

\subsection*{Known functions}
Known functions (besides common math-functions like ``$\exp(\cdot)$'',
``$\sin(\cdot)$'', etc...) are
\begin{itemize}
\item \verb|gamma(x)| is the Gamma-function and \verb|lgamma(x)| is
    its $\log$, and the derivaties \verb|digamma(x)| and
    \verb|trigamma(x)| (see \verb|?gamma| in \verb|R|).
\item \verb|pi| is $\pi$
\item $x^y$ is expressed as either \verb|x^y| or \verb|pow(x;y)|
\end{itemize}

\subsection*{Example}
\verbatiminput{example-expression.R}

\end{document}


% LocalWords:  hyperparameters param gaussian hyperparameter univariate lgamma

%%% Local Variables: 
%%% TeX-master: t
%%% End: 
% LocalWords:  muparser

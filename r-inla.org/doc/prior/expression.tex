\documentclass[a4paper,11pt]{article}
\usepackage[scale={0.8,0.9},centering,includeheadfoot]{geometry}
\usepackage{amstext}
\usepackage{verbatim}
\begin{document}

\section*{``Expression'': a do-it-yourself prior}

This prior allow the user to define an expression for the log-density
of any (univariate) prior, as a function of the corresponding $\theta$
(which is in the internal scale; be aware).

The expression is evaluated using the \texttt{muparser}\footnote{See
    \texttt{http://muparser.sourceforge.net/} for more documentation},
with some local configuration changes to make it more
``\texttt{R}''-like in style.

The format is
\begin{quote}
    \texttt{expression: <statement>; <statement>; ...; return(<value>)}
\end{quote}
where ``\texttt{<statement>}'' is any regular statement (more below)
and ``\texttt{<value>}'' is the value for the log-density of the
prior, evaluated at the ``free parameter(s)''.

The following expression implements the normal prior
\begin{verbatim}
expression: mean = 0; sigma = 1;
        logdens = 1/sqrt(2*pi) * 1/sigma * exp(-0.5*(x-mean)^2/sigma^2);
        return(logdens)
\end{verbatim}
Since \verb|x| is a variable that is not assigned any value (its a
``free parameter''), it must the argument (i.e.\ $\theta$) to this
function. If there are more than one free parameter, then all of them
are assigned to $\theta$ before the expression is evaluated.

\subsection*{Note}
\begin{enumerate}
\item \verb|return (x)| (with a space before ``(.)'') is not allowed,
    it must be \verb|return(x)|.
\item You need a ``;'' to terminate each expression, a newline DOES
    NOT terminate an expression!
\item You can use a ``\verb|_|'' in variable-names,
    like \verb|log_precision = ...|; see the following example. 
\end{enumerate}

\subsection*{Known functions}
Known functions (besides common math-functions like ``exp'',
``sin'', etc...) are 
\begin{itemize}
\item \verb|gamma(x)| is the Gamma-function and \verb|lgamma(x)| is
    its log (see \verb|?gamma| in \verb|R|).
\item \verb|pi| is $\pi$
\item $x^y$ is expressed as  \verb|x^y| or \verb|pow(x;y)|
\end{itemize}

\subsection*{Example}
\verbatiminput{example-expression.R}

\subsection*{Notes}

None.

\end{document}


% LocalWords:  hyperparameters param gaussian hyperparameter univariate lgamma

%%% Local Variables: 
%%% TeX-master: t
%%% End: 
% LocalWords:  muparser

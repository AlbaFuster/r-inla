\documentclass[a4paper,11pt]{article}
\usepackage[scale={0.8,0.9},centering,includeheadfoot]{geometry}
\usepackage{amstext}
\usepackage{listings}
\begin{document}

\section*{Sigmoidal effect of a covariate}

\subsection*{Parametrization}

This model implements a non-linear effect of a positive covariate $x$
as a part of the linear predictor. It comes in two variants,
\emph{sigmoidal}
\begin{displaymath}
    \beta \frac{x^{k}}{x^{k} + a^{k}}\qquad\text{or}\qquad \beta \frac{z}{z+1}
\end{displaymath}
with $z    = (x/a)^{k}$, 
 $x\ge 0$, $k>0$ and $a>0$, and the \emph{reverse-sigmoidal}
\begin{displaymath}
    \beta \frac{a^{k}}{x^{k} + a^{k}} \qquad\text{or}\qquad \beta \frac{1}{z+1}.
\end{displaymath}
Here, $a$ is the halflife parameter, $k$ the shape-parameter and
$\beta$ the scaling.

\subsection*{Hyperparameters}

This model has three hyperparameters, the scaling $\beta$, halflife
$a$ and shape $k$,
\begin{displaymath}
    \theta_1 = \beta \qquad \theta_2 = \log(a) \qquad  \theta_3 = \log(k)
\end{displaymath}
and the priors are given for $\theta_1, \theta_2$ and $\theta_3$.


\subsection*{Specification}

\begin{verbatim}
    f(x, model="sigm",    hyper = ..., precision = <precision>)
    f(x, model="revsigm", hyper = ..., precision = <precision>)
\end{verbatim}
where \texttt{precision} is the precision for the tiny noise used to
implement this as a latent model. 

\subsubsection*{Hyperparameter specification and default values}
%% DO NOT EDIT!
%% This file is generated automatically from models.R
\begin{description}
	\item[doc] Sigmoidal effect of a covariate
	\item[hyper]\ 
	 \begin{description}
	 	\item[theta1]\ 
	 	 \begin{description}
	 	 	\item[hyperid] 38001
	 	 	\item[name] beta
	 	 	\item[short.name] b
	 	 	\item[initial] 1
	 	 	\item[fixed] FALSE
	 	 	\item[prior] normal
	 	 	\item[param] 1 10
	 	 	\item[to.theta] \verb!function(x) x!
	 	 	\item[from.theta] \verb!function(x) x!
	 	 \end{description}
	 	\item[theta2]\ 
	 	 \begin{description}
	 	 	\item[hyperid] 38002
	 	 	\item[name] loghalflife
	 	 	\item[short.name] halflife
	 	 	\item[initial] 3
	 	 	\item[fixed] FALSE
	 	 	\item[prior] loggamma
	 	 	\item[param] 3 1
	 	 	\item[to.theta] \verb!function(x) log(x)!
	 	 	\item[from.theta] \verb!function(x) exp(x)!
	 	 \end{description}
	 	\item[theta3]\ 
	 	 \begin{description}
	 	 	\item[hyperid] 38003
	 	 	\item[name] logshape
	 	 	\item[short.name] shape
	 	 	\item[initial] 0
	 	 	\item[fixed] FALSE
	 	 	\item[prior] loggamma
	 	 	\item[param] 10 10
	 	 	\item[to.theta] \verb!function(x) log(x)!
	 	 	\item[from.theta] \verb!function(x) exp(x)!
	 	 \end{description}
	 \end{description}
	\item[constr] FALSE
	\item[nrow.ncol] FALSE
	\item[augmented] FALSE
	\item[aug.factor] 1
	\item[aug.constr] 
	\item[n.div.by] 
	\item[n.required] FALSE
	\item[set.default.values] FALSE
	\item[status] experimental
	\item[pdf] sigm
\end{description}


\subsection*{Example}

\begin{verbatim}
sigm = function(x, halflife, shape = 1)
{
    xx = (x/halflife)^shape
    return (xx/(1.0+xx))
}
revsigm = function(x, halflife, shape = 1)
{
    xx = (x/halflife)^shape
    return (1.0/(1.0+xx))
}

n = 1000
lambda = 10
s=0.01
x = rpois(n, lambda = lambda)
halflife = lambda
shape = 2

y = sigm(x, halflife, shape) + rnorm(n, sd = s)
r = inla(y ~ -1 + f(x, model="sigm"),
        data = data.frame(y, x),
        family = "gaussian",
        control.family = list(
                hyper = list(
                        prec = list(
                                initial = log(1/s^2),
                                fixed = TRUE))))
summary(r)

y = revsigm(x, halflife, shape) + rnorm(n, sd = s)
r = inla(y ~ -1 + f(x, model="revsigm"),
        data = data.frame(y, x),
        family = "gaussian",
        control.family = list(
                hyper = list(
                        prec = list(
                                initial = log(1/s^2),
                                fixed = TRUE))))
summary(r)
\end{verbatim}

\subsection*{Notes}
None

\end{document}

% LocalWords: 

%%% Local Variables: 
%%% TeX-master: t
%%% End: 


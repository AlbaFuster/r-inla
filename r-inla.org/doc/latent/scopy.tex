\documentclass[a4paper,11pt]{article}
\usepackage[scale={0.8,0.9},centering,includeheadfoot]{geometry}
\usepackage{amstext}
\usepackage{amsmath}
\usepackage{listings}
\usepackage{verbatim}
\usepackage{block}

\begin{document}

\section*{Smooth-Copy of another model component: "scopy"}

This model is a generalization of \texttt{copy}, please refer to
\texttt{inla.doc("copy")} first.

This describes the way to copy another model component with an
optional smooth/spline scaling, like with
\begin{displaymath}
    \eta = u + v
\end{displaymath}
where $v$ is a smooth copy of $u$ (component-wise)
\begin{displaymath}
    v = \beta(z)\times\text{copy}(u)
\end{displaymath}
where $\beta(z)$, a smooth/spline function of the covariate $z$. The
smooth scaling is done \textbf{component-wise} for $u$, so if $u$ are
defined with domain $(1, 2, \ldots, m)$, i.e.\
$u=(u_1, u_2, \ldots, u_m)$, then $z$ must be
$z=(z_1, z_2, \ldots, z_m)$, so that
\begin{displaymath}
    v_i = \beta(z_i) u_i, \qquad i=1, 2, \ldots, m.
\end{displaymath}

\subsection*{Hyperparameters}

The hyperparameters are the value of the spline at $n$ fixed (equally
distant) locations, $(l_i, \beta_i)$, for $i=1, \ldots, n$. Let
$l_M = (l_n+l_1)/2$ be the mid-points of the locations and
$l_L=(l_n-l_1)$ its length. These $\beta$-parameters defines the
interpolating spline (of second order). Let $Q$ be the
scaled-precision matrix for RW2 and define the eigen decomposition, as
\begin{displaymath}
    Q = \sum_{i=1}^{n} \lambda_i v_i v_i^{T}
\end{displaymath}
where $\lambda_{n-1}=\lambda_{n}=0$, assuming descreasing eigenvalues.
We can use $v_{n} = (1, \ldots, 1)^{T}$ and
$v_{n-1} = (-0.5, \ldots, 0.5)^{T}$.
Define the matrix with scaled $v_i$' as columns,
\begin{displaymath}
    W = \begin{bmatrix}
          v_{n}^{T} &|& v_{n-1}^{T} &|&  \frac{1}{\sqrt{\lambda_{n-2}}} v^{T}_{n-2} &|& \ldots &|& \frac{1}{\sqrt{\lambda_{1}}} v^{T}_{1}
        \end{bmatrix}.
\end{displaymath}
Let the columns vectors of $W$ be $w_1, w_2, \ldots$. 
We parameterize the spline at locations $l_1, \ldots, l_n$, as
\beta=(\beta_1, \ldots, \beta_n)$, as
\begin{displaymath}
    \beta = \sum_{i=1}^{n} \theta_i w_i, \qquad i=1, \ldots, n,
\end{displaymath}
In this way, $\theta_1$ is the overall mean, $\theta_2$ is the
(dimension-less) slope, and $\theta_{3}, \ldots, \theta_n$ are weights
for the basis-function expansion of the spline that is beyond the
constant and the linear term.

Since $Q$ is scaled, then using independent $N(0,1)$ prior for each
$\theta_{3}, \ldots, \theta_n$ will make the prior deviation from the
mean and slope, also $N(0,1)$ (in an average sense). So only one
common scaling of the prior precisions for these parameters of this
prior are needed to shrink it more, or to shrink it less. For this
reason the prior for $\theta_{j}$, $j=3, \ldots, n$, is \emph{defined}
to be the same as the prior for $\theta_3$, hence only the prior for
$\theta_3$ needs to be specified.

Doing this from within \texttt{R}, we can evaluate the spline at any
point within $[l_1, l_n]$, we can use
\begin{verbatim}
sfun <- splinefun(loc, beta, method = "natural")
new.value <- sfun(new.loc)
\end{verbatim}
The functions \texttt{inla.scopy.summary} and
\texttt{INLA:::inla.scopy.define} can be consulted for further
details.


We can control $n$ and the covariate with
\texttt{control.scopy} within \texttt{f()},
\begin{quote}
    \texttt{control.scopy = list(\\
        covariate = ..., \\
        n = 9})
\end{quote}
where
\begin{description}
\item[covariate] gives the covariate that is used
\item[n] is the number of hyperparameters used in the spline
    ($5 \leq n \leq 15$).
\end{description}
The \texttt{f()}-argument \texttt{precision}, defines how close the
copy is, is similar as for model \texttt{copy}.

The priors for the mean, slope and the deviation from them, are given
by the \texttt{hyper}-argument. See also the example.

\clearpage
\section*{Spesification}
\documentclass[a4paper,11pt]{article}
\usepackage[scale={0.8,0.9},centering,includeheadfoot]{geometry}
\usepackage{amstext}
\usepackage{amsmath}
\usepackage{listings}
\usepackage{verbatim}
\usepackage{block}

\begin{document}

\section*{Smooth-Copy of another model component: "scopy"}

This model is a generalization of \texttt{copy}, please refer to
\texttt{inla.doc("copy")} first.

This describes the way to copy another model component with an
optional smooth/spline scaling, like with
\begin{displaymath}
    \eta = u + v
\end{displaymath}
where $v$ is a smooth copy of $u$ (component-wise)
\begin{displaymath}
    v = \beta(z)\times\text{copy}(u)
\end{displaymath}
where $\beta(z)$, a smooth/spline function of the covariate $z$. The
smooth scaling is done \textbf{component-wise} for $u$, so if $u$ are
defined with domain $(1, 2, \ldots, m)$, i.e.\
$u=(u_1, u_2, \ldots, u_m)$, then $z$ must be
$z=(z_1, z_2, \ldots, z_m)$, so that
\begin{displaymath}
    v_i = \beta(z_i) u_i, \qquad i=1, 2, \ldots, m.
\end{displaymath}

\subsection*{Hyperparameters}

The hyperparameters are the value of the spline at $n$ fixed (equally
distant) locations, $(l_i, \beta_i)$, for $i=1, \ldots, n$. Let
$l_M = (l_n+l_1)/2$ be the mid-points of the locations and
$l_L=(l_n-l_1)$ its length. These $\beta$-parameters defines the
interpolating spline (of second order). Let $Q$ be the
scaled-precision matrix for RW2 and define the eigen decomposition, as
\begin{displaymath}
    Q = \sum_{i=1}^{n} \lambda_i v_i v_i^{T}
\end{displaymath}
where $\lambda_{n-1}=\lambda_{n}=0$, assuming descreasing eigenvalues.
We can use $v_{n} = (1, \ldots, 1)^{T}$ and
$v_{n-1} = (-0.5, \ldots, 0.5)^{T}$.
Define the matrix with scaled $v_i$' as columns,
\begin{displaymath}
    W = \begin{bmatrix}
          v_{n}^{T} &|& v_{n-1}^{T} &|&  \frac{1}{\sqrt{\lambda_{n-2}}} v^{T}_{n-2} &|& \ldots &|& \frac{1}{\sqrt{\lambda_{1}}} v^{T}_{1}
        \end{bmatrix}.
\end{displaymath}
Let the columns vectors of $W$ be $w_1, w_2, \ldots$. 
We parameterize the spline at locations $l_1, \ldots, l_n$, as
\beta=(\beta_1, \ldots, \beta_n)$, as
\begin{displaymath}
    \beta = \sum_{i=1}^{n} \theta_i w_i, \qquad i=1, \ldots, n,
\end{displaymath}
In this way, $\theta_1$ is the overall mean, $\theta_2$ is the
(dimension-less) slope, and $\theta_{3}, \ldots, \theta_n$ are weights
for the basis-function expansion of the spline that is beyond the
constant and the linear term.

Since $Q$ is scaled, then using independent $N(0,1)$ prior for each
$\theta_{3}, \ldots, \theta_n$ will make the prior deviation from the
mean and slope, also $N(0,1)$ (in an average sense). So only one
common scaling of the prior precisions for these parameters of this
prior are needed to shrink it more, or to shrink it less. For this
reason the prior for $\theta_{j}$, $j=3, \ldots, n$, is \emph{defined}
to be the same as the prior for $\theta_3$, hence only the prior for
$\theta_3$ needs to be specified.

Doing this from within \texttt{R}, we can evaluate the spline at any
point within $[l_1, l_n]$, we can use
\begin{verbatim}
sfun <- splinefun(loc, beta, method = "natural")
new.value <- sfun(new.loc)
\end{verbatim}
The functions \texttt{inla.scopy.summary} and
\texttt{INLA:::inla.scopy.define} can be consulted for further
details.


We can control $n$ and the covariate with
\texttt{control.scopy} within \texttt{f()},
\begin{quote}
    \texttt{control.scopy = list(\\
        covariate = ..., \\
        n = 9})
\end{quote}
where
\begin{description}
\item[covariate] gives the covariate that is used
\item[n] is the number of hyperparameters used in the spline
    ($5 \leq n \leq 15$).
\end{description}
The \texttt{f()}-argument \texttt{precision}, defines how close the
copy is, is similar as for model \texttt{copy}.

The priors for the mean, slope and the deviation from them, are given
by the \texttt{hyper}-argument. See also the example.

\clearpage
\section*{Spesification}
\documentclass[a4paper,11pt]{article}
\usepackage[scale={0.8,0.9},centering,includeheadfoot]{geometry}
\usepackage{amstext}
\usepackage{amsmath}
\usepackage{listings}
\usepackage{verbatim}
\usepackage{block}

\begin{document}

\section*{Smooth-Copy of another model component: "scopy"}

This model is a generalization of \texttt{copy}, please refer to
\texttt{inla.doc("copy")} first.

This describes the way to copy another model component with an
optional smooth/spline scaling, like with
\begin{displaymath}
    \eta = u + v
\end{displaymath}
where $v$ is a smooth copy of $u$ (component-wise)
\begin{displaymath}
    v = \beta(z)\times\text{copy}(u)
\end{displaymath}
where $\beta(z)$, a smooth/spline function of the covariate $z$. The
smooth scaling is done \textbf{component-wise} for $u$, so if $u$ are
defined with domain $(1, 2, \ldots, m)$, i.e.\
$u=(u_1, u_2, \ldots, u_m)$, then $z$ must be
$z=(z_1, z_2, \ldots, z_m)$, so that
\begin{displaymath}
    v_i = \beta(z_i) u_i, \qquad i=1, 2, \ldots, m.
\end{displaymath}

\subsection*{Hyperparameters}

The hyperparameters are the value of the spline at $n$ fixed (equally
distant) locations, $(l_i, \beta_i)$, for $i=1, \ldots, n$. Let
$l_M = (l_n+l_1)/2$ be the mid-points of the locations and
$l_L=(l_n-l_1)$ its length. These $\beta$-parameters defines the
interpolating spline (of second order). Let $Q$ be the
scaled-precision matrix for RW2 and define the eigen decomposition, as
\begin{displaymath}
    Q = \sum_{i=1}^{n} \lambda_i v_i v_i^{T}
\end{displaymath}
where $\lambda_{n-1}=\lambda_{n}=0$, assuming descreasing eigenvalues.
We can use $v_{n} = (1, \ldots, 1)^{T}$ and
$v_{n-1} = (-0.5, \ldots, 0.5)^{T}$.
Define the matrix with scaled $v_i$' as columns,
\begin{displaymath}
    W = \begin{bmatrix}
          v_{n}^{T} &|& v_{n-1}^{T} &|&  \frac{1}{\sqrt{\lambda_{n-2}}} v^{T}_{n-2} &|& \ldots &|& \frac{1}{\sqrt{\lambda_{1}}} v^{T}_{1}
        \end{bmatrix}.
\end{displaymath}
Let the columns vectors of $W$ be $w_1, w_2, \ldots$. 
We parameterize the spline at locations $l_1, \ldots, l_n$, as
\beta=(\beta_1, \ldots, \beta_n)$, as
\begin{displaymath}
    \beta = \sum_{i=1}^{n} \theta_i w_i, \qquad i=1, \ldots, n,
\end{displaymath}
In this way, $\theta_1$ is the overall mean, $\theta_2$ is the
(dimension-less) slope, and $\theta_{3}, \ldots, \theta_n$ are weights
for the basis-function expansion of the spline that is beyond the
constant and the linear term.

Since $Q$ is scaled, then using independent $N(0,1)$ prior for each
$\theta_{3}, \ldots, \theta_n$ will make the prior deviation from the
mean and slope, also $N(0,1)$ (in an average sense). So only one
common scaling of the prior precisions for these parameters of this
prior are needed to shrink it more, or to shrink it less. For this
reason the prior for $\theta_{j}$, $j=3, \ldots, n$, is \emph{defined}
to be the same as the prior for $\theta_3$, hence only the prior for
$\theta_3$ needs to be specified.

Doing this from within \texttt{R}, we can evaluate the spline at any
point within $[l_1, l_n]$, we can use
\begin{verbatim}
sfun <- splinefun(loc, beta, method = "natural")
new.value <- sfun(new.loc)
\end{verbatim}
The functions \texttt{inla.scopy.summary} and
\texttt{INLA:::inla.scopy.define} can be consulted for further
details.


We can control $n$ and the covariate with
\texttt{control.scopy} within \texttt{f()},
\begin{quote}
    \texttt{control.scopy = list(\\
        covariate = ..., \\
        n = 9})
\end{quote}
where
\begin{description}
\item[covariate] gives the covariate that is used
\item[n] is the number of hyperparameters used in the spline
    ($5 \leq n \leq 15$).
\end{description}
The \texttt{f()}-argument \texttt{precision}, defines how close the
copy is, is similar as for model \texttt{copy}.

The priors for the mean, slope and the deviation from them, are given
by the \texttt{hyper}-argument. See also the example.

\clearpage
\section*{Spesification}
\documentclass[a4paper,11pt]{article}
\usepackage[scale={0.8,0.9},centering,includeheadfoot]{geometry}
\usepackage{amstext}
\usepackage{amsmath}
\usepackage{listings}
\usepackage{verbatim}
\usepackage{block}

\begin{document}

\section*{Smooth-Copy of another model component: "scopy"}

This model is a generalization of \texttt{copy}, please refer to
\texttt{inla.doc("copy")} first.

This describes the way to copy another model component with an
optional smooth/spline scaling, like with
\begin{displaymath}
    \eta = u + v
\end{displaymath}
where $v$ is a smooth copy of $u$ (component-wise)
\begin{displaymath}
    v = \beta(z)\times\text{copy}(u)
\end{displaymath}
where $\beta(z)$, a smooth/spline function of the covariate $z$. The
smooth scaling is done \textbf{component-wise} for $u$, so if $u$ are
defined with domain $(1, 2, \ldots, m)$, i.e.\
$u=(u_1, u_2, \ldots, u_m)$, then $z$ must be
$z=(z_1, z_2, \ldots, z_m)$, so that
\begin{displaymath}
    v_i = \beta(z_i) u_i, \qquad i=1, 2, \ldots, m.
\end{displaymath}

\subsection*{Hyperparameters}

The hyperparameters are the value of the spline at $n$ fixed (equally
distant) locations, $(l_i, \beta_i)$, for $i=1, \ldots, n$. Let
$l_M = (l_n+l_1)/2$ be the mid-points of the locations and
$l_L=(l_n-l_1)$ its length. These $\beta$-parameters defines the
interpolating spline (of second order). Let $Q$ be the
scaled-precision matrix for RW2 and define the eigen decomposition, as
\begin{displaymath}
    Q = \sum_{i=1}^{n} \lambda_i v_i v_i^{T}
\end{displaymath}
where $\lambda_{n-1}=\lambda_{n}=0$, assuming descreasing eigenvalues.
We can use $v_{n} = (1, \ldots, 1)^{T}$ and
$v_{n-1} = (-0.5, \ldots, 0.5)^{T}$.
Define the matrix with scaled $v_i$' as columns,
\begin{displaymath}
    W = \begin{bmatrix}
          v_{n}^{T} &|& v_{n-1}^{T} &|&  \frac{1}{\sqrt{\lambda_{n-2}}} v^{T}_{n-2} &|& \ldots &|& \frac{1}{\sqrt{\lambda_{1}}} v^{T}_{1}
        \end{bmatrix}.
\end{displaymath}
Let the columns vectors of $W$ be $w_1, w_2, \ldots$. 
We parameterize the spline at locations $l_1, \ldots, l_n$, as
\beta=(\beta_1, \ldots, \beta_n)$, as
\begin{displaymath}
    \beta = \sum_{i=1}^{n} \theta_i w_i, \qquad i=1, \ldots, n,
\end{displaymath}
In this way, $\theta_1$ is the overall mean, $\theta_2$ is the
(dimension-less) slope, and $\theta_{3}, \ldots, \theta_n$ are weights
for the basis-function expansion of the spline that is beyond the
constant and the linear term.

Since $Q$ is scaled, then using independent $N(0,1)$ prior for each
$\theta_{3}, \ldots, \theta_n$ will make the prior deviation from the
mean and slope, also $N(0,1)$ (in an average sense). So only one
common scaling of the prior precisions for these parameters of this
prior are needed to shrink it more, or to shrink it less. For this
reason the prior for $\theta_{j}$, $j=3, \ldots, n$, is \emph{defined}
to be the same as the prior for $\theta_3$, hence only the prior for
$\theta_3$ needs to be specified.

Doing this from within \texttt{R}, we can evaluate the spline at any
point within $[l_1, l_n]$, we can use
\begin{verbatim}
sfun <- splinefun(loc, beta, method = "natural")
new.value <- sfun(new.loc)
\end{verbatim}
The functions \texttt{inla.scopy.summary} and
\texttt{INLA:::inla.scopy.define} can be consulted for further
details.


We can control $n$ and the covariate with
\texttt{control.scopy} within \texttt{f()},
\begin{quote}
    \texttt{control.scopy = list(\\
        covariate = ..., \\
        n = 9})
\end{quote}
where
\begin{description}
\item[covariate] gives the covariate that is used
\item[n] is the number of hyperparameters used in the spline
    ($5 \leq n \leq 15$).
\end{description}
The \texttt{f()}-argument \texttt{precision}, defines how close the
copy is, is similar as for model \texttt{copy}.

The priors for the mean, slope and the deviation from them, are given
by the \texttt{hyper}-argument. See also the example.

\clearpage
\section*{Spesification}
\input{../hyper/latent/scopy.tex}

\clearpage
\subsection*{Example}
Just simulate some data and estimate the parameters back. 
{\small\verbatiminput{example-scopy.R}}

\subsection*{Notes}

This model is experimental.

\end{document}



% LocalWords: 

%%% Local Variables: 
%%% TeX-master: t
%%% End: 


\clearpage
\subsection*{Example}
Just simulate some data and estimate the parameters back. 
{\small\verbatiminput{example-scopy.R}}

\subsection*{Notes}

This model is experimental.

\end{document}



% LocalWords: 

%%% Local Variables: 
%%% TeX-master: t
%%% End: 


\clearpage
\subsection*{Example}
Just simulate some data and estimate the parameters back. 
{\small\verbatiminput{example-scopy.R}}

\subsection*{Notes}

This model is experimental.

\end{document}



% LocalWords: 

%%% Local Variables: 
%%% TeX-master: t
%%% End: 


\clearpage
\subsection*{Example}
Just simulate some data and estimate the parameters back. 
{\small\verbatiminput{example-scopy.R}}

\subsection*{Notes}

This model is experimental.

\end{document}



% LocalWords: 

%%% Local Variables: 
%%% TeX-master: t
%%% End: 

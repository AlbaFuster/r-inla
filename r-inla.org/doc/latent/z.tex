\documentclass[a4paper,11pt]{article}
\usepackage[scale={0.8,0.9},centering,includeheadfoot]{geometry}
\usepackage{amstext}
\usepackage{listings}
\begin{document}

\section*{The z-model}

\subsection*{Parametrization}

The z-model is an implementation of the ``classical'' way to define a
mixed model, through
\begin{displaymath}
    \eta = \ldots + Z z 
\end{displaymath}
where $Z$ is a $n\times m$ matrix and $z$ a vector of length $m$
representing zero-mean ``random effects''. The z-model is defined as
the augmented model
\begin{displaymath}
    \widetilde{z} =
    \begin{pmatrix}
        Zz \\
        z
    \end{pmatrix}
\end{displaymath}
where the precision matrix for $z$ is $\tau C$ where $C > 0$ is a
$m\times m$ (fixed) matrix.

\subsection*{Hyperparameters}

The precision parameter of the z-model is represented as
\begin{displaymath}
    \theta = \log(\tau)
\end{displaymath}
and prior is assigned to $\theta$

\subsection*{Specification}

The z-model is specified inside the {\tt f()} function as
\begin{verbatim}
 f(<whatever>, model="z", Z = <Z>, Cmatrix = <Cmat>, hyper = <hyper>,
precision = <precision>)
\end{verbatim}
where the \texttt{Z}-matrix argument defines the $Z$ matrix and is
required. The \texttt{Cmatrix} defines the $C$ matrix and if not
given, taken to the the diagonal matrix with the appropriate
dimension. The \texttt{precision} parameter defines the precision for
$Zz | z$.

If $Z$ is a $n\times m$ matrix then the $C$ matrix must be $m\times m$
matrix, and $\widetilde z$ has length $n+m$. The $n$ first terms of
$\widetilde z$ is then $Zz$ and the last $m$ terms of $\widetilde z$
is then $z$.

\subsubsection*{Hyperparameter spesification and default values}
%% DO NOT EDIT!
%% This file is generated automatically from models.R
\begin{description}
	\item[hyper]\ 
	 \begin{description}
	 	\item[theta]\ 
	 	 \begin{description}
	 	 	\item[name] log precision
	 	 	\item[short.name] prec
	 	 	\item[initial] 4
	 	 	\item[fixed] FALSE
	 	 	\item[prior] loggamma
	 	 	\item[param] 1 5e-05
	 	 	\item[to.theta] \verb!function(x) log(x)!
	 	 	\item[from.theta] \verb!function(x) exp(x)!
	 	 \end{description}
	 \end{description}
	\item[constr] FALSE
	\item[nrow.ncol] FALSE
	\item[augmented] FALSE
	\item[aug.factor] 1
	\item[aug.constr] 
	\item[n.div.by] 
	\item[n.required] TRUE
	\item[set.default.values] TRUE
	\item[pdf] z
	\item[status] experimental
\end{description}



\subsection*{Example}

\subsection*{Notes}

\end{document}


% LocalWords: 

%%% Local Variables: 
%%% TeX-master: t
%%% End: 


\documentclass[a4paper,11pt]{article}
\usepackage[scale={0.8,0.9},centering,includeheadfoot]{geometry}
\usepackage{amstext}
\usepackage{listings}
\begin{document}

\section*{Besag model for spatial effects}

\subsection*{Parametrization}

The besag model for random vector $\mathbf{x}=(x_1,\dots,x_n)$ is defined as
\begin{equation}\label{eq.besag}
    x_i|x_j,i\neq j,\tau\sim\mathcal{N}(\frac{1}{n_i}\sum_{i\sim j}x_j,\frac{1}{n_i\tau})
\end{equation}

where $n_i$ is the number of neighbours of node $i$, $i\sim j$
indicates that the two nodes $i$ and $j$ are neighbours.  


\subsection*{Hyperparameters}

The precision parameter $\tau$ is represented as
\begin{displaymath}
    \theta_{1} =\log \tau
\end{displaymath}
and the prior is defined on $\theta_{1}$. 

\subsection*{Specification}

The besag model is specified inside the {\tt f()} function as
\begin{verbatim}
 f(<whatever>,model="besag",graph=<graph>,
   hyper=<hyper>, adjust.for.con.comp = TRUE,
   scale.model = FALSE)
\end{verbatim}

The neighbourhood structure of $\mathbf{x}$ is passed to the program
through the {\tt graph} argument.

If the option \verb|adjust.for.con.comp=TRUE| then the model is
adjusted if the graph has more than one connected compoment. This
adjustment can be disabled setting this option to \texttt{FALSE}. If
\verb|adjust.for.con.comp=TRUE| then \texttt{constr=TRUE} is
interpreted as a sum-to-zero constraint on \emph{each} connected
component in the graph and the \texttt{rankdef} parameter is set to
the number of connected components.

The logical option \verb|scale.model| determine if the model should be
scaled to have an average variance (the diagonal of the generalized
inverse) equal to 1. This makes prior spesification much
easier. Default is \verb|FALSE| so that the model is not scaled.


\subsubsection*{Hyperparameter spesification and default values}
\documentclass[a4paper,11pt]{article}
\usepackage[scale={0.8,0.9},centering,includeheadfoot]{geometry}
\usepackage{amstext}
\usepackage{listings}
\begin{document}

\section*{Besag model for spatial effects}

\subsection*{Parametrization}

The besag model for random vector $\mathbf{x}=(x_1,\dots,x_n)$ is defined as
\begin{equation}\label{eq.besag}
    x_i|x_j,i\neq j,\tau\sim\mathcal{N}(\frac{1}{n_i}\sum_{i\sim j}x_j,\frac{1}{n_i\tau})
\end{equation}

where $n_i$ is the number of neighbours of node $i$, $i\sim j$
indicates that the two nodes $i$ and $j$ are neighbours.  


\subsection*{Hyperparameters}

The precision parameter $\tau$ is represented as
\begin{displaymath}
    \theta_{1} =\log \tau
\end{displaymath}
and the prior is defined on $\theta_{1}$. 

\subsection*{Specification}

The besag model is specified inside the {\tt f()} function as
\begin{verbatim}
 f(<whatever>,model="besag",graph.file=<graph file name>
              prior=c(<prior.model.theta1>),
              param=c(<param.prior.theta1>))
\end{verbatim}

The neighbourhood structure of $\mathbf{x}$ is passed to the program
through the {\tt graph.file} argument.  The structure of this file is
described below.

\subsubsection*{Structure of the graph file}

We describe the required format for the graph file using a small
example. Let the file {\tt gra.dat}, relative to a small graph of only
5 elements, be
\begin{lstlisting}[basicstyle=\footnotesize]
    5
    1 1 2
    2 2 1 3
    3 3 2 4 5 
    4 1 3
    5 1 3
\end{lstlisting}
Line 1 declares the total number of nodes in the graph (5), then, in
lines 2-6 each node is described. For example, line 4 states that node
3 has 4 neighbours and these are nodes 2, 4 and 5.

The graph file can either have nodes indexed from 1 to $n$, or from 0
to $n-1$. Note that in the latter case, node $i$ seen from R
corresponds to node $i-1$ in the 0-indexed graph.



\subsection*{Example}

For examples of application of this model see the {\tt Bym}, {\tt Munich}, {\tt Zambia} or {\tt Scotland} examples in Volume I.

\subsection*{Notes}

The besag model is a intrinsic random field.

The model is modified accordingly is the graph has more than one
connected components.

\end{document}


% LocalWords: 

%%% Local Variables: 
%%% TeX-master: t
%%% End: 


\subsection*{Example}

For examples of application of this model see the {\tt Bym}, {\tt
    Munich}, {\tt Zambia} or {\tt Scotland} examples in Volume I.

%%\documentclass{article}
%%\usepackage{amsmath}
%%\begin{document}

\section*{Details on the implementation}

This gives some details of the implementation, which depends on the
following variables
\begin{description}
\item[nc1] Number of connected components in the graph with size 1.
    These nodes, \emph{singletons}, have no neigbours.
\item[nc2] Number of connected components in the graph with size
    $\ge2$.
\item[scale.model] The value of the logical flag, if the model should
    be scaled or not. (Default FALSE)
\item[adjust.for.con.comp] The value of the logical flag if the
    \texttt{constr=TRUE} option should be reinterpreted.
\end{description}

\subsubsection*{The case \texttt{(scale.model==FALSE \&\&
        adjust.for.con.comp == FALSE)}}

The option \texttt{constr=TRUE} is interpreted as a sum-to-zero
constraint over the whole graph. Singletons are given a uniform
distribution on $(-\infty,\infty)$ before the constraint, which may
give a singular posterior.

\subsubsection*{The case \texttt{(scale.model==TRUE \&\&
        adjust.for.con.comp == FALSE)}}

The option \texttt{constr=TRUE} is interpreted as a sum-to-zero
constraint over the whole graph. Let $Q = \tau R$ be the standard
precision matrix from the \texttt{besag}-model with precision
parameter $\tau$. Then $R$, except the singletons, are scaled so that
the geometric mean of the marginal variances is 1, and $R$ is modified
so that singletons have a standard Normal distribution.

\subsubsection*{The case \texttt{(scale.model==FALSE \&\&
        adjust.for.con.comp == TRUE)}}

The option \texttt{constr=TRUE} is interpreted as one sum-to-zero
constraint over each of the \texttt{nc2} connected components of size
$\ge2$. Singletons are given a uniform distribution on
$(-\infty,\infty)$, which may give a singular posterior.

\subsubsection*{The case \texttt{(scale.model==TRUE \&\&
        adjust.for.con.comp == TRUE)}}

The option \texttt{constr=TRUE} is interpreted as \texttt{nc2}
sum-to-zero constraints for each of the connected components of size
$\ge2$. Let $Q = \tau R$ be the standard precision matrix from the
\texttt{besag}-model with precision parameter $\tau$. Then $R$, are
scaled so that the geometric mean of the marginal variances in each
connected component of size $\ge2$ is 1, and modified so that
singletons have a standard Normal distribution.

%%\end{document}


\subsection*{Notes}

The term $\frac{1}{2}\log(|R|^{*})$ of the normalisation constant is
not computed, hence you need to add this part to the log marginal
likelihood estimate, if you need it. Here $R$ is the precision matrix
with a unit precision parameter.

\end{document}


% LocalWords: 

%%% Local Variables: 
%%% TeX-master: t
%%% End: 

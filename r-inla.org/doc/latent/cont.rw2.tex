\documentclass[a4paper,11pt]{article}
\usepackage[scale={0.8,0.9},centering,includeheadfoot]{geometry}
\usepackage{amstext}
\usepackage{listings}
\begin{document}

\section*{Continous random walk model of order $2$ (CRW2)}

\subsection*{Parametrization}

The continous random walk model of order $2$ (CRW2) for the Gaussian
vector $\mathbf{x}=(x_1,\dots,x_n)$ is descibed in the GMRF-book
chapter 3. It is an exact representation of the continous CRW2 model
augmented with its derivaties. The use its the same as for RW2. 

\subsection*{Hyperparameters}

The precision parameter $\tau$ is represented as
\begin{displaymath}
    \theta =\log \tau
\end{displaymath}
and the prior is defined on $\mathbf{\theta}$. Note that $\tau$ is the
precision for the first order increments.

\subsection*{Specification}

The CRW2 model is specified inside the {\tt f()} function as
\begin{verbatim}
 f(<whatever>,model="crw2",values=<values>,hyper = <hyper>)
\end{verbatim}
The (optional) argument {\tt values } is a numeric or factor vector
giving the values assumed by the covariate for which we want the
effect to be estimated. See next example for an application.
 
\subsubsection*{Hyperparameter spesification and defaults}
\documentclass[a4paper,11pt]{article}
\usepackage[scale={0.8,0.9},centering,includeheadfoot]{geometry}
\usepackage{amstext}
\usepackage{listings}
\begin{document}

\section*{Continous random walk model of order $2$ (CRW2)}

\subsection*{Parametrization}

The continous random walk model of order $2$ (CRW2) for the Gaussian
vector $\mathbf{x}=(x_1,\dots,x_n)$ is descibed in the GMRF-book
chapter 3. It is an exact representation of the continous CRW2 model
augmented with its derivaties. The use its the same as for RW2.

\subsection*{Hyperparameters}

The precision parameter $\tau$ is represented as
\begin{displaymath}
    \theta =\log \tau
\end{displaymath}
and the prior is defined on $\mathbf{\theta}$. Note that $\tau$ is the
precision for the first order increments.

\subsection*{Specification}

The CRW2 model is specified inside the {\tt f()} function as
\begin{verbatim}
 f(<whatever>, model="crw2", values=<values>, hyper = <hyper>)
\end{verbatim}
The (optional) argument {\tt values } is a numeric or factor vector
giving the values assumed by the covariate for which we want the
effect to be estimated. See next example for an application.
 
\subsubsection*{Hyperparameter spesification and default values}
\documentclass[a4paper,11pt]{article}
\usepackage[scale={0.8,0.9},centering,includeheadfoot]{geometry}
\usepackage{amstext}
\usepackage{listings}
\begin{document}

\section*{Continous random walk model of order $2$ (CRW2)}

\subsection*{Parametrization}

The continous random walk model of order $2$ (CRW2) for the Gaussian
vector $\mathbf{x}=(x_1,\dots,x_n)$ is descibed in the GMRF-book
chapter 3. It is an exact representation of the continous CRW2 model
augmented with its derivaties. The use its the same as for RW2.

\subsection*{Hyperparameters}

The precision parameter $\tau$ is represented as
\begin{displaymath}
    \theta =\log \tau
\end{displaymath}
and the prior is defined on $\mathbf{\theta}$. Note that $\tau$ is the
precision for the first order increments.

\subsection*{Specification}

The CRW2 model is specified inside the {\tt f()} function as
\begin{verbatim}
 f(<whatever>, model="crw2", values=<values>, hyper = <hyper>)
\end{verbatim}
The (optional) argument {\tt values } is a numeric or factor vector
giving the values assumed by the covariate for which we want the
effect to be estimated. See next example for an application.
 
\subsubsection*{Hyperparameter spesification and default values}
\documentclass[a4paper,11pt]{article}
\usepackage[scale={0.8,0.9},centering,includeheadfoot]{geometry}
\usepackage{amstext}
\usepackage{listings}
\begin{document}

\section*{Continous random walk model of order $2$ (CRW2)}

\subsection*{Parametrization}

The continous random walk model of order $2$ (CRW2) for the Gaussian
vector $\mathbf{x}=(x_1,\dots,x_n)$ is descibed in the GMRF-book
chapter 3. It is an exact representation of the continous CRW2 model
augmented with its derivaties. The use its the same as for RW2.

\subsection*{Hyperparameters}

The precision parameter $\tau$ is represented as
\begin{displaymath}
    \theta =\log \tau
\end{displaymath}
and the prior is defined on $\mathbf{\theta}$. Note that $\tau$ is the
precision for the first order increments.

\subsection*{Specification}

The CRW2 model is specified inside the {\tt f()} function as
\begin{verbatim}
 f(<whatever>, model="crw2", values=<values>, hyper = <hyper>)
\end{verbatim}
The (optional) argument {\tt values } is a numeric or factor vector
giving the values assumed by the covariate for which we want the
effect to be estimated. See next example for an application.
 
\subsubsection*{Hyperparameter spesification and default values}
\input{../hyper/latent/crw2.tex}


\subsection*{Example}

\begin{verbatim}
n=100
z=seq(0,6,length.out=n)
y=sin(z)+rnorm(n,mean=0,sd=0.5)
data=data.frame(y=y,z=z)

formula=y~f(z,model="crw2")
result=inla(formula,data=data,family="gaussian")
\end{verbatim}


\subsection*{Notes}


\begin{itemize}
\item The CRW2 is a intrinsic with rank deficiency 2.
\item The CRW2 model for irregular locations are supported although not
    described here.
\item The $\frac{n-r}{2}\log(|R|^{*})$-part (with $r=2$) of the
    normalisation constant is not computed, hence you need to add this
    part to the log marginal likelihood estimate, if you need it.
\item Usually, you may want to use the model RW2 instead.
\end{itemize}

\end{document}


% LocalWords: 

%%% Local Variables: 
%%% TeX-master: t
%%% End: 



\subsection*{Example}

\begin{verbatim}
n=100
z=seq(0,6,length.out=n)
y=sin(z)+rnorm(n,mean=0,sd=0.5)
data=data.frame(y=y,z=z)

formula=y~f(z,model="crw2")
result=inla(formula,data=data,family="gaussian")
\end{verbatim}


\subsection*{Notes}


\begin{itemize}
\item The CRW2 is a intrinsic with rank deficiency 2.
\item The CRW2 model for irregular locations are supported although not
    described here.
\item The $\frac{n-r}{2}\log(|R|^{*})$-part (with $r=2$) of the
    normalisation constant is not computed, hence you need to add this
    part to the log marginal likelihood estimate, if you need it.
\item Usually, you may want to use the model RW2 instead.
\end{itemize}

\end{document}


% LocalWords: 

%%% Local Variables: 
%%% TeX-master: t
%%% End: 



\subsection*{Example}

\begin{verbatim}
n=100
z=seq(0,6,length.out=n)
y=sin(z)+rnorm(n,mean=0,sd=0.5)
data=data.frame(y=y,z=z)

formula=y~f(z,model="crw2")
result=inla(formula,data=data,family="gaussian")
\end{verbatim}


\subsection*{Notes}


\begin{itemize}
\item The CRW2 is a intrinsic with rank deficiency 2.
\item The CRW2 model for irregular locations are supported although not
    described here.
\item The $\frac{n-r}{2}\log(|R|^{*})$-part (with $r=2$) of the
    normalisation constant is not computed, hence you need to add this
    part to the log marginal likelihood estimate, if you need it.
\item Usually, you may want to use the model RW2 instead.
\end{itemize}

\end{document}


% LocalWords: 

%%% Local Variables: 
%%% TeX-master: t
%%% End: 



\subsection*{Example}

\begin{verbatim}

n=100
z=seq(0,6,length.out=n)
y=sin(z)+rnorm(n,mean=0,sd=0.5)
data=data.frame(y=y,z=z)

formula=y~f(z,model="crw2")
result=inla(formula,data=data,family="gaussian")
\end{verbatim}


\subsection*{Notes}

The CRW2 model is intrinsic with rank deficiency of 2.

The model RW2 is an good (enough) approximation to CRW2 and do not
augment with the derivaties.
\end{document}


% LocalWords: 

%%% Local Variables: 
%%% TeX-master: t
%%% End: 

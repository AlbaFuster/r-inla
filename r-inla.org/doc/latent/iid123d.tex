\documentclass[a4paper,11pt]{article}
\usepackage[scale={0.8,0.9},centering,includeheadfoot]{geometry}
\usepackage{amstext}
\usepackage{listings}
\usepackage{block}
\begin{document}

\section*{Correlated random effects: \texttt{iid1d}, \texttt{iid2d},
    \texttt{iid3d}}

This model is available for dimensions $p=1$, $2$, $3$, $4$ and
$5$. We describe in detail the case for $p=2$, and then the changes
required for $p=1$, $p=3$, $p=4$ and $p=5$

\subsection*{Parametrization}

The 2-dimensional Normal-Wishard model is used if one want to define
two vectors of ``random effects'', $u$ and $v$, say, for which
$(u_{i}, v_{i})$ are iid bivariate Normals
\begin{displaymath}
    \left(
      \begin{array}{c}
          u_{i}\\
          v_{i}
      \end{array}\right)
    \sim \mathcal{N}\left(\mathbf{0}, \mathbf{W}^{-1}\right)
\end{displaymath}
where the  covariance matrix $\mathbf{W}^{-1}$ is
\begin{equation}
    \label{precision}
    \mathbf{W}^{-1} = \left(\begin{array}{cc}
          1/\tau_a & \rho \sqrt{\tau_a\tau_b}\\
          \rho\sqrt{\tau_a\tau_b}&  1/\tau_b
      \end{array}\right)
\end{equation}
and $\tau_{a}$, $\tau_{b}$ and $\rho$ are the hyperparameters. For
these models the precision matrix $\mathbf{W}$ is Wishart distributed
\begin{displaymath}
    \mathbf{W}
    \;\sim\;\text{Wishart}_{p}(r, \mathbf{R}^{-1}), \quad p=2
\end{displaymath}
with density
\begin{displaymath}
    \pi(\mathbf{W}) = c^{-1} |\mathbf{W}|^{(r-(p+1))/2} \exp\left\{
      -\frac{1}{2}\text{Trace}(\mathbf{W}\mathbf{R})\right\}, \quad r > p+1
\end{displaymath}
and
\begin{displaymath}
    c = 2^{(rp)/2} |\mathbf{R}|^{-r/2} \pi^{(p(p-1))/4}\prod_{j=1}^{p}
    \Gamma((r+1-j)/2).
\end{displaymath}
Then,
\begin{displaymath}
    \text{E}(\mathbf{W}) = r\mathbf{R}^{-1}, \quad\text{and}\quad
    \text{E}(\mathbf{W}^{-1}) = \mathbf{R}/(r-(p+1)).
\end{displaymath}

\subsection*{Hyperparameters}

The hyperparameters are
\begin{displaymath}
    \theta = (\log\tau_{a}, \log\tau_{b}, \tilde\rho)
\end{displaymath}
where
\begin{displaymath}
    \rho =
    2\frac{\exp(\tilde\rho)}{\exp(\tilde{\rho})+1} -1
\end{displaymath}

The prior-parameters are
\begin{displaymath}
    (r,R_{11},R_{22},R_{12})
\end{displaymath}
where
\begin{displaymath}
    \mathbf{R}= \left(\begin{array}{cc}
          R_{11} &R_{12}\\
          R_{21} & R_{22}
      \end{array}\right)
\end{displaymath}
and $r_{12}=R_{21}$ due to symmetry.

The {\tt inla} function reports posterior distribution for the
hyperparameters $\tau_a,\tau_b,\rho$ in equation (\ref{precision}).

The prior for $\theta$ is {\bf fixed} to be {\tt wishart2d}


\subsection*{Specification}

The model \texttt{iid2d}
is specified as
\begin{verbatim}
    y ~ f(i, model="iid2d",n = <length>) + ...
\end{verbatim}
and the \texttt{iid2d} model is represented internally as one vector
of length $n$,
\begin{displaymath}
    (u_{1}, u_{2}, \ldots, u_{m}, v_{1}, v_{2}, \ldots, v_{m})
\end{displaymath}
where $n = 2m$, and $n$ is the (required) argument in
\texttt{f()}.

For this model the argument \texttt{constr=TRUE} is interpreted as
\begin{displaymath}
    \sum u_{i} = 0, \qquad\text{and}\qquad \sum v_{i} = 0.
\end{displaymath}

\subsubsection*{Hyperparameter spesification and default values}
\begin{description}
	\item[hyper]\ 
	 \begin{description}
	 	\item[theta1]\ 
	 	 \begin{description}
	 	 	 \item[ name ] precision1 
	 	 	 \item[ short.name ] prec1 
	 	 	 \item[ initial ] 4 
	 	 	 \item[ fixed ] FALSE 
	 	 	 \item[ prior ] wishart2d 
	 	 	 \item[ param ] 4 1 1 0 
	 	 \end{description}
	 	\item[theta2]\ 
	 	 \begin{description}
	 	 	 \item[ name ] precision2 
	 	 	 \item[ short.name ] prec2 
	 	 	 \item[ initial ] 4 
	 	 	 \item[ fixed ] FALSE 
	 	 	 \item[ prior ] none 
	 	 	 \item[ param ]  
	 	 \end{description}
	 	\item[theta3]\ 
	 	 \begin{description}
	 	 	 \item[ name ] correlation 
	 	 	 \item[ short.name ] cor 
	 	 	 \item[ initial ] 4 
	 	 	 \item[ fixed ] FALSE 
	 	 	 \item[ prior ] none 
	 	 	 \item[ param ]  
	 	 \end{description}
	 \end{description}
	 \item[ constr ] FALSE 
	 \item[ nrow.ncol ] FALSE 
	 \item[ augmented ] TRUE 
	 \item[ aug.factor ] 1 
	 \item[ aug.constr ] 1 2 
	 \item[ n.div.by ] 2 
	 \item[ n.required ] TRUE 
	 \item[ set.default.values ] TRUE 
\end{description}



\subsection*{Example}
In this example we implement the model
\begin{displaymath}
    y|\eta\sim\mbox{Pois}(\exp(\eta))
\end{displaymath}
where
\begin{displaymath}
    \eta=a+b+1
\end{displaymath}
and $a$ and $b$ are correlated as described above.

\begin{verbatim}
n = 1000
N = 2*n
rho = 0.5
## set variances
Sigma = matrix(c(1/1, NA, NA, 1/2), 2, 2)
## and the correlation
Sigma[1,2] = Sigma[2,1] = rho*sqrt(Sigma[1,1]*Sigma[2,2])

## need it to simulate data
library(mvtnorm)

if (TRUE)
{
    ## first example
    
    y = yy = rmvnorm(n, sigma=Sigma)
    y = c(y[,1], y[,2])

    i = 1:N
    formula = y ~ f(i, model="iid2d", n=N)

    r = inla(formula, data = data.frame(i,y),
            control.family=list(initial=10,fixed=TRUE))
    print(summary(r))

    print(1/diag(cov(yy)))
    print(cor(yy)[1,2])
}

if (TRUE)
{
    ## second example

    y = yy = rmvnorm(n, sigma=Sigma)
    z = rnorm(n)
    zz = rnorm(n)
    y = y[,1] + z*y[,2] + zz
    i = 1:n
    j = n + 1:n
    formula = y ~ f(i, model="iid2d", n=N) + f(j,z,copy="i") + zz 

    r = inla(formula, data = data.frame(i,j,y,z,zz),
            control.family=list(initial=10,fixed=TRUE),keep=T)
    print(summary(r))
    print(1/diag(cov(yy)))
    print(cor(yy)[1,2])
}
\end{verbatim}

\subsection*{The case $p=1$}

For $p=1$ the
hyperparameter is the marginal precision
\begin{displaymath}
    \theta = \log \tau_{1}
\end{displaymath}
The prior is fixed to be \texttt{wishart1d} with parameters
\begin{quote}
    \emph{param = } $r\;R_{11}$
\end{quote}
where
\begin{displaymath}
    \mathbf{R} =
    \left[\begin{array}{c}
        R_{11}
    \end{array}\right]
\end{displaymath}

\subsubsection*{Hyperparameter spesification and default values}
\begin{description}
	\item[hyper]\ 
	 \begin{description}
	 	\item[theta]\ 
	 	 \begin{description}
	 	 	\item[name] precision
	 	 	\item[short.name] prec
	 	 	\item[initial] 4
	 	 	\item[fixed] FALSE
	 	 	\item[prior] wishart1d
	 	 	\item[param] 2 1e-04
	 	 	\item[to.theta] \verb|function(x) log(x)|
	 	 	\item[from.theta] \verb|function(x) exp(x)|
	 	 \end{description}
	 \end{description}
	\item[constr] FALSE
	\item[nrow.ncol] FALSE
	\item[augmented] FALSE
	\item[aug.factor] 1
	\item[aug.constr] 
	\item[n.div.by] 
	\item[n.required] FALSE
	\item[set.default.values] TRUE
	\item[pdf] iid123d
\end{description}



\subsection*{The case $p=3$}

For $p=3$ the
hyperparameters are
\begin{displaymath}
    \theta = (\log \tau_{1}, \log \tau_{2}, \log \tau_{3},
    \tilde\rho_{12},
    \tilde\rho_{13},
    \tilde\rho_{23})
\end{displaymath}
The prior is fixed to be \texttt{wishart3d} with parameters
\begin{quote}
    \emph{param = } $r\;R_{11}\;R_{22}\;R_{33}\; R_{12}\;
    R_{13}\; R_{23}$
\end{quote}
where
\begin{displaymath}
    \mathbf{R} =
    \left[\begin{array}{ccc}
        R_{11} &R_{12} & R_{13}\\
        R_{12} & R_{22} & R_{23}\\
        R_{13} & R_{23} & R_{33}
    \end{array}\right]
\end{displaymath}
The reported hyperparameters are the marginal precisions $\tau_{1}$,
$\tau_{2}$ and $\tau_{3}$ and the correlations $\rho_{12}$,
$\rho_{13}$ and $\rho_{23}$.

In this case, the internal representation is given as
\begin{displaymath}
    (u_{1}, u_{2}, \ldots, u_{m}, v_{1}, v_{2}, \ldots, v_{m},
    w_{1}, w_{2}, \ldots, w_{m})
\end{displaymath}
where $n=3m$ is a required argument, and where $(u_{i}, v_{i}, w_{i})$
are trivariate iid Normal.

\subsubsection*{Hyperparameter spesification and default values}
\begin{description}
	\item[hyper]\ 
	 \begin{description}
	 	\item[theta1]\ 
	 	 \begin{description}
	 	 	 \item[ name ] precision1 
	 	 	 \item[ short.name ] prec1 
	 	 	 \item[ initial ] 4 
	 	 	 \item[ fixed ] FALSE 
	 	 	 \item[ prior ] wishart3d 
	 	 	 \item[ param ] 7 1 1 1 0 0 0 
	 	 \end{description}
	 	\item[theta2]\ 
	 	 \begin{description}
	 	 	 \item[ name ] precision2 
	 	 	 \item[ short.name ] prec2 
	 	 	 \item[ initial ] 4 
	 	 	 \item[ fixed ] FALSE 
	 	 	 \item[ prior ] none 
	 	 	 \item[ param ]  
	 	 \end{description}
	 	\item[theta3]\ 
	 	 \begin{description}
	 	 	 \item[ name ] precision3 
	 	 	 \item[ short.name ] prec3 
	 	 	 \item[ initial ] 4 
	 	 	 \item[ fixed ] FALSE 
	 	 	 \item[ prior ] none 
	 	 	 \item[ param ]  
	 	 \end{description}
	 	\item[theta4]\ 
	 	 \begin{description}
	 	 	 \item[ name ] correlation12 
	 	 	 \item[ short.name ] cor12 
	 	 	 \item[ initial ] 0 
	 	 	 \item[ fixed ] FALSE 
	 	 	 \item[ prior ] none 
	 	 	 \item[ param ]  
	 	 \end{description}
	 	\item[theta5]\ 
	 	 \begin{description}
	 	 	 \item[ name ] correlation13 
	 	 	 \item[ short.name ] cor13 
	 	 	 \item[ initial ] 0 
	 	 	 \item[ fixed ] FALSE 
	 	 	 \item[ prior ] none 
	 	 	 \item[ param ]  
	 	 \end{description}
	 	\item[theta6]\ 
	 	 \begin{description}
	 	 	 \item[ name ] correlation23 
	 	 	 \item[ short.name ] cor23 
	 	 	 \item[ initial ] 0 
	 	 	 \item[ fixed ] FALSE 
	 	 	 \item[ prior ] none 
	 	 	 \item[ param ]  
	 	 \end{description}
	 \end{description}
	 \item[ constr ] FALSE 
	 \item[ nrow.ncol ] FALSE 
	 \item[ augmented ] TRUE 
	 \item[ aug.factor ] 1 
	 \item[ aug.constr ] 1 2 3 
	 \item[ n.div.by ] 3 
	 \item[ n.required ] TRUE 
	 \item[ set.default.values ] TRUE 
\end{description}


\subsection*{The case $p=4$}

For $p=4$ the
hyperparameters are
\begin{displaymath}
    \theta = (\log \tau_{1}, \log \tau_{2}, \log \tau_{3}, \log \tau_{4},
    \tilde\rho_{12},
    \tilde\rho_{13},
    \tilde\rho_{14},
    \tilde\rho_{23},
    \tilde\rho_{24},
    \tilde\rho_{34})
\end{displaymath}
The prior is fixed to be \texttt{wishart4d} with parameters
\begin{quote}
    \emph{param = } $r\;R_{11}\;R_{22}\;R_{33}\; R_{44}\;
    R_{12}\;
    R_{13}\;
    R_{14}\;
    R_{23}\;
    R_{24}\;
    R_{34}$
\end{quote}
where
\begin{displaymath}
    \mathbf{R} =
    \left[\begin{array}{cccc}
        R_{11} &R_{12} & R_{13} & R_{14}\\
        R_{12} & R_{22} & R_{23} & R_{24}\\
        R_{13} & R_{23} & R_{33} & R_{34}\\
        R_{14} & R_{24} & R_{34} & R_{44}\\
    \end{array}\right]
\end{displaymath}
The reported hyperparameters are the marginal precisions $\tau_{1}$,
$\tau_{2}$, $\tau_{3}$ and $\tau_{4}$, and the correlations
$\rho_{12}$, $\rho_{13}$, $\rho_{14}$, $\rho_{23}$, $\rho_{24}$ and
$\rho_{34}$.

In this case, the internal representation is given as
\begin{displaymath}
    (u_{1}, u_{2}, \ldots, u_{m}, v_{1}, v_{2}, \ldots, v_{m},
    w_{1}, w_{2}, \ldots, w_{m},
    x_{1}, x_{2}, \ldots, x_{m})
\end{displaymath}
where $n=4m$ is a required argument, and where $(u_{i}, v_{i}, w_{i}, x_{i})$
are fourvariate iid Normal.

\subsubsection*{Hyperparameter spesification and default values}
\begin{description}
	\item[hyper]\ 
	 \begin{description}
	 	\item[theta1]\ 
	 	 \begin{description}
	 	 	 \item[ name ] log precision1 
	 	 	 \item[ short.name ] prec1 
	 	 	 \item[ initial ] 4 
	 	 	 \item[ fixed ] FALSE 
	 	 	 \item[ prior ] wishart4d 
	 	 	 \item[ param ] 11 1 1 1 1 0 0 0 0 0 0 
	 	 	 \item[ to.theta ] \verb|| 
	 	 	 \item[ from.theta ] \verb|| 
	 	 \end{description}
	 	\item[theta2]\ 
	 	 \begin{description}
	 	 	 \item[ name ] log precision2 
	 	 	 \item[ short.name ] prec2 
	 	 	 \item[ initial ] 4 
	 	 	 \item[ fixed ] FALSE 
	 	 	 \item[ prior ] none 
	 	 	 \item[ param ]  
	 	 	 \item[ to.theta ] \verb|| 
	 	 	 \item[ from.theta ] \verb|| 
	 	 \end{description}
	 	\item[theta3]\ 
	 	 \begin{description}
	 	 	 \item[ name ] log precision3 
	 	 	 \item[ short.name ] prec3 
	 	 	 \item[ initial ] 4 
	 	 	 \item[ fixed ] FALSE 
	 	 	 \item[ prior ] none 
	 	 	 \item[ param ]  
	 	 	 \item[ to.theta ] \verb|| 
	 	 	 \item[ from.theta ] \verb|| 
	 	 \end{description}
	 	\item[theta4]\ 
	 	 \begin{description}
	 	 	 \item[ name ] log precision4 
	 	 	 \item[ short.name ] prec4 
	 	 	 \item[ initial ] 4 
	 	 	 \item[ fixed ] FALSE 
	 	 	 \item[ prior ] none 
	 	 	 \item[ param ]  
	 	 	 \item[ to.theta ] \verb|| 
	 	 	 \item[ from.theta ] \verb|| 
	 	 \end{description}
	 	\item[theta5]\ 
	 	 \begin{description}
	 	 	 \item[ name ] logit correlation12 
	 	 	 \item[ short.name ] cor12 
	 	 	 \item[ initial ] 0 
	 	 	 \item[ fixed ] FALSE 
	 	 	 \item[ prior ] none 
	 	 	 \item[ param ]  
	 	 	 \item[ to.theta ] \verb|| 
	 	 	 \item[ from.theta ] \verb|| 
	 	 \end{description}
	 	\item[theta6]\ 
	 	 \begin{description}
	 	 	 \item[ name ] logit correlation13 
	 	 	 \item[ short.name ] cor13 
	 	 	 \item[ initial ] 0 
	 	 	 \item[ fixed ] FALSE 
	 	 	 \item[ prior ] none 
	 	 	 \item[ param ]  
	 	 	 \item[ to.theta ] \verb|| 
	 	 	 \item[ from.theta ] \verb|| 
	 	 \end{description}
	 	\item[theta7]\ 
	 	 \begin{description}
	 	 	 \item[ name ] logit correlation14 
	 	 	 \item[ short.name ] cor14 
	 	 	 \item[ initial ] 0 
	 	 	 \item[ fixed ] FALSE 
	 	 	 \item[ prior ] none 
	 	 	 \item[ param ]  
	 	 	 \item[ to.theta ] \verb|| 
	 	 	 \item[ from.theta ] \verb|| 
	 	 \end{description}
	 	\item[theta8]\ 
	 	 \begin{description}
	 	 	 \item[ name ] logit correlation23 
	 	 	 \item[ short.name ] cor23 
	 	 	 \item[ initial ] 0 
	 	 	 \item[ fixed ] FALSE 
	 	 	 \item[ prior ] none 
	 	 	 \item[ param ]  
	 	 	 \item[ to.theta ] \verb|| 
	 	 	 \item[ from.theta ] \verb|| 
	 	 \end{description}
	 	\item[theta9]\ 
	 	 \begin{description}
	 	 	 \item[ name ] logit correlation24 
	 	 	 \item[ short.name ] cor24 
	 	 	 \item[ initial ] 0 
	 	 	 \item[ fixed ] FALSE 
	 	 	 \item[ prior ] none 
	 	 	 \item[ param ]  
	 	 	 \item[ to.theta ] \verb|| 
	 	 	 \item[ from.theta ] \verb|| 
	 	 \end{description}
	 	\item[theta10]\ 
	 	 \begin{description}
	 	 	 \item[ name ] logit correlation34 
	 	 	 \item[ short.name ] cor34 
	 	 	 \item[ initial ] 0 
	 	 	 \item[ fixed ] FALSE 
	 	 	 \item[ prior ] none 
	 	 	 \item[ param ]  
	 	 	 \item[ to.theta ] \verb|| 
	 	 	 \item[ from.theta ] \verb|| 
	 	 \end{description}
	 \end{description}
	 \item[ constr ] FALSE 
	 \item[ nrow.ncol ] FALSE 
	 \item[ augmented ] TRUE 
	 \item[ aug.factor ] 1 
	 \item[ aug.constr ] 1 2 3 4 
	 \item[ n.div.by ] 4 
	 \item[ n.required ] TRUE 
	 \item[ set.default.values ] TRUE 
	 \item[ pdf ] iid123d 
\end{description}



\subsection*{The case $p=5$}

The case $p=5$ follows by a direct extention of $p=3$ and $p=4$, and
is therefore not included.

\subsubsection*{Hyperparameter spesification and default values}
%% DO NOT EDIT!
%% This file is generated automatically from models.R
\begin{description}
	\item[doc] Gaussian random effect in dim=5 with Wishart prior
	\item[hyper]\ 
	 \begin{description}
	 	\item[theta1]\ 
	 	 \begin{description}
	 	 	\item[hyperid] 29001
	 	 	\item[name] log precision1
	 	 	\item[short.name] prec1
	 	 	\item[initial] 4
	 	 	\item[fixed] FALSE
	 	 	\item[prior] wishart5d
	 	 	\item[param] 16 1 1 1 1 1 0 0 0 0 0 0 0 0 0 0
	 	 	\item[to.theta] \verb!function(x) log(x)!
	 	 	\item[from.theta] \verb!function(x) exp(x)!
	 	 \end{description}
	 	\item[theta2]\ 
	 	 \begin{description}
	 	 	\item[hyperid] 29002
	 	 	\item[name] log precision2
	 	 	\item[short.name] prec2
	 	 	\item[initial] 4
	 	 	\item[fixed] FALSE
	 	 	\item[prior] none
	 	 	\item[param] 
	 	 	\item[to.theta] \verb!function(x) log(x)!
	 	 	\item[from.theta] \verb!function(x) exp(x)!
	 	 \end{description}
	 	\item[theta3]\ 
	 	 \begin{description}
	 	 	\item[hyperid] 29003
	 	 	\item[name] log precision3
	 	 	\item[short.name] prec3
	 	 	\item[initial] 4
	 	 	\item[fixed] FALSE
	 	 	\item[prior] none
	 	 	\item[param] 
	 	 	\item[to.theta] \verb!function(x) log(x)!
	 	 	\item[from.theta] \verb!function(x) exp(x)!
	 	 \end{description}
	 	\item[theta4]\ 
	 	 \begin{description}
	 	 	\item[hyperid] 29004
	 	 	\item[name] log precision4
	 	 	\item[short.name] prec4
	 	 	\item[initial] 4
	 	 	\item[fixed] FALSE
	 	 	\item[prior] none
	 	 	\item[param] 
	 	 	\item[to.theta] \verb!function(x) log(x)!
	 	 	\item[from.theta] \verb!function(x) exp(x)!
	 	 \end{description}
	 	\item[theta5]\ 
	 	 \begin{description}
	 	 	\item[hyperid] 29005
	 	 	\item[name] log precision5
	 	 	\item[short.name] prec5
	 	 	\item[initial] 4
	 	 	\item[fixed] FALSE
	 	 	\item[prior] none
	 	 	\item[param] 
	 	 	\item[to.theta] \verb!function(x) log(x)!
	 	 	\item[from.theta] \verb!function(x) exp(x)!
	 	 \end{description}
	 	\item[theta6]\ 
	 	 \begin{description}
	 	 	\item[hyperid] 29006
	 	 	\item[name] logit correlation12
	 	 	\item[short.name] cor12
	 	 	\item[initial] 0
	 	 	\item[fixed] FALSE
	 	 	\item[prior] none
	 	 	\item[param] 
	 	 	\item[to.theta] \verb!function(x) log((1+x)/(1-x))!
	 	 	\item[from.theta] \verb!function(x) 2*exp(x)/(1+exp(x))-1!
	 	 \end{description}
	 	\item[theta7]\ 
	 	 \begin{description}
	 	 	\item[hyperid] 29007
	 	 	\item[name] logit correlation13
	 	 	\item[short.name] cor13
	 	 	\item[initial] 0
	 	 	\item[fixed] FALSE
	 	 	\item[prior] none
	 	 	\item[param] 
	 	 	\item[to.theta] \verb!function(x) log((1+x)/(1-x))!
	 	 	\item[from.theta] \verb!function(x) 2*exp(x)/(1+exp(x))-1!
	 	 \end{description}
	 	\item[theta8]\ 
	 	 \begin{description}
	 	 	\item[hyperid] 29008
	 	 	\item[name] logit correlation14
	 	 	\item[short.name] cor14
	 	 	\item[initial] 0
	 	 	\item[fixed] FALSE
	 	 	\item[prior] none
	 	 	\item[param] 
	 	 	\item[to.theta] \verb!function(x) log((1+x)/(1-x))!
	 	 	\item[from.theta] \verb!function(x) 2*exp(x)/(1+exp(x))-1!
	 	 \end{description}
	 	\item[theta9]\ 
	 	 \begin{description}
	 	 	\item[hyperid] 29009
	 	 	\item[name] logit correlation15
	 	 	\item[short.name] cor15
	 	 	\item[initial] 0
	 	 	\item[fixed] FALSE
	 	 	\item[prior] none
	 	 	\item[param] 
	 	 	\item[to.theta] \verb!function(x) log((1+x)/(1-x))!
	 	 	\item[from.theta] \verb!function(x) 2*exp(x)/(1+exp(x))-1!
	 	 \end{description}
	 	\item[theta10]\ 
	 	 \begin{description}
	 	 	\item[hyperid] 29010
	 	 	\item[name] logit correlation23
	 	 	\item[short.name] cor23
	 	 	\item[initial] 0
	 	 	\item[fixed] FALSE
	 	 	\item[prior] none
	 	 	\item[param] 
	 	 	\item[to.theta] \verb!function(x) log((1+x)/(1-x))!
	 	 	\item[from.theta] \verb!function(x) 2*exp(x)/(1+exp(x))-1!
	 	 \end{description}
	 	\item[theta11]\ 
	 	 \begin{description}
	 	 	\item[hyperid] 29011
	 	 	\item[name] logit correlation24
	 	 	\item[short.name] cor24
	 	 	\item[initial] 0
	 	 	\item[fixed] FALSE
	 	 	\item[prior] none
	 	 	\item[param] 
	 	 	\item[to.theta] \verb!function(x) log((1+x)/(1-x))!
	 	 	\item[from.theta] \verb!function(x) 2*exp(x)/(1+exp(x))-1!
	 	 \end{description}
	 	\item[theta12]\ 
	 	 \begin{description}
	 	 	\item[hyperid] 29012
	 	 	\item[name] logit correlation25
	 	 	\item[short.name] cor25
	 	 	\item[initial] 0
	 	 	\item[fixed] FALSE
	 	 	\item[prior] none
	 	 	\item[param] 
	 	 	\item[to.theta] \verb!function(x) log((1+x)/(1-x))!
	 	 	\item[from.theta] \verb!function(x) 2*exp(x)/(1+exp(x))-1!
	 	 \end{description}
	 	\item[theta13]\ 
	 	 \begin{description}
	 	 	\item[hyperid] 29013
	 	 	\item[name] logit correlation34
	 	 	\item[short.name] cor34
	 	 	\item[initial] 0
	 	 	\item[fixed] FALSE
	 	 	\item[prior] none
	 	 	\item[param] 
	 	 	\item[to.theta] \verb!function(x) log((1+x)/(1-x))!
	 	 	\item[from.theta] \verb!function(x) 2*exp(x)/(1+exp(x))-1!
	 	 \end{description}
	 	\item[theta14]\ 
	 	 \begin{description}
	 	 	\item[hyperid] 29014
	 	 	\item[name] logit correlation35
	 	 	\item[short.name] cor35
	 	 	\item[initial] 0
	 	 	\item[fixed] FALSE
	 	 	\item[prior] none
	 	 	\item[param] 
	 	 	\item[to.theta] \verb!function(x) log((1+x)/(1-x))!
	 	 	\item[from.theta] \verb!function(x) 2*exp(x)/(1+exp(x))-1!
	 	 \end{description}
	 	\item[theta15]\ 
	 	 \begin{description}
	 	 	\item[hyperid] 29015
	 	 	\item[name] logit correlation45
	 	 	\item[short.name] cor45
	 	 	\item[initial] 0
	 	 	\item[fixed] FALSE
	 	 	\item[prior] none
	 	 	\item[param] 
	 	 	\item[to.theta] \verb!function(x) log((1+x)/(1-x))!
	 	 	\item[from.theta] \verb!function(x) 2*exp(x)/(1+exp(x))-1!
	 	 \end{description}
	 \end{description}
	\item[constr] FALSE
	\item[nrow.ncol] FALSE
	\item[augmented] TRUE
	\item[aug.factor] 1
	\item[aug.constr] 1 2 3 4 5
	\item[n.div.by] 5
	\item[n.required] TRUE
	\item[set.default.values] TRUE
	\item[pdf] iid123d
\end{description}


\subsection*{Notes}

The model \texttt{iid1d} is similar to the model \texttt{iid} (and
included for completeness only). The prior for \texttt{iid1d} is fixed
to be Wishart-distributed, which reduces to a Gamma-distribution for
the precision with parameters
\begin{displaymath}
    a = r/2 \qquad\text{and}\qquad b = R_{11}/2
\end{displaymath}
hence
\begin{verbatim}
    y ~ f(i, model="iid1d", hyper = list(theta=list(param=c(3, 4))))
\end{verbatim}
is equivalent to
\begin{verbatim}
    y ~ f(i, model="iid", hyper = list(theta=list(param=c(1.5, 2), prior="loggamma")))
\end{verbatim}



\end{document}


% LocalWords: 

%%% Local Variables: 
%%% TeX-master: t
%%% End: 

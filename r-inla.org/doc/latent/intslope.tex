\documentclass[a4paper,11pt]{article}
\usepackage[scale={0.8,0.9},centering,includeheadfoot]{geometry}
\usepackage{amstext}
\usepackage{listings}
\usepackage{verbatim}
\begin{document}

\section*{Intercept-slope model}

\subsection*{Parametrization}

The intercept-slope model is a convenient re-implementation of a
commonly used construct, where
\begin{displaymath}
    (a,b)
\end{displaymath}
is bi-variate Gaussian with a Wishart prior for the precision
matrix\footnote{The documentation for the model ``iid2d'' gives the
    details of the definition of the parameterization of the precision
    matrix and the Wishart-prior.}, and various forms of
\begin{equation}\label{eq1}%
    \gamma(a + bz),
\end{equation}
where $z$ is a covariate and $\gamma$ is a (random) scaling, goes into
the linear predictor. Replicates of $(a,b)$ is indexed by
\emph{subject}, $i=1, \ldots, n$, and the various scaling of
Eq.~\ref{eq1} by \emph{strata} $j=1, \ldots, m$, leading to a model
for (a subset of)
\begin{displaymath}
    \left\{\gamma_j(a_i + b_i z_{ij}), \quad i=1, \ldots,n, \quad j=1,\ldots,m\right\},
\end{displaymath}
as not all combinations need to be present.

\subsection*{Hyperparameters}

The hyperparameters are $(\theta_1,\theta_2,\theta_3)$ as in the model
``iid2d'' (related to the precisions of $a$ and $b$, and their
correlation), and
$\theta_4=\gamma_1, \ldots, \theta_{13}=\gamma_{10}$. Since $m$ is
defined in the input, only $\gamma_1, \ldots, \gamma_m$ are used. $m$
is limited to $m \le 10$. \textbf{Please note} that all $\gamma_i$'s
are by default \textbf{fixed} to $1$.


\subsection*{Specification}

The is specified as
\begin{verbatim}
    f(idx, model="intslope", hyper = ...,
      precision = exp(14),
      args.intslope = list(subject=i, strata=j, covariate = z))
\end{verbatim}
The definition of the model is through the \verb|args.intslope|
argument, where \verb|i| and \verb|j| are factors/integers and
\verb|z| is numerical, all with same length $N$, say. The argument
\verb|idx|, index which row that is used for the linear predictor,
hence values of \verb|idx| must take integer values in the interval
$1$ to $N$. The precision argument, defines the tiny small noise added
to each $\gamma(a+bz)$ to avoid a singular joint model. The
\verb|subject| and \verb|strata| argument, is converted internally
into integers $1, 2, \ldots$, using
\begin{verbatim}
    subject = as.numerical(as.factor(subject))
    strata  = as.numerical(as.factor(strata))
\end{verbatim}
and the results is shown after this conversion.


\subsubsection*{Hyperparameter specification and default values}
{\small
\documentclass[a4paper,11pt]{article}
\usepackage[scale={0.8,0.9},centering,includeheadfoot]{geometry}
\usepackage{amstext}
\usepackage{listings}
\usepackage{verbatim}
\begin{document}

\section*{Intercept-slope model}

\subsection*{Parametrization}

The intercept-slope model is a convenient reimplementation of a
commonly used construct, where
\begin{displaymath}
    (a,b)
\end{displaymath}
is bivariate Gaussian with a Wishart prior for the precision
matrix\footnote{The documentation for the model ``iid2d'' gives the
    details of the definition of the parameterisation of the precision
    matrix and the Wishart-prior.}, and various forms of
\begin{equation}\label{eq1}%
    \gamma(a + bz),
\end{equation}
where $z$ is a covariate and $\gamma$ is a (random) scaling, goes into
the linear predictor. Replicates of $(a,b)$ is indexed by
\emph{subject}, $i=1, \ldots, n$, and the various scaling of
Eq.~\ref{eq1} by \emph{strata} $j=1, \ldots, m$, leading to a model
for (a subset of)
\begin{displaymath}
    \left\{\gamma_j(a_i + b_i z_{ij}), \quad i=1, \ldots,n, \quad j=1,\ldots,m\right\},
\end{displaymath}
as not all combinations need to be present.

\subsection*{Hyperparameters}

The hyperparameters are $(\theta_1,\theta_2,\theta_3)$ as in the model
``iid2d'' (related to the precisions of $a$ and $b$, and their
correlation), and
$\theta_4=\gamma_1, \ldots, \theta_{13}=\gamma_{10}$. Since $m$ is
defined in the input, only $\gamma_1, \ldots, \gamma_m$ are used. $m$
is limited to $m \le 10$. Note that $\gamma_1$ is by default
\textbf{fixed} to $1$, due to confounding issues.


\subsection*{Specification}

The is specified as
\begin{verbatim}
    f(idx, model="intslope", hyper = ...,
      precision = exp(14),
      args.intslope = list(subject=i, strata=j, covariate = z))
\end{verbatim}
The defintion of the model is through the \verb|args.intslope|
argument, where \verb|i| and \verb|j| are factors/integers and
\verb|z| is numerical, all with same length $N$, say. The argument
\verb|idx|, index which row that is used, hence values of \verb|idx|
must take integer values in the interval $1$ to $N$. The precision
argument, defines the tiny small noise added to each $\gamma(a+bz)$ to
avoid a singular joint model. The \verb|subject| and \verb|strata|
argument, is converted internally as
\begin{verbatim}
    subject = as.numerical(as.factor(subject))
    strata  = as.numerical(as.factor(strata))
\end{verbatim}
and the results is shown after this conversion.


\subsubsection*{Hyperparameter spesification and default values}
{\small
\documentclass[a4paper,11pt]{article}
\usepackage[scale={0.8,0.9},centering,includeheadfoot]{geometry}
\usepackage{amstext}
\usepackage{listings}
\usepackage{verbatim}
\begin{document}

\section*{Intercept-slope model}

\subsection*{Parametrization}

The intercept-slope model is a convenient reimplementation of a
commonly used construct, where
\begin{displaymath}
    (a,b)
\end{displaymath}
is bivariate Gaussian with a Wishart prior for the precision
matrix\footnote{The documentation for the model ``iid2d'' gives the
    details of the definition of the parameterisation of the precision
    matrix and the Wishart-prior.}, and various forms of
\begin{equation}\label{eq1}%
    \gamma(a + bz),
\end{equation}
where $z$ is a covariate and $\gamma$ is a (random) scaling, goes into
the linear predictor. Replicates of $(a,b)$ is indexed by
\emph{subject}, $i=1, \ldots, n$, and the various scaling of
Eq.~\ref{eq1} by \emph{strata} $j=1, \ldots, m$, leading to a model
for (a subset of)
\begin{displaymath}
    \left\{\gamma_j(a_i + b_i z_{ij}), \quad i=1, \ldots,n, \quad j=1,\ldots,m\right\},
\end{displaymath}
as not all combinations need to be present.

\subsection*{Hyperparameters}

The hyperparameters are $(\theta_1,\theta_2,\theta_3)$ as in the model
``iid2d'' (related to the precisions of $a$ and $b$, and their
correlation), and
$\theta_4=\gamma_1, \ldots, \theta_{13}=\gamma_{10}$. Since $m$ is
defined in the input, only $\gamma_1, \ldots, \gamma_m$ are used. $m$
is limited to $m \le 10$. Note that $\gamma_1$ is by default
\textbf{fixed} to $1$, due to confounding issues.


\subsection*{Specification}

The is specified as
\begin{verbatim}
    f(idx, model="intslope", hyper = ...,
      precision = exp(14),
      args.intslope = list(subject=i, strata=j, covariate = z))
\end{verbatim}
The defintion of the model is through the \verb|args.intslope|
argument, where \verb|i| and \verb|j| are factors/integers and
\verb|z| is numerical, all with same length $N$, say. The argument
\verb|idx|, index which row that is used, hence values of \verb|idx|
must take integer values in the interval $1$ to $N$. The precision
argument, defines the tiny small noise added to each $\gamma(a+bz)$ to
avoid a singular joint model. The \verb|subject| and \verb|strata|
argument, is converted internally as
\begin{verbatim}
    subject = as.numerical(as.factor(subject))
    strata  = as.numerical(as.factor(strata))
\end{verbatim}
and the results is shown after this conversion.


\subsubsection*{Hyperparameter spesification and default values}
{\small
\documentclass[a4paper,11pt]{article}
\usepackage[scale={0.8,0.9},centering,includeheadfoot]{geometry}
\usepackage{amstext}
\usepackage{listings}
\usepackage{verbatim}
\begin{document}

\section*{Intercept-slope model}

\subsection*{Parametrization}

The intercept-slope model is a convenient reimplementation of a
commonly used construct, where
\begin{displaymath}
    (a,b)
\end{displaymath}
is bivariate Gaussian with a Wishart prior for the precision
matrix\footnote{The documentation for the model ``iid2d'' gives the
    details of the definition of the parameterisation of the precision
    matrix and the Wishart-prior.}, and various forms of
\begin{equation}\label{eq1}%
    \gamma(a + bz),
\end{equation}
where $z$ is a covariate and $\gamma$ is a (random) scaling, goes into
the linear predictor. Replicates of $(a,b)$ is indexed by
\emph{subject}, $i=1, \ldots, n$, and the various scaling of
Eq.~\ref{eq1} by \emph{strata} $j=1, \ldots, m$, leading to a model
for (a subset of)
\begin{displaymath}
    \left\{\gamma_j(a_i + b_i z_{ij}), \quad i=1, \ldots,n, \quad j=1,\ldots,m\right\},
\end{displaymath}
as not all combinations need to be present.

\subsection*{Hyperparameters}

The hyperparameters are $(\theta_1,\theta_2,\theta_3)$ as in the model
``iid2d'' (related to the precisions of $a$ and $b$, and their
correlation), and
$\theta_4=\gamma_1, \ldots, \theta_{13}=\gamma_{10}$. Since $m$ is
defined in the input, only $\gamma_1, \ldots, \gamma_m$ are used. $m$
is limited to $m \le 10$. Note that $\gamma_1$ is by default
\textbf{fixed} to $1$, due to confounding issues.


\subsection*{Specification}

The is specified as
\begin{verbatim}
    f(idx, model="intslope", hyper = ...,
      precision = exp(14),
      args.intslope = list(subject=i, strata=j, covariate = z))
\end{verbatim}
The defintion of the model is through the \verb|args.intslope|
argument, where \verb|i| and \verb|j| are factors/integers and
\verb|z| is numerical, all with same length $N$, say. The argument
\verb|idx|, index which row that is used, hence values of \verb|idx|
must take integer values in the interval $1$ to $N$. The precision
argument, defines the tiny small noise added to each $\gamma(a+bz)$ to
avoid a singular joint model. The \verb|subject| and \verb|strata|
argument, is converted internally as
\begin{verbatim}
    subject = as.numerical(as.factor(subject))
    strata  = as.numerical(as.factor(strata))
\end{verbatim}
and the results is shown after this conversion.


\subsubsection*{Hyperparameter spesification and default values}
{\small
\input{../hyper/latent/intslope.tex}
}

\clearpage
\subsection*{Example}
{\small\verbatiminput{example-intslope.R}}

\subsection*{Notes}
\begin{itemize}
\item With $n_s=\max($subject$)$, the internal storage of this model
    is
    \begin{displaymath}
        \left(\gamma_{j_1} (a_{i_1} + z_{1}b_{i_{1}}), \ldots, 
        \gamma_{j_N} (a_{i_N} + z_{N}b_{i_N}), 
        a_1, \ldots, a_{n_s}, b_1, \ldots, b_{n_s}\right),
    \end{displaymath}
    i.e.\ a vector of length $N+2n_s$.
\end{itemize}

\end{document}


% LocalWords: 

%%% Local Variables: 
%%% TeX-master: t
%%% End: 

}

\clearpage
\subsection*{Example}
{\small\verbatiminput{example-intslope.R}}

\subsection*{Notes}
\begin{itemize}
\item With $n_s=\max($subject$)$, the internal storage of this model
    is
    \begin{displaymath}
        \left(\gamma_{j_1} (a_{i_1} + z_{1}b_{i_{1}}), \ldots, 
        \gamma_{j_N} (a_{i_N} + z_{N}b_{i_N}), 
        a_1, \ldots, a_{n_s}, b_1, \ldots, b_{n_s}\right),
    \end{displaymath}
    i.e.\ a vector of length $N+2n_s$.
\end{itemize}

\end{document}


% LocalWords: 

%%% Local Variables: 
%%% TeX-master: t
%%% End: 

}

\clearpage
\subsection*{Example}
{\small\verbatiminput{example-intslope.R}}

\subsection*{Notes}
\begin{itemize}
\item With $n_s=\max($subject$)$, the internal storage of this model
    is
    \begin{displaymath}
        \left(\gamma_{j_1} (a_{i_1} + z_{1}b_{i_{1}}), \ldots, 
        \gamma_{j_N} (a_{i_N} + z_{N}b_{i_N}), 
        a_1, \ldots, a_{n_s}, b_1, \ldots, b_{n_s}\right),
    \end{displaymath}
    i.e.\ a vector of length $N+2n_s$.
\end{itemize}

\end{document}


% LocalWords: 

%%% Local Variables: 
%%% TeX-master: t
%%% End: 

}

\clearpage
\subsection*{Example}
{\small\verbatiminput{example-intslope.R}}

\subsection*{Notes}
\begin{itemize}
\item With $n_s=\max($subject$)$, the internal storage of this model
    is
    \begin{displaymath}
        \left(\gamma_{j_1} (a_{i_1} + z_{1}b_{i_{1}}), \ldots, 
        \gamma_{j_N} (a_{i_N} + z_{N}b_{i_N}), 
        a_1, \ldots, a_{n_s}, b_1, \ldots, b_{n_s}\right),
    \end{displaymath}
    i.e.\ a vector of length $N+2n_s$.
\end{itemize}

\end{document}


% LocalWords:  Parametrization Wishart iid covariate Eq ij Hyperparameters args

%%% Local Variables: 
%%% TeX-master: t
%%% End: 
% LocalWords:  hyperparameters intslope idx bz Hyperparameter

\documentclass[a4paper,11pt]{article}
\usepackage[scale={0.8,0.9},centering,includeheadfoot]{geometry}
\usepackage{amstext}
\usepackage{listings}
\begin{document}

\section*{Random walk model of order $2$ (RW2)}

\subsection*{Parametrization}
The random walk model of order $2$ (RW2) for the Gaussian vector
$\mathbf{x}=(x_1,\dots,x_n)$ is constructed assuming independent
second-orderincrements:
\[
\Delta^2 x_i = x_i-2\ x_{i+1}+x_{i+2}\sim\mathcal{N}(0,\tau^{-1})
\]
The density for $\mathbf{x}$ is derived from its $n-2$ second-order
increments as
\begin{eqnarray}
    \pi(\mathbf{x}|\tau) &\propto& \tau^{(n-2)/2} \exp\left\{-\frac{\tau}{2} \sum (\Delta^2 x_i)^2\right\}\\
    & = &\tau^{(n-2)/2}\exp\left\{-\frac{1}{2}\mathbf{x}^T\mathbf{Q}\mathbf{x} \right\}
\end{eqnarray}
where $\mathbf{Q}=\tau\mathbf{R}$ and $\mathbf{R}$ is the structure
matrix reflecting the neighbourhood structure of the model.

It is also possible to define a {\it cyclic} version of the RW2 model.

\subsection*{Hyperparameters}

The precision parameter $\tau$ is represented as
\begin{displaymath}
    \theta =\log \tau
\end{displaymath}
and the prior is defined on $\theta$.

\subsection*{Specification}

The RW2 model is specified inside the {\tt f()} function as
\begin{verbatim}
 f(<whatever>, model="rw2", values=<values>,
   cyclic=FALSE, scale.model = FALSE)
\end{verbatim}
The (optional) argument {\tt values } is a numeric or factor vector
giving the values assumed by the covariate for which we want the
effect to be estimated. See the example for RW1 for an application.

The logical option \verb|scale.model| determine if the model should be
scaled to have an average variance (the diagonal of the generalized
inverse) equal to 1. This makes prior spesification much
easier. Default is \verb|FALSE| so that the model is not scaled.


\subsubsection*{Hyperparameter spesification and default values}
\begin{description}
	\item[hyper]\ 
	 \begin{description}
	 	\item[theta]\ 
	 	 \begin{description}
	 	 	\item[name] log precision
	 	 	\item[short.name] prec
	 	 	\item[prior] loggamma
	 	 	\item[param] 1 5e-05
	 	 	\item[initial] 4
	 	 	\item[fixed] FALSE
	 	 	\item[to.theta] \verb|function(x) log(x)|
	 	 	\item[from.theta] \verb|function(x) exp(x)|
	 	 \end{description}
	 \end{description}
	\item[constr] TRUE
	\item[nrow.ncol] FALSE
	\item[augmented] FALSE
	\item[aug.factor] 1
	\item[aug.constr] 
	\item[n.div.by] 
	\item[n.required] FALSE
	\item[set.default.values] FALSE
	\item[pdf] rw2
\end{description}



\subsection*{Example}

\begin{verbatim}

n=100
z=seq(0,6,length.out=n)
y=sin(z)+rnorm(n,mean=0,sd=0.5)
data=data.frame(y=y,z=z)

formula=y~f(z,model="rw2")
result=inla(formula,data=data,family="gaussian")
\end{verbatim}


\subsection*{Notes}

\begin{itemize}
\item The RW2 is a intrinsic with rank deficiency 2.
\item The RW2 model for irregular locations are supported although not
    described here.
\item The $\frac{n-r}{2}\log(|R|^{*})$-part (with $r=2$) of the
    normalisation constant is not computed, hence you need to add this
    part to the log marginal likelihood estimate, if you need it.
\end{itemize}
\end{document}


% LocalWords: 

%%% Local Variables: 
%%% TeX-master: t
%%% End: 

\documentclass[a4paper,11pt]{article}
\usepackage[scale={0.8,0.9},centering,includeheadfoot]{geometry}
\usepackage{amstext}
\usepackage{listings}
\usepackage{verbatim}
\def\opening\null
\usepackage{block}
\begin{document}

\section*{Correlated random effects: \texttt{iidkd}}

This model is available for dimensions $k=2$, to $10$. We describe in
detail the case for $k=3$ as other ones are similar. This model do the
same as models \texttt{iid2d}, \texttt{iid3d}, \texttt{iid4d},
\texttt{iid5d}, but uses a different and more efficient parameterisation. 

\subsection*{Parametrization}

The $(k=3)$-dimensional Normal-Wishard model is used if one want to define
three vectors of ``random effects'', $u$ and $v$ and $w$, say, for which
$(u_{i}, v_{i}, w_i)$ are iid bivariate Normals
\begin{displaymath}
    \left(
      \begin{array}{c}
        u_{i}\\
        v_{i}\\
        w_{i}
      \end{array}\right)
    \sim \mathcal{N}\left(\mathbf{0}, \mathbf{W}^{-1}\right)
\end{displaymath}
where the  covariance matrix $\mathbf{W}^{-1}$ is parameterised as
$\mathbf{W}=\mathbf{L}\mathbf{L}^{T}$, where
\begin{equation}
    \label{precision}
    \mathbf{L} = \left(\begin{array}{ccc}
                         \exp(\theta_1) & & \\
                         \theta_4 & \exp(\theta_2) & \\
                         \theta_5 & \theta_6 & \exp(\theta_3)
      \end{array}\right)
\end{equation}
and $\theta_1, \theta_2, \theta_3, \theta_4, \theta_5, \theta_6$ can
take any value. The number of hyperparameters are $k(k+1)/2$, which is
$3$, $6$, $10$, $15$, $21$, $28$, $36$, $45$, $55$, for
$k=2, 3, 4, 5, 6, 7, 8, 9, 10$.

For these models the precision matrix $\mathbf{W}$ is Wishart
distributed
\begin{displaymath}
    \mathbf{W}
    \;\sim\;\text{Wishart}_{k}(r, \mathbf{R}^{-1}), 
\end{displaymath}
with density
\begin{displaymath}
    \pi(\mathbf{W}) = c^{-1} |\mathbf{W}|^{(r-(k+1))/2} \exp\left\{
      -\frac{1}{2}\text{Trace}(\mathbf{W}\mathbf{R})\right\}, \quad r > k+1
\end{displaymath}
and
\begin{displaymath}
    c = 2^{(rk)/2} |\mathbf{R}|^{-r/2} \pi^{(k(k-1))/4}\prod_{j=1}^{k}
    \Gamma((r+1-j)/2).
\end{displaymath}
Then,
\begin{displaymath}
    \text{E}(\mathbf{W}) = r\mathbf{R}^{-1}, \quad\text{and}\quad
    \text{E}(\mathbf{W}^{-1}) = \mathbf{R}/(r-(k+1)).
\end{displaymath}

\subsection*{Hyperparameters}

The hyperparameters are
$\theta_1, \theta_2, \theta_3, \theta_4, \theta_5, \theta_6$.

The prior-parameters are
\begin{displaymath}
    (r,R_{1}, R_{2}, R_{3}, R_{4}, R_{5}, R_{6})
\end{displaymath}
where 
\begin{displaymath}
    \mathbf{R}= \left(
      \begin{array}{ccc}
        R_{1} & R_{4} & R_{5}\\
        R_{4} & R_{2} & R_{6} \\
        R_{5} & R_{6} & R_{3}
      \end{array}\right)
\end{displaymath}

The {\tt inla} function reports posterior distribution for the
hyperparameters $\{\theta_i\}$, and the conversion into interpretable
quantities can be done using simulation as described below.

The prior for $\theta$ is {\bf fixed} to be {\tt wishartkd}, and
number of prior parameters required are $1 + k(k+1)/2$. By default the
prior-parameters are
\begin{displaymath}
    (r=100, \underbrace{1, \ldots, 1}_{k\;\text{times}}, 0, \ldots, 0)
\end{displaymath}


\subsection*{Specification}

The model \texttt{iidkd}
is specified as
\begin{verbatim}
    y ~ f(i, model="iidkd", order=3, n = <length>) + ...
\end{verbatim}
where $\text{order}=k=3$, and the \texttt{iidkd} model is represented
internally as one vector of length $n$,
\begin{displaymath}
    (u_{1}, u_{2} \ldots, u_{m}, v_{1}, v_{2}, \ldots, v_{m}, w_{1},
    w_{2}, \ldots, w_{m})
\end{displaymath}
where $n = 3m$, and $n$ is the (required) argument in
\texttt{f()}.

For this model the argument \texttt{constr=TRUE} is interpreted as $3$
sum-to-zero constraints
\begin{displaymath}
    \sum u_{i} = 0, \quad \sum v_{i} = 0 \quad\text{and}\quad \sum w_{i} = 0.
\end{displaymath}

\subsubsection*{Hyperparameter spesification and default values}

(\textbf{Note:} The value ``$2468.8642$'' is just a code for ``replace
this by the default value''. As the default value depends on
\texttt{order}, the was the easy way out for the moment.)

%% DO NOT EDIT!
%% This file is generated automatically from models.R
\begin{description}
	\item[doc] Gaussian random effect in dim=k with Wishart prior
	\item[hyper]\ 
	 \begin{description}
	 	\item[theta1]\ 
	 	 \begin{description}
	 	 	\item[hyperid] 29101
	 	 	\item[name] theta1
	 	 	\item[short.name] theta1
	 	 	\item[initial] 1048576
	 	 	\item[fixed] FALSE
	 	 	\item[prior] wishartkd
	 	 	\item[param] 21 1048576 1048576 1048576 1048576 1048576 1048576 1048576 1048576 1048576 1048576 1048576 1048576 1048576 1048576 1048576 1048576 1048576 1048576 1048576 1048576 1048576 1048576 1048576 1048576 1048576 1048576 1048576 1048576 1048576 1048576 1048576 1048576 1048576 1048576 1048576 1048576 1048576 1048576 1048576 1048576 1048576 1048576 1048576 1048576 1048576 1048576 1048576 1048576 1048576 1048576 1048576 1048576 1048576 1048576 1048576 1048576 1048576 1048576 1048576 1048576 1048576 1048576 1048576 1048576 1048576 1048576 1048576 1048576 1048576 1048576 1048576 1048576 1048576 1048576 1048576 1048576 1048576 1048576 1048576 1048576 1048576 1048576 1048576 1048576 1048576 1048576 1048576 1048576 1048576 1048576 1048576 1048576 1048576 1048576 1048576 1048576 1048576 1048576 1048576 1048576 1048576 1048576 1048576 1048576 1048576 1048576 1048576 1048576 1048576 1048576 1048576 1048576 1048576 1048576 1048576 1048576 1048576 1048576 1048576 1048576 1048576 1048576 1048576 1048576 1048576 1048576 1048576 1048576 1048576 1048576 1048576 1048576 1048576 1048576 1048576 1048576 1048576 1048576 1048576 1048576 1048576 1048576 1048576 1048576 1048576 1048576 1048576 1048576 1048576 1048576 1048576 1048576 1048576 1048576 1048576 1048576 1048576 1048576 1048576 1048576 1048576 1048576 1048576 1048576 1048576 1048576 1048576 1048576 1048576 1048576 1048576 1048576 1048576 1048576 1048576 1048576 1048576 1048576 1048576 1048576 1048576 1048576 1048576 1048576 1048576 1048576 1048576 1048576 1048576 1048576 1048576 1048576 1048576 1048576 1048576 1048576 1048576 1048576 1048576 1048576 1048576 1048576 1048576 1048576 1048576 1048576 1048576 1048576 1048576 1048576
	 	 	\item[to.theta] \verb!function(x) x!
	 	 	\item[from.theta] \verb!function(x) x!
	 	 \end{description}
	 	\item[theta2]\ 
	 	 \begin{description}
	 	 	\item[hyperid] 29102
	 	 	\item[name] theta2
	 	 	\item[short.name] theta2
	 	 	\item[initial] 1048576
	 	 	\item[fixed] FALSE
	 	 	\item[prior] none
	 	 	\item[param] 
	 	 	\item[to.theta] \verb!function(x) x!
	 	 	\item[from.theta] \verb!function(x) x!
	 	 \end{description}
	 	\item[theta3]\ 
	 	 \begin{description}
	 	 	\item[hyperid] 29103
	 	 	\item[name] theta3
	 	 	\item[short.name] theta3
	 	 	\item[initial] 1048576
	 	 	\item[fixed] FALSE
	 	 	\item[prior] none
	 	 	\item[param] 
	 	 	\item[to.theta] \verb!function(x) x!
	 	 	\item[from.theta] \verb!function(x) x!
	 	 \end{description}
	 	\item[theta4]\ 
	 	 \begin{description}
	 	 	\item[hyperid] 29104
	 	 	\item[name] theta4
	 	 	\item[short.name] theta4
	 	 	\item[initial] 1048576
	 	 	\item[fixed] FALSE
	 	 	\item[prior] none
	 	 	\item[param] 
	 	 	\item[to.theta] \verb!function(x) x!
	 	 	\item[from.theta] \verb!function(x) x!
	 	 \end{description}
	 	\item[theta5]\ 
	 	 \begin{description}
	 	 	\item[hyperid] 29105
	 	 	\item[name] theta5
	 	 	\item[short.name] theta5
	 	 	\item[initial] 1048576
	 	 	\item[fixed] FALSE
	 	 	\item[prior] none
	 	 	\item[param] 
	 	 	\item[to.theta] \verb!function(x) x!
	 	 	\item[from.theta] \verb!function(x) x!
	 	 \end{description}
	 	\item[theta6]\ 
	 	 \begin{description}
	 	 	\item[hyperid] 29106
	 	 	\item[name] theta6
	 	 	\item[short.name] theta6
	 	 	\item[initial] 1048576
	 	 	\item[fixed] FALSE
	 	 	\item[prior] none
	 	 	\item[param] 
	 	 	\item[to.theta] \verb!function(x) x!
	 	 	\item[from.theta] \verb!function(x) x!
	 	 \end{description}
	 	\item[theta7]\ 
	 	 \begin{description}
	 	 	\item[hyperid] 29107
	 	 	\item[name] theta7
	 	 	\item[short.name] theta7
	 	 	\item[initial] 1048576
	 	 	\item[fixed] FALSE
	 	 	\item[prior] none
	 	 	\item[param] 
	 	 	\item[to.theta] \verb!function(x) x!
	 	 	\item[from.theta] \verb!function(x) x!
	 	 \end{description}
	 	\item[theta8]\ 
	 	 \begin{description}
	 	 	\item[hyperid] 29108
	 	 	\item[name] theta8
	 	 	\item[short.name] theta8
	 	 	\item[initial] 1048576
	 	 	\item[fixed] FALSE
	 	 	\item[prior] none
	 	 	\item[param] 
	 	 	\item[to.theta] \verb!function(x) x!
	 	 	\item[from.theta] \verb!function(x) x!
	 	 \end{description}
	 	\item[theta9]\ 
	 	 \begin{description}
	 	 	\item[hyperid] 29109
	 	 	\item[name] theta9
	 	 	\item[short.name] theta9
	 	 	\item[initial] 1048576
	 	 	\item[fixed] FALSE
	 	 	\item[prior] none
	 	 	\item[param] 
	 	 	\item[to.theta] \verb!function(x) x!
	 	 	\item[from.theta] \verb!function(x) x!
	 	 \end{description}
	 	\item[theta10]\ 
	 	 \begin{description}
	 	 	\item[hyperid] 29110
	 	 	\item[name] theta10
	 	 	\item[short.name] theta10
	 	 	\item[initial] 1048576
	 	 	\item[fixed] FALSE
	 	 	\item[prior] none
	 	 	\item[param] 
	 	 	\item[to.theta] \verb!function(x) x!
	 	 	\item[from.theta] \verb!function(x) x!
	 	 \end{description}
	 	\item[theta11]\ 
	 	 \begin{description}
	 	 	\item[hyperid] 29111
	 	 	\item[name] theta11
	 	 	\item[short.name] theta11
	 	 	\item[initial] 1048576
	 	 	\item[fixed] FALSE
	 	 	\item[prior] none
	 	 	\item[param] 
	 	 	\item[to.theta] \verb!function(x) x!
	 	 	\item[from.theta] \verb!function(x) x!
	 	 \end{description}
	 	\item[theta12]\ 
	 	 \begin{description}
	 	 	\item[hyperid] 29112
	 	 	\item[name] theta12
	 	 	\item[short.name] theta12
	 	 	\item[initial] 1048576
	 	 	\item[fixed] FALSE
	 	 	\item[prior] none
	 	 	\item[param] 
	 	 	\item[to.theta] \verb!function(x) x!
	 	 	\item[from.theta] \verb!function(x) x!
	 	 \end{description}
	 	\item[theta13]\ 
	 	 \begin{description}
	 	 	\item[hyperid] 29113
	 	 	\item[name] theta13
	 	 	\item[short.name] theta13
	 	 	\item[initial] 1048576
	 	 	\item[fixed] FALSE
	 	 	\item[prior] none
	 	 	\item[param] 
	 	 	\item[to.theta] \verb!function(x) x!
	 	 	\item[from.theta] \verb!function(x) x!
	 	 \end{description}
	 	\item[theta14]\ 
	 	 \begin{description}
	 	 	\item[hyperid] 29114
	 	 	\item[name] theta14
	 	 	\item[short.name] theta14
	 	 	\item[initial] 1048576
	 	 	\item[fixed] FALSE
	 	 	\item[prior] none
	 	 	\item[param] 
	 	 	\item[to.theta] \verb!function(x) x!
	 	 	\item[from.theta] \verb!function(x) x!
	 	 \end{description}
	 	\item[theta15]\ 
	 	 \begin{description}
	 	 	\item[hyperid] 29115
	 	 	\item[name] theta15
	 	 	\item[short.name] theta15
	 	 	\item[initial] 1048576
	 	 	\item[fixed] FALSE
	 	 	\item[prior] none
	 	 	\item[param] 
	 	 	\item[to.theta] \verb!function(x) x!
	 	 	\item[from.theta] \verb!function(x) x!
	 	 \end{description}
	 	\item[theta16]\ 
	 	 \begin{description}
	 	 	\item[hyperid] 29116
	 	 	\item[name] theta16
	 	 	\item[short.name] theta16
	 	 	\item[initial] 1048576
	 	 	\item[fixed] FALSE
	 	 	\item[prior] none
	 	 	\item[param] 
	 	 	\item[to.theta] \verb!function(x) x!
	 	 	\item[from.theta] \verb!function(x) x!
	 	 \end{description}
	 	\item[theta17]\ 
	 	 \begin{description}
	 	 	\item[hyperid] 29117
	 	 	\item[name] theta17
	 	 	\item[short.name] theta17
	 	 	\item[initial] 1048576
	 	 	\item[fixed] FALSE
	 	 	\item[prior] none
	 	 	\item[param] 
	 	 	\item[to.theta] \verb!function(x) x!
	 	 	\item[from.theta] \verb!function(x) x!
	 	 \end{description}
	 	\item[theta18]\ 
	 	 \begin{description}
	 	 	\item[hyperid] 29118
	 	 	\item[name] theta18
	 	 	\item[short.name] theta18
	 	 	\item[initial] 1048576
	 	 	\item[fixed] FALSE
	 	 	\item[prior] none
	 	 	\item[param] 
	 	 	\item[to.theta] \verb!function(x) x!
	 	 	\item[from.theta] \verb!function(x) x!
	 	 \end{description}
	 	\item[theta19]\ 
	 	 \begin{description}
	 	 	\item[hyperid] 29119
	 	 	\item[name] theta19
	 	 	\item[short.name] theta19
	 	 	\item[initial] 1048576
	 	 	\item[fixed] FALSE
	 	 	\item[prior] none
	 	 	\item[param] 
	 	 	\item[to.theta] \verb!function(x) x!
	 	 	\item[from.theta] \verb!function(x) x!
	 	 \end{description}
	 	\item[theta20]\ 
	 	 \begin{description}
	 	 	\item[hyperid] 29120
	 	 	\item[name] theta20
	 	 	\item[short.name] theta20
	 	 	\item[initial] 1048576
	 	 	\item[fixed] FALSE
	 	 	\item[prior] none
	 	 	\item[param] 
	 	 	\item[to.theta] \verb!function(x) x!
	 	 	\item[from.theta] \verb!function(x) x!
	 	 \end{description}
	 	\item[theta21]\ 
	 	 \begin{description}
	 	 	\item[hyperid] 29121
	 	 	\item[name] theta21
	 	 	\item[short.name] theta21
	 	 	\item[initial] 1048576
	 	 	\item[fixed] FALSE
	 	 	\item[prior] none
	 	 	\item[param] 
	 	 	\item[to.theta] \verb!function(x) x!
	 	 	\item[from.theta] \verb!function(x) x!
	 	 \end{description}
	 	\item[theta22]\ 
	 	 \begin{description}
	 	 	\item[hyperid] 29122
	 	 	\item[name] theta22
	 	 	\item[short.name] theta22
	 	 	\item[initial] 1048576
	 	 	\item[fixed] FALSE
	 	 	\item[prior] none
	 	 	\item[param] 
	 	 	\item[to.theta] \verb!function(x) x!
	 	 	\item[from.theta] \verb!function(x) x!
	 	 \end{description}
	 	\item[theta23]\ 
	 	 \begin{description}
	 	 	\item[hyperid] 29123
	 	 	\item[name] theta23
	 	 	\item[short.name] theta23
	 	 	\item[initial] 1048576
	 	 	\item[fixed] FALSE
	 	 	\item[prior] none
	 	 	\item[param] 
	 	 	\item[to.theta] \verb!function(x) x!
	 	 	\item[from.theta] \verb!function(x) x!
	 	 \end{description}
	 	\item[theta24]\ 
	 	 \begin{description}
	 	 	\item[hyperid] 29124
	 	 	\item[name] theta24
	 	 	\item[short.name] theta24
	 	 	\item[initial] 1048576
	 	 	\item[fixed] FALSE
	 	 	\item[prior] none
	 	 	\item[param] 
	 	 	\item[to.theta] \verb!function(x) x!
	 	 	\item[from.theta] \verb!function(x) x!
	 	 \end{description}
	 	\item[theta25]\ 
	 	 \begin{description}
	 	 	\item[hyperid] 29125
	 	 	\item[name] theta25
	 	 	\item[short.name] theta25
	 	 	\item[initial] 1048576
	 	 	\item[fixed] FALSE
	 	 	\item[prior] none
	 	 	\item[param] 
	 	 	\item[to.theta] \verb!function(x) x!
	 	 	\item[from.theta] \verb!function(x) x!
	 	 \end{description}
	 	\item[theta26]\ 
	 	 \begin{description}
	 	 	\item[hyperid] 29126
	 	 	\item[name] theta26
	 	 	\item[short.name] theta26
	 	 	\item[initial] 1048576
	 	 	\item[fixed] FALSE
	 	 	\item[prior] none
	 	 	\item[param] 
	 	 	\item[to.theta] \verb!function(x) x!
	 	 	\item[from.theta] \verb!function(x) x!
	 	 \end{description}
	 	\item[theta27]\ 
	 	 \begin{description}
	 	 	\item[hyperid] 29127
	 	 	\item[name] theta27
	 	 	\item[short.name] theta27
	 	 	\item[initial] 1048576
	 	 	\item[fixed] FALSE
	 	 	\item[prior] none
	 	 	\item[param] 
	 	 	\item[to.theta] \verb!function(x) x!
	 	 	\item[from.theta] \verb!function(x) x!
	 	 \end{description}
	 	\item[theta28]\ 
	 	 \begin{description}
	 	 	\item[hyperid] 29128
	 	 	\item[name] theta28
	 	 	\item[short.name] theta28
	 	 	\item[initial] 1048576
	 	 	\item[fixed] FALSE
	 	 	\item[prior] none
	 	 	\item[param] 
	 	 	\item[to.theta] \verb!function(x) x!
	 	 	\item[from.theta] \verb!function(x) x!
	 	 \end{description}
	 	\item[theta29]\ 
	 	 \begin{description}
	 	 	\item[hyperid] 29129
	 	 	\item[name] theta29
	 	 	\item[short.name] theta29
	 	 	\item[initial] 1048576
	 	 	\item[fixed] FALSE
	 	 	\item[prior] none
	 	 	\item[param] 
	 	 	\item[to.theta] \verb!function(x) x!
	 	 	\item[from.theta] \verb!function(x) x!
	 	 \end{description}
	 	\item[theta30]\ 
	 	 \begin{description}
	 	 	\item[hyperid] 29130
	 	 	\item[name] theta30
	 	 	\item[short.name] theta30
	 	 	\item[initial] 1048576
	 	 	\item[fixed] FALSE
	 	 	\item[prior] none
	 	 	\item[param] 
	 	 	\item[to.theta] \verb!function(x) x!
	 	 	\item[from.theta] \verb!function(x) x!
	 	 \end{description}
	 	\item[theta31]\ 
	 	 \begin{description}
	 	 	\item[hyperid] 29131
	 	 	\item[name] theta31
	 	 	\item[short.name] theta31
	 	 	\item[initial] 1048576
	 	 	\item[fixed] FALSE
	 	 	\item[prior] none
	 	 	\item[param] 
	 	 	\item[to.theta] \verb!function(x) x!
	 	 	\item[from.theta] \verb!function(x) x!
	 	 \end{description}
	 	\item[theta32]\ 
	 	 \begin{description}
	 	 	\item[hyperid] 29132
	 	 	\item[name] theta32
	 	 	\item[short.name] theta32
	 	 	\item[initial] 1048576
	 	 	\item[fixed] FALSE
	 	 	\item[prior] none
	 	 	\item[param] 
	 	 	\item[to.theta] \verb!function(x) x!
	 	 	\item[from.theta] \verb!function(x) x!
	 	 \end{description}
	 	\item[theta33]\ 
	 	 \begin{description}
	 	 	\item[hyperid] 29133
	 	 	\item[name] theta33
	 	 	\item[short.name] theta33
	 	 	\item[initial] 1048576
	 	 	\item[fixed] FALSE
	 	 	\item[prior] none
	 	 	\item[param] 
	 	 	\item[to.theta] \verb!function(x) x!
	 	 	\item[from.theta] \verb!function(x) x!
	 	 \end{description}
	 	\item[theta34]\ 
	 	 \begin{description}
	 	 	\item[hyperid] 29134
	 	 	\item[name] theta34
	 	 	\item[short.name] theta34
	 	 	\item[initial] 1048576
	 	 	\item[fixed] FALSE
	 	 	\item[prior] none
	 	 	\item[param] 
	 	 	\item[to.theta] \verb!function(x) x!
	 	 	\item[from.theta] \verb!function(x) x!
	 	 \end{description}
	 	\item[theta35]\ 
	 	 \begin{description}
	 	 	\item[hyperid] 29135
	 	 	\item[name] theta35
	 	 	\item[short.name] theta35
	 	 	\item[initial] 1048576
	 	 	\item[fixed] FALSE
	 	 	\item[prior] none
	 	 	\item[param] 
	 	 	\item[to.theta] \verb!function(x) x!
	 	 	\item[from.theta] \verb!function(x) x!
	 	 \end{description}
	 	\item[theta36]\ 
	 	 \begin{description}
	 	 	\item[hyperid] 29136
	 	 	\item[name] theta36
	 	 	\item[short.name] theta36
	 	 	\item[initial] 1048576
	 	 	\item[fixed] FALSE
	 	 	\item[prior] none
	 	 	\item[param] 
	 	 	\item[to.theta] \verb!function(x) x!
	 	 	\item[from.theta] \verb!function(x) x!
	 	 \end{description}
	 	\item[theta37]\ 
	 	 \begin{description}
	 	 	\item[hyperid] 29137
	 	 	\item[name] theta37
	 	 	\item[short.name] theta37
	 	 	\item[initial] 1048576
	 	 	\item[fixed] FALSE
	 	 	\item[prior] none
	 	 	\item[param] 
	 	 	\item[to.theta] \verb!function(x) x!
	 	 	\item[from.theta] \verb!function(x) x!
	 	 \end{description}
	 	\item[theta38]\ 
	 	 \begin{description}
	 	 	\item[hyperid] 29138
	 	 	\item[name] theta38
	 	 	\item[short.name] theta38
	 	 	\item[initial] 1048576
	 	 	\item[fixed] FALSE
	 	 	\item[prior] none
	 	 	\item[param] 
	 	 	\item[to.theta] \verb!function(x) x!
	 	 	\item[from.theta] \verb!function(x) x!
	 	 \end{description}
	 	\item[theta39]\ 
	 	 \begin{description}
	 	 	\item[hyperid] 29139
	 	 	\item[name] theta39
	 	 	\item[short.name] theta39
	 	 	\item[initial] 1048576
	 	 	\item[fixed] FALSE
	 	 	\item[prior] none
	 	 	\item[param] 
	 	 	\item[to.theta] \verb!function(x) x!
	 	 	\item[from.theta] \verb!function(x) x!
	 	 \end{description}
	 	\item[theta40]\ 
	 	 \begin{description}
	 	 	\item[hyperid] 29140
	 	 	\item[name] theta40
	 	 	\item[short.name] theta40
	 	 	\item[initial] 1048576
	 	 	\item[fixed] FALSE
	 	 	\item[prior] none
	 	 	\item[param] 
	 	 	\item[to.theta] \verb!function(x) x!
	 	 	\item[from.theta] \verb!function(x) x!
	 	 \end{description}
	 	\item[theta41]\ 
	 	 \begin{description}
	 	 	\item[hyperid] 29141
	 	 	\item[name] theta41
	 	 	\item[short.name] theta41
	 	 	\item[initial] 1048576
	 	 	\item[fixed] FALSE
	 	 	\item[prior] none
	 	 	\item[param] 
	 	 	\item[to.theta] \verb!function(x) x!
	 	 	\item[from.theta] \verb!function(x) x!
	 	 \end{description}
	 	\item[theta42]\ 
	 	 \begin{description}
	 	 	\item[hyperid] 29142
	 	 	\item[name] theta42
	 	 	\item[short.name] theta42
	 	 	\item[initial] 1048576
	 	 	\item[fixed] FALSE
	 	 	\item[prior] none
	 	 	\item[param] 
	 	 	\item[to.theta] \verb!function(x) x!
	 	 	\item[from.theta] \verb!function(x) x!
	 	 \end{description}
	 	\item[theta43]\ 
	 	 \begin{description}
	 	 	\item[hyperid] 29143
	 	 	\item[name] theta43
	 	 	\item[short.name] theta43
	 	 	\item[initial] 1048576
	 	 	\item[fixed] FALSE
	 	 	\item[prior] none
	 	 	\item[param] 
	 	 	\item[to.theta] \verb!function(x) x!
	 	 	\item[from.theta] \verb!function(x) x!
	 	 \end{description}
	 	\item[theta44]\ 
	 	 \begin{description}
	 	 	\item[hyperid] 29144
	 	 	\item[name] theta44
	 	 	\item[short.name] theta44
	 	 	\item[initial] 1048576
	 	 	\item[fixed] FALSE
	 	 	\item[prior] none
	 	 	\item[param] 
	 	 	\item[to.theta] \verb!function(x) x!
	 	 	\item[from.theta] \verb!function(x) x!
	 	 \end{description}
	 	\item[theta45]\ 
	 	 \begin{description}
	 	 	\item[hyperid] 29145
	 	 	\item[name] theta45
	 	 	\item[short.name] theta45
	 	 	\item[initial] 1048576
	 	 	\item[fixed] FALSE
	 	 	\item[prior] none
	 	 	\item[param] 
	 	 	\item[to.theta] \verb!function(x) x!
	 	 	\item[from.theta] \verb!function(x) x!
	 	 \end{description}
	 	\item[theta46]\ 
	 	 \begin{description}
	 	 	\item[hyperid] 29146
	 	 	\item[name] theta46
	 	 	\item[short.name] theta46
	 	 	\item[initial] 1048576
	 	 	\item[fixed] FALSE
	 	 	\item[prior] none
	 	 	\item[param] 
	 	 	\item[to.theta] \verb!function(x) x!
	 	 	\item[from.theta] \verb!function(x) x!
	 	 \end{description}
	 	\item[theta47]\ 
	 	 \begin{description}
	 	 	\item[hyperid] 29147
	 	 	\item[name] theta47
	 	 	\item[short.name] theta47
	 	 	\item[initial] 1048576
	 	 	\item[fixed] FALSE
	 	 	\item[prior] none
	 	 	\item[param] 
	 	 	\item[to.theta] \verb!function(x) x!
	 	 	\item[from.theta] \verb!function(x) x!
	 	 \end{description}
	 	\item[theta48]\ 
	 	 \begin{description}
	 	 	\item[hyperid] 29148
	 	 	\item[name] theta48
	 	 	\item[short.name] theta48
	 	 	\item[initial] 1048576
	 	 	\item[fixed] FALSE
	 	 	\item[prior] none
	 	 	\item[param] 
	 	 	\item[to.theta] \verb!function(x) x!
	 	 	\item[from.theta] \verb!function(x) x!
	 	 \end{description}
	 	\item[theta49]\ 
	 	 \begin{description}
	 	 	\item[hyperid] 29149
	 	 	\item[name] theta49
	 	 	\item[short.name] theta49
	 	 	\item[initial] 1048576
	 	 	\item[fixed] FALSE
	 	 	\item[prior] none
	 	 	\item[param] 
	 	 	\item[to.theta] \verb!function(x) x!
	 	 	\item[from.theta] \verb!function(x) x!
	 	 \end{description}
	 	\item[theta50]\ 
	 	 \begin{description}
	 	 	\item[hyperid] 29150
	 	 	\item[name] theta50
	 	 	\item[short.name] theta50
	 	 	\item[initial] 1048576
	 	 	\item[fixed] FALSE
	 	 	\item[prior] none
	 	 	\item[param] 
	 	 	\item[to.theta] \verb!function(x) x!
	 	 	\item[from.theta] \verb!function(x) x!
	 	 \end{description}
	 	\item[theta51]\ 
	 	 \begin{description}
	 	 	\item[hyperid] 29151
	 	 	\item[name] theta51
	 	 	\item[short.name] theta51
	 	 	\item[initial] 1048576
	 	 	\item[fixed] FALSE
	 	 	\item[prior] none
	 	 	\item[param] 
	 	 	\item[to.theta] \verb!function(x) x!
	 	 	\item[from.theta] \verb!function(x) x!
	 	 \end{description}
	 	\item[theta52]\ 
	 	 \begin{description}
	 	 	\item[hyperid] 29152
	 	 	\item[name] theta52
	 	 	\item[short.name] theta52
	 	 	\item[initial] 1048576
	 	 	\item[fixed] FALSE
	 	 	\item[prior] none
	 	 	\item[param] 
	 	 	\item[to.theta] \verb!function(x) x!
	 	 	\item[from.theta] \verb!function(x) x!
	 	 \end{description}
	 	\item[theta53]\ 
	 	 \begin{description}
	 	 	\item[hyperid] 29153
	 	 	\item[name] theta53
	 	 	\item[short.name] theta53
	 	 	\item[initial] 1048576
	 	 	\item[fixed] FALSE
	 	 	\item[prior] none
	 	 	\item[param] 
	 	 	\item[to.theta] \verb!function(x) x!
	 	 	\item[from.theta] \verb!function(x) x!
	 	 \end{description}
	 	\item[theta54]\ 
	 	 \begin{description}
	 	 	\item[hyperid] 29154
	 	 	\item[name] theta54
	 	 	\item[short.name] theta54
	 	 	\item[initial] 1048576
	 	 	\item[fixed] FALSE
	 	 	\item[prior] none
	 	 	\item[param] 
	 	 	\item[to.theta] \verb!function(x) x!
	 	 	\item[from.theta] \verb!function(x) x!
	 	 \end{description}
	 	\item[theta55]\ 
	 	 \begin{description}
	 	 	\item[hyperid] 29155
	 	 	\item[name] theta55
	 	 	\item[short.name] theta55
	 	 	\item[initial] 1048576
	 	 	\item[fixed] FALSE
	 	 	\item[prior] none
	 	 	\item[param] 
	 	 	\item[to.theta] \verb!function(x) x!
	 	 	\item[from.theta] \verb!function(x) x!
	 	 \end{description}
	 	\item[theta56]\ 
	 	 \begin{description}
	 	 	\item[hyperid] 29156
	 	 	\item[name] theta56
	 	 	\item[short.name] theta56
	 	 	\item[initial] 1048576
	 	 	\item[fixed] FALSE
	 	 	\item[prior] none
	 	 	\item[param] 
	 	 	\item[to.theta] \verb!function(x) x!
	 	 	\item[from.theta] \verb!function(x) x!
	 	 \end{description}
	 	\item[theta57]\ 
	 	 \begin{description}
	 	 	\item[hyperid] 29157
	 	 	\item[name] theta57
	 	 	\item[short.name] theta57
	 	 	\item[initial] 1048576
	 	 	\item[fixed] FALSE
	 	 	\item[prior] none
	 	 	\item[param] 
	 	 	\item[to.theta] \verb!function(x) x!
	 	 	\item[from.theta] \verb!function(x) x!
	 	 \end{description}
	 	\item[theta58]\ 
	 	 \begin{description}
	 	 	\item[hyperid] 29158
	 	 	\item[name] theta58
	 	 	\item[short.name] theta58
	 	 	\item[initial] 1048576
	 	 	\item[fixed] FALSE
	 	 	\item[prior] none
	 	 	\item[param] 
	 	 	\item[to.theta] \verb!function(x) x!
	 	 	\item[from.theta] \verb!function(x) x!
	 	 \end{description}
	 	\item[theta59]\ 
	 	 \begin{description}
	 	 	\item[hyperid] 29159
	 	 	\item[name] theta59
	 	 	\item[short.name] theta59
	 	 	\item[initial] 1048576
	 	 	\item[fixed] FALSE
	 	 	\item[prior] none
	 	 	\item[param] 
	 	 	\item[to.theta] \verb!function(x) x!
	 	 	\item[from.theta] \verb!function(x) x!
	 	 \end{description}
	 	\item[theta60]\ 
	 	 \begin{description}
	 	 	\item[hyperid] 29160
	 	 	\item[name] theta60
	 	 	\item[short.name] theta60
	 	 	\item[initial] 1048576
	 	 	\item[fixed] FALSE
	 	 	\item[prior] none
	 	 	\item[param] 
	 	 	\item[to.theta] \verb!function(x) x!
	 	 	\item[from.theta] \verb!function(x) x!
	 	 \end{description}
	 	\item[theta61]\ 
	 	 \begin{description}
	 	 	\item[hyperid] 29161
	 	 	\item[name] theta61
	 	 	\item[short.name] theta61
	 	 	\item[initial] 1048576
	 	 	\item[fixed] FALSE
	 	 	\item[prior] none
	 	 	\item[param] 
	 	 	\item[to.theta] \verb!function(x) x!
	 	 	\item[from.theta] \verb!function(x) x!
	 	 \end{description}
	 	\item[theta62]\ 
	 	 \begin{description}
	 	 	\item[hyperid] 29162
	 	 	\item[name] theta62
	 	 	\item[short.name] theta62
	 	 	\item[initial] 1048576
	 	 	\item[fixed] FALSE
	 	 	\item[prior] none
	 	 	\item[param] 
	 	 	\item[to.theta] \verb!function(x) x!
	 	 	\item[from.theta] \verb!function(x) x!
	 	 \end{description}
	 	\item[theta63]\ 
	 	 \begin{description}
	 	 	\item[hyperid] 29163
	 	 	\item[name] theta63
	 	 	\item[short.name] theta63
	 	 	\item[initial] 1048576
	 	 	\item[fixed] FALSE
	 	 	\item[prior] none
	 	 	\item[param] 
	 	 	\item[to.theta] \verb!function(x) x!
	 	 	\item[from.theta] \verb!function(x) x!
	 	 \end{description}
	 	\item[theta64]\ 
	 	 \begin{description}
	 	 	\item[hyperid] 29164
	 	 	\item[name] theta64
	 	 	\item[short.name] theta64
	 	 	\item[initial] 1048576
	 	 	\item[fixed] FALSE
	 	 	\item[prior] none
	 	 	\item[param] 
	 	 	\item[to.theta] \verb!function(x) x!
	 	 	\item[from.theta] \verb!function(x) x!
	 	 \end{description}
	 	\item[theta65]\ 
	 	 \begin{description}
	 	 	\item[hyperid] 29165
	 	 	\item[name] theta65
	 	 	\item[short.name] theta65
	 	 	\item[initial] 1048576
	 	 	\item[fixed] FALSE
	 	 	\item[prior] none
	 	 	\item[param] 
	 	 	\item[to.theta] \verb!function(x) x!
	 	 	\item[from.theta] \verb!function(x) x!
	 	 \end{description}
	 	\item[theta66]\ 
	 	 \begin{description}
	 	 	\item[hyperid] 29166
	 	 	\item[name] theta66
	 	 	\item[short.name] theta66
	 	 	\item[initial] 1048576
	 	 	\item[fixed] FALSE
	 	 	\item[prior] none
	 	 	\item[param] 
	 	 	\item[to.theta] \verb!function(x) x!
	 	 	\item[from.theta] \verb!function(x) x!
	 	 \end{description}
	 	\item[theta67]\ 
	 	 \begin{description}
	 	 	\item[hyperid] 29167
	 	 	\item[name] theta67
	 	 	\item[short.name] theta67
	 	 	\item[initial] 1048576
	 	 	\item[fixed] FALSE
	 	 	\item[prior] none
	 	 	\item[param] 
	 	 	\item[to.theta] \verb!function(x) x!
	 	 	\item[from.theta] \verb!function(x) x!
	 	 \end{description}
	 	\item[theta68]\ 
	 	 \begin{description}
	 	 	\item[hyperid] 29168
	 	 	\item[name] theta68
	 	 	\item[short.name] theta68
	 	 	\item[initial] 1048576
	 	 	\item[fixed] FALSE
	 	 	\item[prior] none
	 	 	\item[param] 
	 	 	\item[to.theta] \verb!function(x) x!
	 	 	\item[from.theta] \verb!function(x) x!
	 	 \end{description}
	 	\item[theta69]\ 
	 	 \begin{description}
	 	 	\item[hyperid] 29169
	 	 	\item[name] theta69
	 	 	\item[short.name] theta69
	 	 	\item[initial] 1048576
	 	 	\item[fixed] FALSE
	 	 	\item[prior] none
	 	 	\item[param] 
	 	 	\item[to.theta] \verb!function(x) x!
	 	 	\item[from.theta] \verb!function(x) x!
	 	 \end{description}
	 	\item[theta70]\ 
	 	 \begin{description}
	 	 	\item[hyperid] 29170
	 	 	\item[name] theta70
	 	 	\item[short.name] theta70
	 	 	\item[initial] 1048576
	 	 	\item[fixed] FALSE
	 	 	\item[prior] none
	 	 	\item[param] 
	 	 	\item[to.theta] \verb!function(x) x!
	 	 	\item[from.theta] \verb!function(x) x!
	 	 \end{description}
	 	\item[theta71]\ 
	 	 \begin{description}
	 	 	\item[hyperid] 29171
	 	 	\item[name] theta71
	 	 	\item[short.name] theta71
	 	 	\item[initial] 1048576
	 	 	\item[fixed] FALSE
	 	 	\item[prior] none
	 	 	\item[param] 
	 	 	\item[to.theta] \verb!function(x) x!
	 	 	\item[from.theta] \verb!function(x) x!
	 	 \end{description}
	 	\item[theta72]\ 
	 	 \begin{description}
	 	 	\item[hyperid] 29172
	 	 	\item[name] theta72
	 	 	\item[short.name] theta72
	 	 	\item[initial] 1048576
	 	 	\item[fixed] FALSE
	 	 	\item[prior] none
	 	 	\item[param] 
	 	 	\item[to.theta] \verb!function(x) x!
	 	 	\item[from.theta] \verb!function(x) x!
	 	 \end{description}
	 	\item[theta73]\ 
	 	 \begin{description}
	 	 	\item[hyperid] 29173
	 	 	\item[name] theta73
	 	 	\item[short.name] theta73
	 	 	\item[initial] 1048576
	 	 	\item[fixed] FALSE
	 	 	\item[prior] none
	 	 	\item[param] 
	 	 	\item[to.theta] \verb!function(x) x!
	 	 	\item[from.theta] \verb!function(x) x!
	 	 \end{description}
	 	\item[theta74]\ 
	 	 \begin{description}
	 	 	\item[hyperid] 29174
	 	 	\item[name] theta74
	 	 	\item[short.name] theta74
	 	 	\item[initial] 1048576
	 	 	\item[fixed] FALSE
	 	 	\item[prior] none
	 	 	\item[param] 
	 	 	\item[to.theta] \verb!function(x) x!
	 	 	\item[from.theta] \verb!function(x) x!
	 	 \end{description}
	 	\item[theta75]\ 
	 	 \begin{description}
	 	 	\item[hyperid] 29175
	 	 	\item[name] theta75
	 	 	\item[short.name] theta75
	 	 	\item[initial] 1048576
	 	 	\item[fixed] FALSE
	 	 	\item[prior] none
	 	 	\item[param] 
	 	 	\item[to.theta] \verb!function(x) x!
	 	 	\item[from.theta] \verb!function(x) x!
	 	 \end{description}
	 	\item[theta76]\ 
	 	 \begin{description}
	 	 	\item[hyperid] 29176
	 	 	\item[name] theta76
	 	 	\item[short.name] theta76
	 	 	\item[initial] 1048576
	 	 	\item[fixed] FALSE
	 	 	\item[prior] none
	 	 	\item[param] 
	 	 	\item[to.theta] \verb!function(x) x!
	 	 	\item[from.theta] \verb!function(x) x!
	 	 \end{description}
	 	\item[theta77]\ 
	 	 \begin{description}
	 	 	\item[hyperid] 29177
	 	 	\item[name] theta77
	 	 	\item[short.name] theta77
	 	 	\item[initial] 1048576
	 	 	\item[fixed] FALSE
	 	 	\item[prior] none
	 	 	\item[param] 
	 	 	\item[to.theta] \verb!function(x) x!
	 	 	\item[from.theta] \verb!function(x) x!
	 	 \end{description}
	 	\item[theta78]\ 
	 	 \begin{description}
	 	 	\item[hyperid] 29178
	 	 	\item[name] theta78
	 	 	\item[short.name] theta78
	 	 	\item[initial] 1048576
	 	 	\item[fixed] FALSE
	 	 	\item[prior] none
	 	 	\item[param] 
	 	 	\item[to.theta] \verb!function(x) x!
	 	 	\item[from.theta] \verb!function(x) x!
	 	 \end{description}
	 	\item[theta79]\ 
	 	 \begin{description}
	 	 	\item[hyperid] 29179
	 	 	\item[name] theta79
	 	 	\item[short.name] theta79
	 	 	\item[initial] 1048576
	 	 	\item[fixed] FALSE
	 	 	\item[prior] none
	 	 	\item[param] 
	 	 	\item[to.theta] \verb!function(x) x!
	 	 	\item[from.theta] \verb!function(x) x!
	 	 \end{description}
	 	\item[theta80]\ 
	 	 \begin{description}
	 	 	\item[hyperid] 29180
	 	 	\item[name] theta80
	 	 	\item[short.name] theta80
	 	 	\item[initial] 1048576
	 	 	\item[fixed] FALSE
	 	 	\item[prior] none
	 	 	\item[param] 
	 	 	\item[to.theta] \verb!function(x) x!
	 	 	\item[from.theta] \verb!function(x) x!
	 	 \end{description}
	 	\item[theta81]\ 
	 	 \begin{description}
	 	 	\item[hyperid] 29181
	 	 	\item[name] theta81
	 	 	\item[short.name] theta81
	 	 	\item[initial] 1048576
	 	 	\item[fixed] FALSE
	 	 	\item[prior] none
	 	 	\item[param] 
	 	 	\item[to.theta] \verb!function(x) x!
	 	 	\item[from.theta] \verb!function(x) x!
	 	 \end{description}
	 	\item[theta82]\ 
	 	 \begin{description}
	 	 	\item[hyperid] 29182
	 	 	\item[name] theta82
	 	 	\item[short.name] theta82
	 	 	\item[initial] 1048576
	 	 	\item[fixed] FALSE
	 	 	\item[prior] none
	 	 	\item[param] 
	 	 	\item[to.theta] \verb!function(x) x!
	 	 	\item[from.theta] \verb!function(x) x!
	 	 \end{description}
	 	\item[theta83]\ 
	 	 \begin{description}
	 	 	\item[hyperid] 29183
	 	 	\item[name] theta83
	 	 	\item[short.name] theta83
	 	 	\item[initial] 1048576
	 	 	\item[fixed] FALSE
	 	 	\item[prior] none
	 	 	\item[param] 
	 	 	\item[to.theta] \verb!function(x) x!
	 	 	\item[from.theta] \verb!function(x) x!
	 	 \end{description}
	 	\item[theta84]\ 
	 	 \begin{description}
	 	 	\item[hyperid] 29184
	 	 	\item[name] theta84
	 	 	\item[short.name] theta84
	 	 	\item[initial] 1048576
	 	 	\item[fixed] FALSE
	 	 	\item[prior] none
	 	 	\item[param] 
	 	 	\item[to.theta] \verb!function(x) x!
	 	 	\item[from.theta] \verb!function(x) x!
	 	 \end{description}
	 	\item[theta85]\ 
	 	 \begin{description}
	 	 	\item[hyperid] 29185
	 	 	\item[name] theta85
	 	 	\item[short.name] theta85
	 	 	\item[initial] 1048576
	 	 	\item[fixed] FALSE
	 	 	\item[prior] none
	 	 	\item[param] 
	 	 	\item[to.theta] \verb!function(x) x!
	 	 	\item[from.theta] \verb!function(x) x!
	 	 \end{description}
	 	\item[theta86]\ 
	 	 \begin{description}
	 	 	\item[hyperid] 29186
	 	 	\item[name] theta86
	 	 	\item[short.name] theta86
	 	 	\item[initial] 1048576
	 	 	\item[fixed] FALSE
	 	 	\item[prior] none
	 	 	\item[param] 
	 	 	\item[to.theta] \verb!function(x) x!
	 	 	\item[from.theta] \verb!function(x) x!
	 	 \end{description}
	 	\item[theta87]\ 
	 	 \begin{description}
	 	 	\item[hyperid] 29187
	 	 	\item[name] theta87
	 	 	\item[short.name] theta87
	 	 	\item[initial] 1048576
	 	 	\item[fixed] FALSE
	 	 	\item[prior] none
	 	 	\item[param] 
	 	 	\item[to.theta] \verb!function(x) x!
	 	 	\item[from.theta] \verb!function(x) x!
	 	 \end{description}
	 	\item[theta88]\ 
	 	 \begin{description}
	 	 	\item[hyperid] 29188
	 	 	\item[name] theta88
	 	 	\item[short.name] theta88
	 	 	\item[initial] 1048576
	 	 	\item[fixed] FALSE
	 	 	\item[prior] none
	 	 	\item[param] 
	 	 	\item[to.theta] \verb!function(x) x!
	 	 	\item[from.theta] \verb!function(x) x!
	 	 \end{description}
	 	\item[theta89]\ 
	 	 \begin{description}
	 	 	\item[hyperid] 29189
	 	 	\item[name] theta89
	 	 	\item[short.name] theta89
	 	 	\item[initial] 1048576
	 	 	\item[fixed] FALSE
	 	 	\item[prior] none
	 	 	\item[param] 
	 	 	\item[to.theta] \verb!function(x) x!
	 	 	\item[from.theta] \verb!function(x) x!
	 	 \end{description}
	 	\item[theta90]\ 
	 	 \begin{description}
	 	 	\item[hyperid] 29190
	 	 	\item[name] theta90
	 	 	\item[short.name] theta90
	 	 	\item[initial] 1048576
	 	 	\item[fixed] FALSE
	 	 	\item[prior] none
	 	 	\item[param] 
	 	 	\item[to.theta] \verb!function(x) x!
	 	 	\item[from.theta] \verb!function(x) x!
	 	 \end{description}
	 	\item[theta91]\ 
	 	 \begin{description}
	 	 	\item[hyperid] 29191
	 	 	\item[name] theta91
	 	 	\item[short.name] theta91
	 	 	\item[initial] 1048576
	 	 	\item[fixed] FALSE
	 	 	\item[prior] none
	 	 	\item[param] 
	 	 	\item[to.theta] \verb!function(x) x!
	 	 	\item[from.theta] \verb!function(x) x!
	 	 \end{description}
	 	\item[theta92]\ 
	 	 \begin{description}
	 	 	\item[hyperid] 29192
	 	 	\item[name] theta92
	 	 	\item[short.name] theta92
	 	 	\item[initial] 1048576
	 	 	\item[fixed] FALSE
	 	 	\item[prior] none
	 	 	\item[param] 
	 	 	\item[to.theta] \verb!function(x) x!
	 	 	\item[from.theta] \verb!function(x) x!
	 	 \end{description}
	 	\item[theta93]\ 
	 	 \begin{description}
	 	 	\item[hyperid] 29193
	 	 	\item[name] theta93
	 	 	\item[short.name] theta93
	 	 	\item[initial] 1048576
	 	 	\item[fixed] FALSE
	 	 	\item[prior] none
	 	 	\item[param] 
	 	 	\item[to.theta] \verb!function(x) x!
	 	 	\item[from.theta] \verb!function(x) x!
	 	 \end{description}
	 	\item[theta94]\ 
	 	 \begin{description}
	 	 	\item[hyperid] 29194
	 	 	\item[name] theta94
	 	 	\item[short.name] theta94
	 	 	\item[initial] 1048576
	 	 	\item[fixed] FALSE
	 	 	\item[prior] none
	 	 	\item[param] 
	 	 	\item[to.theta] \verb!function(x) x!
	 	 	\item[from.theta] \verb!function(x) x!
	 	 \end{description}
	 	\item[theta95]\ 
	 	 \begin{description}
	 	 	\item[hyperid] 29195
	 	 	\item[name] theta95
	 	 	\item[short.name] theta95
	 	 	\item[initial] 1048576
	 	 	\item[fixed] FALSE
	 	 	\item[prior] none
	 	 	\item[param] 
	 	 	\item[to.theta] \verb!function(x) x!
	 	 	\item[from.theta] \verb!function(x) x!
	 	 \end{description}
	 	\item[theta96]\ 
	 	 \begin{description}
	 	 	\item[hyperid] 29196
	 	 	\item[name] theta96
	 	 	\item[short.name] theta96
	 	 	\item[initial] 1048576
	 	 	\item[fixed] FALSE
	 	 	\item[prior] none
	 	 	\item[param] 
	 	 	\item[to.theta] \verb!function(x) x!
	 	 	\item[from.theta] \verb!function(x) x!
	 	 \end{description}
	 	\item[theta97]\ 
	 	 \begin{description}
	 	 	\item[hyperid] 29197
	 	 	\item[name] theta97
	 	 	\item[short.name] theta97
	 	 	\item[initial] 1048576
	 	 	\item[fixed] FALSE
	 	 	\item[prior] none
	 	 	\item[param] 
	 	 	\item[to.theta] \verb!function(x) x!
	 	 	\item[from.theta] \verb!function(x) x!
	 	 \end{description}
	 	\item[theta98]\ 
	 	 \begin{description}
	 	 	\item[hyperid] 29198
	 	 	\item[name] theta98
	 	 	\item[short.name] theta98
	 	 	\item[initial] 1048576
	 	 	\item[fixed] FALSE
	 	 	\item[prior] none
	 	 	\item[param] 
	 	 	\item[to.theta] \verb!function(x) x!
	 	 	\item[from.theta] \verb!function(x) x!
	 	 \end{description}
	 	\item[theta99]\ 
	 	 \begin{description}
	 	 	\item[hyperid] 29199
	 	 	\item[name] theta99
	 	 	\item[short.name] theta99
	 	 	\item[initial] 1048576
	 	 	\item[fixed] FALSE
	 	 	\item[prior] none
	 	 	\item[param] 
	 	 	\item[to.theta] \verb!function(x) x!
	 	 	\item[from.theta] \verb!function(x) x!
	 	 \end{description}
	 	\item[theta100]\ 
	 	 \begin{description}
	 	 	\item[hyperid] 29200
	 	 	\item[name] theta100
	 	 	\item[short.name] theta100
	 	 	\item[initial] 1048576
	 	 	\item[fixed] FALSE
	 	 	\item[prior] none
	 	 	\item[param] 
	 	 	\item[to.theta] \verb!function(x) x!
	 	 	\item[from.theta] \verb!function(x) x!
	 	 \end{description}
	 	\item[theta101]\ 
	 	 \begin{description}
	 	 	\item[hyperid] 29201
	 	 	\item[name] theta101
	 	 	\item[short.name] theta101
	 	 	\item[initial] 1048576
	 	 	\item[fixed] FALSE
	 	 	\item[prior] none
	 	 	\item[param] 
	 	 	\item[to.theta] \verb!function(x) x!
	 	 	\item[from.theta] \verb!function(x) x!
	 	 \end{description}
	 	\item[theta102]\ 
	 	 \begin{description}
	 	 	\item[hyperid] 29202
	 	 	\item[name] theta102
	 	 	\item[short.name] theta102
	 	 	\item[initial] 1048576
	 	 	\item[fixed] FALSE
	 	 	\item[prior] none
	 	 	\item[param] 
	 	 	\item[to.theta] \verb!function(x) x!
	 	 	\item[from.theta] \verb!function(x) x!
	 	 \end{description}
	 	\item[theta103]\ 
	 	 \begin{description}
	 	 	\item[hyperid] 29203
	 	 	\item[name] theta103
	 	 	\item[short.name] theta103
	 	 	\item[initial] 1048576
	 	 	\item[fixed] FALSE
	 	 	\item[prior] none
	 	 	\item[param] 
	 	 	\item[to.theta] \verb!function(x) x!
	 	 	\item[from.theta] \verb!function(x) x!
	 	 \end{description}
	 	\item[theta104]\ 
	 	 \begin{description}
	 	 	\item[hyperid] 29204
	 	 	\item[name] theta104
	 	 	\item[short.name] theta104
	 	 	\item[initial] 1048576
	 	 	\item[fixed] FALSE
	 	 	\item[prior] none
	 	 	\item[param] 
	 	 	\item[to.theta] \verb!function(x) x!
	 	 	\item[from.theta] \verb!function(x) x!
	 	 \end{description}
	 	\item[theta105]\ 
	 	 \begin{description}
	 	 	\item[hyperid] 29205
	 	 	\item[name] theta105
	 	 	\item[short.name] theta105
	 	 	\item[initial] 1048576
	 	 	\item[fixed] FALSE
	 	 	\item[prior] none
	 	 	\item[param] 
	 	 	\item[to.theta] \verb!function(x) x!
	 	 	\item[from.theta] \verb!function(x) x!
	 	 \end{description}
	 	\item[theta106]\ 
	 	 \begin{description}
	 	 	\item[hyperid] 29206
	 	 	\item[name] theta106
	 	 	\item[short.name] theta106
	 	 	\item[initial] 1048576
	 	 	\item[fixed] FALSE
	 	 	\item[prior] none
	 	 	\item[param] 
	 	 	\item[to.theta] \verb!function(x) x!
	 	 	\item[from.theta] \verb!function(x) x!
	 	 \end{description}
	 	\item[theta107]\ 
	 	 \begin{description}
	 	 	\item[hyperid] 29207
	 	 	\item[name] theta107
	 	 	\item[short.name] theta107
	 	 	\item[initial] 1048576
	 	 	\item[fixed] FALSE
	 	 	\item[prior] none
	 	 	\item[param] 
	 	 	\item[to.theta] \verb!function(x) x!
	 	 	\item[from.theta] \verb!function(x) x!
	 	 \end{description}
	 	\item[theta108]\ 
	 	 \begin{description}
	 	 	\item[hyperid] 29208
	 	 	\item[name] theta108
	 	 	\item[short.name] theta108
	 	 	\item[initial] 1048576
	 	 	\item[fixed] FALSE
	 	 	\item[prior] none
	 	 	\item[param] 
	 	 	\item[to.theta] \verb!function(x) x!
	 	 	\item[from.theta] \verb!function(x) x!
	 	 \end{description}
	 	\item[theta109]\ 
	 	 \begin{description}
	 	 	\item[hyperid] 29209
	 	 	\item[name] theta109
	 	 	\item[short.name] theta109
	 	 	\item[initial] 1048576
	 	 	\item[fixed] FALSE
	 	 	\item[prior] none
	 	 	\item[param] 
	 	 	\item[to.theta] \verb!function(x) x!
	 	 	\item[from.theta] \verb!function(x) x!
	 	 \end{description}
	 	\item[theta110]\ 
	 	 \begin{description}
	 	 	\item[hyperid] 29210
	 	 	\item[name] theta110
	 	 	\item[short.name] theta110
	 	 	\item[initial] 1048576
	 	 	\item[fixed] FALSE
	 	 	\item[prior] none
	 	 	\item[param] 
	 	 	\item[to.theta] \verb!function(x) x!
	 	 	\item[from.theta] \verb!function(x) x!
	 	 \end{description}
	 	\item[theta111]\ 
	 	 \begin{description}
	 	 	\item[hyperid] 29211
	 	 	\item[name] theta111
	 	 	\item[short.name] theta111
	 	 	\item[initial] 1048576
	 	 	\item[fixed] FALSE
	 	 	\item[prior] none
	 	 	\item[param] 
	 	 	\item[to.theta] \verb!function(x) x!
	 	 	\item[from.theta] \verb!function(x) x!
	 	 \end{description}
	 	\item[theta112]\ 
	 	 \begin{description}
	 	 	\item[hyperid] 29212
	 	 	\item[name] theta112
	 	 	\item[short.name] theta112
	 	 	\item[initial] 1048576
	 	 	\item[fixed] FALSE
	 	 	\item[prior] none
	 	 	\item[param] 
	 	 	\item[to.theta] \verb!function(x) x!
	 	 	\item[from.theta] \verb!function(x) x!
	 	 \end{description}
	 	\item[theta113]\ 
	 	 \begin{description}
	 	 	\item[hyperid] 29213
	 	 	\item[name] theta113
	 	 	\item[short.name] theta113
	 	 	\item[initial] 1048576
	 	 	\item[fixed] FALSE
	 	 	\item[prior] none
	 	 	\item[param] 
	 	 	\item[to.theta] \verb!function(x) x!
	 	 	\item[from.theta] \verb!function(x) x!
	 	 \end{description}
	 	\item[theta114]\ 
	 	 \begin{description}
	 	 	\item[hyperid] 29214
	 	 	\item[name] theta114
	 	 	\item[short.name] theta114
	 	 	\item[initial] 1048576
	 	 	\item[fixed] FALSE
	 	 	\item[prior] none
	 	 	\item[param] 
	 	 	\item[to.theta] \verb!function(x) x!
	 	 	\item[from.theta] \verb!function(x) x!
	 	 \end{description}
	 	\item[theta115]\ 
	 	 \begin{description}
	 	 	\item[hyperid] 29215
	 	 	\item[name] theta115
	 	 	\item[short.name] theta115
	 	 	\item[initial] 1048576
	 	 	\item[fixed] FALSE
	 	 	\item[prior] none
	 	 	\item[param] 
	 	 	\item[to.theta] \verb!function(x) x!
	 	 	\item[from.theta] \verb!function(x) x!
	 	 \end{description}
	 	\item[theta116]\ 
	 	 \begin{description}
	 	 	\item[hyperid] 29216
	 	 	\item[name] theta116
	 	 	\item[short.name] theta116
	 	 	\item[initial] 1048576
	 	 	\item[fixed] FALSE
	 	 	\item[prior] none
	 	 	\item[param] 
	 	 	\item[to.theta] \verb!function(x) x!
	 	 	\item[from.theta] \verb!function(x) x!
	 	 \end{description}
	 	\item[theta117]\ 
	 	 \begin{description}
	 	 	\item[hyperid] 29217
	 	 	\item[name] theta117
	 	 	\item[short.name] theta117
	 	 	\item[initial] 1048576
	 	 	\item[fixed] FALSE
	 	 	\item[prior] none
	 	 	\item[param] 
	 	 	\item[to.theta] \verb!function(x) x!
	 	 	\item[from.theta] \verb!function(x) x!
	 	 \end{description}
	 	\item[theta118]\ 
	 	 \begin{description}
	 	 	\item[hyperid] 29218
	 	 	\item[name] theta118
	 	 	\item[short.name] theta118
	 	 	\item[initial] 1048576
	 	 	\item[fixed] FALSE
	 	 	\item[prior] none
	 	 	\item[param] 
	 	 	\item[to.theta] \verb!function(x) x!
	 	 	\item[from.theta] \verb!function(x) x!
	 	 \end{description}
	 	\item[theta119]\ 
	 	 \begin{description}
	 	 	\item[hyperid] 29219
	 	 	\item[name] theta119
	 	 	\item[short.name] theta119
	 	 	\item[initial] 1048576
	 	 	\item[fixed] FALSE
	 	 	\item[prior] none
	 	 	\item[param] 
	 	 	\item[to.theta] \verb!function(x) x!
	 	 	\item[from.theta] \verb!function(x) x!
	 	 \end{description}
	 	\item[theta120]\ 
	 	 \begin{description}
	 	 	\item[hyperid] 29220
	 	 	\item[name] theta120
	 	 	\item[short.name] theta120
	 	 	\item[initial] 1048576
	 	 	\item[fixed] FALSE
	 	 	\item[prior] none
	 	 	\item[param] 
	 	 	\item[to.theta] \verb!function(x) x!
	 	 	\item[from.theta] \verb!function(x) x!
	 	 \end{description}
	 	\item[theta121]\ 
	 	 \begin{description}
	 	 	\item[hyperid] 29221
	 	 	\item[name] theta121
	 	 	\item[short.name] theta121
	 	 	\item[initial] 1048576
	 	 	\item[fixed] FALSE
	 	 	\item[prior] none
	 	 	\item[param] 
	 	 	\item[to.theta] \verb!function(x) x!
	 	 	\item[from.theta] \verb!function(x) x!
	 	 \end{description}
	 	\item[theta122]\ 
	 	 \begin{description}
	 	 	\item[hyperid] 29222
	 	 	\item[name] theta122
	 	 	\item[short.name] theta122
	 	 	\item[initial] 1048576
	 	 	\item[fixed] FALSE
	 	 	\item[prior] none
	 	 	\item[param] 
	 	 	\item[to.theta] \verb!function(x) x!
	 	 	\item[from.theta] \verb!function(x) x!
	 	 \end{description}
	 	\item[theta123]\ 
	 	 \begin{description}
	 	 	\item[hyperid] 29223
	 	 	\item[name] theta123
	 	 	\item[short.name] theta123
	 	 	\item[initial] 1048576
	 	 	\item[fixed] FALSE
	 	 	\item[prior] none
	 	 	\item[param] 
	 	 	\item[to.theta] \verb!function(x) x!
	 	 	\item[from.theta] \verb!function(x) x!
	 	 \end{description}
	 	\item[theta124]\ 
	 	 \begin{description}
	 	 	\item[hyperid] 29224
	 	 	\item[name] theta124
	 	 	\item[short.name] theta124
	 	 	\item[initial] 1048576
	 	 	\item[fixed] FALSE
	 	 	\item[prior] none
	 	 	\item[param] 
	 	 	\item[to.theta] \verb!function(x) x!
	 	 	\item[from.theta] \verb!function(x) x!
	 	 \end{description}
	 	\item[theta125]\ 
	 	 \begin{description}
	 	 	\item[hyperid] 29225
	 	 	\item[name] theta125
	 	 	\item[short.name] theta125
	 	 	\item[initial] 1048576
	 	 	\item[fixed] FALSE
	 	 	\item[prior] none
	 	 	\item[param] 
	 	 	\item[to.theta] \verb!function(x) x!
	 	 	\item[from.theta] \verb!function(x) x!
	 	 \end{description}
	 	\item[theta126]\ 
	 	 \begin{description}
	 	 	\item[hyperid] 29226
	 	 	\item[name] theta126
	 	 	\item[short.name] theta126
	 	 	\item[initial] 1048576
	 	 	\item[fixed] FALSE
	 	 	\item[prior] none
	 	 	\item[param] 
	 	 	\item[to.theta] \verb!function(x) x!
	 	 	\item[from.theta] \verb!function(x) x!
	 	 \end{description}
	 	\item[theta127]\ 
	 	 \begin{description}
	 	 	\item[hyperid] 29227
	 	 	\item[name] theta127
	 	 	\item[short.name] theta127
	 	 	\item[initial] 1048576
	 	 	\item[fixed] FALSE
	 	 	\item[prior] none
	 	 	\item[param] 
	 	 	\item[to.theta] \verb!function(x) x!
	 	 	\item[from.theta] \verb!function(x) x!
	 	 \end{description}
	 	\item[theta128]\ 
	 	 \begin{description}
	 	 	\item[hyperid] 29228
	 	 	\item[name] theta128
	 	 	\item[short.name] theta128
	 	 	\item[initial] 1048576
	 	 	\item[fixed] FALSE
	 	 	\item[prior] none
	 	 	\item[param] 
	 	 	\item[to.theta] \verb!function(x) x!
	 	 	\item[from.theta] \verb!function(x) x!
	 	 \end{description}
	 	\item[theta129]\ 
	 	 \begin{description}
	 	 	\item[hyperid] 29229
	 	 	\item[name] theta129
	 	 	\item[short.name] theta129
	 	 	\item[initial] 1048576
	 	 	\item[fixed] FALSE
	 	 	\item[prior] none
	 	 	\item[param] 
	 	 	\item[to.theta] \verb!function(x) x!
	 	 	\item[from.theta] \verb!function(x) x!
	 	 \end{description}
	 	\item[theta130]\ 
	 	 \begin{description}
	 	 	\item[hyperid] 29230
	 	 	\item[name] theta130
	 	 	\item[short.name] theta130
	 	 	\item[initial] 1048576
	 	 	\item[fixed] FALSE
	 	 	\item[prior] none
	 	 	\item[param] 
	 	 	\item[to.theta] \verb!function(x) x!
	 	 	\item[from.theta] \verb!function(x) x!
	 	 \end{description}
	 	\item[theta131]\ 
	 	 \begin{description}
	 	 	\item[hyperid] 29231
	 	 	\item[name] theta131
	 	 	\item[short.name] theta131
	 	 	\item[initial] 1048576
	 	 	\item[fixed] FALSE
	 	 	\item[prior] none
	 	 	\item[param] 
	 	 	\item[to.theta] \verb!function(x) x!
	 	 	\item[from.theta] \verb!function(x) x!
	 	 \end{description}
	 	\item[theta132]\ 
	 	 \begin{description}
	 	 	\item[hyperid] 29232
	 	 	\item[name] theta132
	 	 	\item[short.name] theta132
	 	 	\item[initial] 1048576
	 	 	\item[fixed] FALSE
	 	 	\item[prior] none
	 	 	\item[param] 
	 	 	\item[to.theta] \verb!function(x) x!
	 	 	\item[from.theta] \verb!function(x) x!
	 	 \end{description}
	 	\item[theta133]\ 
	 	 \begin{description}
	 	 	\item[hyperid] 29233
	 	 	\item[name] theta133
	 	 	\item[short.name] theta133
	 	 	\item[initial] 1048576
	 	 	\item[fixed] FALSE
	 	 	\item[prior] none
	 	 	\item[param] 
	 	 	\item[to.theta] \verb!function(x) x!
	 	 	\item[from.theta] \verb!function(x) x!
	 	 \end{description}
	 	\item[theta134]\ 
	 	 \begin{description}
	 	 	\item[hyperid] 29234
	 	 	\item[name] theta134
	 	 	\item[short.name] theta134
	 	 	\item[initial] 1048576
	 	 	\item[fixed] FALSE
	 	 	\item[prior] none
	 	 	\item[param] 
	 	 	\item[to.theta] \verb!function(x) x!
	 	 	\item[from.theta] \verb!function(x) x!
	 	 \end{description}
	 	\item[theta135]\ 
	 	 \begin{description}
	 	 	\item[hyperid] 29235
	 	 	\item[name] theta135
	 	 	\item[short.name] theta135
	 	 	\item[initial] 1048576
	 	 	\item[fixed] FALSE
	 	 	\item[prior] none
	 	 	\item[param] 
	 	 	\item[to.theta] \verb!function(x) x!
	 	 	\item[from.theta] \verb!function(x) x!
	 	 \end{description}
	 	\item[theta136]\ 
	 	 \begin{description}
	 	 	\item[hyperid] 29236
	 	 	\item[name] theta136
	 	 	\item[short.name] theta136
	 	 	\item[initial] 1048576
	 	 	\item[fixed] FALSE
	 	 	\item[prior] none
	 	 	\item[param] 
	 	 	\item[to.theta] \verb!function(x) x!
	 	 	\item[from.theta] \verb!function(x) x!
	 	 \end{description}
	 	\item[theta137]\ 
	 	 \begin{description}
	 	 	\item[hyperid] 29237
	 	 	\item[name] theta137
	 	 	\item[short.name] theta137
	 	 	\item[initial] 1048576
	 	 	\item[fixed] FALSE
	 	 	\item[prior] none
	 	 	\item[param] 
	 	 	\item[to.theta] \verb!function(x) x!
	 	 	\item[from.theta] \verb!function(x) x!
	 	 \end{description}
	 	\item[theta138]\ 
	 	 \begin{description}
	 	 	\item[hyperid] 29238
	 	 	\item[name] theta138
	 	 	\item[short.name] theta138
	 	 	\item[initial] 1048576
	 	 	\item[fixed] FALSE
	 	 	\item[prior] none
	 	 	\item[param] 
	 	 	\item[to.theta] \verb!function(x) x!
	 	 	\item[from.theta] \verb!function(x) x!
	 	 \end{description}
	 	\item[theta139]\ 
	 	 \begin{description}
	 	 	\item[hyperid] 29239
	 	 	\item[name] theta139
	 	 	\item[short.name] theta139
	 	 	\item[initial] 1048576
	 	 	\item[fixed] FALSE
	 	 	\item[prior] none
	 	 	\item[param] 
	 	 	\item[to.theta] \verb!function(x) x!
	 	 	\item[from.theta] \verb!function(x) x!
	 	 \end{description}
	 	\item[theta140]\ 
	 	 \begin{description}
	 	 	\item[hyperid] 29240
	 	 	\item[name] theta140
	 	 	\item[short.name] theta140
	 	 	\item[initial] 1048576
	 	 	\item[fixed] FALSE
	 	 	\item[prior] none
	 	 	\item[param] 
	 	 	\item[to.theta] \verb!function(x) x!
	 	 	\item[from.theta] \verb!function(x) x!
	 	 \end{description}
	 	\item[theta141]\ 
	 	 \begin{description}
	 	 	\item[hyperid] 29241
	 	 	\item[name] theta141
	 	 	\item[short.name] theta141
	 	 	\item[initial] 1048576
	 	 	\item[fixed] FALSE
	 	 	\item[prior] none
	 	 	\item[param] 
	 	 	\item[to.theta] \verb!function(x) x!
	 	 	\item[from.theta] \verb!function(x) x!
	 	 \end{description}
	 	\item[theta142]\ 
	 	 \begin{description}
	 	 	\item[hyperid] 29242
	 	 	\item[name] theta142
	 	 	\item[short.name] theta142
	 	 	\item[initial] 1048576
	 	 	\item[fixed] FALSE
	 	 	\item[prior] none
	 	 	\item[param] 
	 	 	\item[to.theta] \verb!function(x) x!
	 	 	\item[from.theta] \verb!function(x) x!
	 	 \end{description}
	 	\item[theta143]\ 
	 	 \begin{description}
	 	 	\item[hyperid] 29243
	 	 	\item[name] theta143
	 	 	\item[short.name] theta143
	 	 	\item[initial] 1048576
	 	 	\item[fixed] FALSE
	 	 	\item[prior] none
	 	 	\item[param] 
	 	 	\item[to.theta] \verb!function(x) x!
	 	 	\item[from.theta] \verb!function(x) x!
	 	 \end{description}
	 	\item[theta144]\ 
	 	 \begin{description}
	 	 	\item[hyperid] 29244
	 	 	\item[name] theta144
	 	 	\item[short.name] theta144
	 	 	\item[initial] 1048576
	 	 	\item[fixed] FALSE
	 	 	\item[prior] none
	 	 	\item[param] 
	 	 	\item[to.theta] \verb!function(x) x!
	 	 	\item[from.theta] \verb!function(x) x!
	 	 \end{description}
	 	\item[theta145]\ 
	 	 \begin{description}
	 	 	\item[hyperid] 29245
	 	 	\item[name] theta145
	 	 	\item[short.name] theta145
	 	 	\item[initial] 1048576
	 	 	\item[fixed] FALSE
	 	 	\item[prior] none
	 	 	\item[param] 
	 	 	\item[to.theta] \verb!function(x) x!
	 	 	\item[from.theta] \verb!function(x) x!
	 	 \end{description}
	 	\item[theta146]\ 
	 	 \begin{description}
	 	 	\item[hyperid] 29246
	 	 	\item[name] theta146
	 	 	\item[short.name] theta146
	 	 	\item[initial] 1048576
	 	 	\item[fixed] FALSE
	 	 	\item[prior] none
	 	 	\item[param] 
	 	 	\item[to.theta] \verb!function(x) x!
	 	 	\item[from.theta] \verb!function(x) x!
	 	 \end{description}
	 	\item[theta147]\ 
	 	 \begin{description}
	 	 	\item[hyperid] 29247
	 	 	\item[name] theta147
	 	 	\item[short.name] theta147
	 	 	\item[initial] 1048576
	 	 	\item[fixed] FALSE
	 	 	\item[prior] none
	 	 	\item[param] 
	 	 	\item[to.theta] \verb!function(x) x!
	 	 	\item[from.theta] \verb!function(x) x!
	 	 \end{description}
	 	\item[theta148]\ 
	 	 \begin{description}
	 	 	\item[hyperid] 29248
	 	 	\item[name] theta148
	 	 	\item[short.name] theta148
	 	 	\item[initial] 1048576
	 	 	\item[fixed] FALSE
	 	 	\item[prior] none
	 	 	\item[param] 
	 	 	\item[to.theta] \verb!function(x) x!
	 	 	\item[from.theta] \verb!function(x) x!
	 	 \end{description}
	 	\item[theta149]\ 
	 	 \begin{description}
	 	 	\item[hyperid] 29249
	 	 	\item[name] theta149
	 	 	\item[short.name] theta149
	 	 	\item[initial] 1048576
	 	 	\item[fixed] FALSE
	 	 	\item[prior] none
	 	 	\item[param] 
	 	 	\item[to.theta] \verb!function(x) x!
	 	 	\item[from.theta] \verb!function(x) x!
	 	 \end{description}
	 	\item[theta150]\ 
	 	 \begin{description}
	 	 	\item[hyperid] 29250
	 	 	\item[name] theta150
	 	 	\item[short.name] theta150
	 	 	\item[initial] 1048576
	 	 	\item[fixed] FALSE
	 	 	\item[prior] none
	 	 	\item[param] 
	 	 	\item[to.theta] \verb!function(x) x!
	 	 	\item[from.theta] \verb!function(x) x!
	 	 \end{description}
	 	\item[theta151]\ 
	 	 \begin{description}
	 	 	\item[hyperid] 29251
	 	 	\item[name] theta151
	 	 	\item[short.name] theta151
	 	 	\item[initial] 1048576
	 	 	\item[fixed] FALSE
	 	 	\item[prior] none
	 	 	\item[param] 
	 	 	\item[to.theta] \verb!function(x) x!
	 	 	\item[from.theta] \verb!function(x) x!
	 	 \end{description}
	 	\item[theta152]\ 
	 	 \begin{description}
	 	 	\item[hyperid] 29252
	 	 	\item[name] theta152
	 	 	\item[short.name] theta152
	 	 	\item[initial] 1048576
	 	 	\item[fixed] FALSE
	 	 	\item[prior] none
	 	 	\item[param] 
	 	 	\item[to.theta] \verb!function(x) x!
	 	 	\item[from.theta] \verb!function(x) x!
	 	 \end{description}
	 	\item[theta153]\ 
	 	 \begin{description}
	 	 	\item[hyperid] 29253
	 	 	\item[name] theta153
	 	 	\item[short.name] theta153
	 	 	\item[initial] 1048576
	 	 	\item[fixed] FALSE
	 	 	\item[prior] none
	 	 	\item[param] 
	 	 	\item[to.theta] \verb!function(x) x!
	 	 	\item[from.theta] \verb!function(x) x!
	 	 \end{description}
	 	\item[theta154]\ 
	 	 \begin{description}
	 	 	\item[hyperid] 29254
	 	 	\item[name] theta154
	 	 	\item[short.name] theta154
	 	 	\item[initial] 1048576
	 	 	\item[fixed] FALSE
	 	 	\item[prior] none
	 	 	\item[param] 
	 	 	\item[to.theta] \verb!function(x) x!
	 	 	\item[from.theta] \verb!function(x) x!
	 	 \end{description}
	 	\item[theta155]\ 
	 	 \begin{description}
	 	 	\item[hyperid] 29255
	 	 	\item[name] theta155
	 	 	\item[short.name] theta155
	 	 	\item[initial] 1048576
	 	 	\item[fixed] FALSE
	 	 	\item[prior] none
	 	 	\item[param] 
	 	 	\item[to.theta] \verb!function(x) x!
	 	 	\item[from.theta] \verb!function(x) x!
	 	 \end{description}
	 	\item[theta156]\ 
	 	 \begin{description}
	 	 	\item[hyperid] 29256
	 	 	\item[name] theta156
	 	 	\item[short.name] theta156
	 	 	\item[initial] 1048576
	 	 	\item[fixed] FALSE
	 	 	\item[prior] none
	 	 	\item[param] 
	 	 	\item[to.theta] \verb!function(x) x!
	 	 	\item[from.theta] \verb!function(x) x!
	 	 \end{description}
	 	\item[theta157]\ 
	 	 \begin{description}
	 	 	\item[hyperid] 29257
	 	 	\item[name] theta157
	 	 	\item[short.name] theta157
	 	 	\item[initial] 1048576
	 	 	\item[fixed] FALSE
	 	 	\item[prior] none
	 	 	\item[param] 
	 	 	\item[to.theta] \verb!function(x) x!
	 	 	\item[from.theta] \verb!function(x) x!
	 	 \end{description}
	 	\item[theta158]\ 
	 	 \begin{description}
	 	 	\item[hyperid] 29258
	 	 	\item[name] theta158
	 	 	\item[short.name] theta158
	 	 	\item[initial] 1048576
	 	 	\item[fixed] FALSE
	 	 	\item[prior] none
	 	 	\item[param] 
	 	 	\item[to.theta] \verb!function(x) x!
	 	 	\item[from.theta] \verb!function(x) x!
	 	 \end{description}
	 	\item[theta159]\ 
	 	 \begin{description}
	 	 	\item[hyperid] 29259
	 	 	\item[name] theta159
	 	 	\item[short.name] theta159
	 	 	\item[initial] 1048576
	 	 	\item[fixed] FALSE
	 	 	\item[prior] none
	 	 	\item[param] 
	 	 	\item[to.theta] \verb!function(x) x!
	 	 	\item[from.theta] \verb!function(x) x!
	 	 \end{description}
	 	\item[theta160]\ 
	 	 \begin{description}
	 	 	\item[hyperid] 29260
	 	 	\item[name] theta160
	 	 	\item[short.name] theta160
	 	 	\item[initial] 1048576
	 	 	\item[fixed] FALSE
	 	 	\item[prior] none
	 	 	\item[param] 
	 	 	\item[to.theta] \verb!function(x) x!
	 	 	\item[from.theta] \verb!function(x) x!
	 	 \end{description}
	 	\item[theta161]\ 
	 	 \begin{description}
	 	 	\item[hyperid] 29261
	 	 	\item[name] theta161
	 	 	\item[short.name] theta161
	 	 	\item[initial] 1048576
	 	 	\item[fixed] FALSE
	 	 	\item[prior] none
	 	 	\item[param] 
	 	 	\item[to.theta] \verb!function(x) x!
	 	 	\item[from.theta] \verb!function(x) x!
	 	 \end{description}
	 	\item[theta162]\ 
	 	 \begin{description}
	 	 	\item[hyperid] 29262
	 	 	\item[name] theta162
	 	 	\item[short.name] theta162
	 	 	\item[initial] 1048576
	 	 	\item[fixed] FALSE
	 	 	\item[prior] none
	 	 	\item[param] 
	 	 	\item[to.theta] \verb!function(x) x!
	 	 	\item[from.theta] \verb!function(x) x!
	 	 \end{description}
	 	\item[theta163]\ 
	 	 \begin{description}
	 	 	\item[hyperid] 29263
	 	 	\item[name] theta163
	 	 	\item[short.name] theta163
	 	 	\item[initial] 1048576
	 	 	\item[fixed] FALSE
	 	 	\item[prior] none
	 	 	\item[param] 
	 	 	\item[to.theta] \verb!function(x) x!
	 	 	\item[from.theta] \verb!function(x) x!
	 	 \end{description}
	 	\item[theta164]\ 
	 	 \begin{description}
	 	 	\item[hyperid] 29264
	 	 	\item[name] theta164
	 	 	\item[short.name] theta164
	 	 	\item[initial] 1048576
	 	 	\item[fixed] FALSE
	 	 	\item[prior] none
	 	 	\item[param] 
	 	 	\item[to.theta] \verb!function(x) x!
	 	 	\item[from.theta] \verb!function(x) x!
	 	 \end{description}
	 	\item[theta165]\ 
	 	 \begin{description}
	 	 	\item[hyperid] 29265
	 	 	\item[name] theta165
	 	 	\item[short.name] theta165
	 	 	\item[initial] 1048576
	 	 	\item[fixed] FALSE
	 	 	\item[prior] none
	 	 	\item[param] 
	 	 	\item[to.theta] \verb!function(x) x!
	 	 	\item[from.theta] \verb!function(x) x!
	 	 \end{description}
	 	\item[theta166]\ 
	 	 \begin{description}
	 	 	\item[hyperid] 29266
	 	 	\item[name] theta166
	 	 	\item[short.name] theta166
	 	 	\item[initial] 1048576
	 	 	\item[fixed] FALSE
	 	 	\item[prior] none
	 	 	\item[param] 
	 	 	\item[to.theta] \verb!function(x) x!
	 	 	\item[from.theta] \verb!function(x) x!
	 	 \end{description}
	 	\item[theta167]\ 
	 	 \begin{description}
	 	 	\item[hyperid] 29267
	 	 	\item[name] theta167
	 	 	\item[short.name] theta167
	 	 	\item[initial] 1048576
	 	 	\item[fixed] FALSE
	 	 	\item[prior] none
	 	 	\item[param] 
	 	 	\item[to.theta] \verb!function(x) x!
	 	 	\item[from.theta] \verb!function(x) x!
	 	 \end{description}
	 	\item[theta168]\ 
	 	 \begin{description}
	 	 	\item[hyperid] 29268
	 	 	\item[name] theta168
	 	 	\item[short.name] theta168
	 	 	\item[initial] 1048576
	 	 	\item[fixed] FALSE
	 	 	\item[prior] none
	 	 	\item[param] 
	 	 	\item[to.theta] \verb!function(x) x!
	 	 	\item[from.theta] \verb!function(x) x!
	 	 \end{description}
	 	\item[theta169]\ 
	 	 \begin{description}
	 	 	\item[hyperid] 29269
	 	 	\item[name] theta169
	 	 	\item[short.name] theta169
	 	 	\item[initial] 1048576
	 	 	\item[fixed] FALSE
	 	 	\item[prior] none
	 	 	\item[param] 
	 	 	\item[to.theta] \verb!function(x) x!
	 	 	\item[from.theta] \verb!function(x) x!
	 	 \end{description}
	 	\item[theta170]\ 
	 	 \begin{description}
	 	 	\item[hyperid] 29270
	 	 	\item[name] theta170
	 	 	\item[short.name] theta170
	 	 	\item[initial] 1048576
	 	 	\item[fixed] FALSE
	 	 	\item[prior] none
	 	 	\item[param] 
	 	 	\item[to.theta] \verb!function(x) x!
	 	 	\item[from.theta] \verb!function(x) x!
	 	 \end{description}
	 	\item[theta171]\ 
	 	 \begin{description}
	 	 	\item[hyperid] 29271
	 	 	\item[name] theta171
	 	 	\item[short.name] theta171
	 	 	\item[initial] 1048576
	 	 	\item[fixed] FALSE
	 	 	\item[prior] none
	 	 	\item[param] 
	 	 	\item[to.theta] \verb!function(x) x!
	 	 	\item[from.theta] \verb!function(x) x!
	 	 \end{description}
	 	\item[theta172]\ 
	 	 \begin{description}
	 	 	\item[hyperid] 29272
	 	 	\item[name] theta172
	 	 	\item[short.name] theta172
	 	 	\item[initial] 1048576
	 	 	\item[fixed] FALSE
	 	 	\item[prior] none
	 	 	\item[param] 
	 	 	\item[to.theta] \verb!function(x) x!
	 	 	\item[from.theta] \verb!function(x) x!
	 	 \end{description}
	 	\item[theta173]\ 
	 	 \begin{description}
	 	 	\item[hyperid] 29273
	 	 	\item[name] theta173
	 	 	\item[short.name] theta173
	 	 	\item[initial] 1048576
	 	 	\item[fixed] FALSE
	 	 	\item[prior] none
	 	 	\item[param] 
	 	 	\item[to.theta] \verb!function(x) x!
	 	 	\item[from.theta] \verb!function(x) x!
	 	 \end{description}
	 	\item[theta174]\ 
	 	 \begin{description}
	 	 	\item[hyperid] 29274
	 	 	\item[name] theta174
	 	 	\item[short.name] theta174
	 	 	\item[initial] 1048576
	 	 	\item[fixed] FALSE
	 	 	\item[prior] none
	 	 	\item[param] 
	 	 	\item[to.theta] \verb!function(x) x!
	 	 	\item[from.theta] \verb!function(x) x!
	 	 \end{description}
	 	\item[theta175]\ 
	 	 \begin{description}
	 	 	\item[hyperid] 29275
	 	 	\item[name] theta175
	 	 	\item[short.name] theta175
	 	 	\item[initial] 1048576
	 	 	\item[fixed] FALSE
	 	 	\item[prior] none
	 	 	\item[param] 
	 	 	\item[to.theta] \verb!function(x) x!
	 	 	\item[from.theta] \verb!function(x) x!
	 	 \end{description}
	 	\item[theta176]\ 
	 	 \begin{description}
	 	 	\item[hyperid] 29276
	 	 	\item[name] theta176
	 	 	\item[short.name] theta176
	 	 	\item[initial] 1048576
	 	 	\item[fixed] FALSE
	 	 	\item[prior] none
	 	 	\item[param] 
	 	 	\item[to.theta] \verb!function(x) x!
	 	 	\item[from.theta] \verb!function(x) x!
	 	 \end{description}
	 	\item[theta177]\ 
	 	 \begin{description}
	 	 	\item[hyperid] 29277
	 	 	\item[name] theta177
	 	 	\item[short.name] theta177
	 	 	\item[initial] 1048576
	 	 	\item[fixed] FALSE
	 	 	\item[prior] none
	 	 	\item[param] 
	 	 	\item[to.theta] \verb!function(x) x!
	 	 	\item[from.theta] \verb!function(x) x!
	 	 \end{description}
	 	\item[theta178]\ 
	 	 \begin{description}
	 	 	\item[hyperid] 29278
	 	 	\item[name] theta178
	 	 	\item[short.name] theta178
	 	 	\item[initial] 1048576
	 	 	\item[fixed] FALSE
	 	 	\item[prior] none
	 	 	\item[param] 
	 	 	\item[to.theta] \verb!function(x) x!
	 	 	\item[from.theta] \verb!function(x) x!
	 	 \end{description}
	 	\item[theta179]\ 
	 	 \begin{description}
	 	 	\item[hyperid] 29279
	 	 	\item[name] theta179
	 	 	\item[short.name] theta179
	 	 	\item[initial] 1048576
	 	 	\item[fixed] FALSE
	 	 	\item[prior] none
	 	 	\item[param] 
	 	 	\item[to.theta] \verb!function(x) x!
	 	 	\item[from.theta] \verb!function(x) x!
	 	 \end{description}
	 	\item[theta180]\ 
	 	 \begin{description}
	 	 	\item[hyperid] 29280
	 	 	\item[name] theta180
	 	 	\item[short.name] theta180
	 	 	\item[initial] 1048576
	 	 	\item[fixed] FALSE
	 	 	\item[prior] none
	 	 	\item[param] 
	 	 	\item[to.theta] \verb!function(x) x!
	 	 	\item[from.theta] \verb!function(x) x!
	 	 \end{description}
	 	\item[theta181]\ 
	 	 \begin{description}
	 	 	\item[hyperid] 29281
	 	 	\item[name] theta181
	 	 	\item[short.name] theta181
	 	 	\item[initial] 1048576
	 	 	\item[fixed] FALSE
	 	 	\item[prior] none
	 	 	\item[param] 
	 	 	\item[to.theta] \verb!function(x) x!
	 	 	\item[from.theta] \verb!function(x) x!
	 	 \end{description}
	 	\item[theta182]\ 
	 	 \begin{description}
	 	 	\item[hyperid] 29282
	 	 	\item[name] theta182
	 	 	\item[short.name] theta182
	 	 	\item[initial] 1048576
	 	 	\item[fixed] FALSE
	 	 	\item[prior] none
	 	 	\item[param] 
	 	 	\item[to.theta] \verb!function(x) x!
	 	 	\item[from.theta] \verb!function(x) x!
	 	 \end{description}
	 	\item[theta183]\ 
	 	 \begin{description}
	 	 	\item[hyperid] 29283
	 	 	\item[name] theta183
	 	 	\item[short.name] theta183
	 	 	\item[initial] 1048576
	 	 	\item[fixed] FALSE
	 	 	\item[prior] none
	 	 	\item[param] 
	 	 	\item[to.theta] \verb!function(x) x!
	 	 	\item[from.theta] \verb!function(x) x!
	 	 \end{description}
	 	\item[theta184]\ 
	 	 \begin{description}
	 	 	\item[hyperid] 29284
	 	 	\item[name] theta184
	 	 	\item[short.name] theta184
	 	 	\item[initial] 1048576
	 	 	\item[fixed] FALSE
	 	 	\item[prior] none
	 	 	\item[param] 
	 	 	\item[to.theta] \verb!function(x) x!
	 	 	\item[from.theta] \verb!function(x) x!
	 	 \end{description}
	 	\item[theta185]\ 
	 	 \begin{description}
	 	 	\item[hyperid] 29285
	 	 	\item[name] theta185
	 	 	\item[short.name] theta185
	 	 	\item[initial] 1048576
	 	 	\item[fixed] FALSE
	 	 	\item[prior] none
	 	 	\item[param] 
	 	 	\item[to.theta] \verb!function(x) x!
	 	 	\item[from.theta] \verb!function(x) x!
	 	 \end{description}
	 	\item[theta186]\ 
	 	 \begin{description}
	 	 	\item[hyperid] 29286
	 	 	\item[name] theta186
	 	 	\item[short.name] theta186
	 	 	\item[initial] 1048576
	 	 	\item[fixed] FALSE
	 	 	\item[prior] none
	 	 	\item[param] 
	 	 	\item[to.theta] \verb!function(x) x!
	 	 	\item[from.theta] \verb!function(x) x!
	 	 \end{description}
	 	\item[theta187]\ 
	 	 \begin{description}
	 	 	\item[hyperid] 29287
	 	 	\item[name] theta187
	 	 	\item[short.name] theta187
	 	 	\item[initial] 1048576
	 	 	\item[fixed] FALSE
	 	 	\item[prior] none
	 	 	\item[param] 
	 	 	\item[to.theta] \verb!function(x) x!
	 	 	\item[from.theta] \verb!function(x) x!
	 	 \end{description}
	 	\item[theta188]\ 
	 	 \begin{description}
	 	 	\item[hyperid] 29288
	 	 	\item[name] theta188
	 	 	\item[short.name] theta188
	 	 	\item[initial] 1048576
	 	 	\item[fixed] FALSE
	 	 	\item[prior] none
	 	 	\item[param] 
	 	 	\item[to.theta] \verb!function(x) x!
	 	 	\item[from.theta] \verb!function(x) x!
	 	 \end{description}
	 	\item[theta189]\ 
	 	 \begin{description}
	 	 	\item[hyperid] 29289
	 	 	\item[name] theta189
	 	 	\item[short.name] theta189
	 	 	\item[initial] 1048576
	 	 	\item[fixed] FALSE
	 	 	\item[prior] none
	 	 	\item[param] 
	 	 	\item[to.theta] \verb!function(x) x!
	 	 	\item[from.theta] \verb!function(x) x!
	 	 \end{description}
	 	\item[theta190]\ 
	 	 \begin{description}
	 	 	\item[hyperid] 29290
	 	 	\item[name] theta190
	 	 	\item[short.name] theta190
	 	 	\item[initial] 1048576
	 	 	\item[fixed] FALSE
	 	 	\item[prior] none
	 	 	\item[param] 
	 	 	\item[to.theta] \verb!function(x) x!
	 	 	\item[from.theta] \verb!function(x) x!
	 	 \end{description}
	 	\item[theta191]\ 
	 	 \begin{description}
	 	 	\item[hyperid] 29291
	 	 	\item[name] theta191
	 	 	\item[short.name] theta191
	 	 	\item[initial] 1048576
	 	 	\item[fixed] FALSE
	 	 	\item[prior] none
	 	 	\item[param] 
	 	 	\item[to.theta] \verb!function(x) x!
	 	 	\item[from.theta] \verb!function(x) x!
	 	 \end{description}
	 	\item[theta192]\ 
	 	 \begin{description}
	 	 	\item[hyperid] 29292
	 	 	\item[name] theta192
	 	 	\item[short.name] theta192
	 	 	\item[initial] 1048576
	 	 	\item[fixed] FALSE
	 	 	\item[prior] none
	 	 	\item[param] 
	 	 	\item[to.theta] \verb!function(x) x!
	 	 	\item[from.theta] \verb!function(x) x!
	 	 \end{description}
	 	\item[theta193]\ 
	 	 \begin{description}
	 	 	\item[hyperid] 29293
	 	 	\item[name] theta193
	 	 	\item[short.name] theta193
	 	 	\item[initial] 1048576
	 	 	\item[fixed] FALSE
	 	 	\item[prior] none
	 	 	\item[param] 
	 	 	\item[to.theta] \verb!function(x) x!
	 	 	\item[from.theta] \verb!function(x) x!
	 	 \end{description}
	 	\item[theta194]\ 
	 	 \begin{description}
	 	 	\item[hyperid] 29294
	 	 	\item[name] theta194
	 	 	\item[short.name] theta194
	 	 	\item[initial] 1048576
	 	 	\item[fixed] FALSE
	 	 	\item[prior] none
	 	 	\item[param] 
	 	 	\item[to.theta] \verb!function(x) x!
	 	 	\item[from.theta] \verb!function(x) x!
	 	 \end{description}
	 	\item[theta195]\ 
	 	 \begin{description}
	 	 	\item[hyperid] 29295
	 	 	\item[name] theta195
	 	 	\item[short.name] theta195
	 	 	\item[initial] 1048576
	 	 	\item[fixed] FALSE
	 	 	\item[prior] none
	 	 	\item[param] 
	 	 	\item[to.theta] \verb!function(x) x!
	 	 	\item[from.theta] \verb!function(x) x!
	 	 \end{description}
	 	\item[theta196]\ 
	 	 \begin{description}
	 	 	\item[hyperid] 29296
	 	 	\item[name] theta196
	 	 	\item[short.name] theta196
	 	 	\item[initial] 1048576
	 	 	\item[fixed] FALSE
	 	 	\item[prior] none
	 	 	\item[param] 
	 	 	\item[to.theta] \verb!function(x) x!
	 	 	\item[from.theta] \verb!function(x) x!
	 	 \end{description}
	 	\item[theta197]\ 
	 	 \begin{description}
	 	 	\item[hyperid] 29297
	 	 	\item[name] theta197
	 	 	\item[short.name] theta197
	 	 	\item[initial] 1048576
	 	 	\item[fixed] FALSE
	 	 	\item[prior] none
	 	 	\item[param] 
	 	 	\item[to.theta] \verb!function(x) x!
	 	 	\item[from.theta] \verb!function(x) x!
	 	 \end{description}
	 	\item[theta198]\ 
	 	 \begin{description}
	 	 	\item[hyperid] 29298
	 	 	\item[name] theta198
	 	 	\item[short.name] theta198
	 	 	\item[initial] 1048576
	 	 	\item[fixed] FALSE
	 	 	\item[prior] none
	 	 	\item[param] 
	 	 	\item[to.theta] \verb!function(x) x!
	 	 	\item[from.theta] \verb!function(x) x!
	 	 \end{description}
	 	\item[theta199]\ 
	 	 \begin{description}
	 	 	\item[hyperid] 29299
	 	 	\item[name] theta199
	 	 	\item[short.name] theta199
	 	 	\item[initial] 1048576
	 	 	\item[fixed] FALSE
	 	 	\item[prior] none
	 	 	\item[param] 
	 	 	\item[to.theta] \verb!function(x) x!
	 	 	\item[from.theta] \verb!function(x) x!
	 	 \end{description}
	 	\item[theta200]\ 
	 	 \begin{description}
	 	 	\item[hyperid] 29300
	 	 	\item[name] theta200
	 	 	\item[short.name] theta200
	 	 	\item[initial] 1048576
	 	 	\item[fixed] FALSE
	 	 	\item[prior] none
	 	 	\item[param] 
	 	 	\item[to.theta] \verb!function(x) x!
	 	 	\item[from.theta] \verb!function(x) x!
	 	 \end{description}
	 	\item[theta201]\ 
	 	 \begin{description}
	 	 	\item[hyperid] 29301
	 	 	\item[name] theta201
	 	 	\item[short.name] theta201
	 	 	\item[initial] 1048576
	 	 	\item[fixed] FALSE
	 	 	\item[prior] none
	 	 	\item[param] 
	 	 	\item[to.theta] \verb!function(x) x!
	 	 	\item[from.theta] \verb!function(x) x!
	 	 \end{description}
	 	\item[theta202]\ 
	 	 \begin{description}
	 	 	\item[hyperid] 29302
	 	 	\item[name] theta202
	 	 	\item[short.name] theta202
	 	 	\item[initial] 1048576
	 	 	\item[fixed] FALSE
	 	 	\item[prior] none
	 	 	\item[param] 
	 	 	\item[to.theta] \verb!function(x) x!
	 	 	\item[from.theta] \verb!function(x) x!
	 	 \end{description}
	 	\item[theta203]\ 
	 	 \begin{description}
	 	 	\item[hyperid] 29303
	 	 	\item[name] theta203
	 	 	\item[short.name] theta203
	 	 	\item[initial] 1048576
	 	 	\item[fixed] FALSE
	 	 	\item[prior] none
	 	 	\item[param] 
	 	 	\item[to.theta] \verb!function(x) x!
	 	 	\item[from.theta] \verb!function(x) x!
	 	 \end{description}
	 	\item[theta204]\ 
	 	 \begin{description}
	 	 	\item[hyperid] 29304
	 	 	\item[name] theta204
	 	 	\item[short.name] theta204
	 	 	\item[initial] 1048576
	 	 	\item[fixed] FALSE
	 	 	\item[prior] none
	 	 	\item[param] 
	 	 	\item[to.theta] \verb!function(x) x!
	 	 	\item[from.theta] \verb!function(x) x!
	 	 \end{description}
	 	\item[theta205]\ 
	 	 \begin{description}
	 	 	\item[hyperid] 29305
	 	 	\item[name] theta205
	 	 	\item[short.name] theta205
	 	 	\item[initial] 1048576
	 	 	\item[fixed] FALSE
	 	 	\item[prior] none
	 	 	\item[param] 
	 	 	\item[to.theta] \verb!function(x) x!
	 	 	\item[from.theta] \verb!function(x) x!
	 	 \end{description}
	 	\item[theta206]\ 
	 	 \begin{description}
	 	 	\item[hyperid] 29306
	 	 	\item[name] theta206
	 	 	\item[short.name] theta206
	 	 	\item[initial] 1048576
	 	 	\item[fixed] FALSE
	 	 	\item[prior] none
	 	 	\item[param] 
	 	 	\item[to.theta] \verb!function(x) x!
	 	 	\item[from.theta] \verb!function(x) x!
	 	 \end{description}
	 	\item[theta207]\ 
	 	 \begin{description}
	 	 	\item[hyperid] 29307
	 	 	\item[name] theta207
	 	 	\item[short.name] theta207
	 	 	\item[initial] 1048576
	 	 	\item[fixed] FALSE
	 	 	\item[prior] none
	 	 	\item[param] 
	 	 	\item[to.theta] \verb!function(x) x!
	 	 	\item[from.theta] \verb!function(x) x!
	 	 \end{description}
	 	\item[theta208]\ 
	 	 \begin{description}
	 	 	\item[hyperid] 29308
	 	 	\item[name] theta208
	 	 	\item[short.name] theta208
	 	 	\item[initial] 1048576
	 	 	\item[fixed] FALSE
	 	 	\item[prior] none
	 	 	\item[param] 
	 	 	\item[to.theta] \verb!function(x) x!
	 	 	\item[from.theta] \verb!function(x) x!
	 	 \end{description}
	 	\item[theta209]\ 
	 	 \begin{description}
	 	 	\item[hyperid] 29309
	 	 	\item[name] theta209
	 	 	\item[short.name] theta209
	 	 	\item[initial] 1048576
	 	 	\item[fixed] FALSE
	 	 	\item[prior] none
	 	 	\item[param] 
	 	 	\item[to.theta] \verb!function(x) x!
	 	 	\item[from.theta] \verb!function(x) x!
	 	 \end{description}
	 	\item[theta210]\ 
	 	 \begin{description}
	 	 	\item[hyperid] 29310
	 	 	\item[name] theta210
	 	 	\item[short.name] theta210
	 	 	\item[initial] 1048576
	 	 	\item[fixed] FALSE
	 	 	\item[prior] none
	 	 	\item[param] 
	 	 	\item[to.theta] \verb!function(x) x!
	 	 	\item[from.theta] \verb!function(x) x!
	 	 \end{description}
	 \end{description}
	\item[constr] FALSE
	\item[nrow.ncol] FALSE
	\item[augmented] TRUE
	\item[aug.factor] 1
	\item[aug.constr] 1 2 3 4 5 6 7 8 9 10 11 12 13 14 15 16 17 18 19 20
	\item[n.div.by] -1
	\item[n.required] TRUE
	\item[set.default.values] TRUE
	\item[status] experimental
	\item[pdf] iidkd
\end{description}



\subsection*{Example}

Just simulate some data and estimate the parameters back. This is for
\texttt{order=}$4$.

{\small\verbatiminput{example-iidkd.R}}

\end{document}

\documentclass[a4paper,11pt]{article}
\usepackage[scale={0.8,0.9},centering,includeheadfoot]{geometry}
\usepackage{amstext}
\usepackage{listings}
\begin{document}

\section*{Bym model for spatial effects}

\subsection*{Parametrization}

This model is simply a union of the  \lstinline$besag$ model $u$ and a
\lstinline$iid$ model $v$, so that
\begin{displaymath}
    x =
    \begin{pmatrix}
        v + u\\
        u
    \end{pmatrix}
\end{displaymath}
Note that the length of $x$ is $2n$ if the length of $u$ (and $v$) is
$n$. The benefite is that this allows to get the posterior marginals
of the sum of the spatial and iid model; otherwise it offers no
advantages.

\subsection*{Hyperparameters}
The hyperparameters are the precision $\tau_1$ of the \lstinline$iid$
model ($v$) and the precision $\tau_2$ of the \lstinline$besag$ model
($u$).  The precision parameters are represented as
\begin{displaymath}
    \theta=(\theta_1,\theta_2) =(\log \tau_1,\log \tau_2)
\end{displaymath}
and the prior is defined on $\mathbf{\theta}$.

\subsection*{Specification}

The bym model is specified inside the {\tt f()} function as
\begin{verbatim}
 f(<whatever>,model="bym",graph=<graph>,
   hyper=<hyper>, adjust.for.con.comp = TRUE,
   scale.model = FALSE)
\end{verbatim}

The neighbourhood structure of $\mathbf{x}$ is passed to the program
through the {\tt graph} argument.

The option \verb|adjust.for.con.comp| adjust the model if the graph
has more than one connected compoment, and this adjustment can be
disabled setting this option to \texttt{FALSE}. This means that
\texttt{constr=TRUE} is interpreted as a sum-to-zero constraint on
\emph{each} connected component and the \texttt{rankdef} parameter is
set accordingly. 

The logical option \verb|scale.model| determine if the
besag-model-part of the model $u$ should be scaled to have an average
variance (the diagonal of the generalized inverse) equal to 1. This
makes prior spesification much easier. Default is \verb|FALSE| so that
the model is not scaled.


\subsubsection*{Hyperparameter spesification and default values}
\begin{description}
	\item[hyper]\ 
	 \begin{description}
	 	\item[theta1]\ 
	 	 \begin{description}
	 	 	 \item[ name ] log unstructured precision 
	 	 	 \item[ short.name ] prec.unstruct 
	 	 	 \item[ prior ] loggamma 
	 	 	 \item[ param ] 1 0.001 
	 	 	 \item[ initial ] 4 
	 	 	 \item[ fixed ] FALSE 
	 	 	 \item[ to.theta ] \verb|function(x) log(x)| 
	 	 	 \item[ from.theta ] \verb|function(x) exp(x)| 
	 	 \end{description}
	 	\item[theta2]\ 
	 	 \begin{description}
	 	 	 \item[ name ] log spatial precision 
	 	 	 \item[ short.name ] prec.spatial 
	 	 	 \item[ prior ] normal 
	 	 	 \item[ param ] 0 5e-05 
	 	 	 \item[ initial ] 4 
	 	 	 \item[ fixed ] FALSE 
	 	 	 \item[ to.theta ] \verb|function(x) log(x)| 
	 	 	 \item[ from.theta ] \verb|function(x) exp(x)| 
	 	 \end{description}
	 \end{description}
	 \item[ constr ] TRUE 
	 \item[ nrow.ncol ] FALSE 
	 \item[ augmented ] TRUE 
	 \item[ aug.factor ] 2 
	 \item[ aug.constr ] 2 
	 \item[ n.div.by ]  
	 \item[ n.required ] TRUE 
	 \item[ set.default.values ] TRUE 
\end{description}



\subsection*{Example}

For examples of application of this model see the {\tt Bym} example in
Volume I.

\subsection*{Notes}

None

\end{document}


% LocalWords: 

%%% Local Variables: 
%%% TeX-master: t
%%% End: 

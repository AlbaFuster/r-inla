\documentclass{article}
\usepackage{amsmath}
\begin{document}

\title{Saturated DIC}
\maketitle

Let $\theta$ be the hyperparameters and $\eta$ the linear predictor
and $y$ the data. The Saturated DIC in the INLA output, is computed
from the following. Define the expected saturated deviance for $y_i$,
as
\begin{displaymath}
    -2\left(\int \log\pi(y_i|\eta_i,\theta)\; \pi(\eta_i|y,\theta)
      \pi(\theta|y) d\eta_i d\theta
      - \log\pi(y_i|\eta_i^{*}, \theta^{*}) \right)
\end{displaymath}
and the saturated deviance at the expected value, as
\begin{displaymath}
    -2\left(\log\pi(y_i|\eta_i^{e},\theta^{*}) 
      - \log\pi(y_i|\eta_i^{*}, \theta^{*})\right)
\end{displaymath}
where
\begin{displaymath}
    \eta_i^{e}=\text{E}(\eta_i|y)
\end{displaymath}
\begin{displaymath}
    \theta^{*} = \arg\max_{\theta} \pi(\theta|y)
\end{displaymath}
and
\begin{displaymath}
    \eta_i^{*} = \arg\max_{\eta_i}\pi(y_i|\eta_i, \theta^{*})
\end{displaymath}
By taking the sum over all data, we can comupute the (Saturated) DIC.

The mode $\theta^{*}$ is used as for some parameters, posterior
expectation of the hyperparameters in the user-scale, does not make
much sense (the tail is to heavy). So, using the mode seems more
robust and sensible.`

\end{document}



% LocalWords: 

%%% Local Variables: 
%%% TeX-master: t
%%% End: 
